\documentclass[letterpaper]{article}

\usepackage[utf8x]{luainputenc}
\usepackage{aeguill}
%\usepackage{nopageno}
\usepackage{amsmath}
\usepackage{amssymb}
\usepackage{mathrsfs}
\usepackage{fullpage}
\usepackage{fancyhdr}
\setlength{\headheight}{12pt}
\pagestyle{fancy}
\chead{Linear Algebra}
\lhead{September 9, 2015}
\rhead{Jon Allen}
\allowdisplaybreaks

\newcommand{\abs}[1]{\left\lvert #1 \right\rvert}

\begin{document}
\renewcommand{\labelenumi}{1.\arabic{enumi}}
\renewcommand{\labelenumii}{\arabic{enumii}.}
\renewcommand{\labelenumiii}{\alph{enumiii}.}
\begin{enumerate}
\setcounter{enumi}{1}
\item
  \begin{enumerate}
  \item
    For each of the following pairs of vectors {\bfseries x} and {\bfseries y}, calculate $\mathbf{x}\cdot\mathbf{y}$ and the  angle $\theta$ between the vectors.
    \begin{enumerate}
    \setcounter{enumiii}{3}
    \item
      $\mathbf{x}=(1,4,-3),\mathbf{y}=(5,1,3)$
      \begin{align*}
        ||\mathbf{x}||||\mathbf{y}||\cos \theta&=\mathbf{x\cdot y}&
        \cos \theta&=\frac{\mathbf{x\cdot y}}{||\mathbf{x}||||\mathbf{y}||}&
        \theta&=\cos^{-1}\frac{\mathbf{x\cdot y}}{||\mathbf{x}||||\mathbf{y}||}\\
        \theta&=\cos^{-1}\frac{5+4-9}{||\mathbf{x}||||\mathbf{y}||}&
        \theta&=\cos^{-1}0&
        \theta&=\frac{\pi}{2}
      \end{align*}
    \end{enumerate}
  \item
    For each pair in exercise 1, calculate $\text{proj}_{\mathbf{y}}\mathbf{x}$ and $\text{proj}_{\mathbf{x}}\mathbf{y}$
    \begin{enumerate}
    \setcounter{enumiii}{3}
    \item
      The vectors are orthogonal. The projection of either onto the other is the zero vector.
    \end{enumerate}
  \setcounter{enumii}{6}
  \item
    Suppose $\mathbf{x}$, $\mathbf{y}\in \mathbb{R}^n, ||\mathbf{x}||=\sqrt{2}, ||\mathbf{y}||=1$, and the angle between $x$ and $y$ is $3\pi/4$. Show that the vectors $2\mathbf{x}+3\mathbf{y}$ and $\mathbf{x}-\mathbf{y}$ are orthogonal.

    Using proposition 2.1 we see that $(2\mathbf{x}+3\mathbf{y})\cdot(\mathbf{x}-\mathbf{y})=2\mathbf{x}\cdot\mathbf{x}-2\mathbf{x}\cdot\mathbf{y}+3\mathbf{y}\cdot\mathbf{x}-3\mathbf{y}\cdot\mathbf{y}=2||\mathbf{x}||^2+\mathbf{x}\cdot\mathbf{y}-3||\mathbf{y}||^2$. From the definition of the angle between two vectors we can further simplify this expression to $2\cdot 2+||\mathbf{x}||||\mathbf{y}||\cos \frac{3\pi}{4}+3=4+\sqrt{2}\cdot(-\frac{\sqrt{2}}{2})-3=0$ and so the vectors are orthogonal.
  \setcounter{enumii}{9}
  \item
    Let $\mathbf{x}=(1,1,1,\dots,1)\in \mathbb{R}^n$ and $\mathbf{y}=(1,2,3,\dots,n)\in \mathbb{R}^n$. Let $\theta_n$ be the angle between $\mathbf{x}$ and $\mathbf{y}$ in $\mathbb{R}^n$. Find $\displaystyle \lim_{n\to \infty}\theta_n$. (The formulas $1+2+\dots+n=n(n+1)/2$ and $1^2+2^2+\dots+n^2=n(n+1)(2n+1)/6$ may be useful.)

    We know that by definition $\theta_n=\cos^{-1}\frac{\mathbf{x}\cdot\mathbf{y}}{||\mathbf{x}||||\mathbf{y}||}$ and so
    \begin{align*}
      \lim_{n\to\infty}\theta_n&=\lim_{n\to\infty}\cos^{-1}\frac{\mathbf{x}\cdot\mathbf{y}}{||\mathbf{x}||||\mathbf{y}||}\\
      &=\lim_{n\to\infty}\cos^{-1}\frac{1\cdot1+1\cdot2+\dots+1\cdot n}{\sqrt{n}\sqrt{1^2+2^2+\cdots+n^2}}\\
      &=\lim_{n\to\infty}\cos^{-1}\frac{\sqrt{6}n(n+1)}{2\sqrt{n}\sqrt{n(n+1)(2n+1)}}\\
      &=\lim_{n\to\infty}\cos^{-1}\sqrt{\frac{3(n+1)}{2(2n+1)}}
      =\lim_{n\to\infty}\cos^{-1}\sqrt{\frac{3(n+1/2+1/2)}{4(n+1/2)}}\\
      &=\lim_{n\to\infty}\cos^{-1}\sqrt{\frac{3}{4}\left(1+\frac{1/2}{n+1/2}\right)}
      =\cos^{-1}\sqrt{\frac{3}{4}\left(1+\lim_{n\to\infty}\frac{1/2}{n+1/2}\right)}\\
      &=\cos^{-1}\sqrt{\frac{3}{4}\left(1+0\right)}
      =\frac{\pi}{6}
    \end{align*}
  \item
    Suppose $\mathbf{x}, \mathbf{v}_1,\dots,\mathbf{v}_k\in \mathbb{R}^n$ and $\mathbf{x}$ is orthogonal to each of the vectors $\mathbf{v}_1,\dots\mathbf{v}_k$. Show that $\mathbf{x}$ is orthogonal to any linear combination $c_1\mathbf{v}_1+c_2\mathbf{v}_2+\dots+c_k\mathbf{v}_k$.

    We know that $\mathbf{x}$ is orthogonal if and only if the dot product is zero. So lets just find it.
    \begin{align*}
      \mathbf{x}\cdot(c_1\mathbf{v}_1+\dots+c_k\mathbf{v}_k)
      &=\mathbf{x}\cdot(c_1\mathbf{v}_1)+\dots+\mathbf{x}\cdot(c_k\mathbf{v}_k)\\
      &=c_1(\mathbf{x}\cdot\mathbf{v}_1)+\dots+c_k(\mathbf{x}\cdot\mathbf{v}_k)
    \end{align*}
    But then $\mathbf{x}$ is orthogonal to $\mathbf{v}_i$ for all $0<i\le k$. Which leads us to $\mathbf{x}\cdot\mathbf{v}_i=0$ and $\mathbf{x}\cdot(c_1\mathbf{v}_1+\dots+c_k\mathbf{v}_k)=c_1\cdot0+\dots+c_k\cdot0=0$. And we have our result.
  \setcounter{enumii}{12}
  \item
    Use the algebraic properties of the dot product to show that
    \[||\mathbf{x}+\mathbf{y}||^2+||\mathbf{x}-\mathbf{y}||^2=2(||\mathbf{x}||^2+||\mathbf{y}||^2).\]
    Interpret the result geometrically.
    \begin{align*}
      ||\mathbf{x}+\mathbf{y}||^2+||\mathbf{x}-\mathbf{y}||^2
      &=||\mathbf{x}+\mathbf{y}||^2+||\mathbf{x}+(-\mathbf{y})||^2\\
      &=(||\mathbf{x}||^2+2\mathbf{x}\cdot\mathbf{y}+||\mathbf{y}||^2)+(||\mathbf{x}||^2+2\mathbf{x}\cdot(-\mathbf{y})+||-\mathbf{y}||^2)\\
      &=||\mathbf{x}||^2+2\mathbf{x}\cdot\mathbf{y}+||\mathbf{y}||^2+||\mathbf{x}||^2-2\mathbf{x}\cdot\mathbf{y}+||\mathbf{y}||^2\\
      &=||\mathbf{x}||^2+||\mathbf{y}||^2+||\mathbf{x}||^2+||\mathbf{y}||^2\\
      &=2(||\mathbf{x}||^2+||\mathbf{y}||^2)
    \end{align*}
  \item
    Use the dot product to prove the law of cosines: As shown in Figure 2.8.
    \[c^2=a^2+b^2-2ab\cos \theta\]

    Let $\overline{CB}=\mathbf{a}, \overline{CA}=\mathbf{b},$ and $\overline{BA}=\mathbf{c}$.
    Notice that $\mathbf{c}=\mathbf{b}-\mathbf{a}$.
    And so $c=||\mathbf{b}-\mathbf{a}||$, $a=||\mathbf{a}||$, and $b=||\mathbf{b}||$.
    Using corollary 2.3 from the notes and definition 2.9 from the notes we have
    \begin{align*}
      c^2&=||\mathbf{b}-\mathbf{a}||^2\\
      &=||\mathbf{b}||^2-2\mathbf{b}\cdot\mathbf{a}+||\mathbf{a}||^2\\
      &=||\mathbf{a}||^2+||\mathbf{b}||^2-2\mathbf{b}\cdot\mathbf{a}\frac{||\mathbf{a}||||\mathbf{b}||}{||\mathbf{a}||||\mathbf{b}||}\\
      &=a^2+b^2-2ab\frac{\mathbf{a}\cdot\mathbf{b}}{||\mathbf{a}||||\mathbf{b}||}\\
      &=a^2+b^2-2ab\cos \theta\\
    \end{align*}
    Boom.$\Box$
  \setcounter{enumii}{16}
  \item
    If $\mathbf{x}=(x_1,x_2)\in \mathbb{R}^2$, set $\rho(\mathbf{x})=(-x_2,x_1)$.
    \begin{enumerate}
    \item
      Check that $\rho(\mathbf{x})$ is orthogonal to $\mathbf{x}$. (Indeed, $\rho(\mathbf{x})$ is obtained by rotating $\mathbf{x}$ an angle $\pi/2$ counterclockwise.)
      \[\mathbf{x}\cdot\rho(\mathbf{x})=(x_1,x_2)\cdot(-x_2,x_1)=-x_1x_2+x_1x_2=0\]
    \item
      Given $\mathbf{x},\mathbf{y}\in\mathbb{R}^2$,show that $\mathbf{x}\cdot\rho(\mathbf{y})=-\rho(\mathbf{x})\cdot\mathbf{y}$. Interpret this statement geometrically.

      Let $\mathbf{y}=(y_1,y_2)$. Then \[\mathbf{x}\cdot\rho(\mathbf{y})=(x_1,x_2)\cdot(-y_2,y_1)=-x_1y_2+x_2y_1=-(-x_2y_1+x_1y_2)=-\rho(\mathbf{x})\cdot\mathbf{y}\]
    \end{enumerate}
  \item
    Prove the \emph{triangle inequality}: For any vectors $\mathbf{x}, \mathbf{y}\in \mathbb{R}^n, ||\mathbf{x}+\mathbf{y}||\le ||\mathbf{x}||+||\mathbf{y}||$. (\emph{Hint:} Use the dot product to calculate $||\mathbf{x}+\mathbf{y}||^2$.)

    We know from Cauchy-Schwartz that $|\mathbf{x}\cdot\mathbf{y}|\le||\mathbf{x}||||\mathbf{y}|||$ and that $\mathbf{x}\cdot\mathbf{y}\le |\mathbf{x}\cdot\mathbf{y}|$.
    Double both sides and we have $2\mathbf{x}\cdot\mathbf{y}\le 2||\mathbf{x}||||\mathbf{y}||$.
    Of course $\mathbf{x}\cdot\mathbf{x}=||\mathbf{x}||^2\ge 0$ and $\mathbf{y}\cdot\mathbf{y}=||\mathbf{y}||^2\ge 0$ which leads us to $\mathbf{x}\cdot\mathbf{x}+2\mathbf{x}\cdot\mathbf{y}+\mathbf{y}\cdot\mathbf{y}\le ||\mathbf{x}||^2+2||\mathbf{x}||||\mathbf{y}||+||\mathbf{y}||^2$.
    Factoring we get $(\mathbf{x}+\mathbf{y})\cdot(\mathbf{x}+\mathbf{y})=||\mathbf{x}+\mathbf{y}||^2\le (||\mathbf{x}||+||\mathbf{y}||)^2$.
    And finally, because $||\mathbf{x}+\mathbf{y}||\ge 0$ and $||\mathbf{x}||+||\mathbf{y}||\ge 0$ it is safe to say that $||\mathbf{x}+\mathbf{y}||\le ||\mathbf{x}||+||\mathbf{y}||$
  \end{enumerate}
\item
  \begin{enumerate}
  \setcounter{enumii}{3}
  \item
    Find a normal vector to the given hyperplane and use it to find the distance from the origin to the hyperplane.
    \begin{enumerate}
    \item
      $\mathbf{x}=(-1,2)+t(3,2)$
      \begin{align*}
        (-2,3)\cdot\mathbf{x}&=(-2,3)\cdot(-1,2)+t(3,2)\cdot(-2,3)\\
        (-2,3)\cdot\mathbf{x}&=7\\
        ||\text{proj}_{\mathbf{a}}\mathbf{x}_0||&=\frac{|c|}{||\mathbf{a}||}=\frac{7}{\sqrt{4+9}}=\frac{7\sqrt{13}}{13}
      \end{align*}
      And so we have a normal vector of $(-2,3)$ and a distance of $\frac{7\sqrt{13}}{13}$
    \item
      The plane in $\mathbb{R}^3$ given by the equation $2x_1+x_2-x_3=5$
      \begin{align*}
        (2,1,-1)\cdot(x_1,x_2,x_3)&=5\\
        ||\text{proj}_{\mathbf{a}}\mathbf{x}_0||&=\frac{5}{\sqrt{4+1+1}}
      \end{align*}
      And so we have a normal of $(2,1,-1)$ and a distance of $\frac{5\sqrt{6}}{6}$
    \item
      The plane passing through $(1,2,2)$ and orthogonal to the line $\mathbf{x}=(3,1,-1)+t(-1,1,-1)$

      \begin{align*}
        (-1,1,-1)\cdot\mathbf{x}&=(-1,1,-1)\cdot(1,2,2)=-1\\
        ||\text{proj}_{\mathbf{a}}\mathbf{x}_0||&=\frac{1}{\sqrt{1+1+1}}\\
      \end{align*}
      Looks like a normal of $(-1,1,-1)$ and a distance of $\frac{\sqrt{3}}{3}$
    \item
      The plane passing through $(2,-1,1)$ and orthogonal to the  line $\mathbf{x}=(3,1,1)+t(-1,2,1)$

      The normal is $(-1,2,1)$ and has a distance of $\frac{|-2-2+1|}{\sqrt{4+1+1}}=\frac{3\sqrt{6}}{6}$
    \item
      The plane spanned by $(1,1,4)$ and $(2,1,0)$ and passing through $(1,1,2)$

      \begin{align*}
        a_1+a_2+4a_3&=0&
        2a_1+a_2&=0\\
        a_1-2a_1+4a_3&=0&
        4a_3&=a_1\\
        a_3&=1&
        a_1&=4\\
        a_2&=-8&
        \mathbf{a}&=(4,-8,1)
      \end{align*}
      So our normal vector is $(4,-8,1)$ and our distance is $\frac{|4-8+2|}{\sqrt{16+64+1}}=\frac{2}{9}$
    \item
      The plane spanned by $(1,1,1)$ and $(2,1,0)$ and passing through $(3,0,2)$
      \begin{align*}
        a_1+a_2+a_3&=0&
        2a_1+a_2&=0\\
        a_1&=1&
        a_2&=-2\\
        a_3&=1
      \end{align*}
      Normal is $(1,-2,1)$ and distance is $\frac{3+2}{\sqrt{1+4+1}}=\frac{5\sqrt{6}}{6}$.
    \item
      The hyperplane in $\mathbb{R}^4$ spanned by $(1,-1,1,-1),(1,1,-1,-1)$ and $(1,-1,-1,1)$ and passing through $(2,1,0,1)$

      \begin{align*}
        a_1-a_2-a_3+a_4&=0&
        a_1-a_2+a_3-a_4&=0&
        a_1+a_2-a_3-a_4&=0\\
        a_1-a_2+a_3&=a_4&
        a_1+a_2-a_3&=a_4&
        2a_3&=2a_2\\
        a_1-a_2-a_2+(a_1-a_2+a_2)&=0&
        2a_1&=2a_2&
        a4&=a_1+a_1-a_1\\
      \end{align*}
      So we let $(1,1,1,1)$ be the normal vector and $\frac{|2+1+0+1|}{\sqrt{4}}=2$
    \end{enumerate}
  \setcounter{enumii}{5}
  \item
    \begin{enumerate}
    \item
      Give the general solution of the equation $x_1+5x_2-2x_3=0$ in $\mathbb{R}^3$ (as a linear combination of two vectors, as in the text).

      Note that $\mathbf{0}$ is on the plane and so we just need to find two vectors that aren't parallel and are on the plane, and we have our solution. Lets take $(-3,1,1)$ and $(2,0,1)$. Then our equation is $\mathbf{x}=x_2(-3,1,1)+x_3(2,0,1)$
    \item
      Find a specific solution of the equation $x_1+5x_2-2x_3=3$ in $\mathbb{R}^3$; give the general solution.

      $\mathbf{x}=(0,1,1)$ is a specific solution to the equation. Combining the specific solution with part $a$ we have $\mathbf{x}=(0,1,1)+x_2(-3,1,1)+x_3(2,0,1)$
    \item
      Give the general solution of the equation $x_1+5x_2-2x_3+x_4=0$ in $\mathbb{R}^4$. Now give the general solution of the equation $x_1+5x_2-2x_3+x_4=3$
      
      As before we notice that we are going through $\mathbf{0}$, so we find three vectors that are independent and we are golden. We can even steal two of them from the previous part. Take $(-3,1,1,0), (2,0,1,0)$ and $(0,1,1,-3)$. These are obviously all on our hyperplane, and we can easily see that they are independent. So for the first part we have $\mathbf{x}=x_2(-3,1,1,0)+x_3(2,0,1,0)+x_4(0,1,1,-3)$. For the second part we just need a point on the plane. we see that $3+5\cdot0-2\cdot0+0=3$ so we use $(3,0,0,0)$ and this gives us $\mathbf{x}=(3,0,0,0)+x_2(-3,1,1,0)+x_3(2,0,1,0)+x_4(0,1,1,-3)$
    \end{enumerate}
  \item
    The equation $2x_1-3x_2=5$ defines a line in $\mathbb{R}^2$.
    \begin{enumerate}
    \item
      Give a normal vector $\mathbf{a}$ to the line.

      $2x_1-3x_2=(2,-3)\cdot(x_1,x_2)$ so our normal vector is $(2,-3)$
    \item
      Find the distance from the origin to the line by using projection.

      $||\text{proj}_\mathbf{a}\mathbf{x}||=\frac{5}{\sqrt{4+9}}=\frac{5\sqrt{13}}{13}$
    \item
      Find the point on the line closest to the origin by using the parametric equation of the line through $\mathbf{0}$ with direction vector $\mathbf{a}$. Double-check your anser to part $b$.
      \begin{align*}
        \mathbf{x}&=t(2,-3)&2(2t)-3(-3t)&=5\\
        13t&=5&
        ||\mathbf{x}||&=\sqrt{\left(\frac{2\cdot5}{13}\right)^2+\left(\frac{-3\cdot5}{13}\right)^2}
        =\frac{5}{13}\sqrt{13}
      \end{align*}
    \item
      Find the distance from the point $\mathbf{w}=(3,1)$ to the line by using projection.

      \begin{align*}
        \left\lvert||\text{proj}_{\mathbf{a}}\mathbf{w}||-||\text{proj}_{\mathbf{a}}\mathbf{\mathbf{x}}||\right\rvert&=\left\lvert\left\lvert\left\lvert\frac{\mathbf{w}\cdot\mathbf{a}}{||\mathbf{a}||^2}\mathbf{a}\right\rvert\right\rvert-\frac{5\sqrt{13}}{13}\right\rvert\\
        &=\left\lvert\frac{|6-3|}{||(2,-3||}-\frac{5\sqrt{13}}{13}\right\rvert\\
        &=\left\lvert\frac{3}{\sqrt{13}}-\frac{5\sqrt{13}}{13}\right\rvert=\frac{2\sqrt{13}}{13}\\
      \end{align*}
    \item
      Find the point on the line closest to $\mathbf{w}$ by using the parametric equation of the line through $\mathbf{w}$ with direction vector $\mathbf{a}$. Double-check your answer to part $d$

      \begin{align*}
      (3,1)+t(2,-3)=(2t+3,-3t+1)&=\mathbf{x}&
      2(2t+3)-3(-3t+1)&=5\\
      13t&=2\\
      \left\lvert\left\lvert\left(2\frac{2}{13}+3,-3\frac{2}{13}+1\right)-\mathbf{w}\right\rvert\right\rvert&=\sqrt{\frac{4^2+(-6)^2}{13^2}}\\
      &=\frac{\sqrt{2^2(2^2+3^2)}}{13}\\
      &=\frac{2\sqrt{13}}{13}
      \end{align*}
    \end{enumerate}
  \setcounter{enumii}{8}
  \item
    The equation $2x_1+2x_2-3x_3+8x_4=6$ defines a hyperplane in $\mathbb{R}^4$.
    \begin{enumerate}
    \item
      Give a normal vector $\mathbf{a}$ to the hyperplane.

      $2x_1+2x_2-3x_3+8x_4=(2,2,-3,8)\cdot(x_1,x_2,x_3,x_4)$ and so $\mathbf{a}=(2,2,-3,8)$
    \item
      Find the distance from the origin to the hyperplane using projection.

      $||\text{proj}_{\mathbf{a}}\mathbf{x}||=\frac{6}{\sqrt{4+4+9+64}}=\frac{2}{3}$
    \item
      Find the point on the plane closest to the origin by using the parametric equation of the line through $\mathbf{0}$ with direction vector $\mathbf{a}$. Double-check your answer to part $b$.

      \begin{align*}
        \mathbf{x}&=t(2,2,-3,8)&2(2t)+2(2t)-3(-3t)+8(8t)&=6\\
        &&81t&=6\\
        \mathbf{x}_0&=\left\lvert\left\lvert\left(\frac{12}{81},\frac{12}{81},\frac{-18}{81},\frac{48}{81}\right)\right\rvert\right\rvert&=\sqrt{\frac{3^24^2+3^24^2+6^23^2+6^28^2}{81^2}}\\
        &=\frac{2\sqrt{18+18+81+9\cdot64}}{81}&=\frac{2\cdot3\sqrt{81}}{81}\\
        &=\frac{2}{3}
      \end{align*}
    \item
      Find the distance from the point $\mathbf{w}=(1,1,1,1)$ to the hyperplane by using dot products.

      $\left\lvert\frac{\left\lvert\mathbf{w}\cdot\mathbf{a}\right\rvert}{||\mathbf{a}||}-\frac{2}{3}\right\rvert=\left\lvert\frac{\left\lvert 2+2-3+8\right\rvert}{\sqrt{4+4+9+64}}-\frac{2}{3}\right\rvert=\left\lvert\frac{9-6}{9}\right\rvert=\frac{1}{3}$
    \item
      Find the point on the plane closest to $\mathbf{w}$ by using the parametric equation of the line through $\mathbf{w}$ with direction vector $\mathbf{a}$. Double-check your answer to part $d$

      \begin{align*}
        \mathbf{x}&=(1,1,1,1)+t(2,2,-3,8)=(2t+1,2t+1,-3t+1,8t+1)\\
        6&=2(2t+1)+2(2t+1)-3(-3t+1)+8(8t+1)=(4+4+9+64)t+(2+2-3+8)\\
        t&=-\frac{3}{81}=-\frac{1}{27}\\
        ||\mathbf{x}-\mathbf{w}||&=||-\frac{1}{27}(2,2,-3,8)||=\frac{\sqrt{81}}{27}=\frac{1}{3}
      \end{align*}
    \end{enumerate}
  \item
    \begin{enumerate}
    \item
      The equations $x_1=0$ and $x_2=0$ describe planes in $\mathbb{R}^3$ that contain the $x_3$-axis. Write down the Cartesian equation of a general such plane.

      Such a plane would be spanned by a vector that lies on the $x_3$ axis and a non-zero vector that lies on the $x_1\times x_2$ plane. Lets describe the plane by saying $(x_1,x_2,x_3)=\mathbf{x}=s(0,0,1)+t(a,b,0)$. Taking the dot product of both sides with $(-b,a,0)$ gives us $(-b,a,0)\cdot(x_1,x_2,x_3)=0=-bx_1+ax_2$. Of course since our original choices of $a$ and $b$ were arbitrary, we will just rewrite our equation to say $ax_1+b_x2=0$
    \item
      The equations $x_1-x_2=0$ and $x_1-x_3=0$ describe planes in $\mathbb{R}^3$ that contain the line through the origin with direction vector $(1,1,1)$. Write down the cartesian equation of a general such plane.

      Such a plane must contain a non-zero vector on the $x_1\times x_2$ plane. Lets say $(a,b,0)$. This gives us $\mathbf{x}=s(1,1,1)+t(a,b,0)$. Now if we multiply both sides by the vector $(-b,a,b-a)$ then we get $-bx_1+ax_2+(b-a)x_3=0$. As before, our choice of $a$ and $b$ was arbitrary, and so we rewrite our equation to be $ax_1+bx_1-(b+a)x_3=0$
    \end{enumerate}
  \item Don't have to do after all

%    \begin{enumerate}
%    \item
%      Assume $\mathbf{b}$ and $\mathbf{c}$ are nonparallel vectors in $\mathbb{R}^3$. Generalizing the result of Exercise 10, show that the plan $\mathbf{a}\cdot \mathbf{x}=0$ contains the inetersection of the planes $\mathbf{b}\cdot\mathbf{x}=0$ and $\mathbf{c}\cdot\mathbf{x}=0$ if and only if $\mathbf{a}=s\mathbf{b}+t\mathbf{c}$ for some $s,t\in \mathbb{R}$, not both $0$. Describe this result geometrically.
%
%      We assume that $\mathbf{a}=s\mathbf{b}+t\mathbf{c}$ and then $0=\mathbf{a}\mathbf{x}=s\mathbf{b}\mathbf{x}+t\mathbf{c}\mathbf{x}$
%    \item
%      Assume $\mathbf{b}$ and $\mathbf{c}$ are nonparallel vectors in $\mathbb{R}^n$. Formulate a conjecture about which hyperplanes $\mathbf{a}\cdot\mathbf{x}=0$ in $\mathbb{R}^n$ contain the intersection of the hyperplanes $\mathbf{b}\cdot\mathbf{x}=0$ and $\mathbf{c}\cdot\mathbf{x}=0$. Prove as much of your conjecture as you can.
%    \end{enumerate}
  \item
    Suppose $\mathbf{a}\ne \mathbf{0}$ and $\mathcal{P}\subset \mathbb{R}^3$ is the plane through the origin with normal vector $\mathbf{a}$. Suppose $\mathcal{P}$ is spanned by $\mathbf{u}$ and $\mathbf{v}$
    \begin{enumerate}
    \item
      Suppose $\mathbf{u}\cdot\mathbf{v}=0$. Show that for every $\mathbf{x}\in \mathcal{P}$, we have
      \[\mathbf{x}=\text{proj}_{\mathbf{u}}\mathbf{x}+\text{proj}_{\mathbf{v}}\mathbf{x}\]

      Using the definition of a plane and the definition of projection we have the following:
      \begin{align*}
        \text{proj}_{\mathbf{u}}\mathbf{x}+\text{proj}_{\mathbf{v}}\mathbf{x}
        &=\frac{\mathbf{x}\mathbf{u}}{||\mathbf{u}||^2}\mathbf{u}+\frac{\mathbf{x}\mathbf{v}}{||\mathbf{v}||^2}\mathbf{v}\\
        \exists s,t\in \mathbb{R} \text{ such that }\mathbf{x}&=s\mathbf{u}+t\mathbf{v}
      \end{align*}
      Making appropriate substitution gives us
      \begin{align*}
        \text{proj}_{\mathbf{u}}\mathbf{x}+\text{proj}_{\mathbf{v}}\mathbf{x}
        &=\frac{(s\mathbf{u}+t\mathbf{v})\cdot\mathbf{u}}{||\mathbf{u}||^2}\mathbf{u}+\frac{(s\mathbf{u}+t\mathbf{v})\cdot\mathbf{v}}{||\mathbf{v}||^2}\mathbf{v}\\
        &=\frac{s\mathbf{u}\cdot\mathbf{u}+t\mathbf{v}\cdot\mathbf{u}}{||\mathbf{u}||^2}\mathbf{u}+\frac{s\mathbf{u}\cdot\mathbf{v}+t\mathbf{v}\cdot\mathbf{v}}{||\mathbf{v}||^2}\mathbf{v}\\
        &=\frac{s||\mathbf{u}||^2+t\mathbf{0}}{||\mathbf{u}||^2}\mathbf{u}+\frac{s\mathbf{0}+t||\mathbf{v}||^2}{||\mathbf{v}||^2}\mathbf{v}\\
        &=s\mathbf{u}+s\mathbf{v}=\mathbf{x}\\
      \end{align*}
    \item
      Suppose $\mathbf{u}\cdot\mathbf{v}=0$. Show that for every $\mathbf{x}\in\mathbb{R}^3$, we have
      \[\mathbf{x}=\text{proj}_{\mathbf{a}}\mathbf{x}+\text{proj}_{\mathbf{u}}\mathbf{x}+\text{proj}_{\mathbf{v}}\mathbf{x}\]
      (\emph{Hint:} Apply part $a$ to the vector $\mathbf{x}-\text{proj}_{\mathbf{a}}\mathbf{x}$)

      First we note that $\mathbf{x}-\text{proj}_{\mathbf{a}}\mathbf{x}\in \mathcal{P}$ and so
      \begin{align*}
        \mathbf{x}-\text{proj}_{\mathbf{a}}\mathbf{x}
        &=\text{proj}_{\mathbf{u}}(\mathbf{x}-\text{proj}_{\mathbf{a}}\mathbf{x})+\text{proj}_{\mathbf{v}}(\mathbf{x}-\text{proj}_{\mathbf{a}}\mathbf{x})\\
        \text{proj}_{\mathbf{u}}(\mathbf{x}-\text{proj}_{\mathbf{a}}\mathbf{x})
        &=\frac{(\mathbf{x}-\text{proj}_{\mathbf{a}}\mathbf{x})\cdot\mathbf{u}}{||\mathbf{u}||^2}\mathbf{u}\\
        &=\frac{(\mathbf{x}-\frac{\mathbf{x}\cdot\mathbf{a}}{||\mathbf{||a||^2}}\mathbf{a})\cdot\mathbf{u}}{||\mathbf{u}||^2}\mathbf{u}\\
      \end{align*}
      Now of course $\mathbf{a}$ is orthogonal to every vector in $\mathcal{P}$ including $\mathbf{u}$ and so $\mathbf{a}\cdot\mathbf{u}=0$ which leads us to conclude that
      \begin{align*}
        \text{proj}_{\mathbf{u}}(\mathbf{x}-\text{proj}_{\mathbf{a}}\mathbf{x})
        &=\frac{\mathbf{x}\cdot\mathbf{u}}{||\mathbf{u}||^2}\mathbf{u}\\
        &=\text{proj}_{\mathbf{u}}\mathbf{x}
      \end{align*}
      In exactly the same way it can be shown that $\text{proj}_{\mathbf{v}}(\mathbf{x}-\text{proj}_{\mathbf{a}}\mathbf{x})=\text{proj}_{\mathbf{v}}\mathbf{x}$. Thus $\mathbf{x}-\text{proj}_{\mathbf{a}}\mathbf{x}=\text{proj}_{\mathbf{u}}\mathbf{x}+\text{proj}_{\mathbf{v}}\mathbf{x}$ or $\mathbf{x}=\text{proj}_{\mathbf{a}}\mathbf{x}+\text{proj}_{\mathbf{u}}\mathbf{x}+\text{proj}_{\mathbf{v}}\mathbf{x}$
    \item
      Give an example to show the result of part $a$ is false when $\mathbf{u}$ and $\mathbf{v}$ are not orthogonal

      Let $\mathbf{a}=(0,0,1), \mathbf{x}=\mathbf{u}=(1,1,0),$ and $\mathbf{v}=(1,0,0)$. Then $\frac{\mathbf{x}\cdot\mathbf{u}}{||\mathbf{u}||^2}\mathbf{u}+\frac{\mathbf{x}\cdot \mathbf{v}}{||\mathbf{v}||^2}\mathbf{v}=\mathbf{u}+\mathbf{v}=(2,1,0)\ne \mathbf{x}$
    \end{enumerate}
  \item
    Consider the line $\ell$ in $\mathbb{R}^3$ given parametrically by $\mathbf{x}=\mathbf{x}_0+t\mathbf{a}$. Let $\mathcal{P}_0$ denote the plane through the origin with normal vector $\mathbf{a}$ (so it is orthogonal to $\ell$).
    \begin{enumerate}
    \item
      Show that $\ell$ and $\mathcal{P}_0$ intersect in the point $\mathbf{x}_0-\text{proj}_{\mathbf{a}}\mathbf{x}_0$

      Well we know that $\mathbf{a}\cdot\mathbf{x}=0$ defines the plane. And so $\mathbf{a}\cdot\mathbf{x}_0+t\mathbf{a}\cdot\mathbf{a}=0$ and $-t=\frac{\mathbf{a}\cdot\mathbf{x}_0}{||\mathbf{a}||^2}$. Substituting $t$ back in we get $\mathbf{x}=\mathbf{x}_0-\frac{\mathbf{a}\cdot\mathbf{x}_0}{||\mathbf{a}||^2}\mathbf{a}=\mathbf{x}_0-\text{proj}_{\mathbf{a}}\mathbf{x}_0$.
    \item
      Conclude that the distance from the origin to $\ell$ is $||\mathbf{x}_0-\text{proj}_{\mathbf{a}}\mathbf{x}_0||$

      Well the origin is on the plane, and so is our intersection point, so the vector from the origin to the intersection point is on the plane. Further, $\ell$ is orthogonal to the plane, and so $\ell$ must be orthogonal to the $\mathbf{x}_0-\text{proj}_{\mathbf{a}}\mathbf{x}_0$ vector. Thus the closest point to the origin on the line is where it intersects with the plane.
    \end{enumerate}
  \end{enumerate}
\end{enumerate}
\end{document}
