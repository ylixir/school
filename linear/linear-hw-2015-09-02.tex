\documentclass[letterpaper]{article}

\usepackage{fullpage}
\usepackage{nopageno}
\usepackage{amsmath}
\usepackage{amssymb}
\allowdisplaybreaks

\newcommand{\abs}[1]{\left\lvert #1 \right\rvert}

\begin{document}
\title{Homework}
\date{September 2, 2015}
\author{Jon Allen}
\maketitle
%Section 1.1: 5,6(a,c,d,h),7,10(a,d),20-25,28(d)
\renewcommand{\labelenumi}{1.\arabic{enumi}}
\renewcommand{\labelenumii}{\arabic{enumii}.}
\renewcommand{\labelenumiii}{(\alph{enumiii})}
\begin{enumerate}
\item
  \begin{enumerate}
  \setcounter{enumii}{4}
  \item
  Let $\ell$ be the line given parametrically by $\mathbf{x}=(1,3)+t(-2,1),t\in \mathbb{R}$. Which of the following points lie on $\ell$? Give your reasoning.

  No magic, just algebra, if we can work out a true equation it's on the line. If we work out a false equation, it's not.
    \begin{enumerate}
    \item
    $\mathbf{x}=(-1,4)$
    \begin{align*}
      (-1,4)&=(1,3)+t(-2,1) & (-1-1,4-3)=(-2,1)&=t(-2,1) & t&=1
    \end{align*}
    lies on the line
    \item
    $\mathbf{x}=(7,0)$
    \begin{align*}
      (7-1,0-3)=(6,-3)&=t(-2,1)&t&=-3
    \end{align*}
    also lies on the line
    \item
    $\mathbf{x}=(6,2)$
    \begin{align*}
      (6-1,2-3)=(5,-1)&\ne t(-2,1)
    \end{align*}
    \end{enumerate}
  \item
  Find a parametric equation of each of the following lines:
    \begin{enumerate}
    \item
    $3x_1+4x_2=6$
    \begin{align*}
      x_2&=-\frac{3}{4}x_1+\frac{6}{4}\\
      (x_1,x_2)&=(0,\frac{6}{4})+t(-3,4)\\
      \mathbf{x}&=(2,0)+t(-3,4)
    \end{align*}
    \setcounter{enumiii}{2}
    \item
    the line with the slope $2/5$ that passes through $A=(3,1)$
    \begin{align*}
      \mathbf{x}=(3,1)+t(5,2)
    \end{align*}
    \item
    the line through $A=(-2,1)$ parallel to $\mathbf{x}=(1,4)+t(3,5)$
    \begin{align*}
      \mathbf{x}=(-2,1)+t(3,5)
    \end{align*}
    \setcounter{enumiii}{7}
    \item
    the line through $(1, 1, 0, -1)$ parallel to $\mathbf{x}=(2+t,1-2t,3t,4-t)$
    \begin{align*}
      \mathbf{x}&=(2+t,1-2t,3t,4-t)\\
      &=(2,1,0,4)+t(1,-2,3,-1)\\
      \mathbf{x}'&=(1,1,0,-1)+t(1,-2,3,-1)
    \end{align*}
    \end{enumerate}
  \item
  Suppose $\mathbf{x}=\mathbf{x}_0+t\mathbf{v}$ and $\mathbf{y}=\mathbf{y}_0+s\mathbf{w}$ are two parametric representations of the same line $\ell$ in $\mathbb{R}$.
    \begin{enumerate}
    \item
    Show that there is a scalar $t_0$ so that $\mathbf{y}_0=\mathbf{x}_0+t_0\mathbf{v}$

    By definition 2.2 the line goes through $\mathbf{y}_0$ and $\mathbf{x}_0$. Because $\mathbf{y}_0\in\ell=\{\mathbf{x}\in \mathbb{R}^n:\mathbf{x}=\mathbf{x}_0+t\mathbf{v}\text{ for some }t\in \mathbb{R}\}$ then there is some $t_0\in\mathbb{R}$ such that $\mathbf{y}_0=\mathbf{x}=\mathbf{x}_0+t_0\mathbf{v}$
    \item
    Show that $\mathbf{v}$ and $\mathbf{w}$ are parallel.

    Let us choose some point $\mathbf{z}\in \ell$ other than $\mathbf{y}_0$. Then there exists some $t_1,s_1\in \mathbb{R}$ such that $\mathbf{y}_0+s_1\mathbf{w}=\mathbf{z}=\mathbf{x}_0+t_1\mathbf{v}$. We just saw that there exists some $t_0\in \mathbb{R}$ such that $\mathbf{y}_0=\mathbf{x}_0+t_0\mathbf{v}$. So then letting the algebra work itself out:
    \begin{align*}
      \mathbf{y}_0+s_1\mathbf{w}&=\mathbf{x}_0+t_1\mathbf{v}\\
      (\mathbf{x}_0+t_0\mathbf{v})+s_1\mathbf{w}&=\mathbf{x}_0+t_1\mathbf{v}& \text{A1 and A4}\\
      s_1\mathbf{w}&=t_1\mathbf{v}-t_0\mathbf{v}& \text{S1, S3, and S4}\\
      \mathbf{w}&=\frac{t_1-t_0}{s_1}\mathbf{v}
    \end{align*}
    Now obviously $\frac{t_1-t_0}{s_1}\in \mathbb{R}$ and so by definition 1.7 we know that $\mathbf{v}$ and $\mathbf{w}$ are parallel.
    \end{enumerate}
  \setcounter{enumii}{9}
  \item
  Find a parametric equation of each of the following planes:
    \begin{enumerate}
    \item
    the plane containing the point $(-1,0,1)$ and the line $\mathbf{x}=(1,1,1)+t(1,7,-1)$

    \begin{align*}
      (-1,0,1)&=(1,1,1)+t(1,7,-1)+\mathbf{u}&\text{let }t&=0\\
      (-2,-1,-2)&=\mathbf{u}& \text{By A3 and Theorem 11}\\
      \mathcal{P}(\mathbf{x}_0,\mathbf{u},\mathbf{v})&=(1,1,1)+t(1,7,-1)+s(-2,1,-2)
    \end{align*}
    \setcounter{enumiii}{3}
    \item
    the plane in $\mathbb{R}^4$ containing the points $(1,1,-1,4),(2,3,0,1)$ and $(1,2,2,3)$
    \end{enumerate}
  \setcounter{enumii}{19}
  \item
  Assume that $\mathbf{u}$ and $\mathbf{v}$ are parallel vectors in $\mathbb{R}^n$. Prove that $\text{Span}(\mathbf{u},\mathbf{v})$ is a line.

  
  \item
  Suppose $\mathbf{v},\mathbf{w}\in \mathbb{R}^n$ and $c$ is a scalar. Prove that $\text{Span}(\mathbf{v}+c\mathbf{w},\mathbf{w})=\text{Span}(\mathbf{v},\mathbf{w})$. (See the blue box on p. 12.)
  \item
  Suppose  the vectors $\mathbf{v}$ and $\mathbf{w}$ are both linear combinations of $\mathbf{v}_1,\dots,\mathbf{v}_k$.
    \begin{enumerate}
    \item
    Prove that for any scalar $c,c\mathbf{v}$ is a linear combination of $\mathbf{v}_1,\dots,\mathbf{v}_k$.
    \item
    Prove that $\mathbf{v}+\mathbf{w}$ is a linear combination of $\mathbf{v}_1,\dots,\mathbf{v}_k$
    \end{enumerate}
  \item
  Consider the line $\ell: \mathbf{x}=\mathbf{x}_0+r\mathbf{v} (r\in \mathbb{R})$ and the plane $\mathcal{P}: \mathbf{x}=s\mathbf{u}+t\mathbf{v} (s,t\in \mathbb{R})$. Show that if $\ell$ and $\mathcal{P}$ intersect, then $\mathbf{x}_0\in \mathcal{P}$
  \item
  Consider the lines $\ell: \mathbf{x}=\mathbf{x}_0+t\mathbf{v}$ and $\mathit{m}: \mathbf{x}=\mathbf{x}_1+s\mathbf{u}$. Show that $\ell$ and $\mathit{m}$ intersect if and only if $\mathbf{x}_0-\mathbf{x}_1$ lies in $\text{Span}(\mathbf{u},\mathbf{v})$.
  \item
  Suppose $\mathbf{x},\mathbf{y}\in \mathbb{R}^n$ are nonparallel vectors. (Recall definition on p.3.)
    \begin{enumerate}
    \item
    Prove that if $s\mathbf{x}+t\mathbf{y}=\mathbf{0}$ then $s=t=0$. ({\it Hint:} Show that neither $s\ne 0$ nor $t\ne 0$ is possible.)
    \item
    Prove that if $a\mathbf{x}+b\mathbf{y}=c\mathbf{x}+d\mathbf{y}$, then $a=c$ and $b=d$
    \end{enumerate}
  \setcounter{enumii}{27}
  \item
  Verify algebraically that the following properties of vector arithmetic hold. (Do so for $n=2$ if the general case is too intimidating.) Give the geometric interpretation of each property.
    \begin{enumerate}
    \setcounter{enumiii}{3}
    \item
    For each $\mathbf{x}\in \mathbb{R}^n$, there is a vector $-\mathbf{x}$ so that $\mathbf{x}+(-\mathbf{x})=\mathbf{0}$
    \end{enumerate}
  \end{enumerate}
\end{enumerate}
\end{document}
