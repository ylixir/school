\documentclass[letterpaper]{article}

\usepackage[utf8x]{luainputenc}
\usepackage{aeguill}
%\usepackage{nopageno}
\usepackage{amsmath}
\usepackage{amssymb}
\usepackage{mathrsfs}
\usepackage{fullpage}
\usepackage{fancyhdr}
\setlength{\headheight}{12pt}
\pagestyle{fancy}
\chead{Linear Algebra}
\lhead{September 2, 2015}
\rhead{Jon Allen}
\allowdisplaybreaks

\newcommand{\abs}[1]{\left\lvert #1 \right\rvert}

\begin{document}
%Section 1.1: 5,6(a,c,d,h),7,10(a,d),20-25,28(d)
\renewcommand{\labelenumi}{1.\arabic{enumi}}
\renewcommand{\labelenumii}{\arabic{enumii}.}
\renewcommand{\labelenumiii}{(\alph{enumiii})}
\begin{enumerate}
\item
  \begin{enumerate}
  \setcounter{enumii}{4}
  \item
  Let $\ell$ be the line given parametrically by $\mathbf{x}=(1,3)+t(-2,1),t\in \mathbb{R}$. Which of the following points lie on $\ell$? Give your reasoning.

  No magic, just algebra, if we can work out a true equation it's on the line. If we work out a false equation, it's not.
    \begin{enumerate}
    \item
    $\mathbf{x}=(-1,4)$ leads to $(-1,4)=(1,3)+t(-2,1)$ and $(-1-1,4-3)=(-2,1)=t(-2,1)$. If we let $t=1$ then the equation holds, thus the point lies on the line
    \item
    $\mathbf{x}=(7,0)$ leads to $(7-1,0-3)=(6,-3)=t(-2,1)$. So we let $t=-3$ to make the equation hold and find that this point also lies on the line.
    \item
    $\mathbf{x}=(6,2)$ leads to $(6-1,2-3)=(5,-1)\ne t(-2,1)$ and so the point is not on the line.
    \end{enumerate}
  \item
  Find a parametric equation of each of the following lines:
    \begin{enumerate}
    \item
    $3x_1+4x_2=6$
    \begin{align*}
      x_2&=-\frac{3}{4}x_1+\frac{6}{4}\\
      (x_1,x_2)&=(0,\frac{6}{4})+t(-3,4)\\
      \mathbf{x}&=(2,0)+t(-3,4)
    \end{align*}
    \setcounter{enumiii}{2}
    \item
    the line with the slope $2/5$ that passes through $A=(3,1)$
    \begin{align*}
      \mathbf{x}=(3,1)+t(5,2)
    \end{align*}
    \item
    the line through $A=(-2,1)$ parallel to $\mathbf{x}=(1,4)+t(3,5)$
    \begin{align*}
      \mathbf{x}=(-2,1)+t(3,5)
    \end{align*}
    \setcounter{enumiii}{7}
    \item
    the line through $(1, 1, 0, -1)$ parallel to $\mathbf{x}=(2+t,1-2t,3t,4-t)$
    \begin{align*}
      \mathbf{x}&=(2+t,1-2t,3t,4-t)\\
      &=(2,1,0,4)+t(1,-2,3,-1)\\
      \mathbf{x}'&=(1,1,0,-1)+t(1,-2,3,-1)
    \end{align*}
    \end{enumerate}
  \item
  Suppose $\mathbf{x}=\mathbf{x}_0+t\mathbf{v}$ and $\mathbf{y}=\mathbf{y}_0+s\mathbf{w}$ are two parametric representations of the same line $\ell$ in $\mathbb{R}$.
    \begin{enumerate}
    \item
    Show that there is a scalar $t_0$ so that $\mathbf{y}_0=\mathbf{x}_0+t_0\mathbf{v}$

    By definition 2.2 the line goes through $\mathbf{y}_0$ and $\mathbf{x}_0$. Because $\mathbf{y}_0\in\ell=\{\mathbf{x}\in \mathbb{R}^n:\mathbf{x}=\mathbf{x}_0+t\mathbf{v}\text{ for some }t\in \mathbb{R}\}$ then there is some $t_0\in\mathbb{R}$ such that $\mathbf{y}_0=\mathbf{x}=\mathbf{x}_0+t_0\mathbf{v}$
    \item
    Show that $\mathbf{v}$ and $\mathbf{w}$ are parallel.

    We choose the point $\mathbf{y}_0+\mathbf{w}\in \ell$.
    Note that because this point is on our line, then there exists some $t_1\in \mathbb{R}$ such that $\mathbf{y}_0+\mathbf{w}=\mathbf{x}_0+t_1\mathbf{v}$.
    And we have already established that we have some $t_0\in \mathbb{R}$ such that $\mathbf{y}_0=\mathbf{x}_0+t_0\mathbf{v}$. 
    Making the substitution we have $(\mathbf{x}_0+t_0\mathbf{v})+\mathbf{w}=\mathbf{x}_0+t_1\mathbf{v}$. Now using the algebraic properties of vectors and scalars from theorem 1.9:
    \begin{align*}
      (\mathbf{x}_0+t_0\mathbf{v})+\mathbf{w}&=\mathbf{x}_0+t_1\mathbf{v}\\
      \mathbf{w}&=\mathbf{x}_0-\mathbf{x}_0+t_1\mathbf{v}-t_0\mathbf{v}\\
      \mathbf{w}&=(t_1-t_0)\mathbf{v}
    \end{align*}
    We know from the definition of a line that $\mathbf{v}$ and $\mathbf{w}$ are not $\mathbf{0}$. So the above equation means that $\mathbf{v}$ and $\mathbf{w}$ fit the definition of parallel.
    \end{enumerate}
  \setcounter{enumii}{9}
  \item
  Find a parametric equation of each of the following planes:
    \begin{enumerate}
    \item
    the plane containing the point $(-1,0,1)$ and the line $\mathbf{x}=(1,1,1)+t(1,7,-1)$

    We know that the plane contains the points $(-1,0,1)$ and $(1,1,1)$.
    Importantly $(-1,0,1)$ is not on our line.
    So we just pick $(1,1,1)$ to be $\mathbf{x}_0$ and then choose the direction vector from our line to be one of the spanning vectors for the plane.
    Plugging it all into the definition of a plane we get $(-1,0,1)=(1,1,1)+s\mathbf{u}+t(1,7,-1)$.
    Because $(-1,0,1)$ is not on our line, so if we use $t=0$ and $s=1$ then applying theorem 1.9 and 11 should give us a second spanning vector.
    \begin{align*}
      \mathbf{x}&=\mathbf{x}_0+s\mathbf{u}+t\mathbf{v}\\
      (-1,0,1)&=(1,1,1)+\mathbf{u}+0(1,7,-1)\\
      (-1,0,1)-(1,1,1)&=\mathbf{u}\\
      (-2,-1,-2)&=\mathbf{u}&\\
      \mathcal{P}(\mathbf{x}_0,\mathbf{u},\mathbf{v})&=\{\mathbf{x}\in \mathbb{R}^n:\mathbf{x}=(1,1,1)+s(-2,1,-2)+t(1,7,-1)\quad\forall s,t\in \mathbb{R}\}
    \end{align*}
    Note that we cannot multiply $(1,7,-1)$ by any real number and get $(-2,1,-2)$. Thus $(-2,1,-2)$ and $(1,7,-1)$ are not parallel and the above equation fits all the criteria for our plane.
    \setcounter{enumiii}{3}
    \item
    the plane in $\mathbb{R}^4$ containing the points $(1,1,-1,4),(2,3,0,1)$ and $(1,2,2,3)$

    So we should be able to just pick a point from the three for an $\mathbf{x}_0$ or $\mathbf{y}_0$ and then find the equation for lines through this point and the other two points. That will give us our spanning vectors. So lets pick $(1,1,-1,4)=\mathbf{x}_0=\mathbf{y}_0$ and do the algebra.
    \begin{align*}
      \mathbf{x}&=\mathbf{x}_0+t\mathbf{v}&
      \mathbf{y}&=\mathbf{y}_0+s\mathbf{u}\\
      (2,3,0,1)&=(1,1,-1,4)+t\mathbf{v}&
      (1,2,2,3)&=(1,1,-1,4)+s\mathbf{u}\\
      t\mathbf{v}&=(2,3,0,1)-(1,1,-1,4)&
      s\mathbf{u}&=(1,2,2,3)-(1,1,-1,4)\\
      \text{choose }t&=1&
      \text{choose }s&=1\\
      \mathbf{v}&=(1,2,1,-3)&
      \mathbf{u}&=(0,1,3,-1)
    \end{align*}
    I claim that the zero in $\mathbf{u}$ means that it is obviously not parallel to $\mathbf{v}$ which has no zeroes. Thus we have our plane in $\mathcal{P}\left((1,1,-1,4),(1,2,1,-3),(0,1,3,-1)\right)$
    \end{enumerate}
  \setcounter{enumii}{19}
  \item
  Assume that $\mathbf{u}$ and $\mathbf{v}$ are parallel vectors in $\mathbb{R}^n$. Prove that $\text{Span}(\mathbf{u},\mathbf{v})$ is a line.

  First we note that $\text{Span}(\mathbf{u},\mathbf{v})=\{\mathbf{x}\in \mathbb{R}^n:a\mathbf{u}+b\mathbf{v}\quad\forall a,b\in \mathbb{R}\}$.
  Of course we know that because $\mathbf{u}$ and $\mathbf{v}$ are parallel, then $\exists c\in \mathbb{R}$ such that $\mathbf{u}=c\mathbf{v}$.
  And so we have $a\mathbf{u}+b\mathbf{v}=a(c\mathbf{v})+b\mathbf{v}=(ac+b)\mathbf{v}$.
  Let $ac+b=t$ and $\mathbf{x}_0=(0,0,0)$.
  Then $(ac+b)\mathbf{v}=\mathbf{x}_0+t\mathbf{v}$.
  Thus $\text{Span}(\mathbf{u},\mathbf{v})=\{x\in \mathbb{R}^n:\mathbf{x}_0+t\mathbf{v}\}$.
  From the definition of parallel we know that $\mathbf{v}\ne \mathbf{0}$ and so $\text{Span}(\mathbf{u},\mathbf{v})$ fits the definition of a line.
  $\Box$
  
  \item
  Suppose $\mathbf{v},\mathbf{w}\in \mathbb{R}^n$ and $c$ is a scalar. Prove that $\text{Span}(\mathbf{v}+c\mathbf{w},\mathbf{w})=\text{Span}(\mathbf{v},\mathbf{w})$. (See the blue box on p. 12.)

  Sometimes it's easier to let the math speak for itself, and so from the definition of span and from our established algebraic properties:
  \begin{align*}
    \text{Span}(\mathbf{v}+c\mathbf{w})&=\{\mathbf{x}\in \mathbb{R}^n:\mathbf{x}=a(\mathbf{v}+c\mathbf{w})+b(\mathbf{w})\quad \forall a,b\in \mathbb{R}\}\\
    &=\{\mathbf{x}\in \mathbb{R}^n:\mathbf{x}=a\mathbf{v}+(ac+b)\mathbf{w}\quad \forall a,b\in \mathbb{R}\}
  \end{align*}
  Now of course if we fix any $a,c \in \mathbb{R}$ then we know that $\mathbb{R}=\{ac+b:\forall b\in \mathbb{R} \}$.
  Lets just say $d=ac+b$.
  And so:
  \begin{align*}
    \{\mathbf{x}\in \mathbb{R}^n:\mathbf{x}=a\mathbf{v}+(ac+b)\mathbf{w}\quad \forall a,b\in \mathbb{R}\}&=\{\mathbf{x}\in \mathbb{R}^n:\mathbf{x}=a\mathbf{v}+d\mathbf{w}\quad \forall a,d\in \mathbb{R}\}\\
    &=\text{Span}(\mathbf{v},\mathbf{w})
  \end{align*}
  $\Box$
  \item
  Suppose  the vectors $\mathbf{v}$ and $\mathbf{w}$ are both linear combinations of $\mathbf{v}_1,\dots,\mathbf{v}_k$.
    \begin{enumerate}
    \item
    Prove that for any scalar $c,c\mathbf{v}$ is a linear combination of $\mathbf{v}_1,\dots,\mathbf{v}_k$.

    Say $\mathbf{v}=a_1\mathbf{v}_1+\dots+a_k\mathbf{v}_k$ for some $a_1,\dots,a_k\in \mathbb{R}$. Then $c\mathbf{v}$ must be $c(a_1\mathbf{v}_1+\dots+a_k\mathbf{v}_k)=ca_1\mathbf{v}_1+\dots+ca_k\mathbf{v}_k$ and so $c\mathbf{v}$ is a linear combination of $\mathbf{v}_1,\dots,\mathbf{v}_k$.
    \item
    Prove that $\mathbf{v}+\mathbf{w}$ is a linear combination of $\mathbf{v}_1,\dots,\mathbf{v}_k$

    Say $\mathbf{w}=b_1\mathbf{v}_1+\dots+b_k\mathbf{v}_k$. Then $\mathbf{v}+\mathbf{w}=a_1\mathbf{v}_1+\dots+a_k\mathbf{v}_k+b_1\mathbf{v}_1+\dots+b_k\mathbf{v}_k=(a_1+b_1)\mathbf{v}_1+\dots+(a_k+b_k)\mathbf{v}_k$. Naturally this is a linear combination of $\mathbf{v}_1,\dots,\mathbf{v}_k$.
    \end{enumerate}
  \item
  Consider the line $\ell: \mathbf{x}=\mathbf{x}_0+r\mathbf{v} (r\in \mathbb{R})$ and the plane $\mathcal{P}: \mathbf{x}=s\mathbf{u}+t\mathbf{v} (s,t\in \mathbb{R})$. Show that if $\ell$ and $\mathcal{P}$ intersect, then $\mathbf{x}_0\in \mathcal{P}$

  If $\ell\cap\mathcal{P}\ne \emptyset$ then there exists some $r_0,s_0,t_0\in \mathbb{R}$ such that $\mathbf{x}_0+r_0\mathbf{v}=s_0\mathbf{u}+t_0\mathbf{v}$. Of course then $\mathbf{x}_0=s_0\mathbf{u}+(t_0-r_0)\mathbf{v}$. And because $t_0-r_0\in \mathbb{R}$ then $s_0\mathbf{u}+(t_0-r_0)\mathbf{v}\in \mathcal{P}$. So $\mathbf{x}_0$ is clearly on our plane.
  \item
  Consider the lines $\ell: \mathbf{x}=\mathbf{x}_0+t\mathbf{v}$ and $\mathit{m}: \mathbf{x}=\mathbf{x}_1+s\mathbf{u}$. Show that $\ell$ and $\mathit{m}$ intersect if and only if $\mathbf{x}_0-\mathbf{x}_1$ lies in $\text{Span}(\mathbf{u},\mathbf{v})$.

  First let us assume that $\ell\cap \mathit{m}\ne \emptyset$. Then $\exists t_0,s_0\in \mathbb{R}$ such that $\mathbf{x}_0+t_0\mathbf{v}=\mathbf{x}_1+s_0\mathbf{u}$. Simple manipulation leads to $\mathbf{x}_0-\mathbf{x}_1=s_0\mathbf{u}+(-t_0)\mathbf{v}$. And since $\mathbf{x}_0-\mathbf{x}_1$ is a linear combination of $\mathbf{u}$ and $\mathbf{v}$ then $\mathbf{x}_0-\mathbf{x}_1\in \text{Span}(\mathbf{u},\mathbf{v})$

  And if we assume that $\mathbf{x}_0-\mathbf{x}_1\in \text{Span}(\mathbf{u},\mathbf{v})$? Well then $\mathbf{x}_0-\mathbf{x}_1=c_1\mathbf{u}+c_2\mathbf{v}$. For some $c_1,c_2\in \mathbb{R}$. Again, simple manipulation leads to $\mathbf{x}_0+(-c_2)\mathbf{v}=\mathbf{x}_1+c_1\mathbf{u}$. Thus we have found two ways to represent the point where the two lines intersect. $\Box$
  \item
  Suppose $\mathbf{x},\mathbf{y}\in \mathbb{R}^n$ are nonparallel vectors. (Recall definition on p.3.)

    \begin{enumerate}
    \item
    Prove that if $s\mathbf{x}+t\mathbf{y}=\mathbf{0}$ then $s=t=0$. ({\it Hint:} Show that neither $s\ne 0$ nor $t\ne 0$ is possible.)

    We first note that we have no definition of nonparallel, only parallel. Parallel vectors can not be zero, which kind of implies that nonparallel vectors might be. One can easily find a counterexample to the assertion if either vector is zero. I do not think that is the point of the exercise though, so we will assume that neither vector is the zero vector.

    Now let us assume that $s\ne0$. Then with some simple algebra we go from $s\mathbf{x}+t\mathbf{y}=0$ to $\mathbf{x}=-\frac{t}{s}\mathbf{y}$. Thus $\mathbf{x}$ and $\mathbf{y}$ fit the definition of parallel. But they are not parallel, and so we know that $s=0$. Similarly, assuming $t\ne 0$ leads to $\mathbf{y}=-\frac{s}{t}\mathbf{x}$ and so we know that $t=0$. $\Box$
    \item
    Prove that if $a\mathbf{x}+b\mathbf{y}=c\mathbf{x}+d\mathbf{y}$, then $a=c$ and $b=d$

    If we rearrange our equation a little, we arrive at $(a-c)\mathbf{x}=(d-b)\mathbf{y}$. Similarly to above we must now add to our premise that $\mathbf{x}$ and $\mathbf{y}$ are not zero. Now if we assume that $a\ne c$ then we find that $a-c\ne 0$. This allows us to arrange our equation as $\mathbf{x}=\frac{d-b}{a-c}\mathbf{y}$. Similarly, assuming $b\ne d$ leads us to $\mathbf{y}=\frac{a-c}{d-b}\mathbf{x}$. Either way we must conclude that $\mathbf{x}$ and $\mathbf{y}$ are parallel. But this is not true, and so we know that $a=c$ and $b=d$.
    \end{enumerate}
  \setcounter{enumii}{27}
  \item
  Verify algebraically that the following properties of vector arithmetic hold. (Do so for $n=2$ if the general case is too intimidating.) Give the geometric interpretation of each property.
    \begin{enumerate}
    \setcounter{enumiii}{3}
    \item
    For each $\mathbf{x}\in \mathbb{R}^n$, there is a vector $-\mathbf{x}$ so that $\mathbf{x}+(-\mathbf{x})=\mathbf{0}$

    Let $\mathbf{x}=(x_1,\dots,x_n)$. We know that for each $x_i\in \mathbb{R}$ there exists some $-x_i\in \mathbb{R}$ such that $x_i+(-x_i)=0$. Let us define $-\mathbf{x}=(-x_1,\dots,-x_n)$. Then
    \begin{align*}
      \mathbf{x}+(-\mathbf{x})&=(x_1,\dots,x_n)+(-x_1,\dots,-x_n)\\
      &=(x_1+(-x_1),\dots,x_n+(-x_n))\\
      &=(0,\dots,0)=\mathbf{0}
    \end{align*}
    Geometrically, for every vector, there exists another parallel vector with equal magnitude, and opposite direction.
    \end{enumerate}
  \end{enumerate}
\end{enumerate}
\end{document}
