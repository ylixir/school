\documentclass[letterpaper]{article}

\usepackage[utf8x]{luainputenc}
\usepackage{aeguill}
%\usepackage{nopageno}
\usepackage{amsmath}
\usepackage{amssymb}
\usepackage{mathrsfs}
\usepackage{fullpage}
\usepackage{fancyhdr}
\setlength{\headheight}{12pt}
\pagestyle{fancy}
\chead{Linear Algebra}
\lhead{October 2, 2015}
\rhead{Jon Allen}
\allowdisplaybreaks

\newcommand{\abs}[1]{\left\lvert #1 \right\rvert}

%Exercises Section 1.4.
%3(c,d,f,h), 4(b,d,e), 14, 15.
%Exercises Section 1.5.
%You should do all of them but I will collect 13,14,15
\begin{document}
%\renewcommand{\labelenumii}{\alph{enumii}.}
%\renewcommand{\labelenumiii}{\alph{enumiii}.}
%\renewcommand{\labelenumi}{(\arabic{enumi})}
\section*{1.4}
\begin{enumerate}
\setcounter{enumi}{2}
\item
For each of the following matrices $A$, determine its reduced echelon form and give the general solution of $A\mathbf{x}=\mathbf{0}$ in standard form.
  \begin{enumerate}
  \setcounter{enumii}{2}
  \item
  \begin{align*}
    A&=
    \left[\begin{array}{ccc}
      1&2&-1\\
      1&3&1\\
      2&4&3\\
      -1&1&6\\
    \end{array}\right]
    \to
    \left[\begin{array}{ccc|c}
      1&2&-1&0\\
      0&1&2&0\\
      0&0&5&0\\
      0&3&5&0\\
    \end{array}\right]
    \to
    \left[\begin{array}{ccc|c}
      1&2&-1&0\\
      0&1&2&0\\
      0&0&1&0\\
      0&3&5&0\\
    \end{array}\right]
    \to
    \left[\begin{array}{ccc|c}
      1&0&0&0\\
      0&1&0&0\\
      0&0&1&0\\
      0&0&0&0\\
    \end{array}\right]\\
    \mathbf{x}&=\left[\begin{array}{c}0\\0\\0\end{array}\right]
  \end{align*}
  \item
  \begin{align*}
    A&=
    \left[\begin{array}{cccc}
      1&1&1&1\\
      1&2&1&2\\
      1&3&2&4\\
      1&2&2&3\\
    \end{array}\right]
    \to
    \left[\begin{array}{cccc|c}
      1&1&1&1&0\\
      0&1&0&1&0\\
      0&2&1&3&0\\
      0&1&1&2&0\\
    \end{array}\right]
    \to
    \left[\begin{array}{cccc|c}
      1&1&1&1&0\\
      0&1&0&1&0\\
      0&0&1&1&0\\
      0&0&1&1&0\\
    \end{array}\right]
    \to
    \left[\begin{array}{cccc|c}
      1&0&0&-1&0\\
      0&1&0&1&0\\
      0&0&1&1&0\\
      0&0&0&0&0\\
    \end{array}\right]
    \\
    \mathbf{x}&=x_4\left[\begin{array}{c}1\\-1\\-1\\1\end{array}\right]
  \end{align*}
  \setcounter{enumii}{5}
  \item
  \begin{align*}
    A&=
    \left[\begin{array}{ccccc}
      1&2&0&-1&-1\\
      -1&-3&1&2&3\\
      1&-1&3&1&1\\
      2&-3&7&3&4\\
    \end{array}\right]
    \to
    \left[\begin{array}{ccccc}
       1& 2& 0&-1&-1\\
       0&-1& 1& 1& 2\\
       0&-3& 3& 2& 2\\
       0&-7& 7& 5& 6\\
    \end{array}\right]\\
    &\to
    \left[\begin{array}{ccccc}
       1& 2& 0&-1&-1\\
       0& 1&-1&-1&-2\\
       0& 0& 0&-1&-4\\
       0& 0& 0&-2&-8\\
    \end{array}\right]
    \to
    \left[\begin{array}{ccccc}
       1& 2& 0&-1&-1\\
       0& 1&-1&-1&-2\\
       0& 0& 0& 1& 4\\
       0& 0& 0& 0& 0\\
    \end{array}\right]\\
    &\to
    \left[\begin{array}{ccccc}
       1& 0& 2& 0&-1\\
       0& 1&-1&-1&-2\\
       0& 0& 0& 1& 4\\
       0& 0& 0& 0& 0\\
    \end{array}\right]
    \to
    \left[\begin{array}{ccccc}
       1& 0& 2& 0&-1\\
       0& 1&-1& 0& 2\\
       0& 0& 0& 1& 4\\
       0& 0& 0& 0& 0\\
    \end{array}\right]\\
    \mathbf{x}&=\left[\begin{array}{ccccc}
      &&-2x_3&&+x_5\\
      &&x_3&&-2x_5\\
      &&x_3\\
      &&&&-4x_5\\
      &&&&x_5
    \end{array}\right]
    =x_3\left[\begin{array}{c}-2\\1\\1\\0\\0\end{array}\right]
    +x_5\left[\begin{array}{c}1\\-2\\0\\-4\\1\end{array}\right]
  \end{align*}
  \setcounter{enumii}{7}
  \item
  \begin{align*}
    A&=
    \left[\begin{array}{cccccc}
      1&1&0&5&0&-1\\
      0&1&1&3&-2&0\\
      -1&2&3&4&1&-6\\
      0&4&4&12&-1&-7\\
    \end{array}\right]
    \to
    \left[\begin{array}{cccccc|c}
      1&1&0&5&0&-1&0\\
      0&1&1&3&-2&0&0\\
      0&3&3&9&1&-7&0\\
      0&0&0&0&7&-7&0\\
    \end{array}\right]\\
    &\to
    \left[\begin{array}{cccccc|c}
      1&1&0&5&0&-1&0\\
      0&1&1&3&-2&0&0\\
      0&0&0&0&7&-7&0\\
      0&0&0&0&1&-1&0\\
    \end{array}\right]
    \to
    \left[\begin{array}{cccccc|c}
      1&0&-1&2&0&1&0\\
      0&1&1&3&0&-2&0\\
      0&0&0&0&1&-1&0\\
      0&0&0&0&0&0&0\\
    \end{array}\right]\\
    \mathbf{x}&=\left[\begin{array}{ccccc}
    &&x_3&-2x_4&-x_6\\
    &&-x_3&-3x_4&+2x_6\\
    &&x_3\\
    &&&x_4\\
    &&&&x_6\\
    &&&&x_6
    \end{array}\right]
    =x_3\left[\begin{array}{c}1\\-1\\1\\0\\0\\0\end{array}\right]
    +x_4\left[\begin{array}{c}-2\\-3\\0\\1\\0\\0\end{array}\right]
    +x_4\left[\begin{array}{c}-1\\2\\0\\0\\1\\1\end{array}\right]
  \end{align*}
  \end{enumerate}
\item
  \begin{enumerate}
  \setcounter{enumii}{1}
  \item
  \begin{align*}
    &
    \left[\begin{array}{rr|r}
      2&-1&-4\\
      2&1&0\\
      -1&1&3\\
    \end{array}\right]
    \to
    \left[\begin{array}{rr|r}
      2&-1&-4\\
      0&2&4\\
      0&1&2\\
    \end{array}\right]
    \to
    \left[\begin{array}{rr|r}
      2&0&-2\\
      0&1&2\\
      0&0&0\\
    \end{array}\right]
    \to
    \left[\begin{array}{rr|r}
      1&0&-1\\
      0&1&2\\
      0&0&0\\
    \end{array}\right]\\
    \mathbf{x}=\left[\begin{array}{r}-1\\2\end{array}\right]
  \end{align*}
  \setcounter{enumii}{3}
  \item
  \begin{align*}
    &
    \left[\begin{array}{rrr|r}
      2&-1&1& 1\\
      2& 1&3&-1\\
      1& 1&2&-1\\
    \end{array}\right]
    \to
    \left[\begin{array}{rrr|r}
      1& 1& 2&-1\\
      0&-3&-3& 3\\
      0&-1&-1& 1\\
    \end{array}\right]
    \to
    \left[\begin{array}{rrr|r}
      1& 0& 1& 0\\
      0& 1& 1&-1\\
      0& 0& 0& 0\\
    \end{array}\right]\\
    \mathbf{x}&=\left[\begin{array}{r}0\\-1\\0\end{array}\right]+x_3\left[\begin{array}{r}-1\\-1\\1\end{array}\right]
  \end{align*}
  \item
  \begin{align*}
    &
    \left[\begin{array}{rrrr|r}
      1&1&1&1&6\\
      3&3&2&0&17\\
    \end{array}\right]
    \to
    \left[\begin{array}{rrrr|r}
      1&1& 1& 1&6\\
      0&0&-1&-3&-1\\
    \end{array}\right]
    \to
    \left[\begin{array}{rrrr|r}
      1&1& 0&-2& 5\\
      0&0& 1& 3& 1\\
    \end{array}\right]\\
    \mathbf{x}&
    =\left[\begin{array}{r}5\\0\\1\\0\end{array}\right]
    +x_2\left[\begin{array}{r}-1\\1\\0\\0\end{array}\right]
    +x_4\left[\begin{array}{r}2\\0\\-3\\1\end{array}\right]
  \end{align*}
  \end{enumerate}
\setcounter{enumi}{13}
\item
Let $A$ be an $m\times n$ matrix and let $b\in \mathbb{R}^m$.
  \begin{enumerate}
  \item
  Show that if $\mathbf{u}$ and $\mathbf{v}\in \mathbb{R}^n$ are both solutions of $A\mathbf{x}=\mathbf{b}$ then $\mathbf{u}-\mathbf{v}$ is a solution of $A\mathbf{x}=\mathbf{0}$.

  If $A\mathbf{u}=\mathbf{b}=A\mathbf{v}$ then $A\mathbf{u}-A\mathbf{v}=\mathbf{0}$ and by the distributive property $A(\mathbf{u}-\mathbf{v})=\mathbf{0}$ as required.
  \item
  Suppose $\mathbf{u}$ is a solution of $A\mathbf{x}=\mathbf{0}$ and $\mathbf{p}$ is a solution of $A\mathbf{x}=\mathbf{b}$. Show that $\mathbf{u}+\mathbf{p}$ is a solutioon of $A\mathbf{x}=\mathbf{b}$.

  We know that $A\mathbf{u}=\mathbf{0}$ and $A\mathbf{p}=\mathbf{b}$ and so we know that $A(\mathbf{u}+\mathbf{p})=A\mathbf{u}+A\mathbf{p}=\mathbf{0}+\mathbf{b}=\mathbf{b}$
  \end{enumerate}
\item
Prove or give a counterexample:
  \begin{enumerate}
  \item
  If $A$ is an $m\times n$ matrix and $\mathbf{x}\in \mathbb{R}^n$ satisfies $A\mathbf{x}=\mathbf{0}$, then either every entry of $A$ is $0$ or $\mathbf{x}=\mathbf{0}$
  \item
  If $A$ is an $m\times n$ matrix and $A\mathbf{x}=\mathbf{0}$ for every vector $\mathbf{x}\in \mathbb{R}^n$, then every entry of $A$ is 0.

  Let $A=\left[\begin{array}{rrr}1&-1&0\end{array}\right]$ and let $\mathbf{x}=(1,1,10^{100})$. Then $A\mathbf{x}=0$

  Assume $A\ne O$. Then $\exists \mathbf{r}_i(A)$ such that $\mathbf{r}_i(A)\ne \mathbf{0}$. Choose $\mathbf{x}=\mathbf{r}_i(A)$. Now $\text{ent}_{i}(A\mathbf{x})=\sum\limits_{k=1}^n{\text{ent}_k(\mathbf{r}_i(A))^2}$. Note that $\text{ent}_k(\mathbf{r}_i(A))^2\ge 0$. And because $\mathbf{r}_i(A)\ne 0$ then there is at least one non-zero element in that row and so $\text{ent}_{i}(A\mathbf{x}) >0$. Thus we have a contradiction, and so every entry of $A$ is 0.
  \end{enumerate}
\end{enumerate}
\section*{1.5}
\begin{enumerate}
\setcounter{enumi}{12}
\item
Suppose $A$ is an $m\times n$ matrix with rank $m$ and $\mathbf{v}_1,\dots,\mathbf{v}_k\in \mathbb{R}^n$ are vectors with $\text{Span}(\mathbf{v}_1,\dots,\mathbf{v}_k)=\mathbb{R}^n$. Prove that $\text{Span}(A\mathbf{v}_1,\dots,A\mathbf{v}_k)=\mathbb{R}^m$

First we note that $m\le n$ or else $A$ could not have rank $m$.
We choose an arbitrary $\mathbf{b}\in \mathbb{R}^m$. Because the rank of $A$ is no larger than $n$ we know that we can find some $x\in \mathbb{R}^n$ such that $A\mathbf{x}=\mathbf{b}$. Because $x\in \text{Span}(\mathbf{v}_1,\dots,\mathbf{v}_k)$ then $A\mathbf{x}=A(c_1\mathbf{v}_1+\dots+c_k\mathbf{v}_k)=c_1A\mathbf{v}_1+\dots+c_kA\mathbf{v}_k\in \text{Span}(A\mathbf{v}_1,\dots,A\mathbf{v}_k)$.

Thus all elements of $\mathbb{R}^m$ are contained in the Span. Usually to show equality one needs to show the reverse. But I claim that it is obvious that any element of the Span must be in $\mathbb{R}^m$.
\item
Let $A$ be an $m\times n$ matrix with row vectors $\mathbf{A}_1,\dots,\mathbf{A}_m\in\mathbb{R}^n$.
  \begin{enumerate}
  \item
  Suppose $\mathbf{A}_1+\dots+\mathbf{A}_m=\mathbf{0}$. Deduce that $\text{rank}(A)<m$.

  If $\text{rank}A=m$ then
  $\text{rref}(A)=\left[\begin{array}{cccc}
  1&\dots&0&\dots\\
  \vdots&\ddots&\vdots&\dots\\
  0&\dots&1&\dots
  \end{array}\right]$. Now it is clear that adding all the rows of this matrix will give us $(1_1,\dots,1_m,\dots)\ne 0$. But the sum of all the rows in $A$ is in fact $\mathbf{0}$ and so the rank of $A$ is less than $m$.
  \item
  More generally, suppose there is some linear combination $c_1\mathbf{A}_1+\dots+c_m\mathbf{A}_m=\mathbf{0}$ where some $c_i\ne 0$. Show that $\text{rank}(A)<m$.

  If $\text{rank}A=m$ then
  $\text{rref}(A)=\left[\begin{array}{cccc}
  1&\dots&0&\dots\\
  \vdots&\ddots&\vdots&\dots\\
  0&\dots&1&\dots
  \end{array}\right]$. Now it is clear that adding some combination of the rows of this matrix will give us $(c_1,\dots,c_m,\dots)\ne 0$. Now if there exists some $c_i\ne0$ then we have a non-zero vector. But the combination of the rows in $A$ is in fact $\mathbf{0}$ and so the rank of $A$ is less than $m$.
  \end{enumerate}
\item
Let $A$ be an $m\times n$ matrix with column vectors $\mathbf{a}_1,\dots,\mathbf{a}_n\in\mathbb{R}^m.$
  \begin{enumerate}
  \item
    Suppose $\mathbf{a}_1+\dots+\mathbf{a}_n=\mathbf{0}$. Prove that $\text{rank}(A)<n$.

    Recall that $A\left(\begin{array}{c}x_1\\\vdots\\x_n\end{array}\right)=x_1\mathbf{a}_1+\dots+x_n\mathbf{a}_n$. Since $\mathbf{a}_1+\dots+\mathbf{a}_n=\mathbf{0}$ we have $A\left(\begin{array}{c}1\\\vdots\\1\end{array}\right)=\mathbf{0}$ and so we have more than one solution for $A\mathbf{x}=\mathbf{0}$ and so $\text{rank}(A)<n$.
  \item
    More generally, suppose there is some linear combination $c_1\mathbf{a}_1+\dots+c_n\mathbf{a}_n=\mathbf{0}$ where some $c_i\ne 0$. Prove that $\text{rank}(A)<n$.

    Recall that $A\left(\begin{array}{c}x_1\\\vdots\\x_n\end{array}\right)=x_1\mathbf{a}_1+\dots+x_n\mathbf{a}_n$. Since $c_1\mathbf{a}_1+\dots+c_n\mathbf{a}_n=\mathbf{0}$ we have $A\left(\begin{array}{c}c_1\\\vdots\\c_n\end{array}\right)=\mathbf{0}$. Because $\exists c_i\ne0$ then $\left(\begin{array}{c}c_1\\\vdots\\c_n\end{array}\right)\ne \mathbf{0}$ and so we have more than one solution for $A\mathbf{x}=\mathbf{0}$ and so $\text{rank}(A)<n$.
  \end{enumerate}
\end{enumerate}
\end{document}
