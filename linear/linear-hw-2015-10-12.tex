\documentclass[letterpaper]{article}

\usepackage[utf8x]{luainputenc}
\usepackage{aeguill}
%\usepackage{nopageno}
\usepackage{amsmath}
\usepackage{amssymb}
\usepackage{mathrsfs}
\usepackage{fullpage}
\usepackage{fancyhdr}
\setlength{\headheight}{12pt}
\pagestyle{fancy}
\chead{Linear Algebra}
\lhead{October 12, 2015}
\rhead{Jon Allen}
\allowdisplaybreaks

\newcommand{\abs}[1]{\left\lvert #1 \right\rvert}

\begin{document}
%\renewcommand{\labelenumii}{\alph{enumii}.}
%\renewcommand{\labelenumiii}{\alph{enumiii}.}
%\renewcommand{\labelenumi}{(\arabic{enumi})}
\section*{3.1}
%6, 8, 9(b,c), 11, 15, 16.
\begin{enumerate}
\setcounter{enumi}{5}
\item
  \begin{enumerate}
  \item
  Let $U$ and $V$ be subspaces of $\mathbb{R}^n$. Define the \emph{intersection} of $U$ and $V$ to be
  \[U\cap V=\{\mathbf{x}\in \mathbb{R}^n:\mathbf{x}\in U\text{ and }\mathbf{x}\in V\}\]
  Show that $U\cap V$ is a subspace of $\mathbb{R}^n$. Give two examples.

  We know that $\mathbf{0}\in U$ and $\mathbf{0}\in V$. And so $\mathbf{0}\in U\cap V$. Now if we take any $\mathbf{u},\mathbf{v}\in U\cap V$ then $\mathbf{u},\mathbf{v}\in U$ and $\mathbf{u},\mathbf{v}\in V$ and so $\mathbf{u}+\mathbf{v}\in U$ and $\mathbf{u}+\mathbf{v}\in V$. Thus $\mathbf{u}+\mathbf{v}\in  U\cap V$. Similarly $\alpha\mathbf{u}\in U$ and $\alpha\mathbf{u}\in V$ for any $\alpha\in \mathbb{R}$. Thus $\alpha\mathbf{u}\in U\cap V$.
  \item
  Is $U\cup V=\{\mathbf{x}\in \mathbb{R}^n:\mathbf{x}\in U\text{ or }\mathbf{x}\in V\}$ always a subspace of $\mathbb{R}^n$? Give a proof or counterexample.

  Let $V=\{3n:\forall n\in \mathbb{Z}\}$ and $U=\{2n:\forall n\in \mathbb{Z}\}$. Then $3\in V$ and $2\in U$. Therefore $3,2\in U\cup V$. But $3+2=5\not\in V$ and $5\not\in U$. Therefore $5\not\in U\cup V$ and so we have a counterexample.
  \end{enumerate}
\setcounter{enumi}{7}
\item
Let $\mathbf{v}_1,\dots,\mathbf{v}_k\in \mathbb{R}^n$ and let $\mathbf{v}\in \mathbb{R}^n$. Prove that $\text{Span}(\mathbf{v}_1,\dots,\mathbf{v}_k)=\text{Span}(\mathbf{v}_1,\dots,\mathbf{v}_k,\mathbf{v})$ if and only if $\mathbf{v}\in \text{Span}(\mathbf{v}_1,\dots,\mathbf{v}_k)$.

Define $A=\text{Span}(\mathbf{v}_1,\dots,\mathbf{v}_k)$ and $B=\text{Span}(\mathbf{v}_1,\dots,\mathbf{v}_k,\mathbf{v})$.

We know that $\mathbf{v}\in B$ and so if we assume that $A=B$ then $\mathbf{v}\in A$ follows immediately. Now lets assume that $\mathbf{v}\in A$. Then for any $\mathbf{x}\in A$ we have $\mathbf{x}=\alpha_1\mathbf{v}_1+\dots+\alpha_k\mathbf{v}_k+0\mathbf{v}\in B$.
Now, because $\mathbf{v}=\alpha_1\mathbf{v}_1+\dots+\alpha_k\mathbf{v}_k$ Then any element $\mathbf{x}\in B$ we have $\mathbf{x}=\beta_1\mathbf{v}_1+\dots+\beta_k\mathbf{v}_k+\beta\mathbf{v}=\beta_1\mathbf{v}_1+\dots+\beta_k\mathbf{v}_k+\beta(\alpha_1\mathbf{v}_1+\dots+\alpha_k\mathbf{v}_k)=(\beta_1+\beta\alpha_1)\mathbf{v}_1+\dots+(\beta_k+\beta\alpha_k)\mathbf{v}_k\in A$.
$\Box$
\item
Determine the intersection of the subspaces $\mathcal{P}_1$ and $\mathcal{P}_2$ in each case:
  \begin{enumerate}
  \setcounter{enumii}{1}
  \item
    $\mathcal{P}_1=\text{Span}\left((1,2,2),(0,1,1)\right),\mathcal{P}_2=\text{Span}\left((2,1,1),(1,0,0)\right)$
    \begin{align*}
      a\left(\begin{array}{r}1\\2\\2\end{array}\right)
      +b\left(\begin{array}{r}0\\1\\1\end{array}\right)
      -c\left(\begin{array}{r}2\\1\\1\end{array}\right)
      -d\left(\begin{array}{r}1\\0\\0\end{array}\right)
      =0\Rightarrow
      \left[\begin{array}{rrrr}
      1&0&2&1\\
      2&1&1&0\\
      2&1&1&0
      \end{array}\right]\Rightarrow
      \left[\begin{array}{rrrr}
      1&0&2&1\\
      0&1&-3&-2\\
      0&0&0&0
      \end{array}\right]
    \end{align*}
    And so we have two free variables. Which means given any element in $\mathcal{P}_1$ we can find that element in $\mathcal{P}_2$ and the other way around. That is $\mathcal{P}_1=\mathcal{P}_2$
  \item
    $\mathcal{P}_1=\text{Span}\left((1,0,-1),(1,2,3)\right),\mathcal{P}_2=\{\mathbf{x}:x_1-x_2+x_3=0\}$
    Converting $\mathcal{P}_2$ to standard form we have $\mathbf{x}=\left(\begin{array}{c}x_1\\x_1+x_3\\x_3\end{array}\right)$ or $\mathcal{P}_2=\text{Span}\{(1,1,0),(0,1,1)\}$. And so we put everything in an array as above
    \[
    \left[\begin{array}{rrrr}
    1&1&-1&0\\
    0&2&-1&-1\\
    -1&3&0&-1
    \end{array}\right]
    \left[\begin{array}{rrrr}
    1&1&-1&0\\
    0&2&-1&-1\\
    0&4&-1&-1
    \end{array}\right]
    \left[\begin{array}{rrrr}
    1&1&-1&0\\
    0&2&0&0\\
    0&2&-1&-1
    \end{array}\right]
    \left[\begin{array}{rrrr}
    1&1&-1&0\\
    0&1&0&0\\
    0&0&1&1
    \end{array}\right]
    \left[\begin{array}{rrrr}
    1&0&0&1\\
    0&1&0&0\\
    0&0&1&1
    \end{array}\right]
    \]
    Which gives us one free variable, and so the two planes intersect on a line. We notice that $(1,0,-1)=(1,1,0)-(0,1,1)$ and so since $(1,0,-1)$ is in $P_1$ and in $\mathcal{P}_2$ so any element in $\text{Span}(1,0,-1)$ is in the intersection, thus we have found our line.
  \end{enumerate}
\setcounter{enumi}{10}
\item
Suppose $V$ and $W$ are orthogonal subspaces of $\mathbb{R}^n$, i.e., $\mathbf{v}\cdot\mathbf{w}=0$ for every $\mathbf{v}\in V$ and every $\mathbf{w}\in W$.
Prove that $V\cap W=\{\mathbf{0}\}$.

Start by assuming there exists some element  $\mathbf{u}\in V\cap W$ such that $\mathbf{u}\ne \mathbf{0}$.
Then because $\mathbf{u}\in V$ and $\mathbf{u}\in W$ we know that $\mathbf{u}\cdot\mathbf{u}=\mathbf{0}$.
And of course $\mathbf{u}\cdot\mathbf{u}=||\mathbf{u}||^2$.
But $\mathbf{u}\ne 0$ and so $||\mathbf{u}||^2>0$ and $\mathbf{u}\cdot\mathbf{u}\ne 0$ which leaves us with a contradiction.
We must assume then that the intersection of these sets contains only $\mathbf{0}$.
\setcounter{enumi}{14}
\item
Let $A$ be an $m\times n$ matrix. Let $V\subset \mathbb{R}^n$ and $W\subset \mathbb{R}^m$ be subspaces.
  \begin{enumerate}
  \item
  Show that $E=\{\mathbf{x}\in \mathbb{R}^n:A\mathbf{x}\in W\}$ is a subspace of $\mathbb{R}^n$.

  Of course $A\mathbf{0}=\mathbf{0}\in W$. Let us choose $\mathbf{x},\mathbf{y}\in E$. Then we have $A\mathbf{x}\in W$ and $A\mathbf{y}\in W$ and so $A\mathbf{x}+A\mathbf{y}\in W$. But $A\mathbf{x}+A\mathbf{y}=A(\mathbf{x}+\mathbf{y})$ and so $\mathbf{x}+\mathbf{y}\in E$. And if $A\mathbf{x}\in W$ then $\alpha A\mathbf{x}\in W$ for any $\alpha\in \mathbb{R}$. So because $\alpha A\mathbf{x}=A(\alpha\mathbf{x})$ then we have $\alpha\mathbf{x}\in E$. And so $E$ is a subspace.
  \item
  Show that $F=\{\mathbf{y}\in \mathbb{R}^m:\mathbf{y}=A\mathbf{x}\text{ for some }\mathbf{x}\in V\}$ is a subspace of $\mathbb{R}^m$

  Observe that $\mathbf{0}=A\mathbf{0}$ and $\mathbf{0}\in V$. And so $\mathbf{0}\in F$. Now, let us take $\mathbf{u},\mathbf{v}\in F$. Then $\mathbf{u}=A\mathbf{x}_1$ and $\mathbf{v}=A\mathbf{x}_2$ where $\mathbf{x}_1,\mathbf{x}_2\in V$. Now $\mathbf{u}+\mathbf{v}=A\mathbf{x}_1+A\mathbf{x}_2=A(\mathbf{x}_1+\mathbf{x}_2)$. But $\mathbf{x}_1+\mathbf{x}_2\in V$ and so $\mathbf{u}+\mathbf{v}\in F$. And if we choose $\alpha\in \mathbb{R}$ then $\alpha\mathbf{u}=\alpha A\mathbf{x}=A(\alpha\mathbf{x})$ for some $\mathbf{x}\in V$. And since $\mathbf{x}\in V$ then $\alpha\mathbf{x}\in V$ and so $\alpha\mathbf{u}\in F$. Thus $F$ is a subspace.
  \end{enumerate}
\item
Suppose $A$ is a symmetric $n\times n$ matrix. Let $V\subset \mathbb{R}^n$ be a  subspace with the property that $A\mathbf{x}\in V$ for every $\mathbf{x}\in V$. Show that $A\mathbf{y}\in V^\perp$ for all $\mathbf{y}\in V^\perp$

We choose some $\mathbf{y}\in V^\perp$ and $\mathbf{x}\in V$. Then $\mathbf{x}\cdot\mathbf{y}=\mathbf{0}$. Now if $\mathbf{x}A\mathbf{y}=\mathbf{0}$ then $A\mathbf{y}\in V^\perp$. Of course $\mathbf{x}A\mathbf{y}=A^T\mathbf{x}^T\mathbf{y}=(A\mathbf{x})\mathbf{y}$. But $A\mathbf{x}\in V$ and so $(A\mathbf{x})\mathbf{y}=0$ therefore $A\mathbf{y}\in V^\perp$
\end{enumerate}
\end{document}
