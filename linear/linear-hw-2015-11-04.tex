\documentclass[letterpaper]{article}

\usepackage[utf8x]{luainputenc}
\usepackage{aeguill}
%\usepackage{nopageno}
\usepackage{amsmath}
\usepackage{amssymb}
\usepackage{mathrsfs}
\usepackage{fullpage}
\usepackage{fancyhdr}
\setlength{\headheight}{12pt}
\pagestyle{fancy}
\chead{Linear Algebra}
\lhead{November 4, 2015}
\rhead{Jon Allen}
\allowdisplaybreaks

\newcommand{\abs}[1]{\left\lvert #1 \right\rvert}
\newcommand{\proj}{\text{proj}}
\newcommand{\rank}{\text{rank}}
\newcommand{\Span}{\text{Span}}

\begin{document}
%\renewcommand{\labelenumii}{\alph{enumii}.}
%\renewcommand{\labelenumiii}{\alph{enumiii}.}
%\renewcommand{\labelenumi}{(\arabic{enumi})}
\begin{enumerate}
\item
Let $T:\mathbb{R}^3\to\mathbb{R}^2$ be given by
\[T\left(\begin{array}{r}x\\y\\z\end{array}\right)
  =\left(\begin{array}{r}x-2y\\2x+y\end{array}\right)\]
  \begin{enumerate}
  \item
  Find $A_T$

  $\left(\begin{array}{rrr}1&-2&0\\2&1&0\end{array}\right)$
  \item
  Is $T$ 1-1? If not, does there exist a 1-1 map $P:\mathbb{R}^3\to\mathbb{R}^2$? Justify.

  No, and no, $A_T\in \mathcal{M}_{2\times 3}$ which means that $\dim(\text{Null}(A_T))\ge 1$. And so $\{0\}$ is a proper subset of $\text{Null}(A_T)$ and always will be regardless of our choice of $A_T$.
  \item
  Is $T$ onto? Justify.

  Yep, it's onto. The column space clearly has dimension 2, and so it is equivalent to $\mathbb{R}^2$
  \item
  Find $\dim(\ker T)$ and $\dim(\text{Im} T)$

  The dimension of the column space is the same as the dimension of the image, and as we just pointed out, $\dim(\text{Col} A_T)=2=\dim(\text{Im} T)$. And the null space dimension is the number of columns of $A_T$ minus the dimension of the column space, and so $\dim(\text{Null} T)=1$
  \end{enumerate}
\item
Let $S:\mathbb{R}^2\to\mathbb{R}^4$ be given by
\[S\left(\begin{array}{r}x\\y\end{array}\right)
  =\left(\begin{array}{r}3x\\y-2x\\2x\\x+2y\end{array}\right)\]
Find a formula for $(S\circ T):\mathbb{R}^3\to\mathbb{R}^4$
\[\left(\begin{array}{rr}3&0\\-2&1\\2&0\\1&2\end{array}\right)
  \left(\begin{array}{rrr}1&-2&0\\2&1&0\end{array}\right)
 =\left(\begin{array}{rrr}3&-6&0\\0&5&0\\2&-3&0\\5&0&0\end{array}\right)
\]
\item
Prove theorem 11: If $T:\mathbb{R}^n\to\mathbb{R}^n$ is a bijective linear transformation, then $T^{-1}:R^n\to R^n$ is a linear transformation. Moreover, $A_{T^{-1}}=A_T^{-1}$

We know that $T(\mathbf{x}+\mathbf{y})=T(\mathbf{x})+T\mathbf{y}$. If we apply $T^{-1}$ to both sides of the equation, then we have $T^{-1}(T(\mathbf{x})+T(\mathbf{y}))=T^{-1}(T(\mathbf{x}+\mathbf{y}))=\mathbf{x}+\mathbf{y}=T^{-1}(T(\mathbf{x}))+T^{-1}(T(\mathbf{y}))$. Now we we observe $cT^{-1}(T(\mathbf{x}))=c\mathbf{x}=T^{-1}(T(c\mathbf{x}))$ and so $T^{-1}$ is linear. Now we know that $T\circ T^{-1}$ maps any $\mathbf{x}\in \mathbb{R}^n$ to itself. And so if $A_{T^{-1}\circ T}\mathbf{x}=\mathbf{x}$ then $A_{T^{-1}\circ T}=I_n$. And so $A_{T^{-1}}A_T=I_n$. This means that $A_{T^{-1}=A_T^{-1}}$
\item
Let $T:\mathbb{R}^3\to\mathbb{R}^3$ be given by
\[T\left(\begin{array}{r}x\\y\\z\end{array}\right)
=\left(\begin{array}{r}3x+3y+z\\3x+3y+z\\2x+4y+z\end{array}\right)\]
Prove that $T$ is an invertible map (1-1 and onto) and find a formula for $T^{-1}:\mathbb{R}^3\to\mathbb{R}^3$.
\[
\left(\begin{array}{rrr|rrr}
2&3&1&1&0&0\\
3&3&1&0&1&0\\
2&4&1&0&0&1\\
\end{array}\right)
\Rightarrow
\left(\begin{array}{rrr|rrr}
2&3&1& 1& 0&0\\
1&0&0&-1& 1&0\\
0&1&0&-1& 0&1\\
\end{array}\right)
\Rightarrow
\left(\begin{array}{rrr|rrr}
0&3&1& 3&-2&0\\
1&0&0&-1& 1&0\\
0&1&0&-1& 0&1\\
\end{array}\right)\]
\[
\Rightarrow
\left(\begin{array}{rrr|rrr}
0&0&1& 6&-2&-3\\
1&0&0&-1& 1& 0\\
0&1&0&-1& 0& 1\\
\end{array}\right)
\Rightarrow
\left(\begin{array}{rrr|rrr}
1&0&0&-1& 1& 0\\
0&1&0&-1& 0& 1\\
0&0&1& 6&-2&-3\\
\end{array}\right)
\]
So Null$(A_T)=\{\mathbf{0}\}$ and $\dim(\text{Col}(A_T))=3$ or $\dim(\text{Col}(A_T))=\mathbb{R}^3$. Thus $T$ is 1-1 and onto.
\[T^{-1}
\left(\begin{array}{r}x\\y\\z\end{array}\right)=
\left(\begin{array}{rrr}
-1& 1& 0\\
-1& 0& 1\\
 6&-2&-3\\
\end{array}\right)
\left(\begin{array}{r}x\\y\\z\end{array}\right)
= \left(\begin{array}{c}
-x+ y\\
-x+ z\\
6x-2y-3z
\end{array}\right)
\]
\item
Construct a linear transformation $\rho:\mathbb{R}^2\to\mathbb{R}^2$ that rotates every vector through an angle of $\theta=\frac{\pi}{2}$. Find the standard matrix $A_\rho$ of the transformation and verify that $\rho$ really does rotate the plane through $\theta=\frac{\pi}{2}$.
\begin{align*}
  \rho\left(\begin{array}{c}x\\y\end{array}\right)
  &=\left(\begin{array}{r}-y\\x\end{array}\right)&
  A_\rho
  &=\left(\begin{array}{rr}0&-1\\1&0\end{array}\right)\\
  r&=\sqrt{x^2+y^2}&
  \left(\begin{array}{r}x\\y\end{array}\right)
  &=\left(\begin{array}{r}r\cos\theta\\r\sin\theta\end{array}\right)\\
  \rho\left(\begin{array}{c}x\\y\end{array}\right)
  &= \left(\begin{array}{rr}0&-1\\1&0\end{array}\right)
  \left(\begin{array}{r}r\cos\theta\\r\sin\theta\end{array}\right)
  &&=\left(\begin{array}{r}-r\sin\theta\\r\cos\theta\end{array}\right)\\
  &=\left(\begin{array}{r}r\sin-\theta\\r\cos-\theta\end{array}\right)
  &&=\left(\begin{array}{r}r\sin\left(\theta+\frac{\pi}{2}\right)\\r\cos\left(\theta+\frac{\pi}{2}\right)\end{array}\right)\\
\end{align*}
\item
Let $B=\{\mathbf{v}_1,\dots,\mathbf{v}_n\}$ be a \emph{basis} of $\mathbb{R}^n$. Let $\{\mathbf{w}_1,\dots,\mathbf{w}_n\}$ be a \emph{list} of $n$ vectors in $\mathbb{R}^m$. Prove the following statements.
  \begin{enumerate}
  \item
  There exists a linear transformation $T:\mathbb{R}^n\to\mathbb{R}^m$ such that $T(\mathbf{v}_i)=\mathbf{w}_i$ for each index $i\le n$.

  We say $B$ is the matrix $\left(\begin{array}{ccc}\mathbf{v}_1&\dots&\mathbf{v}_n\end{array}\right)$ and $C$ is the matrix $\left(\begin{array}{ccc}\mathbf{v}_1&\dots&\mathbf{v}_n\end{array}\right)$. Now because $\{\mathbf{v}_1,\dots,\mathbf{v}_n\}$ are linearly independent, we know that $B$ has $\rank n$ and is therefore nonsingular, or it has an inverse. Now $C=CI_n=(CB^{-1})B$. Thus we see that $CB^{-1}\mathbf{v}_i=\mathbf{w}_i$ and we have found our transformation.
  \item
  If $S:\mathbb{R}^n\to\mathbb{R}^m$ is another linear transformation such that $S(\mathbf{v}_i)=\mathbf{w}_i$ for each index $i\le n$, then  $S=T$.

  If $S\ne T$ then there exists some $x\in \mathbb{R}^n$ such that $T(\mathbf{x})\ne S(\mathbf{x})$. Because the $\mathbf{v}_i$'s form a basis for $\mathbb{R}^n$, we can rewrite $\mathbf{x}=a_1\mathbf{v}_1+\dots+a_n\mathbf{v}_n$. And so we have
  \begin{align*}
    T(a_1\mathbf{v}_1+\dots+a_n\mathbf{v}_n)
    &\ne S(a_1\mathbf{v}_1+\dots+a_n\mathbf{v}_n)\\
    a_1T(\mathbf{v}_1)+\dots+a_nT(\mathbf{v}_n)
    &\ne a_1S(\mathbf{v}_1)+\dots+a_nS(\mathbf{v}_n)\\
    a_1\mathbf{w}_1+\dots+a_n\mathbf{w}_n
    &\ne a_1\mathbf{w}_1+\dots+a_n\mathbf{w}_n
  \end{align*}
  That is absurd, so we must conclude that $T=S$
  \item
  $T$ is onto if and only if $\{\mathbf{w}_1,\dots,\mathbf{w}_n\}$ spans $\mathbb{R}^m$.

  We can express any element in the image of $T$ as a linear combination of the images of the elements of $B$ under $T$ which means each element in the image of $T$ can be expressed as a linear combination of $\{\mathbf{w}_1,\dots,\mathbf{w}_n\}$. Now if $T$ is onto, then every element in $\mathbb{R}^m$ must then be in the span of our $\mathbf{w}$'s. We note that every linear combination of our $\mathbf{w}$'s can be expressed at a linear combination of our $T(\mathbf{v})$'s and so every linear combination of $\mathbf{w}$ is in the image of $T$. Now if $T$ is not onto, then there is an element in $\mathbb{R}^m$ that is not in the image of $T$ and therefore not in the span of our $\mathbf{w}$'s.
  \item
  $T$ is 1-1 iff $\{\mathbf{w}_1,\dots,\mathbf{w}_n\}$ is a linearly independent subset of $\mathbb{R}^m$.

  If $T$ is not 1-1 then the nullspace of $A_T$ is not trivial and there exists some $0=A_T(a_1\mathbf{v}_1+\dots+a_n\\mathbf{v}_n)=a_1A_T\mathbf{v}_1+\dots+a_nA_t\mathbf{v}_n=a_1\mathbf{w}_1+\dots+a_n\mathbf{w}_n$ where at least one $a_i\ne 0$. Thus our $\mathbf{w}$'s are not linearly independent. Now if our $\mathbf{w}$'s are not linearly independent, then there exists at least one $a_i\ne0$ such that $0=a_1\mathbf{w}_1+\dots+a_n\mathbf{w}_n=A_T(a_1\mathbf{v}_1+\dots+a_n\mathbf{v}_n)$ and we have found a nonzero element in the nullspace of $A_T$ and so $T$ is not 1-1.
  \item
  $T$ is a bijection iff $\{\mathbf{w}_1,\dots,\mathbf{w}_n\}$ is a basis of $\mathbb{R}^m$

  From c and d we know that if $T$ is 1-1 and onto then our $\mathbf{w}$'s are a linearly independent and span $\mathbb{R}^m$. Also if our $\mathbf{w}$'s are linearly independent and span $\mathbb{R}^m$ then $T$ is 1-1 and onto. Thus by the definitions of basis and bijection we have our result.
  \end{enumerate}
\item
Let $T:\mathbb{R}^n\to\mathbb{R}^m$ be a linear transformation. Prove the following statements.

We say $A_T$ is the $m \times n$ matrix that is associate with $T$
  \begin{enumerate}
  \item
  $\dim(\text{Im}T)\le n$.

  $\text{Im}T$ is the column space of $A_T$ which is the rank of $A_T$ which is less than $n$
  \item
  $n=\dim(\ker T)+\dim(\text{Im}T)$

  $\ker T$ is the nullspace of $A_T$ and $\dim(\text{Im}T)$ is the rank of $A_T$. The dimension of the rank plus the dimension of the nullspace is the width of a matrix.
  \item
  If $T$ is 1-1 then $n\le m$

  If $T$ is 1-1 then the nullspace of $A_T$ is trivial, and so $\rank(A_T)=n$. Because the rank cannot be greater than either of a matrix's dimensions, then $n\le m$
  \item
  If $T$ is onto then $m\le n$

  If $T$ is onto then $\rank(A_T)=\dim(\text{Im})= \dim(\mathbb{R}^m)$ and since the rank cannot be greater than either of the dimensions then $m\le n$.
  \item
  If $n=m$, then $T$ is onto if and only if $T$ is a bijection if and only if $T$ is 1-1.

  We always assume that $n=m$. Now if $T$ is onto then $\text{Im}T=\mathbb{R}^m=\text{Col}(A_T)$. Now because $\dim(\text{Null}(A_T)+\dim(\text{Col}T)=m=n$ then $\dim(\text{Null}(A_T))=0$ and $\text{Null}(A_T))=\{0\}$. And so $T$ is onto. Now if $T$ is 1-1 then $\text{Null}(A_T)=\{0\}$ and because $\dim(\text{Null}(A_T)+\dim(\text{Col}T)=m=n$ then $\text{Col}A_T=\mathbb{R}^m$ and so $T$ is onto. The bijective bit in the middle comes for free since $T$ is 1-1 iff $T$ is onto.
  \end{enumerate}
\end{enumerate}
\end{document}
