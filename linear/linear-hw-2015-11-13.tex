\documentclass[letterpaper]{article}

\usepackage[utf8x]{luainputenc}
\usepackage{aeguill}
%\usepackage{nopageno}
\usepackage{amsmath}
\usepackage{amssymb}
\usepackage{mathrsfs}
\usepackage{fullpage}
\usepackage{fancyhdr}
\setlength{\headheight}{12pt}
\pagestyle{fancy}
\chead{Linear Algebra}
\lhead{November 4, 2015}
\rhead{Jon Allen}
\allowdisplaybreaks

\newcommand{\abs}[1]{\left\lvert #1 \right\rvert}
\newcommand{\proj}{\text{proj}}
\newcommand{\rank}{\text{rank}}
\newcommand{\Span}{\text{Span}}

\begin{document}
%\renewcommand{\labelenumii}{\alph{enumii}.}
%\renewcommand{\labelenumiii}{\alph{enumiii}.}
%\renewcommand{\labelenumi}{(\arabic{enumi})}
\begin{enumerate}
\item
Let
\[
B=
\left\{\left(\begin{array}{r}1\\2\end{array}\right),\left(\begin{array}{r}3\\-1\end{array}\right)\right\}
\text{ and }
C=\left\{\left(\begin{array}{r}1\\0\\0\end{array}\right),\left(\begin{array}{r}1\\1\\0\end{array}\right),\left(\begin{array}{r}1\\1\\1\end{array}\right)\right\}
\]
be bases of $\mathbb{R}^2$ and $\mathbb{R}^3$ (resp). Let $T:\mathbb{R}^2\to \mathbb{R}^3$ be given by
\[T\left(\begin{array}{r}x\\y\end{array}\right)=\left(\begin{array}{c}x+2y\\-x\\y\end{array}\right)\]
and let
\[\mathbf{x}=\left(\begin{array}{r}-7\\7\end{array}\right)\]
  \begin{enumerate}
  \item
  Find the image $T(\mathbf{x})$ of the vector $\mathbf{x}$ under the action of $T$.

  $T(\mathbf{x})=\left(\begin{array}{r}7\\7\\7\end{array}\right)$
  \item
  Find the change of basis matrices $P=M(1_{\mathbb{R}^2},S_2,B)$ and $Q=M(1_{\mathbb{R}^3},S_3,C)$.

  $ P=\left[\begin{array}{rr}1&3\\2&-1\end{array}\right]^{-1}
  \Rightarrow
  \left[\begin{array}{rr|rr}1&3&1&0\\2&-1&0&1\end{array}\right]
  \Rightarrow
  \left[\begin{array}{rr|rr}
    1& 3& 1& 0\\
    0&-7&-2& 1
  \end{array}\right]
  \Rightarrow
  \left[\begin{array}{rr|rr}
    1& 0&1/7&3/7\\
    0& 1&2/7&-1/7
  \end{array}\right] $

  $Q=
  \left[\begin{array}{rrr}
    1&1&1\\
    0&1&1\\
    0&0&1
  \end{array}\right]^{-1}
  \Rightarrow
  \left[\begin{array}{rrr|rrr}
    1&1&1&1&0&0\\
    0&1&1&0&1&0\\
    0&0&1&0&0&1
  \end{array}\right]
  \Rightarrow
  \left[\begin{array}{rrr|rrr}
    1&0&0&1&-1& 0\\
    0&1&0&0& 1&-1\\
    0&0&1&0& 0& 1
  \end{array}\right]
  $

  $\displaystyle P=\frac{1}{7}
  \left[\begin{array}{rr}
    1&3\\
    2&-1
  \end{array}\right]
  \qquad
  Q=
  \left[\begin{array}{rrr}
    1&-1& 0\\
    0& 1&-1\\
    0& 0& 1
  \end{array}\right] $
  \item
  Use part (b) to compute $[\mathbf{x}]_B$ and $[T(\mathbf{x})]_C$.

  $\displaystyle[\mathbf{x}]_B=P\mathbf{x}=\frac{1}{7}\left(\begin{array}{r}14\\-21\end{array}\right)=\left(\begin{array}{r}2\\-3\end{array}\right)$

  $\displaystyle[T(\mathbf{x})]_C=QT(\mathbf{x})=\left(\begin{array}{r}0\\0\\7\end{array}\right)$

  \item
  Find $M(T,B,C)$ using the method of Example 7.
  \begin{align*}
    T\left(\begin{array}{r}1\\2\end{array}\right)=\left(\begin{array}{r}5\\-1\\2\end{array}\right)
    &=c_{11}\left(\begin{array}{r}1\\0\\0\end{array}\right)
    +c_{21}\left(\begin{array}{r}1\\1\\0\end{array}\right)
    +c_{31}\left(\begin{array}{r}1\\1\\1\end{array}\right)\\
    &=6\left(\begin{array}{r}1\\0\\0\end{array}\right)
    +-3\left(\begin{array}{r}1\\1\\0\end{array}\right)
    +2\left(\begin{array}{r}1\\1\\1\end{array}\right)\\
    T\left(\begin{array}{r}3\\-1\end{array}\right)=\left(\begin{array}{r}1\\-3\\-1\end{array}\right)
    &=c_{12}\left(\begin{array}{r}1\\0\\0\end{array}\right)
    +c_{22}\left(\begin{array}{r}1\\1\\0\end{array}\right)
    +c_{32}\left(\begin{array}{r}1\\1\\1\end{array}\right)\\
    &=4\left(\begin{array}{r}1\\0\\0\end{array}\right)
    +-2\left(\begin{array}{r}1\\1\\0\end{array}\right)
    +-1\left(\begin{array}{r}1\\1\\1\end{array}\right)\\
    M(T,B,C)&=\left[\begin{array}{rr}6&4\\-3&-2\\2&-1\end{array}\right]
  \end{align*}
  \item
  Find $M(T,B,C)$ using the method of example 10.

  $M(T,B,C)=QA_TP^{-1}=
  \left[\begin{array}{rrr}1&-1&0\\0&1&-1\\0&0&1\end{array}\right]
  \left[\begin{array}{rr}1&2\\-1&0\\0&1\end{array}\right]
  \left[\begin{array}{rr}1&3\\2&-1\end{array}\right]
  =
  \left[\begin{array}{rr}6&4\\-3&-2\\2&-1\end{array}\right]
  $
  \item
  Check our answer in part (c) by verifying that $M(T,B,C)[\mathbf{x}]_B=[T(\mathbf{x})]_C$.
  
  $
  \left[\begin{array}{rr}6&4\\-3&-2\\2&-1\end{array}\right]
  \left(\begin{array}{r}2\\-3\end{array}\right)
  =
  \left(\begin{array}{r}0\\0\\7\end{array}\right)
  $
  \end{enumerate}
\item
Let
\[
B=
\left\{\left(\begin{array}{r}-3\\2\end{array}\right),\left(\begin{array}{r}4\\-2\end{array}\right)\right\}
\text{ and }
C=
\left\{\left(\begin{array}{r}-1\\2\end{array}\right),\left(\begin{array}{r}2\\-2\end{array}\right)\right\}
\]
be bases of $\mathbb{R}^2$. If
\[M(T,B,B)=\left[\begin{array}{rr}-2&7\\-3&7\end{array}\right]\]
find $M(T,B,B)$.

Maybe we are meant to find $M(T,C,C)$?
\begin{align*}
  M(T,B,B)&=\left[\begin{array}{rr}-2&7\\-3&7\end{array}\right]\\
  &=\left[\begin{array}{rr}-3&4\\2&-2\end{array}\right]^{-1}
  T
  \left[\begin{array}{rr}-3&4\\2&-2\end{array}\right]\\
  T&=\left[\begin{array}{rr}-3&4\\2&-2\end{array}\right]
  \left[\begin{array}{rr}-2&7\\-3&7\end{array}\right]
  \left[\begin{array}{rr}-3&4\\2&-2\end{array}\right]^{-1}\\
  T&=\left[\begin{array}{rr}-3&4\\2&-2\end{array}\right]
  \left[\begin{array}{rr}-2&7\\-3&7\end{array}\right]
  \left[\begin{array}{rr}1&2\\1&\frac{3}{2}\end{array}\right]\\
  M(T,C,C)&=\left[\begin{array}{rr}-1&2\\2&-2\end{array}\right]^{-1}
  \left[\begin{array}{cc} 1 & -\frac{3}{2} \\ 2 & 4 \end{array}\right]
  \left[\begin{array}{rr}-1&2\\2&-2\end{array}\right]\\
  &=\left[\begin{array}{rr}2&1\\-1&3\end{array}\right]
\end{align*}
\item
Prove Theorem 11. Let $B_1,B_2$ be bases for $\mathbb{R}^n$ and let $C_1,C_2$ be bases for $\mathbb{R}^m$. If $T:\mathbb{R}^n\to\mathbb{R}^m$ is a linear transformation, then $\rank(M(T,B_1,C_1))=\rank(M(T,B_2,C_2))$.

Now $M(T,B_1,C_1)[\mathbf{x}]_{B_1}=[T(\mathbf{x})]_{C_1}$
and $M(T,B_2,C_2)[\mathbf{x}]_{B_2}=[T(\mathbf{x})]_{C_2}$.
But $[T(\mathbf{x})]_{C_1}$ and $[T(\mathbf{x})]_{C_2}$

Let $B_1=\{\mathbf{v}_{11},\dots,\mathbf{v}_{1n}\}$
and $B_2=\{\mathbf{v}_{21},\dots,\mathbf{v}_{2n}\}$
while $C_1=\{\mathbf{w}_{11},\dots,\mathbf{w}_{1m}\}$
and $C_2=\{\mathbf{w}_{21},\dots,\mathbf{w}_{2m}\}$
\item
Let $T:\mathbb{R}^n\to\mathbb{R}^n$ be a linear transformation and suppose that $V\le \mathbb{R}^n$ is invariant under $T$ (that is $T(V)\subseteq V$). Prove that there exists a basis $B$ such that
\[M(T,B,B)=\left[\begin{array}{rr}A&B\\0&C\end{array}\right]\]
where $A$ is a $\dim(V)\times\dim(V)$ matrix.
\item
Let $T:\mathbb{R}^n\to\mathbb{R}^m$ be a linear transformation of $\rank(T)=r$. Prove there exist bases $B,C$ of $\mathbb{R}^n,\mathbb{R}^m$ (resp) such that
\[M(T,B,C)=\left[\begin{array}{rr}I_R&0\\0&0\end{array}\right]\]
Do you see how this establishes the Rank Nullity Theorem?
%such that $B=PAP^{-1}$. If $A~B$, we say that $A$ is similar to $B$. This relation is a special case of Example 10 where $B=PAQ^{-1}$
\item
Prove that $\sim$ is an equivalence relation on $\mathcal{M}_n$

If we have some matrix $A$ then $A=I_nAI_n=I_nAI_n^{-1}$ and so we have reflexivity. Now if $B=PAP^{-1}$ then $P^{-1}BP=A$ and so if $A\sim B$ then $B\sim A$. Now if $C=QBQ^{-1}$ and $B=PAP^{-1}$ then $C=(QP)A(P^{-1}Q^{-1})$. Because $QP$ and $Q^{-1}P^{-1}$ are inverses, we know that $C\sim A$. Thus we have an equivalence relation.
\item
Prove that $A\sim A'$ if and only if $A$ and $A'$ represent the same linear transformation $T:\mathbb{R}^n\to\mathbb{R}^n$.  That is, $A\sim A'$ if and only if $A=M(T,B,B)$ and $A'=M(T,B',B')$ for some bases $B,B'$ of $\mathbb{R}^n$.

If $A=M(T,B,B)$ then $A=M(I_n,E_n,B)M(T,E_n,E_n)M(I_n,E_n,B)^{-1}$. And $A'=M(T,B',B')=M(I_n,E_n,B')M(T,E_n,E_n)M(I_n,E_n,B')^{-1}$. Thus $A\sim M(T,E_n,E_n)\sin A'$ and so by transitivity we have $A\sim A'$.

If $A\sim A'$ then
\item
Prove that if $A\sim B$, then $A^{-1}\sim B^{-1}$ and $A^T\sim B^T$.

If $A\sim B$ then $A=PBP^{-1}$ for some $P$. Now $PBP^{-1}PB^{-1}P^{-1}=I$ and so $A^{-1}=PB^{-1}P^{-1}$. Thus $A^{-1}\sim B^{-1}$

If $A\sim B$ then $A=PBP^{-1}$.  Taking the transpose of both sides we have $A^T=(PBP^{-1})^T=(P^{-1})^T(PB)^T=(P^{-1})^TB^TP^T$ and so $A^T\sim B^T$
\end{enumerate}
\end{document}
