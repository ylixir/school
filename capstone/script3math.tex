\documentclass[11pt,reqno]{amsart}
\usepackage{amsfonts,amssymb,amscd,amsmath,mathrsfs,amsthm}
\usepackage[pdftex]{graphicx}

\setlength{\oddsidemargin}{0.25in}  %please do not change
\setlength{\evensidemargin}{0.25in} %please do not change
\setlength{\marginparwidth}{0in} %please do not change
\setlength{\marginparsep}{0in} %please do not change
\setlength{\marginparpush}{0in} %please do not change
\setlength{\topmargin}{0in} %please do not change

\setlength{\footskip}{.3in} %please do not change
\setlength{\textheight}{8.75in} %please do not change
\setlength{\textwidth}{6in} %please do not change
\setlength{\parskip}{4pt} %please do not change

%
% The following lines set the environments for theorems, propositions, and definitions. 
%

\theoremstyle{plain}
\newtheorem{thm}{Theorem}[section]
\newtheorem{prop}{Proposition}

%  Alternatively, can use
%  \newtheorem{thm}{Theorem}[section]
%  to create theorems with numbering depending on the sections.
%
%  \newtheorem{prop}[thm]{Proposition} will generate propositions that use the same counter as theorems.

\theoremstyle{definition}
\newtheorem*{defi}{Definition}


\newcommand{\ab}[2]{\left(\frac{#1}{#2}\right)}

%  Mathematical sets; R, C, ...

\begin{document}

\section{Theorem environments}\label{intro}


\begin{abstract}
This is the abstract for the article.
\end{abstract}

This is an introduction to latex. A file with symbols and an introduction to latex is posted on blackboard. When creating your own latex file for your project, please use the commands before \textbackslash{}begin\{document\}. This will set the margins in such a way that the different files can easily be combined into one file afterwards.

The following are templates for theorems and propositions. Change the newtheorem arguments to create different looks and counters.

\begin{thm}\label{th1} This is a sample theorem.
\end{thm}

\begin{proof} This is an empty proof.
\end{proof}

\begin{defi} This is a definition.
\end{defi}

\begin{prop} This is an example of a proposition.
\end{prop}

\begin{thm}\label{th2} This is the second sample theorem.
\end{thm}

\section{References and counters}


We give a brief example how theorem counters are referred to. We defined theorems \ref{th1}, \ref{th2} in Section \ref{intro}. A counter that is referred to must have been defined with the \textbackslash{}label command at some point in the latex file.

\begin{thm} This is a theorem in the second section. The [section] optional parameter in the environment definition will change the counter so that it depends on the section.
\end{thm}


\end{document}
