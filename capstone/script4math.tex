\documentclass[11pt,reqno]{amsart}
\usepackage{amsfonts,amssymb,amscd,amsmath,mathrsfs,amsthm}
\usepackage[pdftex]{graphicx}

\setlength{\oddsidemargin}{0.25in}  %please do not change
\setlength{\evensidemargin}{0.25in} %please do not change
\setlength{\marginparwidth}{0in} %please do not change
\setlength{\marginparsep}{0in} %please do not change
\setlength{\marginparpush}{0in} %please do not change
\setlength{\topmargin}{0in} %please do not change

\setlength{\footskip}{.3in} %please do not change
\setlength{\textheight}{8.75in} %please do not change
\setlength{\textwidth}{6in} %please do not change
\setlength{\parskip}{4pt} %please do not change

%
% The following lines set the environments for theorems, propositions, and definitions. 
%

\theoremstyle{plain}
\newtheorem{thm}{Theorem}[section]
\newtheorem{prop}{Proposition}

%  Alternatively, can use
%  \newtheorem{thm}{Theorem}[section]
%  to create theorems with numbering depending on the sections.
%
%  \newtheorem{prop}[thm]{Proposition} will generate propositions that use the same counter as theorems.

\theoremstyle{definition}
\newtheorem*{defi}{Definition}


\newcommand{\ab}[2]{\left(\frac{#1}{#2}\right)}

%  Mathematical sets; R, C, ...

\begin{document}

\section{List environments}\label{intro}

Create the following list

\begin{enumerate}
\item The main list environments are enumerate and itemize.

\item Separate items are created with the command \textbackslash{}item.

\item Counters change automatically when generating sublists, or they can be changed manually.
\begin{enumerate}

\item First and
\item second item
\end{enumerate}
\item This is the fourth item

\end{enumerate}
Generate the following itemized list.

\begin{itemize}

\item Item 1
\item Item 2 contains another list:

\begin{itemize}
\item Item 3
\item Item 4 
\end{itemize}
\end{itemize}

Change counters: 

\renewcommand{\theenumi}{(\Roman{enumi})}  % \theenumi is the command to print the formatted string related to the counter
\renewcommand{\labelenumi}{\theenumi}            % \labelenumi formats the counter enumi. The syntax given here guarantees that references to
                                                                               % these counters are printed correctly as well.


\begin{enumerate}
\item The main list environments are enumerate and itemize.

\item Separate items are created with the command \textbackslash{}item.

\item Counters change automatically when generating sublists, or they can be changed manually.
\begin{enumerate}

\item First and
\item second item
\end{enumerate}
\item This is the fourth item

\end{enumerate}


\end{document}
