\documentclass[11pt]{amsart}
%  vim:spell:spelllang=en
%\documentclass[letterpaper]{article}

\usepackage{amsfonts,amssymb,amscd,amsmath,mathrsfs,amsthm,tikz}
\usepackage{forest}
\usepackage{cancel}
\usepackage{graphicx}
\usepackage{titlesec}
\usepackage{gnuplottex}
\usepackage[utf8]{inputenc}

% Uncomment the following line in order to include graphics

%\usepackage[pdftex]{graphicx}

%
%  The following are margin definitions used in the publication at the end of term.
%

\setlength{\oddsidemargin}{0.25in}  %please do not change
\setlength{\evensidemargin}{0.25in} %please do not change
\setlength{\marginparwidth}{0in} %please do not change
\setlength{\marginparsep}{0in} %please do not change
\setlength{\marginparpush}{0in} %please do not change
\setlength{\topmargin}{0in} %please do not change

\setlength{\footskip}{.3in} %please do not change
\setlength{\textheight}{8.75in} %please do not change
\setlength{\textwidth}{6in} %please do not change
\setlength{\parskip}{4pt} %please do not change

\theoremstyle{plain}
\newtheorem{thm}{Theorem}
\newtheorem{prop}{Proposition}
\newtheorem{lemma}{Lemma}
\newtheorem{cor}{Corollary}
\newtheorem{fact}{Fact}
\newtheorem{exa}{Example}

\theoremstyle{definition}
\newtheorem{defi}{Definition}
%\renewcommand\qedsymbol{$\blacksquare$}

\titleformat{\section}{\centering\bfseries}{\thesection.}{.5em}{}
\titleformat{\subsection}{\centering\bfseries}{\thesubsection.}{.5em}{}
\titleformat{\title}{\centering}{\thesection.}{.5em}{}

\begin{document}

%\centerline{\bf{\LARGE{Sample Capstone Paper}}}
%\medskip
%\centerline{\large{by Math Student}}
%\bigskip
\title{A Study of Numerical Semigroups, Markov Bases, and Gaussian Integers}
\author{Jon Allen}
\begin{abstract}
In this article will study Markov bases, numerical semigroups and Gaussian
integers. We will study the relationship between these objects, and study  the
usefulness of maps between these objects.
\end{abstract}
\maketitle
\section{Introduction}
This paper is broken into four main sections. They are: Numerical Semigroups,
Bijections, Markov Bases, Gaussian Integers.

Instead of simply heading straight for a definition of numerical semigroups,
we will instead construct them from scratch. It will be necessary to understand
the reasoning behind the construction of these groups if we wish to create an
analogous construction with Gaussian integers later. After we have constructed
this semigroup, we will then define it and show some examples. Finally we will
discuss Frobenius numbers, which are related to the fact that numerical
semigroups have a finite complement.

After we discuss numerical semigroups, then we will talk about the kinds of things
which are in bijection with them. These objects include lattice ideals, and Markov
bases. This leads us to ask what exactly Markov Ideals are.

The third section will define Markov bases and present a theorem that is central
to the usefulness of these objects. We our better understanding of Markov bases
we will revisit the map between these and numerical semigroups before we move
onto Gaussian integers.

The cocept of numerical semigroups leads us to ask what kind of other similar
constructions we can create. We examine what happens when we extend this object
from $\mathbb{N}$ to $\mathbb{N}\times \mathbb{N}$. Here we will encounter an
exciting and surpising fact.

%We will proceed by first going over some basic maps between elements of
%$\mathbb{N}^n$ and monomials. We will also touch on similar maps from
%$\mathbb{Z}^n$ and binomials.
%TM: What monomials? Binomials?

We will then touch on Markov bases and their relationships to the numerical
semigroups we will be looking at.

We will also look at what numeric semigroups are, along with the logic of their
construction. We will later use this information and logic to create a similar
construction with the Gaussian integers.

We will finally look the structure of the numerical semigroups and see a
surprising result in the Gaussian analogue of these.

\section{Numerical Semigroups}
%We will go into a little greater depth here than we would ordinarily need to, but we will use some of this logic to make some decisions later when we examine Gaussian integers.

We begin with a set of positive integers $A=\{n_1,\dots,n_k\}$. We can form an
additive semigroup $S\subset \mathbb{Z}$ with elements of the form
$a_1n_1+\cdots+{a_k}n_k\in S$ where $a_i\in \mathbb{N}$. It is sometimes useful
to represent $A$ as a vector $\mathbf{n}\in \mathbb{N}^k$ and $S$ as it's vector
space. We say that $S=\langle A\rangle=\{\mathbf{n}\cdot\mathbf{a}:
\forall\mathbf{a}\in \mathbb{N}^k\}$. Suppose that $n_i=0$ for some $i$.
Then $n_i$ does not contribute to $S$ and $\langle A\rangle \cong \langle
A\setminus \{n_i\}\rangle$ as a semigroup. Going forward we assume that
$n_i\ne 0$ for all $i$.

As before, we let $A=\{n_1,\dots,n_k\}$. We will say $c=\gcd(n_1,\dots,n_k)$ and
$A'=\{n_1/c,\dots,n_k/c\}$. Consider the following function.
\begin{align*}
  \varphi:&\langle A\rangle \to \langle A'\rangle\\
  &n_i\mapsto \frac{n_i}{c}
\end{align*}
Now because $\varphi$ maps the generators of $\langle A\rangle$ to the generators
of $\langle A'\rangle$ and is invertible, we know that $\langle A\rangle \cong
\langle A'\rangle $. Further, we know that $\gcd(A')=1$. Henceforth, we will
restrict ourselves to semigroups whose generators have a greatest common
denominator of one. This restriction has some interesting consequences.

\begin{lemma}
  Given $A=\{n_1,\dots,n_k:n\in \mathbb{N}\}$ with $\gcd(A)$, the set of linear
  combinations of $A$ is the set of all integers.
  \begin{proof}
    We say $E=\{a_1n_1+\dots+a_kn_k>0:a_i\in \mathbb{Z}\text{ and }n_i\in A\}$.
    Obviously $n_1\in E$. Because $E$ has at least one element and a lower
    bound, there must be a smallest element in $E$. We say the smallest element
    of $E$ is $s=a_1n_1+\dots+a_kn_k$. We choose some $n_i\in A$. We note that
    $s\le n_i$ and divide $n_i$ by $s$ to obtain $n_i=sq+r$ where
    $q,r\in \mathbb{N}$ and $0\le r<s$. This means that
    \begin{align*}
      r&=n_i-sq\\
      &=n_i-(a_1n_1+\dots+a_kn_k)q\\
      &=n_i-qa_1n_1-\dots-qa_kn_k\\
      &=-qa_1n_1-\dots-qa_in_i+n_i-\dots-a_kn_k\\
      &=-qa_1n_1-\dots-(qa_i-1)n_i-\dots-a_kn_k
    \end{align*}
    Thus $r\in E\cup \{0\}$. But $r<s$ and $s$ is the smallest element of $E$ so
    $r=0$. This means that $n_i=sq$ for all $n_i$, but the only number that
    divides all $n_i$ is $1$, and so $s=1$. And since every multiple of $s$ is
    a linear combination of the elements of $A$, we have our result.
  \end{proof}
\end{lemma}
Of course studying a more complicated version of $\mathbb{Z}$ isn't very
interesting. This is why we have been limiting the generators of our semigroups
to $\mathbb{N}$. There is another much less obvious property that we can obtian
from this result however. It will allow us to create an upper bound for the
complement in $\mathbb{N}$ of our semigroups. The immediate consequence of this
is that the complement in $\mathbb{N}$ of these semigroups is finite.
\begin{thm}
A semigroup generated by a set of positive integers with a
greatest common denominator of one has a finite complement in $\mathbb{N}$.
\begin{proof}
Let $A=\{n_1,\dots,n_k\}$ with $\gcd(A)=1$ and let $S=\langle A\rangle$. We
know from our previous lemma that we can find some $a_1,\dots,a_k\in \mathbb{Z}$
such that $1=a_1n_1+\dots+a_kn_k$. Let $\mathbf{a^+}=\{a_i:a_i>0\}$ and
$\mathbf{a^-}=\{a_j:a_j<0\}$. Then $1=\sum\limits_{a_i\in\mathbf{a^+}}{a_in_i}+
\sum\limits_{a_i\in\mathbf{a^-}}{a_jn_j}$ or $1-\sum\limits_{a_j\in\mathbf{a^-}}
{a_jn_j}=\sum\limits_{a_i\in\mathbf{a^+}}{a_in_i}$. Now because $a_j<0$ for all
$a_j\in \mathbf{a^-}<0$ and $a_j<0$ for all $a_j\in \mathbf{a^+}>0$ then $\sum\limits_{a_i\in\mathbf{a^+}}
{a_in_i}\in \langle A\rangle $ and $-\sum\limits_{a_i\in\mathbf{a^-}}{a_in_i}\in
\langle A\rangle $. So if we say $c=-\sum\limits_{a_i\in\mathbf{a^-}}{a_in_i}$
then we have found $c$ and $c+1$ which are both elements of $ \langle A\rangle$.

We claim that for any $n\ge (c-1)(c+1)$ then $n\in \langle A\rangle$. To verify
this claim, we show that $c|n$. Dividing $n$ by $c$ leads us to $n=cq+r$ where
$q,r\in \mathbb{N}$ and $0\le r<c$. We know that $cq+r\ge(c-1)(c+1)=(c-1)c+(c-1)$.
Further $r\ge (c-1)$. This means that $cq\ge (c-1)c$ or $q\ge c-1\ge r$. But
$n=cq+r=qc+r+rc-rc=(q-r)c+r(c+1)$. We know that $q\ge r\ge 0$ and so $q-r\ge 0$
and $r\ge 0$. We also know that $c$ and $c+1$ are both in $ \langle A\rangle$.
This means that $n\in \langle A\rangle$. Thus
$\mathbb{N}\ \langle A\rangle\subset \{n\in \mathbb{N}:n<(c+1)(c-1)\}$, which is
finite.

\end{proof}
\end{thm}
%TODO get citations straight
\begin{defi}\cite{rosales}
  A numerical semigroup (NSG) is a nonempty subset $S$ of $\mathbb{N}$ that is closed
under addition, contains the zero element, and whose complement in $\mathbb{N}$
is finite.
\end{defi}

\subsection{Examples}
%TODO numerical semigroups
\begin{exa}
  The semigroup generated by $\{1\}$ is $\{0,1,2,\dots \}=\mathbb{N}$. Obviously
  $\mathbb{N}\setminus \mathbb{N}=\emptyset$ which is finite, thus $\mathbb{N}$
  is a NSG.
\end{exa}
\begin{exa}
  The semigroup generated by $\{2\}$ is $\{0,2,4,\dots \}$. The complement of
  this set is all odd natural numbers. The complement of this semigroup in
  $\mathbb{N}$ is not finite, thus our semigroup is not a NSG.
\end{exa}
\begin{exa}
  The semigroup generated by $\{2,3\}$ is $A=\{0,2,3,4,\dots \}$. Obviously
  $\mathbb{N}\setminus A=\{1\}$ which is finite, thus $A$ is a NSG.
\end{exa}
\begin{exa}
  The semigroup generated by $\{6,10,15\}$ is $B=\{0, 6, 10, 12, 15, 16, 18,
  20, 21, 22,\\ 24, 25, 26, 27, 28, 30, \dots \}$. $\mathbb{N}\setminus
  B=\{1,2,3,4,7,8,9,11,13,
  14,17,19,23,29\}$ is finite, and $B$ is a NSG.
\end{exa}
\subsection{Frobenius Numbers}
The fact that NSGs have a finite complement in $\mathbb{N}$ has
a consequence. Namely that for any semigroup $S$ there exists some minimal
natural number $F(S)$ such that if $n>F(S)$ then $n\in S$. In words, this number
$F(S)$ is the largest natural number that is not in our semigroup. This is
called the Frobenius number, while the number $F(S)+1$ is referred to as
the conductor\cite{rosales}.

We take a numerical semigroup $S$ generated by the set $A=\{ n_1,\dots,n_k\}$.
As we saw in the proof for Theorem 1, if we have two consecutive numbers $c,c+1\in S$ then $c^2-1$ provides an upper bound for $F(S)$.
How much better can we do?
There are some formulas for edge cases, but there are no known formulas for every case.
Here we present an algorithm for finding the Frobenius number of any NSG.

The structure of an NSG has some trivial but very useful properties.
The first thing of note, is that if we find the conductor than  we have found the Frobenius number and vice versa.
If we choose some $s_1\in S$, then there exists at least one $s_2$ such that
$s_1<s_2\le s_2+\max A$.
In fact, for any $n_i\in A,s_1\in S$ we know that $s_1<s_1+n_i\le s_1+\max A$.
We use this fact to provide working interval for our algorithm of $[s_i,s_i+\max A]$.
This will be much more convenient than working in the entire solution space of $[0,c^2-1]$.

We have restricted ourself to a very small window in which to work. We should have a method to know whether we have found our number without counting all the way to $c^2-1$.
Ideally we can be sure of our result by only looking at our window which is at most $\max(A)$ elements.

Let us suppose we have found $\min A$ sequential elements in $S$ starting with $s_1\in S$.
We know that $s_1+i\in S$ for all $0\le i\le \min(A)-1$.
Choose any $s_2>s_1$ and $s_2=s_1+j$ for some $j\in \mathbb{N}$.
If we divide $j$ by $\min A$ then we obtain $j=q\cdot\min(A)+r$ where $0\le r<\min A$.
From the definition of a semigroup, we know that if $s_1\in S$ then $s_1+q\cdot\min(A)\in S$.
It also follows that $s_1+r\in S$.
Putting thse two facts together, we have $(s_1+r)+q\cdot\min(A)\in S$.
This means that $s_2\in S$.

We have found a termination criteria for our algorithm.
The first element of the first $\min A$ sequential elements of a NSG is the conductor of that NSG.  We begin with known a lower bound for our conductor.
Increasing this bound until we find a sequence of $\min A$ consecutive elements in $S$ will provide an upper bound and thus our conductor.

We know that every NSG contains zero.
Zero is also the smallest element of any NSG and so it makes a decent first candidate for our conductor.
We say $c_1=0$ and begin our algorithm by constructing a working set $F_1=\{c_1+n_1,\dots, c_1+n_k\}$.

We say $c_i=c_1$ and $F_i=F_1$ and we begin iterating our algorithm.
As we iterate we will seek to ensure that $|F_i|\le \max A$.

We check to see if the smallest $\min A$ elements of $F_i$ are sequential.
If they are, then we have met our termination criteria, and $F(S)=c_i-1$.
If we have not met the termination criteria, then we choose our next conductor.
We know that we have accounted for all elements in our semigroup at this point up to $c_i+\min A$.
It is then safe to discard any elements of our semigroup below $\min F_i$.
We choose our next lower bound for our conductor to be $c_{i+1}=\min F_i$ and define $F_i'=F_i\setminus\{\min F_i\}$.

We have assumed that $c_i+\min A$ elements are accounted for, and so we need to ensure this for the next iteration.
Thus we say $F_{i+1}=F_i'\cup \{\min F_i+n_j:\forall n_j\in A\}$. Now we are ready to iterate again, and so go to the termination criteria step for $F_{i+1}$

The maximum time this algorithm takes to find the Frobenius number is a linear
multiple of the number of elements in the semigroup and the size of the
Frobenius number. There may be faster ways of finding this number, but for our
purposes, it is simple to implement and lends itself to a great deal of optimization when run on binary computer system.\cite{frobmask}

The following is a pseudocode implementation of the algorithm. Given a numerical semigroup $S=\langle A\rangle$ where $A=\{n_1,\dots,n_k\}$, the algorithm takes $A$ as input and produces the conductor $c$ of the NSG as output.
\begin{align*}
  &\text{Input: } \langle A\rangle=\langle n_1,\dots,n_k\rangle\\
  &\text{Output: a conductor } c \text{ for } \langle A\rangle
\end{align*}
\section{Bijections}
The things we need to understand is a common map that the literature takes for
granted. There is a straightforward map between $\mathbb{N}^k$ and monomials
$k[x_1,\dots,x_k]$. We choose some $\mathbf{\alpha}\in \mathbb{N}^k$ such that
$\mathbf{\alpha}=(\alpha_1,\dots,\alpha_k)$. Now we define a map for
$\varphi:\mathbb{N}^n\to k[x_1,\dots,x_k]$ such that $\mathbf{\alpha}\mapsto
\mathbf{x}^\alpha$.

\begin{lemma}
Let $S$ be a NSG.  Then $\varphi(S)$ is the set of monic
monomials of a monomial ideal of $k[x]$, where $\varphi$ is as above.
\begin{proof}
  We take a numerical semigroup $S=\langle n_1,\dots,n_k\rangle$. Under
$\varphi$ this set maps to $\{x^{s}:s\in S\}$. Of course $s=a_1n_1+\cdots+a_kn_k$
where $a_i\in \mathbb{N}$. Of course multiplying monomials is simply a matter
of adding exponents, and so if $I=\langle x^{n_1},\dots,x^{n_k}\rangle$ then
$\varphi:S\to I$
\end{proof}
\end{lemma}

We can do a similar map with binomials.
In the binomial case we start with some $\mathbf{z}\in \mathbb{Z}^n$.
We define $\mathbf{z}=(z_1,\dots,z_n)$.
We also define $\mathbf{z}^+$ and $\mathbf{z}^-$ as follows:
\begin{align*}
  \mathbf{z}^+&=\mathbf{z}\vee\mathbf{0}&
  \mathbf{z}^-&=-(\mathbf{z}\wedge\mathbf{0})
\end{align*}
Now we can map an element of $\mathbb{Z}^n$ to the binomials over field $k$ if
we define $\varphi:\mathbb{Z}^n\to k[x_1\dots x_n]$ as $\mathbf{z}\mapsto
\mathbf{x}^{\mathbf{z}^+}-\mathbf{x}^{\mathbf{z}^-}$

\subsection{Markov Bases}
As we will see in Theorem 2 we can map a NSG to a Markov
basis\cite{bernd}.
Each element $a$ of a numerical semigroup $S=\langle n_1,\dots,n_k\rangle$ takes
the form $a=a_1n_1+\dots+a_kn_k$. Now we examine the vectors $(a_1,\dots,a_k)$.
Note that any given $a\in S$ may have more than one vector associated with it.
\begin{exa}
  $S=\langle 3,4,5\rangle$. Notice that $8=2\cdot 4=3+5$. Thus we have two
  vectors associated with $8\in S$.
\end{exa}
We are looking for elements of our NSG which have multiple
associated but disconnected vectors. The vectors ${\bf a}=(a_1,\dots,a_k)$
and ${\bf b}=(b_1,\dots,b_k)$ are connected if there exists some $i$
such that $a_i>0$ and $b_i>0$. Furthermore, if ${\bf a}$ is connected to
${\bf b}$ and ${\bf b}$ is connected to ${\bf c}$ then ${\bf a}$ is connected to
${\bf c}$. Once we have found two disconnected vectors associated with an
element of our NSG, then we subtract them to find an element
of our Markov basis. We continue until we have found the elements of our Markov
basis guaranteed by Theorem 2.

\begin{exa}
  We will find the Markov basis which corresponds to the NSG
$\langle 3,4,5\rangle$. First we generate a list of vectors which correspond
to the elements of our NSG.
\[\left[\begin{array}{r|rrr}
&3&4&5\\
\hline
3&1&0&0\\
4&0&1&0\\
5&0&0&1\\
6&2&0&0\\
7&1&1&0\\
8&1&0&1\\
8&0&2&0\\
9&3&0&0\\
9&0&1&1\\
10&2&1&0\\
10&0&0&2\\
\end{array}\right]
\]
Now we see that $8,9,10$ all have two associated but disconnected vectors. We
subtract these vectors to obtain
\[
  \left[\begin{array}{rrr}
  3&-1&-1\\
  -1&2&-1\\
  -2&-1&2\\
  \end{array}\right]
\]
Which is our Markov basis.
\end{exa}

It is convenient to write our Markov basis as a matrix, whose rows consist of
the elements of the Markov basis.
\subsection{Smith Normal Form}
We choose a Markov basis matrix $M$ and it's associated numerical semigroup $S$.
Every row $v$ of $M$ consists of two vectors $a,b$ where $v=a-b$. Let $S=
\langle n_1,\dots,n_k\rangle$ and define the vector $s=(n_1,\dots,n_k)$. Now
$s\cdot a=s\cdot b$. And so $s\cdot v=0$. This means that if we have $M$, we can
find our numerical semigroup $S$ by finding a nontrivial solution to $Ms=0$.

This suggests that we could use the Smith normal form of our Markov basis matrix
to find our NSG.
\begin{defi}
  If we are given some matrix $M$ whose entries are in a principal ideal domain,
  then we can find some matrices $U,V,B$ such that
\[UAV=B=
\left[\begin{array}{cccc}
  a_1&\cdots&0\\
  \vdots&\ddots&\vdots \\
  0&\cdots&a_r&
\end{array}\right]
\text{ with }a_i|a_{i+1}\]
We call $B$ the \emph{Smith normal form} of $A$.\cite{adkins}
\end{defi}

\begin{exa}
  Let us find the Smith normal form of
  $
  A=\left[\begin{array}{rrr}
  3&-1&-1\\
  -1&2&-1\\
  -2&-1&2\\
  \end{array}\right]
  $. We start with $U'AV'=I_AAI_A$. We then use the standard row and column
  operations on $A$ while doing reflecting the row operations on $U'$ and the
  column operations on $V'$ to find our Smith normal form, along with the
  matrices which give us this form. Note that we are dealing with integers here,
  fractions are not allowed
  \begin{align*}
  \left[\begin{array}{rrr}
  1&0&0\\
  0&1&0\\
  0&0&1\\
  \end{array}\right]
  &
  \left[\begin{array}{rrr}
  3&-1&-1\\
  -1&2&-1\\
  -2&-1&2\\
  \end{array}\right]
  \left[\begin{array}{rrr}
  1&0&0\\
  0&1&0\\
  0&0&1\\
  \end{array}\right]\\
  &\Downarrow\\
  \left[\begin{array}{rrr}
  1&0&0\\
  0&1&0\\
  1&1&1\\
  \end{array}\right]
  &
  \left[\begin{array}{rrr}
  3&-1&-1\\
  -1&2&-1\\
  0&0&0\\
  \end{array}\right]
  \left[\begin{array}{rrr}
  1&0&0\\
  0&1&0\\
  0&0&1\\
  \end{array}\right]\\
  &\Downarrow\\
  \left[\begin{array}{rrr}
  1&2&0\\
  0&1&0\\
  1&1&1\\
  \end{array}\right]
  &
  \left[\begin{array}{rrr}
  1&3&-3\\
  -1&2&-1\\
  0&0&0\\
  \end{array}\right]
  \left[\begin{array}{rrr}
  1&0&0\\
  0&1&0\\
  0&0&1\\
  \end{array}\right]\\
  &\Downarrow\\
  \left[\begin{array}{rrr}
  1&2&0\\
  1&3&0\\
  1&1&1\\
  \end{array}\right]
  &
  \left[\begin{array}{rrr}
  1&3&-3\\
  0&5&-4\\
  0&0&0\\
  \end{array}\right]
  \left[\begin{array}{rrr}
  1&0&0\\
  0&1&0\\
  0&0&1\\
  \end{array}\right]\\
  &\Downarrow\\
  \left[\begin{array}{rrr}
  1&2&0\\
  1&3&0\\
  1&1&1\\
  \end{array}\right]
  &
  \left[\begin{array}{rrr}
  1&0&-3\\
  0&1&-4\\
  0&0&0\\
  \end{array}\right]
  \left[\begin{array}{rrr}
  1&0&0\\
  0&1&0\\
  0&1&1\\
  \end{array}\right]\\
  &\Downarrow\\
  \left[\begin{array}{rrr}
  1&2&0\\
  1&3&0\\
  1&1&1\\
  \end{array}\right]
  &
  \left[\begin{array}{rrr}
  1&0&0\\
  0&1&0\\
  0&0&0\\
  \end{array}\right]
  \left[\begin{array}{rrr}
  1&0&3\\
  0&1&4\\
  0&1&5\\
  \end{array}\right]\\
\end{align*}
And so
$\left[\begin{array}{rrr} 1&2&0\\ 1&3&0\\ 1&1&1\\ \end{array}\right]
\left[\begin{array}{rrr} 3&-1&-1\\ -1&2&-1\\ -2&-1&2\\ \end{array}\right]
\left[\begin{array}{rrr} 1&0&3\\ 0&1&4\\ 0&1&5\\ \end{array}\right]
=\left[\begin{array}{rrr} 1&0&0\\0&1&0\\0&0&0 \end{array}\right]
$

\end{exa}
We can use the software package Xcas\cite{xcas} to easily compute these matrices.
\subsection{Numerical Semigroups in Lattice Theory}
\begin{defi}
\cite{stanley}
A \emph{lattice} is a partially ordered set in which every two elements have a
unique least upper bound and a unique greatest lower bound.
\end{defi}
\begin{exa}
$\mathbb{N}^2$ is a lattice with supremum and infinum for any two elements which
belong to it. Notice that $(1,2)$ and $(2,1)$ have a lower bound of $(1,1)$ and
an upper bound of $(2,2)$.
\end{exa}
\begin{exa}
We can form a lattice if we order $\mathbb{N}$ by division. The least common
multiple forms a least upper bound and an greatest lower bound is formed by the
greatest common denominator.
\end{exa}
The fundamental theorem of Markov bases (which we will discuss in the next section)
claims that there is a bijection between a lattice ideal and a Markov basis. As
we have already seen, Markov bases are in bijection with NSGs.

Thus we see that every NSG is in bijection with a lattice ideal.

\begin{exa}
  We begin with a lattice ideal
  \[
  I_\Lambda=
  \begin{cases}
    x^3-yz\\
    y^2-xz\\
    z^2-xy
  \end{cases}
  \]
  Now this ideal is a set of binomials which we can map in the usual way to a set
  of vectors.
  \[
  \begin{cases}
    x^3-yz\\
    y^2-xz\\
    z^2-x^2y
  \end{cases}
  \Rightarrow
  \left[\begin{array}{rrr}
    3&-1&-1\\
    -1&2&-1\\
    -1&-1&2
  \end{array}\right]
  \]
  The key here, is that we are guaranteed by Theorem 2, that these vectors are
  actually a Markov basis. And as we saw in the previous two sections, this basis
  corresponds to the numerical semigroup $\langle 3,4,5\rangle$. And so we see
  \[
  \begin{cases}
    x^3-yz\\
    y^2-xz\\
    z^2-xy
  \end{cases}
  \mapsto\langle 3,4,5\rangle
  \]
  The reverse map is similarly straightforward.
\end{exa}

\section{Markov Bases}
We have been speaking of Markov bases, but we have not explored what they are or
where they come from. In this section we will explore their history and their
relevance.

Markov bases play a central role in the recent field of algebraic statistics.
One of the earliest papers on this field\cite{bernd} introduced the idea of a
Markov basis for log linear statistical models\cite{markstats} and related them
to commutative algebra. This work has been applied in many fields and has been
particularly active in computational biology. However, we will be glossing over
the statistical role of these bases and will instead focus on their algebraic
properties. First we need to get a few definitions and some notation out of the
way.

We can represent a numerical semigroup $S=\langle n_1,\dots,n_k \rangle$ as a
matrix $A=\left[\begin{array}{rrr}n_1&\cdots&n_k\end{array}\right]$. Then for
every element $s\in S$ we can say $s=Au$ for some $u\in \mathbb{N}^k$.

\begin{defi}
  \cite{bernd}
  The set of tables
  \[\mathcal{F}(u)=\left\{v\in \mathbb{N}^k:Av=Au\right\}\]
  is called the \emph{fiber} of a contingency table $u\in \mathcal{T}(n)$ with
  respect to the model $\mathcal{M}_A$
\end{defi}

Contingency tables and matrix models are specific to statistics. The thing we
should take from this definition, is that a fiber $\mathcal{F}(u)$ of an element
$Au$ of our semigroup $A$ is the set of vectors $\left\{v\in \mathbb{N}^k:Av=Au
\right\}$
\begin{exa}
  For the numerical semigroup $A=\left[\begin{array}{rrr}3&4&5\end{array}\right]$
  we will find the fiber corresponding to the element $Au=8$.

  We need to find all solutions to the equation $\left[\begin{array}{rrr}3&4&5
\end{array}\right]\left[\begin{array}{rrr}x,y,z \end{array}\right]=8$ or
$3x+4y+5z=8$ where $x,y,z\in \mathbb{N}$. We find that $3+5$ and $4\cdot 2$ are
the only two possible solutions, and so for $Au=8$ we see that $\mathcal{F}(u)=
\{(1,0,1),(0,2,0)\}$.
\end{exa}

Before we give the definiton of a Markov basis, we note that the literature often refers to the elements of a Markov basis as \emph{moves}\cite[p.16]{bernd}
\begin{defi}
\cite{bernd}
Let $\mathcal{M}_A$ be the log-linear model associated with a matrix $A$ whose integer kernel we denote by $\ker_\mathbb{Z}(A)$.
A finite subset $\mathcal{B}\subset\ker_{\mathbb{Z}}(A)$ is a \emph{Markov basis} for $\mathcal{M}_A$ if for all $u\in \mathcal{T}(n)$ and all pairs $v,v'\in \mathcal{F}(u)$ there exists a sequence $u_1,\dots,u_L\in  \mathcal{B}$ such that
\[v'=v+\sum\limits_{k=1}^L{u_k}\text{ and }v+\sum\limits_{k=1}^l{u_k}\ge 0\text{ for all }l=1,\dots,L.\]
\end{defi}

Now that we have some more insight as to what a Markov basis is, the process in
example 8 should make more sense.
\begin{exa}
  We will find the Markov basis which corresponds to the numerical semigroup
$A=\langle 3,4,5\rangle$. First we generate the fibers which correspond to the
elements of our NSG.
\[\left[\begin{array}{r|rrr}
&3&4&5\\
\hline
3&1&0&0\\
4&0&1&0\\
5&0&0&1\\
6&2&0&0\\
7&1&1&0\\
8&1&0&1\\
8&0&2&0\\
9&3&0&0\\
9&0&1&1\\
10&2&1&0\\
10&0&0&2\\
\end{array}\right]
\]
We are particularly interested in fibers with more than one element. These are
the fibers associated with the elements $8,9$ and $10$. From the definition of
the Markov basis, we know that the product of $A$ and an element of the Markov
basis is $0$. Furthermore, we know that we can add a sequence of elements of the
Markov basis to any of the elements of a fiber to obtain any other element of
that fiber. Taking the difference of two elements from the same fiber will
meet both of these criteria.

Thus if $Au=8$ then $\mathcal{F}(u)=\{(1,0,1),(0,2,0)\}$ and so
$(-1,2,-1)\in\mathcal{B}$.
If $Av=9$ then $\mathcal{F}(v)=\{(3,0,0),(0,1,1)\}$ and so
$(3,-1,-1)\in\mathcal{B}$.
And if $Aw=10$ then $\mathcal{F}(w)=\{(2,1,0),(0,0,2)\}$ and so
$(-2,-1,2)\in\mathcal{B}$.
And so we have built our Markov basis.
\[
  \mathcal{B}=\left[\begin{array}{rrr}
  3&-1&-1\\
  -1&2&-1\\
  -2&-1&2\\
  \end{array}\right]
\]
\end{exa}


We also have the fundamental theorem of Markov bases which provides a direct relation
to a lattice ideal.
\begin{thm}
\cite[p.~54]{aoki}
A finite set of moves $\mathcal{B}$ is a Markov basis for $A$ if and only if the set of binomials $\{p^{\mathbf{z}^+}-p^{\mathbf{z}^-}|\mathbf{z}\in \mathcal{B}\}$ generates the toric ideal $I_A$.
\end{thm}

\section{Gaussian Integers}
Notice that the real and complex parts of Gaussian integers do not interact under addition. Now let us take the linear combination of some finite set of Gaussian integers.

If we do not restrict ourselves to positive coefficients, then we wind up with semigroups that span the entire number line along multiples of a greatest common denominator. This is as uninteresting now as it was with NSGs, and so we will only look at positive values from here.

Now let us take some ``Gaussian semigroup'' $\mathbf{z}=\langle x_1+y_1i,\dots,
x_n+{y_n}i\rangle$ where $x_i,y_i\ge 0$. Notice that this semigroup is actually
just the direct sum of two NSGs. Say $\mathbf{z}=\mathbf{x}\oplus\mathbf{y}=\langle x_1,\dots,x_n\rangle\oplus\langle y_1,\dots,y_n\rangle$.

Now as you may have guessed from our choice of notation, we are going to think of this direct sum as a Cartesian product. Now we know that the semigroup $\mathbf{x}$ has some Frobenius number after which every number is in the semigroup.

\begin{gnuplot}
set multiplot
set xrange [0:10]
set yrange [0:10]
set trange [0:10]
set parametric
plot 5,t; plot 6,t; plot 7,t; plot 8,t; plot 9,t; plot 10,t
\end{gnuplot}

And if we add in $\mathbf{y}$ then we have

\begin{gnuplot}
set multiplot
set xrange [0:10]
set yrange [0:10]
set trange [0:10]
set parametric
plot t,5; plot t,6; plot t,7; plot t,8; plot t,9; plot t,10
plot 5,t; plot 6,t; plot 7,t; plot 8,t; plot 9,t; plot 10,t
\end{gnuplot}

Now our intuition and our inspection of the graph leads us to believe that we can
easily come up with an analog of a Frobenius number in $\mathbb{N}$ to a Gaussian
Frobenius number in $\mathbb{N}[i]$. We say that the Frobenius number for $\bf{x}$
is $F(\bf{x})$ and the Frobenius number for $\bf{y}$ is $F(\bf{x})$. Now let us
choose some $F(\bf{x}<x\in \bf{x}$. There are only a finite number of combinations
that equal $x$. That means that for any $x$ there are only a finitely many number
of $y\in \bf{y}$ such that $x+yi\in \bf{z}$. But this means that for any point in
our Gaussian semigroup, we can find an infinite number of Gaussian integers greater
than that point which are not in our semigroup. The surprise here is that not only
is there no analog to a Frobenius number but element is this semigroup remain sparse
over the entire $\mathbb{N}^2$ lattice.
\bibliographystyle{amsplain}
\bibliography{bib}


\end{document}
