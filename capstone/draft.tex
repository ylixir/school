\documentclass[11pt]{amsart}
%  vim:spell:spelllang=en
%\documentclass[letterpaper]{article}

\usepackage{amsfonts,amssymb,amscd,amsmath,mathrsfs,amsthm,tikz}
\usepackage{forest}
\usepackage{cancel}
\usepackage{graphicx}
\usepackage{titlesec}
%\usepackage{gnuplottex}
\usepackage[utf8]{inputenc}
\usepackage{multirow}
\usepackage{luacode}
\usepackage{url}
\usetikzlibrary{arrows.meta}
\def\UrlBreaks{\do\/}
\allowdisplaybreaks
% Uncomment the following line in order to include graphics

%\usepackage[pdftex]{graphicx}

%
%  The following are margin definitions used in the publication at the end of term.
%

\setlength{\oddsidemargin}{0.25in}  %please do not change
\setlength{\evensidemargin}{0.25in} %please do not change
\setlength{\marginparwidth}{0in} %please do not change
\setlength{\marginparsep}{0in} %please do not change
\setlength{\marginparpush}{0in} %please do not change
\setlength{\topmargin}{0in} %please do not change

\setlength{\footskip}{.3in} %please do not change
\setlength{\textheight}{8.75in} %please do not change
\setlength{\textwidth}{6in} %please do not change
\setlength{\parskip}{4pt} %please do not change

\theoremstyle{plain}
\newtheorem{thm}{Theorem}
\newtheorem{prop}{Proposition}
\newtheorem{lemma}{Lemma}
\newtheorem{cor}{Corollary}
\newtheorem{fact}{Fact}
\newtheorem{exa}{Example}

\theoremstyle{definition}
\newtheorem{defi}{Definition}
%\renewcommand\qedsymbol{$\blacksquare$}

\titleformat{\section}{\centering\bfseries}{\thesection.}{.5em}{}
\titleformat{\subsection}{\centering\bfseries}{\thesubsection.}{.5em}{}
\titleformat{\title}{\centering}{\thesection.}{.5em}{}

\begin{document}

%\centerline{\bf{\LARGE{Sample Capstone Paper}}}
%\medskip
%\centerline{\large{by Math Student}}
%\bigskip
\title{A Study of Numerical Semigroups, Markov Bases, and Gaussian Integers}
\author{Jon Allen}
\begin{abstract}
In this article will study Markov bases, numerical semigroups and Gaussian
integers. We will study the relationship between these objects, and study  the
usefulness of maps between these objects.
\end{abstract}
\maketitle
\section{Introduction}
This paper is broken into four main sections. They are: Numerical Semigroups,
Bijections, Markov Bases, Gaussian Integers.

Instead of simply heading straight for a definition of numerical semigroups,
we will instead construct them from scratch. It will be necessary to understand
the reasoning behind the construction of these groups if we wish to create an
analogous construction with Gaussian integers later. After we have constructed
this semigroup, we will then define it and show some examples. Finally we will
discuss Frobenius numbers, which are related to the fact that numerical
semigroups have a finite complement.

After we discuss numerical semigroups, then we will talk about the kinds of things
which are in bijection with them. These objects include lattice ideals, and Markov
bases. This leads us to ask what exactly Markov Ideals are.

The third section will define Markov bases and present a theorem that is central
to the usefulness of these objects. We our better understanding of Markov bases
we will revisit the map between these and numerical semigroups before we move
onto Gaussian integers.

The cocept of numerical semigroups leads us to ask what kind of other similar
constructions we can create. We examine what happens when we extend this object
from $\mathbb{N}$ to $\mathbb{N}\times \mathbb{N}$. Here we will encounter an
exciting and surpising fact.

%We will proceed by first going over some basic maps between elements of
%$\mathbb{N}^n$ and monomials. We will also touch on similar maps from
%$\mathbb{Z}^n$ and binomials.
%TM: What monomials? Binomials?

We will then touch on Markov bases and their relationships to the numerical
semigroups we will be looking at.

We will also look at what numeric semigroups are, along with the logic of their
construction. We will later use this information and logic to create a similar
construction with the Gaussian integers.

We will finally look the structure of the numerical semigroups and see a
surprising result in the Gaussian analogue of these.

\section{Numerical Semigroups}
%We will go into a little greater depth here than we would ordinarily need to, but we will use some of this logic to make some decisions later when we examine Gaussian integers.

We begin with a set $A=\{n_1,\dots,n_k:n_i\in \mathbb{N}\}$. We can form an
additive semigroup $S\subseteq \mathbb{N}$ with elements of the form
$a_1n_1+\cdots+{a_k}n_k\in S$ where $a_i\in \mathbb{N}$. It is sometimes useful
to represent $A$ as a vector $\mathbf{n}\in \mathbb{N}^k$ and $S$ as a span over vectors in $\mathbb{N}^k$. We say that $S=\langle A\rangle=\{\mathbf{n}\cdot\mathbf{a}:
\forall\mathbf{a}\in \mathbb{N}^k\}$. Suppose that $n_i=0$ for some $n_i\in A$.
Then $n_i$ does not contribute to $S$ and $\langle A\rangle \cong \langle
A\setminus \{n_i\}\rangle$ as a semigroup. Going forward we assume that
$n_i\ne 0$ for all $i$.

As before, we let $A=\{n_1,\dots,n_k\}$. We will say $c=\gcd(n_1,\dots,n_k)$ and
$A'=\{\frac{n_1}{c},\dots,\frac{n_k}{c}\}$. Consider the following function:
\begin{align*}
  \varphi:&\langle A\rangle \to \langle A'\rangle\\
  &n_i\mapsto \frac{n_i}{c}
\end{align*}
Now because $\varphi$ maps the generators of $\langle A\rangle$ to the generators
of $\langle A'\rangle$ and is invertible, we know that $\langle A\rangle \cong
\langle A'\rangle $. Further, we know that $\gcd(A')=1$.
\begin{exa}
  The semigroup $\langle2\rangle$ is $\{0,2,4,\dots \}$. This semigroup is isomorphic to $\langle 2/2\rangle=\langle1\rangle=\{0,1,2,\dots\}=\mathbb{N}$
\end{exa}

Henceforth, we will
restrict ourselves to semigroups whose generators have a greatest common
denominator of one. This restriction has some interesting consequences.

\begin{lemma}
  Given $A=\{n_1,\dots,n_k\}$ with $\gcd(A)=1$, the set of linear
  combinations of $A$ is the set of all integers.
  \begin{proof}
    We say $E=\{a_1n_1+\dots+a_kn_k>0:a_i\in \mathbb{Z}\text{ and }n_i\in A\}$.
    Obviously $n_1\in E$. Because $E$ has at least one element and a lower
    bound, there must be a smallest element in $E$. We say the smallest element
    of $E$ is $s=a_1n_1+\dots+a_kn_k$. We choose some $n_i\in A$. We note that
    $s\le n_i$ and divide $n_i$ by $s$ to obtain $n_i=sq+r$ where
    $q,r\in \mathbb{N}$ and $0\le r<s$. This means that
    \begin{align*}
      r&=n_i-sq\\
      &=n_i-(a_1n_1+\dots+a_kn_k)q\\
      &=n_i-qa_1n_1-\dots-qa_kn_k\\
      &=-qa_1n_1-\dots-qa_in_i+n_i-\dots-a_kn_k\\
      &=-qa_1n_1-\dots-(qa_i-1)n_i-\dots-a_kn_k
    \end{align*}
    Thus $r\in E\cup \{0\}$. But $r<s$ and $s$ is the smallest element of $E$ so
    $r=0$. This means that $n_i=sq$ for all $n_i$, but the only number that
    divides all $n_i$ is $1$, and so $s=1$. And since every multiple of $s$ is
    a linear combination of the elements of $A$, we have our result.
  \end{proof}
\end{lemma}
Of course studying a more complicated version of $\mathbb{Z}$ is not very
interesting. This is why we have been limiting the generators of our semigroups
to $\mathbb{N}$. There is another much less obvious property that we can obtian
from this result however. It will allow us to create an upper bound for the
complement in $\mathbb{N}$ of our semigroups. The immediate consequence of this
is that the complement in $\mathbb{N}$ of these semigroups is finite.
\begin{thm}
A semigroup $S=\langle A\rangle$ where $A=\{n_1,\dots,n_k\}$ with $\gcd(A)=1$ has a finite complement in $\mathbb{N}$.
\begin{proof}
Let $S=\langle A\rangle$.
We know from our previous lemma that we can find some $a_1,\dots,a_k\in \mathbb{Z}$ such that $1=a_1n_1+\dots+a_kn_k$.
Let $\mathbf{a^+}=\{a_i:a_i>0\}$ and $\mathbf{a^-}=\{a_j:a_j<0\}$.
Then $1=\sum\limits_{a_i\in\mathbf{a^+}}{a_in_i}+\sum\limits_{a_i\in\mathbf{a^-}}{a_jn_j}$ or $1-\sum\limits_{a_j\in\mathbf{a^-}}
{a_jn_j}=\sum\limits_{a_i\in\mathbf{a^+}}{a_in_i}$. Now because $a_j<0$ for all
$a_j\in \mathbf{a^-}<0$ and $a_j<0$ for all $a_j\in \mathbf{a^+}>0$ then $\sum\limits_{a_i\in\mathbf{a^+}}
{a_in_i}\in \langle A\rangle $ and $-\sum\limits_{a_i\in\mathbf{a^-}}{a_in_i}\in
\langle A\rangle $. So if we say $c=-\sum\limits_{a_i\in\mathbf{a^-}}{a_in_i}$
then we have found $c$ and $c+1$ which are both elements of $ \langle A\rangle$.

We claim that for any $n\ge (c-1)(c+1)$ then $n\in \langle A\rangle$. To verify
this claim, we show that $c|n$. Dividing $n$ by $c$ leads us to $n=cq+r$ where
$q,r\in \mathbb{N}$ and $0\le r<c$. We know that $cq+r\ge(c-1)(c+1)=(c-1)c+(c-1)$.
Further $r\ge (c-1)$. This means that $cq\ge (c-1)c$ or $q\ge c-1\ge r$. But
$n=cq+r=qc+r+rc-rc=(q-r)c+r(c+1)$. We know that $q\ge r\ge 0$ and so $q-r\ge 0$
and $r\ge 0$. We also know that $c$ and $c+1$ are both in $ \langle A\rangle$.
This means that $n\in \langle A\rangle$. Thus
$\mathbb{N}\ \langle A\rangle\subset \{n\in \mathbb{N}:n<(c+1)(c-1)\}$, which is
finite.

\end{proof}
\end{thm}
\begin{exa}
  The semigroup generated by $\{2,3\}$ is $A=\{0,2,3,4,\dots \}$. Obviously
  $\mathbb{N}\setminus A=\{1\}$ which is finite.
\end{exa}
Now that we have constructed this object ans explored the logic behind it, we give it a name and a formal definition.
%TODO get citations straight
\begin{defi}\cite{rosales}
  A numerical semigroup (NSG) is a nonempty subset $S$ of $\mathbb{N}$ that is closed
under addition, contains the zero element, and whose complement in $\mathbb{N}$
is finite.
\end{defi}
One thing to note is that $\mathbb{N}$ itself is a numerical semigroup.
\begin{exa}
  The semigroup generated by $\{1\}$ is $\{0,1,2,\dots \}=\mathbb{N}$. Obviously
  $\mathbb{N}\setminus \mathbb{N}=\emptyset$ which is finite, thus $\mathbb{N}$
  is a NSG.
\end{exa}
This idea leads to the following observation: while the elements of the generating set for a NSG are required to have a greatest common denominator of 1, any subset of the generating set need not meet that restriction. This means that a subsemigroup of an NSG may not actually be an NSG.
\begin{exa}
  The semigroup generated by $\{6,10,15\}$ is $$B=\{0, 6, 10, 12, 15, 16, 18,
  20, 21, 22,\\ 24, 25, 26, 27, 28, 30, \dots \}$$. We see that $$\mathbb{N}\setminus
  B=\{1,2,3,4,7,8,9,11,13,
  14,17,19,23,29\}$$ is finite, and $B$ is a NSG.

  The subsemigroup of $<6,10,15>$ generated by $\{6,10\}$ is $\{0,6,10,12,16,18,20,22,\dots\}$. This semigroup, has an infinite complement in $\mathbb{N}$ and is therefore not a NSG, even though it is a subset of a NSG.
\end{exa}
\subsection{Frobenius Numbers}
The fact that NSGs have a finite complement in $\mathbb{N}$ has
a consequence. Namely that for any semigroup $S$ there exists some minimal
natural number $F(S)$ such that if $n>F(S)$ then $n\in S$. In words, this number
$F(S)$ is the largest natural number that is not in our semigroup. This is
called the Frobenius number, while the number $F(S)+1$ is referred to as
the conductor\cite{rosales}.

We take a numerical semigroup $S$ generated by the set $A=\{ n_1,\dots,n_k\}$. We let $m=\min(A)$ and $M=\max(A)$.
As we saw in the proof for Theorem 1, if we have two consecutive numbers $c,c+1\in S$ then $c^2-1$ provides an upper bound for $F(S)$.
How much better can we do?
There are formulas for some specific cases, but there are no known formulas for every case.
Here we present an algorithm for finding the Frobenius number of any NSG.

The structure of an NSG has some trivial but very useful properties.
The first thing of note, is that if we find the conductor than  we have found the Frobenius number and vice versa.
If we choose some $s_1\in S$, then there exists at least one $s_2$ such that
$s_1<s_2\le s_2+M$.
In fact, for any $n_i\in A,s_1\in S$ we know that $s_1<s_1+n_i\le s_1+M$.
We use this fact to provide a working interval for our algorithm of $[s_i,s_i+M]$.

Let us suppose we have found $m$ sequential elements in $S$ starting with $s_1\in S$.
We know that $s_1+i\in S$ for all $0\le i\le m-1$.
Choose any $s_2>s_1$. Then $s_2=s_1+j$ for some $j\in \mathbb{N}$.
If we divide $j$ by $\min A$ then we obtain $j=q\cdot\min(A)+r$ where $0\le r<\min A$.
From the definition of a semigroup, we know that if $s_1\in S$ then $s_1+q\cdot\min(A)\in S$.
It also follows that $s_1+r\in S$.
Putting thse two facts together, we have $(s_1+r)+q\cdot\min(A)\in S$.
This means that $s_2\in S$.

Thus the smallest number in $A$ which begins a consecutive sequence of $m$ elements in $A$ is the conductor.
Our algorithm begins with a known lower bound for our conductor.
We increase this bound until we find a sequence of $\min A$ consecutive elements in $S$ will. This reveals our conductor.

We know that every NSG contains zero.
Zero is also the smallest element of any NSG and so it is our first candidate for the conductor.
We say $c_1=0$ and begin our algorithm by constructing a working set $F_1=\{c_1+n_1,\dots, c_1+n_k\}$.

We say $c_i=c_1$ and $F_i=F_1$ and we begin iterating our algorithm.
As we iterate we will seek to ensure that $|F_i|\le \max A$.

We check to see if the smallest $\min A$ elements of $F_i$ are sequential.
If they are, then we have met our termination criteria, and $F(S)=c_i-1$.
If we have not met the termination criteria, then we choose our next conductor candidate.
We know that we have accounted for all elements in our semigroup at this point up to $c_i+\min A$.
It is then safe to discard any elements of our semigroup below $\min F_i$.
We choose our next lower bound for our conductor to be $c_{i+1}=\min F_i$ and define $F_i'=F_i\setminus\{\min F_i\}$.

We have assumed that $c_i+\min A$ elements are accounted for, and so we need to ensure this for the next iteration.
Thus we say $F_{i+1}=F_i'\cup \{\min F_i+n_j:\forall n_j\in A\}$. Now we are ready to iterate again, and so go to the termination criteria step for $F_{i+1}$

The maximum time this algorithm takes to find the Frobenius number is a linear
multiple of the number of elements in the semigroup and the size of the
Frobenius number. There may be faster ways of finding this number, but for our
purposes, it is simple to implement and lends itself to a great deal of optimization when run on binary computer system.\cite{frobmask}

The following is a pseudocode implementation of the algorithm. Given a numerical semigroup $S=\langle A\rangle$ where $A=\{n_1,\dots,n_k\}$, the algorithm takes $A$ as input and produces the conductor $c$ of the NSG as output.

\texttt{
\noindent
\\INPUT: $A=\{n_1,\dots,n_k\}$ where $\gcd(n_1,\dots,n_k)=1$\\
\hspace*{\fill}\\
\hspace*{1em}$c:=0$ \\
\hspace*{1em}$F:=A$ \\
\hspace*{1em}WHILE $[c+1,c+\min(A)-1]\not\subset F$ \\
\hspace*{2em}$c:=\min(F)$\\
\hspace*{2em}$F:=F\setminus\{\min(F)\}$\\
\hspace*{2em}$F:=F\cup\{c+n_i:\forall n_i\in A\}$\\
\hspace*{1em}ENDWHILE\\
\hspace*{\fill}\\
OUTPUT: The conductor of $\langle A\rangle$ is $c$
}

\begin{exa}
We let $\langle A\rangle=\langle 5,7,9\rangle$. Then $c_1=0$ and $F_1=\{5,7,9\}$ while $\min(A)=5$. We iterate our algorithm in the table below:

\begin{center}
\begin{tabular}{l|l|l}
  Conductor&Working NSG Window&Termination\\
  \hline
  $c_1=0$&$F_1=\{5,7,9\}$&$\{1,2,3,4\}\not\subset F_1$\\
  $c_2=5$&$F_2=\{7,9,10,12,14\}$&$\{6,7,8,9\}\not\subset F_2$\\
  $c_3=7$&$F_3=\{9,10,12,14,15\}$&$\{8,9,10,11\}\not\subset F_3$\\
  $c_4=9$&$F_4=\{10,12,14,15,16,18\}$&$\{10,11,12,13\}\not\subset F_4$\\
  $c_5=10$&$F_5=\{12,14,15,16,17,18,19\}$&$\{11,12,13,14\}\not\subset F_5$\\
  $c_6=12$&$F_6=\{14,15,16,17,18,19,21\}$&$\{12,13,14,15\}\not\subset F_6$\\
  $c_7=14$&$F_7=\{15,16,17,18,19,21,23\}$&$\{15,16,17,18\}\subset F_7$\\
\end{tabular}
\end{center}
And so we see that $13$ is the Frobenius number for $\langle 5,7,9\rangle$
\end{exa}
\section{Markov Bases}
The semigroups that we have been examining are relatively simple objects. In the upcoming section on bijections we will see how we can relate them to less straightforward algebraic and combinatorial objects. However in order to have that discussion we will need to understand a tool called the Markov basis.

Markov bases play a central role in the recent field of algebraic statistics.
A seminal paper\cite{markstats} on this field introduced the idea of a
Markov basis for log linear statistical models and related them
to commutative algebra. This work has been applied in many fields and has been
particularly active in computational biology\cite{bernd}. However, we will be glossing over
the statistical role of these bases and will instead focus on their algebraic
properties. First we need to get a few definitions and some notation out of the
way.

Coming from statistics, Markov bases were developed to be used with tables.
Tables are naturally expressed as matrices, and the Markov basis
revolves around this. We noted earlier that a NSG can be represented as the
span of vectors in $\mathbb{N}^k$. We can similarly represent a numerical semigroup $S=\langle n_1,\dots,n_k \rangle$ as a
matrix $A=\left[\begin{array}{rrr}n_1&\cdots&n_k\end{array}\right]$. Then for
every element $s\in S$ we can say $s=Au$ for some $u\in \mathbb{N}^k$.

\begin{defi}
  \cite{bernd}
  The set of tables
  \[\mathcal{F}(u)=\left\{v\in \mathbb{N}^k:Av=Au\right\}\]
  is called the \emph{fiber} of a contingency table $u\in \mathcal{T}(n)$ with
  respect to the model $\mathcal{M}_A$
\end{defi}

Contingency tables and matrix models are specific to statistics. The thing we
should take from this definition, is that a fiber $\mathcal{F}(u)$ of an element
$Au$ of our semigroup $A$ is the set of vectors $\left\{v\in \mathbb{N}^k:Av=Au
\right\}$
\begin{exa}
  For the numerical semigroup $A=\left[\begin{array}{rrr}3&4&5\end{array}\right]$
  we will find the fiber corresponding to the element $Au=8$.

  We need to find all solutions to the equation $\left[\begin{array}{rrr}3&4&5
\end{array}\right]\left[\begin{array}{rrr}x,y,z \end{array}\right]=8$ or
$3x+4y+5z=8$ where $x,y,z\in \mathbb{N}$. We find that $3+5$ and $4\cdot 2$ are
the only two possible solutions, and so for $Au=8$ we see that $\mathcal{F}(u)=
\{(1,0,1),(0,2,0)\}$.
\end{exa}

Before we give the definiton of a Markov basis, we note that the literature often refers to the elements of a Markov basis as \emph{moves}\cite[p.16]{bernd}
\begin{defi}
\cite{bernd}
Let $\mathcal{M}_A$ be the log-linear model associated with a matrix $A$ whose integer kernel we denote by $\ker_\mathbb{Z}(A)$.
A finite subset $\mathcal{B}\subset\ker_{\mathbb{Z}}(A)$ is a \emph{Markov basis} for $\mathcal{M}_A$ if for all $u\in \mathcal{T}(n)$ and all pairs $v,v'\in \mathcal{F}(u)$ there exists a sequence $u_1,\dots,u_L\in  \mathcal{B}$ such that
\[v'=v+\sum\limits_{k=1}^L{u_k}\text{ and }v+\sum\limits_{k=1}^l{u_k}\ge 0\text{ for all }l=1,\dots,L.\]
\end{defi}

\begin{exa}
  We will find the Markov basis which corresponds to the numerical semigroup
$A=\langle 3,4,5\rangle$. First we generate the fibers which correspond to the
elements of our NSG.
\begin{center}
\textnormal{
\begin{tabular}{|r|rrr|}
\hline
&3&4&5\\
\hline
3&1&0&0\\
4&0&1&0\\
5&0&0&1\\
6&2&0&0\\
7&1&1&0\\
8&1&0&1\\
8&0&2&0\\
9&3&0&0\\
9&0&1&1\\
10&2&1&0\\
10&0&0&2\\
\hline
\end{tabular}
}
\end{center}
We are particularly interested in fibers with more than one element. These are
the fibers associated with the elements $8,9$ and $10$. From the definition of
the Markov basis, we know that the product of $A$ and an element of the Markov
basis is $0$. Furthermore, we know that we can add a sequence of elements of the
Markov basis to any of the elements of a fiber to obtain any other element of
that fiber. Taking the difference of two elements from the same fiber will
meet both of these criteria.

Thus if $Au=8$ then $\mathcal{F}(u)=\{(1,0,1),(0,2,0)\}$ and so
$(-1,2,-1)\in\mathcal{B}$.
If $Av=9$ then $\mathcal{F}(v)=\{(3,0,0),(0,1,1)\}$ and so
$(3,-1,-1)\in\mathcal{B}$.
And if $Aw=10$ then $\mathcal{F}(w)=\{(2,1,0),(0,0,2)\}$ and so
$(-2,-1,2)\in\mathcal{B}$.
And so we have built our Markov basis.
\[
  \mathcal{B}=\left[\begin{array}{rrr}
  3&-1&-1\\
  -1&2&-1\\
  -2&-1&2\\
  \end{array}\right]
\]
We acknowledge that we have given no explanation as to why we stopped searching for elements of the Markov basis. In general there is no easy termination criteria when searching for these. However, we will come back to this example later, and show that in this case we have found all the elements of our basis.
\end{exa}


We also have the fundamental theorem of Markov bases which provides a direct relation
to a lattice ideal.
\begin{thm}
\cite[p.~54]{aoki}
A finite set of moves $\mathcal{B}$ is a Markov basis for $A$ if and only if the set of binomials $\{p^{\mathbf{z}^+}-p^{\mathbf{z}^-}|\mathbf{z}\in \mathcal{B}\}$ generates the toric ideal $I_A$.
\end{thm}

\section{Bijections}
As we mentioned in the last section, NSGs are simple to study.
It would be useful if we could consider more complicated objects as NSGs.
One such object is the lattice ideal. We will begin by looking at common mappings for monomials and binomials. We will continue on to connect these mappings to our NSGs. This will allow consider lattice ideals as NSGs.
\subsection{Ideals}
Consider
$\varphi:\mathbb{N}^n\to k[x_1,\dots,x_k]$ given by $\mathbf{\alpha}\mapsto
\mathbf{x}^\alpha$.

\begin{lemma}
Let $S$ be a NSG.  Then $\varphi(S)$ is the set of monic
monomials of a monomial ideal of $k[x]$, where $\varphi$ is as above.
\begin{proof}
  Let $S=\langle n_1,\dots,n_k\rangle$ and let $I=\langle x^{n_1},\dots,x^{n_k}\rangle$.
  We choose $M\in I$ and $s\in S$ where
  $M=\prod\limits_{i=1}^k{\left(x_i^{n_i}\right)^{a_i}}$ while
  $s=\sum\limits_{i=1}^k{a_in_i}$.
  We observe that $\varphi(s)=\varphi(\sum\limits_{i=1}^k{a_in_i})=\prod\limits_{i=1}^k{x_i^{a_in_i}}=\prod\limits_{i=1}^k{\left(x_i^{n_i}\right)^{a_i}}=M$.
\end{proof}
\end{lemma}

If we look at $\mathbb{Z}$ instead of $\mathbb{N}$ then we have an analogous map with binomials. In the binomial case we start with some $\mathbf{z}\in \mathbb{Z}^n$.
We define $\mathbf{z}=(z_1,\dots,z_n)$.
We say that $\mathbf{z}^+=\mathbf{z}\vee\mathbf{0}$
gives us a vector where $z_i^+=\max(z_i,0)$. Similarly
$\mathbf{z^-}=\mathbf{z}\wedge \mathbf{0}$ is a vector where $z_i^-=\min(z_i,0)$.
We can map an element of $\mathbb{Z}^n$ to a binomials over field $k$ by $\varphi:\mathbb{Z}^n\to k[x_1\dots x_n]$ where $\mathbf{z}\mapsto
\mathbf{x}^{\mathbf{z}^+}-\mathbf{x}^{\mathbf{z}^-}$.

If we wish for the map to be bijective, then we need some addition restrictions. Let us assume that $k$ has characteristic 2. We assume that $x-y=x+y$ for any $x,y\in k$. Then $\varphi(1,-1)=x-y=x+y=\varphi(1,1)$ and $\varphi(1,1)=x+y=y-x=\varphi(-1,1)$. This map is obviously not one to one, and so we must restrict ourselves to fields who have characteristic other than 2.

Now consider the binomial $x+x^2$. This binomial is not in the image of $\varphi$. Bijection requires surjection, and so we restrict our codomain to be the set of pure binomials in $k[x_1,\dots,x_n]$

\subsection{Markov Bases}
We wish to move from $\mathbb{N}$ to $k[x_1,\dots,x_n]$. The tool we need for this is the Markov basis. These bases are troublesome to compute, but theorem 2 tells us that they provide a map from $\mathbb{N}$ to $\mathbb{Z}^n$. This combined with our above discussion on binomials will give us all that we need to complete this map.

Each element $a$ of a numerical semigroup $S=\langle n_1,\dots,n_k\rangle$ takes
the form $a=a_1n_1+\dots+a_kn_k$. Now we examine the vectors $(a_1,\dots,a_k)$.
Note that any given $a\in S$ may have more than one vector associated with it, and the set of these vectors make up the fiber over $a$.
\begin{exa}
  $S=\langle 3,4,5\rangle$.
  Notice that $8=2\cdot 4=(3,4,5)\cdot(0,2,0)$ and $8=3+5=(3,4,5)\cdot(1,0,1)$.
  Thus the vectors $(0,2,0)$ and $(1,0,1)$ make up the fiber over $8\in S$.
\end{exa}
The vectors ${\bf a}=(a_1,\dots,a_k)$
and ${\bf b}=(b_1,\dots,b_k)$ are connected if there exists some $i$
such that $a_i>0$ and $b_i>0$. Furthermore, if ${\bf a}$ is connected to
${\bf b}$ and ${\bf b}$ is connected to ${\bf c}$ then ${\bf a}$ is connected to
${\bf c}$.

We are looking for elements of our NSG which have multiple
associated but disconnected vectors. Once we have found two disconnected vectors associated with an
element of our NSG, then we subtract them to find an element
of our Markov basis. We continue until we have found the elements of our Markov
basis guaranteed by Theorem 2.

\begin{exa}
  We will find the Markov basis which corresponds to the NSG
$\langle 3,4,5\rangle$. First we generate a list of vectors which correspond
to the elements of our NSG.
\begin{center}
\textnormal{
\begin{tabular}{|r|rrr|}
\hline
&3&4&5\\
\hline
3&1&0&0\\
4&0&1&0\\
5&0&0&1\\
6&2&0&0\\
7&1&1&0\\
8&1&0&1\\
8&0&2&0\\
9&3&0&0\\
9&0&1&1\\
10&2&1&0\\
10&0&0&2\\
\hline
\end{tabular}
\begin{tabular}{|r|rrr|}
\hline
&3&4&5\\
\hline
11&2&0&1\\
11&1&2&0\\
12&4&0&0\\
12&0&3&0\\
12&1&1&1\\
13&3&1&0\\
13&1&0&2\\
13&0&2&1\\
14&2&2&0\\
14&0&1&2\\
14&3&0&1\\
\hline
\end{tabular}
}
\end{center}
Now we see that $8,9,10$ all have two associated but disconnected vectors. The fibers of $11,12,13$ and $14$ are all connected. We further observe that the Frobenius number of this semigroup is 2. Now the fiber of 12 contains the special vector $(1,1,1)$. If we choose any element $n>14$ of our semigroup, then we can say $n=12+n'$. We know that $n'$ is also in our semigroup because $2$ is our Frobenius number. Thus the fiber of every element $n>14$ contains a vector that is the sum of $(1,1,1)$ and some other vector. Thus the fibers of all elements greater than $10$ are in fact connected. And so we can subtract the vectors in the fibers of $8,9,10$ to obtain all the elements of our Markov basis.
\[
  \left[\begin{array}{rrr}
  3&-1&-1\\
  -1&2&-1\\
  -2&-1&2\\
  \end{array}\right]
\]
Which is our Markov basis.
\end{exa}

It is convenient to write our Markov basis as a matrix, whose rows consist of
the elements of the Markov basis.

\subsection{Smith Normal Form}
Now we have found a map from NSGs to pure binomial ideals of $k[x_1,\dots,x_n]$ where $k$ is not characteristic 2. If we wish to consider the NSG and the binomial ideals the same objects, then given a Markov basis, we should be able to find an associated NSG.
Let $M$ be the matrix with rows corresponding to the vectors of a Markov basis.
Every row $v$ of $M$ consists of two vectors $a,b$ where $v=a-b$. Let $S=
\langle n_1,\dots,n_k\rangle$ and define the vector $s=(n_1,\dots,n_k)$. Now
if $S$ is the semigroup associate with $M$ then by  construction $s\cdot a=s\cdot b$. And so $s\cdot v=0$. This means that if we have $M$, we can
find our numerical semigroup $S$ by finding a nontrivial solution to $Ms=0$.

The usual tools of linear algebra are not useful here. The major problem is that we are dealing with matrices in $\mathbb{Z}$ instead of $\mathbb{R}$ and so we do not have division available in general. The tool we need is the Smith normal form.
\begin{defi}
  If we are given some matrix $A$ whose entries are in a principal ideal domain,
  then we can find some matrices $U,V,B$ such that
\[UAV=B=
\left[\begin{array}{cccc}
  b_1&\cdots&0\\
  \vdots&\ddots&\vdots \\
  0&\cdots&b_r&
\end{array}\right]
\text{ with }b_i|b_{i+1}\]
We call $B$ the \emph{Smith normal form} of $A$.\cite{adkins}
\end{defi}
If there exists some $s\ne \mathbf{0}, M\in \mathcal{M}_{q\times r}$ such that $Ms=\mathbf{0}$ then $\text{rank}(M)<r$. And so if we have $UMV=B$ where $B$ is the Smith normal form of $M$ then $B=\left[\begin{array}{ccccc}
  b_1&\cdots&0&0\\
  \vdots&\ddots&\vdots&\vdots \\
  0&\cdots&b_{r-1}&0\\
  0&\cdots&0&0\\
\end{array}\right]
$
. Because the last column of $B$ is $\mathbf{0}$ we know that if $\mathbf{v}$ is the last column of $V$ then $M\mathbf{v}=\mathbf{0}$. The semigroup generated by the coordinates of $\mathbf{v}$ must be isomorphic to a unique numerical semigroup generated by the coordinates of some vector $\mathbf{s}=\alpha\mathbf{v}$ where $\alpha\in \mathbb{Q}$ and the greatest common divisor of the coordinates of $s$ is 1.
\begin{exa}
  Let us find the Smith normal form of
  $
  A=\left[\begin{array}{rrr}
  3&-1&-1\\
  -1&2&-1\\
  -2&-1&2\\
  \end{array}\right]
  $. We start with $U'AV'=I_AAI_A$. We then use the standard row and column
  operations on $A$ while recording the row operations on $U'$ and the
  column operations on $V'$ to find our Smith normal form, along with the
  $U$ and $V$. Remember we are working over $\mathbb{Z}$ and so we will only be adding, subtracting, and multiplying, not dividing.
  \begin{align*}
  \left[\begin{array}{rrr}
  1&0&0\\
  0&1&0\\
  0&0&1\\
  \end{array}\right]
  &
  \left[\begin{array}{rrr}
  3&-1&-1\\
  -1&2&-1\\
  -2&-1&2\\
  \end{array}\right]
  \left[\begin{array}{rrr}
  1&0&0\\
  0&1&0\\
  0&0&1\\
  \end{array}\right]\\
  &\Downarrow\\
  \left[\begin{array}{rrr}
  1&0&0\\
  0&1&0\\
  1&1&1\\
  \end{array}\right]
  &
  \left[\begin{array}{rrr}
  3&-1&-1\\
  -1&2&-1\\
  0&0&0\\
  \end{array}\right]
  \left[\begin{array}{rrr}
  1&0&0\\
  0&1&0\\
  0&0&1\\
  \end{array}\right]\\
  &\Downarrow\\
  \left[\begin{array}{rrr}
  1&2&0\\
  0&1&0\\
  1&1&1\\
  \end{array}\right]
  &
  \left[\begin{array}{rrr}
  1&3&-3\\
  -1&2&-1\\
  0&0&0\\
  \end{array}\right]
  \left[\begin{array}{rrr}
  1&0&0\\
  0&1&0\\
  0&0&1\\
  \end{array}\right]\\
  &\Downarrow\\
  \left[\begin{array}{rrr}
  1&2&0\\
  1&3&0\\
  1&1&1\\
  \end{array}\right]
  &
  \left[\begin{array}{rrr}
  1&3&-3\\
  0&5&-4\\
  0&0&0\\
  \end{array}\right]
  \left[\begin{array}{rrr}
  1&0&0\\
  0&1&0\\
  0&0&1\\
  \end{array}\right]\\
  &\Downarrow\\
  \left[\begin{array}{rrr}
  1&2&0\\
  1&3&0\\
  1&1&1\\
  \end{array}\right]
  &
  \left[\begin{array}{rrr}
  1&0&-3\\
  0&1&-4\\
  0&0&0\\
  \end{array}\right]
  \left[\begin{array}{rrr}
  1&0&0\\
  0&1&0\\
  0&1&1\\
  \end{array}\right]\\
  &\Downarrow\\
  \left[\begin{array}{rrr}
  1&2&0\\
  1&3&0\\
  1&1&1\\
  \end{array}\right]
  &
  \left[\begin{array}{rrr}
  1&0&0\\
  0&1&0\\
  0&0&0\\
  \end{array}\right]
  \left[\begin{array}{rrr}
  1&0&3\\
  0&1&4\\
  0&1&5\\
  \end{array}\right]\\
\end{align*}
And so
$\left[\begin{array}{rrr} 1&2&0\\ 1&3&0\\ 1&1&1\\ \end{array}\right]
\left[\begin{array}{rrr} 3&-1&-1\\ -1&2&-1\\ -2&-1&2\\ \end{array}\right]
\left[\begin{array}{rrr} 1&0&3\\ 0&1&4\\ 0&1&5\\ \end{array}\right]
=\left[\begin{array}{rrr} 1&0&0\\0&1&0\\0&0&0 \end{array}\right]
$

Observe that the Markov basis was the basis obtained in a previous example from the NSG $\langle3,4,5\rangle$. We see that the last column of $V$ is $(3,4,5)$ which corresponds exactly to the NSG we began with.
\end{exa}
Calculating these matrices is tedious, time consuming and error prone. Fortunately computers well suited to these kinds of calculations. One software package that can easily compute these matrices is Xcas\cite{xcas}.

\subsection{Numerical Semigroups in Lattice Theory}
We have already discussed NSGs, Markov basis, and binomials. Now we examine lattice ideals, which can be expressed as binomial ideals.
\begin{defi}
\cite{stanley}
A \emph{lattice} is a partially ordered set in which every two elements have a
unique least upper bound and a unique greatest lower bound.
\end{defi}
\begin{exa}
The space $\mathbb{N}^2$ is a lattice with supremum and infinum for any two elements which
belong to it. Notice that $(1,2)$ and $(2,1)$ have a lower bound of $(1,1)$ and
an upper bound of $(2,2)$.
\end{exa}
\begin{exa}
We can form a lattice if we order $\mathbb{N}$ by division. The least common
multiple forms a least upper bound and an greatest lower bound is formed by the
greatest common denominator.
\end{exa}
The fundamental theorem of Markov bases (theorem 2)
states that there is a bijection between a lattice ideal and a Markov basis. As
we have already seen, Markov bases are in bijection with NSGs.

Thus we see that every NSG corresponds to a lattice ideal.

\begin{exa}
  We begin with a lattice ideal in $k[x,y,x]$ with $\text{char}(k)\ne 2$
  \[
  I_\Lambda=\langle x^3-yz,y^2-xz,z^2-xy\rangle
  \]
  We can map this generating set to $\mathbb{Z}^3$ in the usual way.
  \[
  \begin{cases}
    x^3-yz\\
    y^2-xz\\
    z^2-x^2y
  \end{cases}
  \Rightarrow
  \left[\begin{array}{rrr}
    3&-1&-1\\
    -1&2&-1\\
    -1&-1&2
  \end{array}\right]
  \]
  Theorem 2 is the here. It guarantees that these vectors are
  actually a Markov basis. As we saw in the previous two sections, this basis
  corresponds to the numerical semigroup $\langle 3,4,5\rangle$. So we see
  \[
  \begin{cases}
    x^3-yz\\
    y^2-xz\\
    z^2-xy
  \end{cases}
  \mapsto\langle 3,4,5\rangle
  \]
  The reverse map is similarly straightforward.
\end{exa}


\section{Gaussian Integers}
The complex numbers are a natural extension of the real numbers as the Gaussian integers are a natural extension of the real integers. We consider the corresponding extension of the natural numbers. In this section, we will explore the NSG construction in terms of the Gaussian integers and discuss the similarities and differences.

We take two NSGs $\langle A\rangle, \langle B\rangle$ where $A=\{x_1,\dots,x_n\}$ and $B=\{y_1,\dots,y_n\}$.
These semigroups can be used to form a ``Gaussian numerical semigroup'' $\langle G\rangle$ where $G=\{x_1+y_1i,\dots,x_n+y_ni\}$.
We observe that the we can generate every number greater than the Frobenius number in the real part and in the imaginary part.
However, unlike the component NSGs, the GNSG never becomes dense.
This is because any $n>F(\langle A\rangle)$ there are a finite number of ways in which we can combine the elements of $A$ to add up to $n$.
Combining the elements of $B$ in the same ways will then only produce a finite number of outputs.
This means that for any $n\in \langle A\rangle$ we can only generate a finite number of $n+mi\in \langle G\rangle$.
Surprisingly, we have the opposite case that we have in the NSG.
Rather than having a finite complement in any infinite subset of the positive Gaussian integers, we have an infinite complement.
The GNSG remains sparse over the entire $\mathbb{N}^2$ lattice.

\begin{exa}
We consider $\langle 2,7\rangle$ and $\langle 3,5\rangle$. These NSGs have Frobenius numbers $5$ and $7$ respectively\cite{frobmask}. The following is a plot in $\mathbb{N}^2$ of the numbers generated by $\langle 2+3i,7+5i\rangle$.

\begin{center}
\begin{tikzpicture}[scale=.1]
  \draw[-{Latex[scale=1.5]}] (0,0) -- (0,50);
  \draw[-{Latex[scale=1.5]}] (0,0) -- (50,0);
\begin{luacode}
  for i=10,40,10 do
    tex.print('\\draw (-.5,'..i..') -- (.5,'..i..') node[left=5pt] {$'..i..'$};')
    tex.print('\\draw ('..i..',-.5) -- ('..i..',.5) node[below=5pt] {$'..i..'$};')
  end
  for i=0,20 do -- the (2,3)
    for j=0,7 do -- the (7,5)
      if 2*i+7*j>50 or 3*i+5*j>50 then break end
      tex.print('\\draw ('..2*i+7*j..','..3*i+5*j..') circle[radius=5pt];')
    end
  end
\end{luacode}
\end{tikzpicture}
\end{center}
\end{exa}
This notion of a GNSG does impose a sort of ordering on our elements that is not present in the original numerical semigroup. This ordering is caused by linking elements of our two NSGs in the generation of the GNSG. In addition, combining two NSGs with differently sized generating sets could be problematic.

This problems can be overcome with the  creation of a semigroup in the form of $\langle A\oplus B\rangle$. Combining NSGs whose generating sets are different sizes is explicitly handled in this case, and the two semigroups are allowed to generate their elements independently. Unfortunately we still have the same problem of only being able to generate any number in a finite number of ways. This leaves us with the same problem of the semigroup remaining sparse over all of $\mathbb{N}^2$
\begin{exa}
We consider $\langle 2,7\rangle$ and $\langle 3,5\rangle$. These NSGs have Frobenius numbers $5$ and $7$ respectively\cite{frobmask}. The following is a plot in $\mathbb{N}^2$ of the numbers generated by $\langle 2+3i,7+5i,2+5i,7+3i\rangle$.

\begin{center}
\begin{tikzpicture}[scale=.1]
  \draw[-{Latex[scale=1.5]}] (0,0) -- (0,50);
  \draw[-{Latex[scale=1.5]}] (0,0) -- (50,0);
\begin{luacode}
  for i=10,40,10 do
    tex.print('\\draw (-.5,'..i..') -- (.5,'..i..') node[left=5pt] {$'..i..'$};')
    tex.print('\\draw ('..i..',-.5) -- ('..i..',.5) node[below=5pt] {$'..i..'$};')
  end
  for i=0,20 do -- the (2,3)
    for j=0,7 do -- the (7,5)
      for k=0,10 do -- the (2,5)
        for l=0,7 do -- the (7,3)
      if 2*i+7*j+2*k+7*l>50 or 3*i+5*j+5*k+3*l>50 then break end
      tex.print('\\draw ('..2*i+7*j+2*k+7*l..','..3*i+5*j+5*k+3*l..') circle[radius=5pt];')
        end
      end
    end
  end
\end{luacode}
\end{tikzpicture}
\end{center}
\end{exa}
If we wish to create a semigroup in $\mathbb{N}[i]$ which becomes dense, then we must allow the semigroup generators to include zero for the real and complex components.
Thus our semigroup would be $\langle x_1,\dots,x_n,y_1i,\dots,y_ni\rangle$.
But this is just the direct sum of our original NSGs.
We can find any point $x+yi\in \langle A\rangle\oplus \langle B\rangle$ so long as $x>F(A)$ and $y>F(B)$.
\begin{exa}
We consider $\langle 2,7\rangle$ and $\langle 3,5\rangle$. These NSGs have Frobenius numbers $5$ and $7$ respectively\cite{frobmask}. The following is a plot in $\mathbb{N}^2$ of the numbers generated by $\langle 3i,5i,2,7\rangle$.

\begin{center}
\begin{tikzpicture}[scale=.1]
  \draw[-{Latex[scale=1.5]}] (0,0) -- (0,50);
  \draw[-{Latex[scale=1.5]}] (0,0) -- (50,0);
\begin{luacode}
  for i=10,40,10 do
    tex.print('\\draw (-.5,'..i..') -- (.5,'..i..') node[left=5pt] {$'..i..'$};')
    tex.print('\\draw ('..i..',-.5) -- ('..i..',.5) node[below=5pt] {$'..i..'$};')
  end
  for i=0,50,2 do
    for j=0,50,7 do
      for k=0,50,3 do
        for l=0,50,5 do
      if i+j>50 or k+l>50 then break end
      tex.print('\\draw ('..i+j..','..k+l..') circle[radius=5pt];')
        end
      end
    end
  end
\end{luacode}
\end{tikzpicture}
\end{center}
\end{exa}
\section{Conclusion}
Extending the concept of a NSG into the Gaussian integers breaks everything we have seen so far. There may be ways to fix this problem and we have discussed two alternative approaches to this problem. We do not know if any of the relationships between NSGs and lattice ideals and Markov bases can be applied in this new space. Going forward it remains to examine these Gaussian semigroups and see if we can find analogues to lattices and Markov bases in the Gaussian integers.
\bibliographystyle{amsplain}
\bibliography{bib}


\end{document}
