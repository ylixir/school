
% $Header: /cvsroot/latex-beamer/latex-beamer/solutions/conference-talks/conference-ornate-20min.en.tex,v 1.6 2004/10/07 20:53:08 tantau Exp $
\PassOptionsToPackage{pdfpagelabels=false}{hyperref}
\documentclass
%[handout] % Enable this opption if you want a compact printout of the talk for distribution.
[notheorems]{beamer}

% This file is a solution template for:

% - Talk at a conference/colloquium.
% - Talk length is about 20min.
% - Style is ornate.



% Copyright 2004 by Till Tantau <tantau@users.sourceforge.net>.
%
% In principle, this file can be redistributed and/or modified under
% the terms of the GNU Public License, version 2.
%
% However, this file is supposed to be a template to be modified
% for your own needs. For this reason, if you use this file as a
% template and not specifically distribute it as part of a another
% package/program, I grant the extra permission to freely copy and
% modify this file as you see fit and even to delete this copyright
% notice. 


\mode<presentation>
{
  \usetheme{Warsaw}
	\usecolortheme[RGB={0,102,51}]{structure}
  % or ...

  %\setbeamercovered{transparent}
  % or whatever (possibly just delete it)
}


\usepackage[english]{babel}
% or whatever

%\usepackage[latin1]{inputenc}
\usepackage[utf8]{inputenc}
% or whatever

%\usepackage{times}
%\usepackage[T1]{fontenc}
\usepackage{pgf,tikz}
\usepackage{color}
\usepackage{xcolor}
\usepackage{cases}

\def\mathunderline#1#2{\color{#1}\underline{{\color{black}#2}}\color{black}}
\usetikzlibrary{arrows}
\usetikzlibrary{calc}
\newcommand\scalemath[2]{\scalebox{#1}{\mbox{\ensuremath{\displaystyle #2}}}}
\newcommand{\R}{{\mathbb R}}
\newcommand{\Q}{{\mathbb Q}}
\newcommand{\C}{{\mathbb C}}
\newcommand{\N}{{\mathbb N}}
\newcommand{\Z}{{\mathbb Z}}
\newcommand{\Fr}{{\text{Fr}}}
\newcommand{\LCM}{\text{LCM}}
\newcommand{\multideg}{\text{multideg}}
\newcommand{\ord}{\text{ord}}
\newcommand{\Irr}{\text{Irr}}
%\newtheorem{conjecture}[lem]{Conjecture}
% Or whatever. Note that the encoding and the font should match. If T1
% does not look nice, try deleting the line with the fontenc.

\definecolor{ffffff}{rgb}{1.0,1.0,1.0}
\definecolor{wwwwqq}{rgb}{0.4,0.4,0}
\definecolor{zzqqtt}{rgb}{0.6,0,0.2}
\definecolor{zzttqq}{rgb}{0.6,0.2,0}
\definecolor{qqwwcc}{rgb}{0,0.4,0.8}
\definecolor{fffftt}{rgb}{1,1,0.2}
\definecolor{qqqqff}{rgb}{0,0,1}
\definecolor{qqzzqq}{rgb}{0,0.6,0}
\definecolor{ffqqqq}{rgb}{1,0,0}
\definecolor{uququq}{rgb}{0.25,0.25,0.25}
\definecolor{qqffqq}{rgb}{0.0,1.0,0.0}
\definecolor{qqqqff}{rgb}{0.0,0.0,1.0}
\definecolor{qqwuqq}{rgb}{0.0,0.39215686274509803,0.0}
\definecolor{uuuuuu}{rgb}{0.26666666666666666,0.26666666666666666,0.26666666666666666}

\newtheorem{lem}{Lemma}[section]
\newtheorem{cor}[lem]{Corollary}
\newtheorem{prop}[lem]{Proposition}
\newtheorem{theorem}[lem]{Theorem}
\newtheorem{conjecture}[lem]{Conjecture}
\newtheorem{example}[lem]{Example}
\newtheorem{definition}[lem]{Definition}


\title[Numerical Semigroups, Lattice Ideals, and Markov Bases] % (optional, use only with long paper titles)
{Numerical Semigroups, Lattice Ideals, and Markov Bases}


\author[Allen] % (optional, use only with lots of authors)
{Jon Allen}
% - Give the names in the same order as the appear in the paper.
% - Use the \inst{?} command only if the authors have different
%   affiliation.

\institute[NDSU] % (optional, but mostly needed)
{
  %
  Department of Mathematics\\
  North Dakota State University\\
  Fargo, ND}
  
% - Use the \inst command only if there are several affiliations.
% - Keep it simple, no one is interested in your street address.

\date[AMS2007] % (optional, should be abbreviation of conference name)
{Capstone Presentation, North Dakota State University, December 2015}
% - Either use conference name or its abbreviation.
% - Not really informative to the audience, more for people (including
%   yourself) who are reading the slides online

 



% If you have a file called "university-logo-filename.xxx", where xxx
% is a graphic format that can be processed by latex or pdflatex,
% resp., then you can add a logo as follows:

% \pgfdeclareimage[height=0.5cm]{university-logo}{university-logo-filename}
% \logo{\pgfuseimage{university-logo}}



% Delete this, if you do not want the table of contents to pop up at
% the beginning of each subsection:



% If you wish to uncover everything in a step-wise fashion, uncomment
% the following command: 

%\beamerdefaultoverlayspecification{<+->}


\begin{document}

\begin{frame}
  \titlepage
\end{frame}

%\begin{frame}
  %\frametitle{Outline}
%  \tableofcontents
  % You might wish to add the option [pausesections]
%\end{frame}


% Structuring a talk is a difficult task and the following structure
% may not be suitable. Here are some rules that apply for this
% solution: 

% - Exactly two or three sections (other than the summary).
% - At *most* three subsections per section.
% - Talk about 30s to 2min per frame. So there should be between about
%   15 and 30 frames, all told.

% - A conference audience is likely to know very little of what you
%   are going to talk about. So *simplify*!
% - In a 20min talk, getting the main ideas across is hard
%   enough. Leave out details, even if it means being less precise than
%   you think necessary.
% - If you omit details that are vital to the proof/implementation,
%   just say so once. Everybody will be happy with that.
\section{Overview of Numerical Semigroup}
\begin{frame}\frametitle{Overview}
A numerical semigroup is a nonempty subset $S$ of $\mathbb{N}$ that is closed under addition, contains the zero element, and whose complement in $\mathbb{N}$ is finite.
\pause
\begin{itemize}
\item It is closed under addition
\pause
\item It is generated from positive (nonzero) integers
\pause
\item The greatest common factor of its generators is 1
\end{itemize}
\end{frame}
\begin{frame}\frametitle{Example}
Let $S$ be the numerical semigroup generated by $\{n_1,\dots,n_k\}$ with $n_i\in \mathbb{N}\setminus \{0\}$. Then the elements of $S$ are $a_1n_1+\dots a_kn_k$ for all $a_i\in \mathbb{N}$.
\pause
\begin{example}
The semigroup generated by $\{3,4,5\}$ is $\{3,4,5,6,7,8,\dots\}$
\end{example}
\end{frame}
\section{Making Markov}
\begin{frame}\frametitle{Table}
We can make a table of $\langle 3, 4, 5\rangle$ rows corresponding to the coefficients of the generators.
\pause
\begin{example}
\begin{center}
\begin{tabular}{|r|rrr|}
\hline
&3&4&5\\
\hline
3&1&0&0\\\pause
4&0&1&0\\
5&0&0&1\\
6&2&0&0\\
7&1&1&0\\
8&1&0&1\\\pause
8&0&2&0
\end{tabular}
\pause
\begin{tabular}{|r|rrr|}
\hline
&3&4&5\\
\hline
9&3&0&0\\
9&0&1&1\\
10&2&1&0\\
10&0&0&2\\\pause
11&2&0&1\\
11&1&2&0
\end{tabular}
%\hline
\end{center}
\end{example}
\end{frame}

\begin{frame}\title{Fibers}
A fiber consists of the different linear combinations of generators that result in an element of our semigroup.
\end{frame}
\begin{frame}\title{Moves}
Moves happen when elements of fibers are 'disconnected'.
\pause
\begin{example}
\begin{align*}
  10&=2\cdot3+1\cdot 4+0\cdot 5&11&=2\cdot 3+0\cdot 4+1\cdot 5\\
  10&=0\cdot3+0\cdot 4+2\cdot 5&11&=1\cdot 3+2\cdot 4+0\cdot 5\\
\end{align*}
\end{example}
\end{frame}
\begin{frame}\title{Moves}
Moves are the elements of the Markov basis and are the difference of disconnected elements of fibers.
\pause
\begin{example}
\begin{tabular}{|r|rrr|rrr}
\hline
&3&4&5&&&\\
\hline
8&1&0&1&&&\\
8&0&2&0&-1&2&-1\\
9&3&0&0&&&\\
9&0&1&1&3&-1&-1\\
10&2&1&0&&&\\
10&0&0&2&-2&-1&2
\end{tabular}
\end{example}
\end{frame}
\section{Integer Lattice}

\begin{frame}\title{Integer Lattice}
We have an easy bijection between our Markov basis and an integer lattice.
\pause
\begin{example}
\[
\left[\begin{array}{rrr}3&-1&-1\\-1&2&-1\\-2&-1&2\end{array}\right]
\Leftrightarrow
\begin{cases}x^3-yz\\y^2-xz\\z^2-x^2y\end{cases}
\]
\end{example}
\end{frame}
\section{Smith Normal Form}
\begin{frame}
We have actually explicitly built our Markov basis to be the null space of the numerical semigroup basis.
\pause
If we can find some vector $x$ such that
$\left[\begin{array}{rrr}3&-1&-1\\-1&2&-1\\-2&-1&2\end{array}\right]x=0$
Then we will have found our semigroup!
\end{frame}
\begin{frame}
What we need is the Smith Normal Form.
\[UAV=\left[\begin{array}{rrr}3&-1&-1\\-1&2&-1\\-2&-1&2\end{array}\right]x=0
\]
\end{frame}
\begin{frame}
We start with identity matrices on either side of our Markov matrix.

The procedure is similar to finding and inverse matrix, (except the
Markov matrix is singular).
\pause

We reduce our Markov matrix, mirroring column and row operations in
the adjacent matrices.
\pause

We can't use anything but integers for our row and column operations!
\begin{align*}
  \left[\begin{array}{rrr}
  1&0&0\\
  0&1&0\\
  0&0&1\\
  \end{array}\right]
  &
  \left[\begin{array}{rrr}
  3&-1&-1\\
  -1&2&-1\\
  -2&-1&2
  \end{array}\right]
  \left[\begin{array}{rrr}
  1&0&0\\
  0&1&0\\
  0&0&1\\
  \end{array}\right]
\end{align*}
\end{frame}
\begin{frame}
Row operations on the left
\begin{align*}
  \left[\begin{array}{rrr}
  1&2&0\\
  1&3&0\\
  1&1&1\\
  \end{array}\right]
  &
  \left[\begin{array}{rrr}
  1&3&-3\\
  0&5&-4\\
  0&0&0\\
  \end{array}\right]
  \left[\begin{array}{rrr}
  1&0&0\\
  0&1&0\\
  0&0&1\\
  \end{array}\right]
\end{align*}
\end{frame}
\begin{frame}
Column operations on the right
\begin{align*}
    \left[\begin{array}{rrr}
  1&2&0\\
  1&3&0\\
  1&1&1\\
  \end{array}\right]
  &
  \left[\begin{array}{rrr}
  1&0&0\\
  0&1&0\\
  0&0&0\\
  \end{array}\right]
  \left[\begin{array}{rrr}
  1&0&3\\
  0&1&4\\
  0&1&5\\
  \end{array}\right]
\end{align*}
\end{frame}
\begin{frame}
\begin{center}
\LARGE Thank You!
\end{center}
\end{frame}
\end{document}


