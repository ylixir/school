\documentclass[letterpaper]{article}

\usepackage{fullpage}
\usepackage{nopageno}
\usepackage{amsmath}
\usepackage{amssymb}
\usepackage[utf8]{inputenc}
\usepackage{aeguill}
\usepackage{setspace}
\usepackage{apacite}
\allowdisplaybreaks

\newcommand{\abs}[1]{\left\lvert #1 \right\rvert}

\begin{document}
\title{Formal Outline\\ Informative presentation on Évariste Galois}
\date{March 6, 2015}
\author{Jon Allen}
\maketitle
\renewcommand{\labelenumi}{\Roman{enumi}}
%\renewcommand{\labelenumii}{\arabic{enumii}.}
\renewcommand{\labelenumii}{\Alph{enumii}.}
%\renewcommand{\labelenumiii}{(\alph{enumiii})}
%\renewcommand\thefigure{4.\arabic{figure}}

\begin{description}
\item[Introduction]$\quad$

  \begin{enumerate}
  \item
  \emph{Attention Getter}: ``He dies young whom the gods love.'' -- Menander
  \item
  \emph{Speaker Credibility}:
  My major is math and so in the course of my studies I've learned about various mathematicians from history. One of the most interesting is Galois.
  \item
  \emph{Listener Relevance}:
  Galois is one of the craziest mathematicians to ever live. He spent quite a bit of his life expelled, or in jail. In fact he spent his entire adult life...well he didn't really have an adult life.
  \item
  \emph{Thesis}:
  We will explore the life of Galois, touching on his genius, but we will try to avoid boring math stuff.
  \item
  \emph{Preview}:
  We will first examine the world Galois lived in, then we will talk about his childhood and early life, then we will talk about his death and legacy.
  \end{enumerate}
\item[Body]$\quad$

\begin{enumerate}
  \item
  \emph{First Main Point}: In order to understand Galois, we must understand the time he lived in. Namely nineteenth century France.

  \begin{enumerate}
    \item
    We can't really talk about the early 1800's without mentioning Napoleon. Napoleon invested a great deal of resources into French universities, and as a result Galois' peers were the greatest mathematicians in the world. \cite{gods}
    \item
    Unfortunately Waterloo happened 4 years after Galois was born, leaving France in a kind of political limbo. France had decided that it didn't want a king to tell them how to live, but they had a pretty decent time of it with Napoleon. And so they decided on a kind of constitutional monarchy. They put a king on the throne, but created a couple legislative bodies and decided the courts would be independent, along with the press. \cite{france}
    \item
    In 1824 the first of these kings died, and was succeeded by his brother. People very quickly became unhappy with this new king. By 1830 he had been thrown out an replaced. This third king was not bad, but the genie was already out of the bottle. Again.
  \end{enumerate}
  \renewcommand{\labelenumi}{\emph{Transition}:}
  \item
  The parents of Galois were actually quite liberal, and involved in politics. In that time and place that meant they were Republicans. This early influence no doubt shaped his choices later in life.
  \renewcommand{\labelenumi}{\Roman{enumi}}
  \setcounter{enumi}{1}
  \item
  \emph{Second Main Point}: The early life and education of Galois.
  \begin{enumerate}
    \item
    Galois was born in 1811, the same year as Napoleon's son. \cite{gods} His father was the mayer of Bourg-la-Reine. His mother had a very strong literary background, and apparently did quite a good job home schooling him. Even turning down a college that tried to recruit him at 10.
    \item
    At 12 he finally went to the School of Louis the Great. He won awards for Latin and stuff, but quickly got bored and started reading original mathematical papers like novel.
    \item
    When he was 17 he tried to get into the best mathematics school in France. He didn't prepare though and didn't make the cut. Instead he went to a school that literally translates to the Normal School.
  \end{enumerate}
  \renewcommand{\labelenumi}{\emph{Transition}:}
  \item
  That was basically the beginning of the end for Galois. At this point he starts to make fundamental discoveries about the nature of mathematics. Things are also really starting to heat up politically.
  \renewcommand{\labelenumi}{\Roman{enumi}}
  \setcounter{enumi}{2}
  \item
  \emph{Third Main Point}: The life work and the politics of Galois.
  \begin{enumerate}
    \item
    At this point Galois, despite attending a second rate school is publishing papers that are solving problems that are several centuries old, and laying down the foundations for...well for algebra.
    \item
    In 1829 his dad kills himself after a dispute with a priest...um...And people keep dying with, or losing the papers that Galois is writing.
    \item
    When Galois is 19 he is kicked out of school for calling the head of his school a political coward. He then joins an artillery unit in the National Guard. This artillery unit is then disbanded for fear that it will overthrow the government. And then he is thrown in jail for conspiracy to overthrow the government.
    \item
    When Galois gets out of jail he threatens the kings life and shortly thereafter ends up in mysterious duel. Allegedly over love, but probably over politics. The night before the duel, knowing that he is going to die the next day, he puts pen to paper and writes down a some of the ideas he has not been able to publish. Hermann Weyl said that work ``is perhaps the most substantial piece of writing in the whole literature of mankind.''\cite{abstract}
  \end{enumerate}
  \renewcommand{\labelenumi}{\emph{Transition}:}
  \item
  And at twenty years old, one of the greatest minds we have ever produced is dead. Sacrificing an unimaginable wealth of knowledge in service of liberty.
  \renewcommand{\labelenumi}{\Roman{enumi}}
\end{enumerate}
\item[Conclusion]$\quad$
\begin{enumerate}
  \item
  \emph{Thesis Restatement}:
  We have looked very briefly at his life, focussing mostly on the color and only mentioning the brilliance.
  \item
  \emph{Main Point Summary}:
  Galois was truly a product the times he lived. Born to political and well educated parents, into a world dominated by intellectuals and politics. From his childhood through to his death, he embodied the best of all of these.
  \item
  \emph{Clincher}:
  The really great minds are often unnamed. We don't study Greek Government. We study democracy. We don't study Einsteinian Science. We study physics. We don't study Galois Theory. We study algebra.

\begin{quote}
Death is a Dialogue between

The Spirit and the Dust.

"Dissolve" says Death--The Spirit "Sir

I have another Trust"--
\end{quote}--Emily Dickinson
\end{enumerate}
\end{description}


\newpage
\bibliographystyle{apacite}
\bibliography{speech-bib-2015-03-06}
\end{document}
