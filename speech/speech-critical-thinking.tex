\documentclass[letterpaper,12pt]{article}

\usepackage[in]{fullpage}
\usepackage{nopageno}
\usepackage{amsmath}
\usepackage{amssymb}
\usepackage[utf8]{inputenc}
\usepackage{aeguill}
\usepackage{setspace}
\usepackage{url}
\usepackage{apacite}
%\usepackage[T1]{fontenc}
%\usepackage{mathptmx}
\usepackage{setspace}
\usepackage{fontspec}
\setmainfont{DejaVu Serif}
\mathchardef\UrlBreakPenalty=9999
\mathchardef\UrlBigBreakPenalty=9999
\allowdisplaybreaks

\newcommand{\abs}[1]{\left\lvert #1 \right\rvert}

\begin{document}
\title{Critical Thinking Essay}
\date{April 24, 2015}
\author{Jon Allen}
\maketitle
\doublespacing
\renewcommand{\labelenumi}{\Roman{enumi}}
%\renewcommand{\labelenumii}{\arabic{enumii}.}
\renewcommand{\labelenumii}{\Alph{enumii}.}
%\renewcommand{\labelenumiii}{(\alph{enumiii})}
%\renewcommand\thefigure{4.\arabic{figure}}

For the subject of this essay, I have chosen a speech on birth control by Margaret Sanger \cite{speech}. The speech was given in 1921 and it is a persuasive speech arguing that using birth control is both moral and necessary. I think that most of this speech remains relevant today. However, some of Sanger's arguments are specious at best. It's much easier to rip apart fallacies than it is to support strong arguments\dots so that is what we will do. We will first examine how Sanger uses post hoc argumentation to link morality to the development of technology. We will then see how she uses the bandwagon technique to group people who use birth control and people with enviable characteristics. She then cleverly slides from this grouping to a scary cause and effect implication.

A post hoc argument claims that because something happened before another thing, that first thing caused the second thing \cite{text}. In this speech Sanger discusses how action depends on thought. Her argument is as our brains develop, then our actions become less reckless, more responsible, and more moral. The conclusion she draws is that morality  is a direct cause of brain development. This premise is then used to argue that advances in womens rights are the natural result of a developing society. Of course the irony here is that it trivializes the gains that feminists like Sanger fought and suffered for. This fallacy runs deep in her arguments. She uses it to explain why we no longer practice infanticide or abortion. I'll be happy to agree with her on infanticide, but a century of hindsight shows that regardless of how one might feel about abortion, it has not disappeared with the progression of our society, but has returned. Later she again makes this link with morality intelligence. She speaks of a society divided into 3 groups. Two of which are responsible and intelligent, the third which is not intelligent and irresponsible and reckless. This linking of intelligence to behavior starts as only a small problem\dots Lets take a look at how she groups us intelligent people together with her through the bandwagon technique.

This technique attempts to paint people who are not in agreement with a point of view as someone whom one does not wish to be \cite{text}. Near the beginning of her speech she mentions a group who disagrees with her point of view, and points out that they are ``a disgrace to liberty-loving people''. This group  is actually the police \cite{origin}. Of course we all love liberty. We certainly don't want to be a disgrace to liberty lovers. Seriously, the word liberty should always be met with skepticism. She then goes on to break society into three groups. Those who are intelligent and wealthy (count me in with those guys!), those who are intelligent and responsible, but ignorant--notice how she links intelligence with responsibility and therefore morality. And the third group, the irresponsible and reckless. Of course the diseased and feeble-minded are in the irresponsible and reckless group. Intelligence implies responsibility and morality after all. Now we have set the stage for the most horrifying aspect of Sanger's argument.

Now she points out that it's immoral to fill the earth with disease, poverty, and misery. But the moral people are using birth control and the immoral people are not. Also the immoral people are poor, diseased, stupid. We have a classic cause and effect fallacy here. People who don't use birth control cause our civilizations problems. Not to mention ``race deterioration''. Drop the mic.

I'm a little constrained by the requirements of this essay, so I couldn't take it exactly where I wanted to. I certainly haven't touched on the good things Sanger stood for. But I think I pointed to the most striking reason that this speech is not effective. From my point of view her entire argument is completely undermined by two things. First she subscribes to that classic American ideal that morality is the same as wealth. That's bad but following that argument out to the natural eugenic conclusion is horrifying. In our post Hitler, post Stephen Hawking world I hope we are moving past ideas like this.
\newpage
\begingroup
%\raggedright
\bibliographystyle{apacite}
\bibliography{speech-bib-critical}
\endgroup
\end{document}
