%shell-escape
\documentclass[letterpaper]{article}

\usepackage{amsmath}
\usepackage[utf8x]{luainputenc}
\usepackage{amssymb}
\usepackage{gnuplottex}
\usepackage{fancyhdr}
\pagestyle{fancy}
%\chead{Real Analysis II Homework}
\rhead{Jon Allen}
\lhead{Real Analysis 2}
\allowdisplaybreaks

\newcommand{\abs}[1]{\left\lvert #1 \right\rvert}

\begin{document}

\renewcommand{\labelenumi}{\Alph{enumi}.}
%\renewcommand{\labelenumii}{\arabic{enumii}.}
\renewcommand{\labelenumii}{(\alph{enumii})}
\subsection*{9.1}
\begin{enumerate}
\setcounter{enumi}{1}
\item
%9.1 B
Show that every subset of a discrete metric space is both open and closed.

We have a discrete metric $d$ on a set $X$. Now we take $U\subset X$. For any $x\in U$ we have $B(x,r)\subset U$ if $r\le 1$ because the ball will contain only the point $x$. Note that this is trivially true even if $U=\emptyset$ because there is no $x\in U$ that does not have a ball around it. Now because our choice of $U$ was arbitrary we know that all subsets of $X$ are open. And the complements of any subsets of $X$ are themselves subsets of $X$, and so they are open. But they are the complement of an open set, and so  they must be closed. Thus every subset of a discrete metric space is both open and closed.
\setcounter{enumi}{3}
\item
%9.1 D
Prove Theorem 9.1.7

Let $f$ map a metric space $(X,\rho)$ into a metric space $(Y,\rho)$. The
following are equivalent:
\renewcommand{\labelenumii}{(\arabic{enumii})}
  \begin{enumerate}
    \item
    $f$ is continuous on $X$;
    \item
    for every sequence $(x_n)$ with $\lim\limits_{n\to\infty}x_n=a\in X$, we
    have $\lim\limits_{n\to\infty}f(x_n)=f(x)$; and
    \item
    $f^{-1}(U)=\{x\in X:f(x)\in U\}$ is open in $X$ for every open set $U$ in $Y$.
  \end{enumerate}

  We start by assuming that $f$ is continuous on $X$. Now we know that for every $\delta>0$ there exists some $N$ such that $\rho(x_n,a)<\delta$ for all $n\ge N$. Thus for every 
\renewcommand{\labelenumii}{(\alph{enumii})}
\setcounter{enumi}{7}
\item
%9.1 H
Two metrics $\rho$ and $\sigma$ on a set $X$ are {\bf equivalent} if there are
constants $0<c<C$ such that $c\rho(x,y)\le\sigma(x,y)\le C\rho(x,y)$ for all
$x,y\in X$
  \begin{enumerate}
    \item
    Prove that equivalent metrics are topologically equivalent
    \item
    Prove that equivalent metrics have the saame Cauchy sequences

    We begin with some Cauchy sequence $(x_n)\in \rho$.
    Then for every $\varepsilon/C>0$ there exists some $N$ such that $\rho(x_i,x_j)<\varepsilon/C$.
    But $\sigma(x_i,x_j)\le C\rho(x_i,x_j)<\varepsilon$ and so the sequence is Cauchy in $\sigma$.
    Now let us assume that our sequence is Cauchy in $\sigma$.
    Then for every $c\varepsilon>0$ there exists some $N$ such that $c\rho(x_i,x_j)\le \sigma(x_i,x_j)<c\varepsilon$ and so certainly $\rho(x_i,x_j)<\varepsilon$.
    \item
    Give examples of topologically equivalent metrics that are not equivalent
  \end{enumerate}
\setcounter{enumi}{10}
\item
%9.1 K
Recall the 2-adic metric of examples 9.1.2 (4) and 9.1.5 (4). Extend it to
$\mathbb{Q}$ by setting $\rho_2(a/b,a/b)=0$ and, if $a/b\ne c/d$, then
$\rho_2(a/b,c/d)=2^{-e}$, where $e$ is the unique integer such that
$a/b-c/d=2^e(f/g)$ and both $f$ and $g$ are odd integers
  \begin{enumerate}
    \item
    Prove that $\rho_2$ is a metric on $\mathbb{Q}$

    if $a/b\ne c/d$ then $a/b-c/d=\frac{ad-cb}{db}$. Now $ad-cb=2^{i}f$ for some odd $f$ and $db=2^{j}g$ for some odd $g$. Then $a/b-c/d=2^{i-j}(f/g)$. Of course $2^{i-j}$ is non-zero and so $\rho_2(a/b,c/d)\ne 0$.

    Now we assume that $a/b-c/b=2^e\frac{f}{g}$. Then $c/d-a/b=2^e(-f/g)$ and so $\rho_2(x,y)=\rho)2(y,x)$.

    And finally, if $\rho_2(a/b,c/d)=2^{-i+l}$, $\rho_2(a/b,e/f)=2^{-k+l}$ and $\rho_2(c/d,e/f)=2^{-j+l}$ then $a/b-c/d=(adf-bcf)/bdf$ and $c/d-e/f=(bcf-bde)/bdf$ while $a/b-e/f=(adf-bde)/bdf=(adf-bcf)/bdf+(bcf-bde)/bdf$. Now we see that $\rho_2(a/b,e/f)=2^{-i-j+l}\le 2^{-i-j+2l}=2^{-i+l}+2^{-j+l}$
    \item
    Show that the sequence of integers $a_n=(1-(-2)^n)/3$ converges in
$(\mathbb{Q},\rho_2)$
    \item
    Find the limit of $\displaystyle \frac{n!}{n!+1}$ in this metric.
  \end{enumerate}
\end{enumerate}
\end{document}
