%shell-escape
\documentclass[letterpaper]{article}

\usepackage{amsmath}
\usepackage[utf8x]{luainputenc}
\usepackage{amssymb}
\usepackage{gnuplottex}
\usepackage{fancyhdr}
\pagestyle{fancy}
\lhead{September 28, 2015}
\chead{Real Analysis II Homework}
\rhead{Jon Allen}
\allowdisplaybreaks

\newcommand{\abs}[1]{\left\lvert #1 \right\rvert}

\begin{document}

\renewcommand{\labelenumi}{\Alph{enumi}.}
%\renewcommand{\labelenumii}{\arabic{enumii}.}
\renewcommand{\labelenumii}{(\alph{enumii})}
\section*{8.3}
%abdeh
\begin{enumerate}
\item
%8.3 A
For $x\in [-1,1]$ let $F(x)=\int_0^1{x(1-x^2y^2)^{-1/2}\mathrm{d}y}$. Show that $F'(x)=(1-x^2)^{-1/2}$ and deduce that $F(x)=\arcsin(x)$.

We let $\displaystyle f(x,y)=\frac{x}{\sqrt{1-x^2y^2}}$.
Then $\displaystyle \frac{\partial}{\partial x}f(x,t)=(1-x^2y^2)^{-1/2}-\frac{1}{2}x(1-x^2y^2)^{-3/2}(-2xy^2)=\frac{1-x^2y^2+x^2y^2}{(1-x^2y^2)^{3/2}}=(1-x^2y^2)^{-3/2}$.
We note that we can find critical points when $x\in \{-1,0,1\}$ and $y\in\{0,1\}$.
As $x\to 0$ and $y\to 0$ we see that $1-x^2y^2\to 1$, and so as we approach zero on the numerator of $f(x,y)$ then the denominator approaches $1$.
As we go away from zero, then the numerator increases, and the denominator decreases, and so $f(x)$ increases.
Similarly, $\frac{\partial}{\partial x}f(x,y)$ is at a minimum when $(x,y)=(0,0)$ and increases as we go away from zero.
Further, as $y\to 1$ and $x^2\to 1$ then $1-x^2y^2\to 0$.
Thus we have a discontinuous point when $y=1$ and $x^2=1$.
This problem occurs in both the function and it's partial derivative.
The exercise is practically begging us to use Leibniz's Rule, but this rule is only defined on continuous functions.
Let us reformulate the question as follows: 
\[F(x)=\lim_{n\to 1}\int_0^n{\frac{x}{\sqrt{1-x^2y^2}}\mathrm{d}y}\]
Now, because $0\le n< 1$ we have a continuous function we can work with and so by Leibniz's Rule we have
\[F'(x)=\lim_{n\to 1}\int_0^n{(1-x^2y^2)^{-3/2}\mathrm{d}y}\]
The trick here is to notice that $-1<xy<1$ and so we can make the substitution $xy=\sin u$ and $\mathrm{d}y=\frac{\cos u}{x}\mathrm{d}u$. Of course $u=\arcsin xy$ and $\arcsin x\cdot0=0$. Let us assign $m=\arcsin xn$
\begin{align*}
  \lim_{n\to 1}\int_0^n{(1-x^2y^2)^{-3/2}\;\mathrm{d}y}
  &=\frac{1}{x}\lim_{n\to 1}\int_0^m{(1-\sin^2u)^{-3/2}cos u\;\mathrm{d}u}\\
  &=\frac{1}{x}\lim_{n\to 1}\int_0^m{\frac{1}{\cos^2 u}\;\mathrm{d}u}\\
  &=\frac{1}{x}\lim_{n\to 1}\int_0^m{\frac{\sin^2u+\cos^2u}{\cos^2 u}\;\mathrm{d}u}\\
  &=\frac{1}{x}\lim_{n\to 1}\left[\int_0^m{\frac{\sin^2u}{\cos^2 u}\;\mathrm{d}u}+\int_0^m{1\;\mathrm{d}u}\right]\\
\end{align*}
And continuing with substitution by parts:
\begin{align*}
  w&=\sin u&\mathrm{d}w&=\cos u\;\mathrm{d}u\\
  \mathrm{d}v&=\frac{\sin u}{\cos^2u}\mathrm{d}u&v&=\frac{1}{\cos u}
\end{align*}
\begin{align*}
  \lim_{n\to 1}\int_0^n{(1-x^2y^2)^{-3/2}\;\mathrm{d}y}
  &=\frac{1}{x}\lim_{n\to 1}\left[\left.\frac{\sin u}{\cos u}\right|_0^m-\int_0^m{\frac{\cos u}{\cos u}\;\mathrm{d}u}+\int_0^m{1\;\mathrm{d}u}\right]\\
  &=\lim_{n\to 1}\left.\frac{\sin u}{x\cos u}\right|_0^m
  =\lim_{n\to 1}\left.\frac{\sin u}{x\sqrt{1-\sin^2u}}\right|_0^m\\
  &=\lim_{n\to 1}\left.\frac{y}{\sqrt{1-x^2y^2}}\right|_0^n\\
  F'(x)
  &=\lim_{n\to 1}\left(\frac{n}{\sqrt{1-x^2n^2}}-\frac{0}{\sqrt{1-x^20^2}}\right)\\
  &=\frac{1}{\sqrt{1-x^2}}\\
\end{align*}
Integrating $F'(x)$ by using the trig substitution of $x=\sin u$ gives us
\begin{align*}
  \int{\frac{1}{\sqrt{1-x^2}}\;\mathrm{d}x}
  &=\int{\frac{1}{\sqrt{1-\sin^2u}}\cos u\;\mathrm{d}u}\\
  &=\int{\frac{\cos u}{\sqrt{\cos^2u}}\;\mathrm{d}u}=\int{\;\mathrm{d}u}\\
  &=u+C=\arcsin x+C
\end{align*}
Thus we have $F(x)=\arcsin x+C$. Substituting in $x=0$ for both versions of $F(x)$ gives us $F(0)=\int_0^1{0\cdot(1-0)^{-1/2}\;\mathrm{d}y}=0=\arcsin 0+C=0+C$. And so we have $F(x)=\arcsin x$
\item
%8.3 B
For $n\ge 1$, define function $f_n$ on $[0,\infty)$ by

\[
f_n(x)
  =\begin{cases}
  e^{-x}&\text{for}\quad0\le x\le n,\\
  e^{-2n}(e^n+n-x)&\text{for}\quad n\le x\le n+e^n,\\
  0&\text{for}\quad x\ge n+e^n,\\
  \end{cases}
\]
  \begin{enumerate}
  \item
  Find the pointwise limit $f$ of $(f_n)$. Show that the convergence is uniform on $[0,\infty)$

  Obviously for any $x$ we know that for all $k\in \mathbb{N}$ where $k>x$ we have $f_k(x)=e^{-x}$ and so our pointwise limit is just $e^{-x}$.

  And just as obvious, if we raise any number greater than zero to any power we must get a number greater than zero. Thus $e^{-x}>0$. And we can immediately conclude that $\frac{\mathrm{d}}{\mathrm{d}x}e^{-x}=-e^{-x}< 0$. Because the derivative is always smaller than zero and the function is greater than zero we have a monotonically decreasing function that is bounded below by $0$. 

  Let us move onto the second piece of our function.
  Note that $e^{-2n}(e^n-n+x)=\frac{1}{e^n}+\frac{n}{e^{2n}}-\frac{x}{e^{2n}}$.
  Notice that this is a linear function of $x$ with a negative slope of $-e^{-2n}$.
  It is clear that $e^{-2n}(e^n-n+x)=e^{-x}$ when $x=n$.
  And when $x=n+e^n$ then we have $e^{-2n}(e^n+n-(n+e^n))=0$.
  And so, this piece of the function is bounded above by $\frac{1}{e^n}$ and below by $0$.

  Now let us choose some $\varepsilon>0$ and some $N\in \mathbb{N}$ such that $e^{-N}<\varepsilon$. We let $k\ge N$. Now if $x\le N$ then $||f_k(x)-f(x)||=0$ and if $x>N$ then we know that $0\le f(x)\le \frac{1}{e^N}$ and $0\le f_k(x)\le \frac{1}{e^N}$. And so $||f_k(x)-f(x)||\le \left\lvert\frac{1}{e^n}-0\right\rvert<\varepsilon$.

  And so we have uniform continuity.
  \item
  Compute $\displaystyle \int_0^\infty{f(x)\mathrm{d}x}$ and $\displaystyle \lim_{n\to\infty}\int_0^\infty{f_n(x)\mathrm{d}x}$.
  \begin{align*}
    \int_0^\infty{f(x)\;\mathrm{d}x}
    &=\lim_{k\to\infty}\int_0^k{e^{-x}\;\mathrm{d}x}\\
    &=\lim_{k\to\infty}-\int_0^k{-e^{-x}\;\mathrm{d}x}\\
    &=\left.\lim_{k\to\infty}-e^{-x}\right\rvert_0^k\\
    &=\lim_{k\to\infty}\left(-e^{-k}+e^{0}\right)=1\\
    \lim_{n\to\infty}\int_0^\infty{f_n(x)\mathrm{d}x}
    &=\lim_{n\to\infty}\left[\int_0^n{e^{-x}\mathrm{d}x}+\int_n^{n+e^n}{e^{-2n}(e^n+n-x)\;\mathrm{d}x}+\int_{n+e^n}^\infty{0\;\mathrm{d}x}\right]\\
    &=\lim_{n\to\infty}\left(\left.-e^{-x}\right\rvert_0^n+e^{-2n}\left[x(e^{n}+n)-\frac{x^2}{2}\right]_n^{n+e^n}+0\right)\\
    &=\lim_{n\to\infty}\left(e^{-2n}\left[(n+e^n)(e^{n}+n)-\frac{(n+e^n)^2}{2}\right.\right.\\&\qquad\qquad\left.\left.-n(e^{n}+n)+\frac{n^2}{2}\right]-e^{-n}+1\right)\\
    &=\lim_{n\to\infty}\left(e^{-2n}\left[n^2+2ne^n+e^{2n}-\frac{n^2}{2}-ne^n-\frac{e^{2n}}{2}\right.\right.\\&\qquad\qquad\left.\left.-ne^{n}-n^2+\frac{n^2}{2}\right]-e^{-n}+1\right)\\
    &=\lim_{n\to\infty}\left(e^{-2n}e^{2n}-e^{-2n}\frac{e^{2n}}{2}-e^{-n}+1\right)\\
    &=\lim_{n\to\infty}\left(1-\frac{1}{2}-e^{-n}+1\right)=\frac{3}{2}
  \end{align*}
  \item Why does this not contradict theorem 8.3.1?

  Because the theorem is for closed intervals. $[0,\infty)$ is not closed.
  \end{enumerate}
\setcounter{enumi}{3}
\item
%8.3 D
Find $\displaystyle \lim_{n\to\infty}\int_0^\pi{\frac{\sin nx}{nx}\mathrm{d}x}$ {\scshape Hint:} Find the limit of the integral over $[\varepsilon,\pi]$ and estimate the rest.
\item
%8.3 E
Define $\displaystyle f(x)=\int_0^\pi{\frac{\sin xt}{t}\mathrm{d}t}$.
  \begin{enumerate}
  \item
  Prove that this integral is defined.
  \item
  Compute $f'(x)$ explicitly.
  \item
  Prove that $f'$ is continuous at 0.
  \end{enumerate}
\setcounter{enumi}{7}
\item
%8.3 H
Suppose that $f\in \mathbb{C}^2[0,1]$ such that $f''(x)+bf'(x)+cf(x)=0, f(0)=0$, and $f'(x)=1$. Let $d(x)$ be continuous on $[0,1]$ and define $g(x)=\int_0^x{f(x-d)d(t)\mathrm{d}t}$. Prove that $g(0)=g'(0)=0$ and $g''(x)+bg'(x)=cg(x)=d(x)$
\end{enumerate}
\end{document}
