\documentclass[letterpaper]{article}

\usepackage[utf8]{inputenc}
\usepackage{fullpage}
\usepackage{nopageno}
\usepackage{amsmath}
\usepackage{amssymb}
\allowdisplaybreaks

\newcommand{\abs}[1]{\left\lvert #1 \right\rvert}

\begin{document}
\title{Notes}
\date{6 fevrier, 2015}
\maketitle
\section*{quiz}
\begin{enumerate}
\item
$\int_{[a,b]}{f\;\mathrm{d}m}=\inf\limits_{\psi\ge f\text{ and simple}}\psi\;\mathrm{d}m$
\item
$\{f_n\}\to F$ pointwise if $f_n(x)=f(x)$ for all $x$
\end{enumerate}
\section*{thm}
if $f$ is bounded and riemann integrable on $[a,b]$ then $\int_{[a,b]}f\;\mathrm{d}m$ exists and $\int_{[a,b]}{f\;\mathrm{d}m}=\int_{a}^b{f(x)\;\mathrm{d}x}$
\subsubsection*{proof}
let $\varepsilon>0$. there is a partition $P=\{x_0,\dots,x_n\}$ such that $U(P,f)-L(P,f)<\varepsilon$. let
\begin{align*}
  \psi(x)&=\sum\limits_{k=1}^n{M_k\chi_{[x_{k-1},x_k]}}\\
  \varphi&=\sum\limits_{k=1}^n{m_k\chi_{[x_{k-1},x_k]}}\\
\end{align*}
and $\varphi(x)\le f(x)\le \psi(x)$ for all $x$.
\begin{align*}
  \int_{[a,b]}{\psi(x)\;\mathrm{d}m}&=\sum\limits_{k=1}^n{M_km*([x_{k-1},x_k}=\sum\limits_{k=1}^n{M_k\Delta_k}=U(P,f)\\
  \int_{[a,b]}{\varphi(x)\;\mathrm{d}m}&=\sum\limits_{k=1}^n{m_km*([x_{k-1},x_k}=\sum\limits_{k=1}^n{m_k\Delta_k}=L(P,f)\\
\end{align*}
and $\inf\int{\theta\;\mathrm{d}m}\le\int_{[a,b]}{\psi\;\mathrm{d}m}=U(P,f)$
and $\sup\int{\tau\;\mathrm{d}m}\ge\int_{[a,b]}{\varphi\;\mathrm{d}m}=U(P,f)$

then $0\le \inf\int\theta-\sup\int\tau\le U(P,f)-L(P,f)<\varepsilon$

let $\varepsilon\to 0$ we let $0\le \inf\int\theta-\sup\int\tau\le 0$ so $\inf=\sup$. and they are all equal to $\int_{a}^b{f(x)\;\mathrm{d}x}$

we note that continuous implies measurable.

\section*{sequences of functions}
motivation: we would love for all functions to be polynomials. not all are though. lets approximate. approximations are not created equally. sometimes can approximate by other ``nice'' functions like trig functions.

$\{f_n\}_{n=1}^\infty$ we say that $f_n\to f$ pointwise if for every $x$ $\{f_n(x)\}_{n=1}^\infty\to f(x)$. (sequence convergence).

\subsection*{examples}
\begin{enumerate}
\item
$f_n:[0,1]\to[0,1]$ and $f_n(x)=x^n$. $\lim\limits_{n\to\infty}x^n=f(x)=\begin{cases}0&x<1\\1&x=1\end{cases}$.

note that $f_n$ is continuous, but $f$ is not.
\item
$f_n(x)=\frac{1}{n}\sin nx$ and $f_n\to0$ but $f_n'=\cos nx$ which diverges for all $x$.
\item
$n\chi_{(0,\frac{1}{n}]}\to 0$. but $\int{n\chi_{(0,\frac{1}{n}]}}=1$

\end{enumerate}
\subsection*{to fix this}
$||f-g||_\infty=\sup\limits_{k\in S}\{|f(x)-g(x)|\}$ which is the furthest apart $f$ and $g$ are.

define {\bfseries uniform convergence} as $\lim_{n\to\infty}||f_n-f||_\infty=0$

$x^n\not\to\lim x^n$ uniformally

\subsection*{fact}
$f_n\to f$ uniformally on $S$ implies $f_n\to f$ pointwise on $S$.
\subsubsection*{proof}
we fix $x\in S$. $|f_n(x)-f(x)|\le \sup\{|f_n(x)-f(x)|\}=||f_n-f||_\infty\to 0$ and so pointwise convergence by  squeeze theorem

\section*{dini's theorem}
$\{f_n\}, f:[a,b]\to\mathbb{R}$ (it's compact) with $f_n\le f_{n+1}$ for all $n$ and $f_n\to f$ pointwise, then $f_n\to f$ uniformally.
\subsubsection*{proof}
$g_n=f-f_n$ and then $g_n\le g_{n+1}$ and $0\le g_n\le g_1$ for all $n$. and $g_n\to0$ pointwise. if $g_n$ converges to 0 uniformally then $f_n$ converges to $f$ uniformally.

$||g_n-0||_\infty=||f-f_n||\infty$

to be continued
\end{document}
