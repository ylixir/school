\documentclass[letterpaper]{article}

\usepackage[utf8]{luainputenc}
\usepackage{fullpage}
\usepackage{nopageno}
\usepackage{amsmath}
\usepackage{amssymb}
\allowdisplaybreaks

\newcommand{\abs}[1]{\left\lvert #1 \right\rvert}

\begin{document}
\title{Notes}
\date{4 mars, 2015}
\maketitle
\begin{align*}
  \log x\\
  \sum\limits_{k=1}^\infty{\frac{(-1)^{k+1}(x-1)^k}{k}}
\end{align*}
radius of convergence is 1 and interval of convergence is $(0,2]$ from last time


in general the $n$th derivative looks like $f^{(n)}=\frac{(-1)^{n+1}(n-1)!}{x^n}$. 

we restrict our $[a,b]$ such that $0<a\le b$.

now $|f^{(n_1)}(x)|=|f^{(n+1)}(a)|=\frac{(n)!}{a^n}$

now with taylors thm
$|R_n(x)|\le \frac{n!}{a^n}(\frac{|x-1|^n}{(n+1)!}$

take the limit of n to inf and
\begin{align*}
  \lim |R_n|&=\lim \frac{|x-1|^n}{a^nn}\to 0
\end{align*}

and so $\log x=\sum\limits_{k=1}^\infty{\frac{(-1)^{k+1}}{k!}(x-1)^k}$

\subsection*{example}
$f(x)=\begin{cases}e^{1/x^2}&x\ne 0\\0&x=0\end{cases}$

\begin{align*}
  f^{(n)}(x)&=\frac{q_n(x)}{x^{3n}}e^{-1/x^2}&x&\ne 0
\end{align*}
with $q_n(x)$ a polynomial of degree less than or equal to 2n
first derivative at zero is zero because $\lim_{x\to 0}\frac{f(x)-0}{x-0}=\frac{e^{-1/x^2}}{x}$ which is zero with l'hopitals rule. same with inductive step fr $n$th degree. but then the power series at 0 converges to $P_n(x)=0$!


\section*{can we approximate a continuous function by a polynomial?}

yes, but....

not usually with power series.

\section*{thm}
if $f$ is continuous on $[a,b]$ then for any $\epsilon>0$ there is a polynomial $p_\epsilon$ such that $||f-p_\epsilon||_\infty<\epsilon$

the book does three proofs of this. bernstein, chebychev, and a general proof. stone-weierstrass thm.

\subsection*{corollary}
if $f$ is continuous on $[0,1]$ and $\int_0^1f(x)x^n\;\mathrm{d}x$ for all $n$ theen $f(x)=0$.

\subsubsection*{proof}
$\int_0^1{|f(x)|^2\;\mathrm{d}x}=\lim\int_0^1f(x)p_n(x)\;\mathrm{d}x$ with $p_n(x)$ being the polynomial approximation from weierstrass approximation theorem.

\begin{align*}
  \lim a_n\int f(x)x^n+a_{n-1}\int\dots+a_0\int f(x)x^0\;\mathrm{d}x=0
\end{align*}
\end{document}

