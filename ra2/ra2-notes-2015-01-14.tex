\documentclass[letterpaper]{article}

%\usepackage{fullpage}
%\usepackage{nopageno}
\usepackage{amsmath}
\usepackage{amssymb}
\allowdisplaybreaks

\newcommand{\abs}[1]{\left\lvert #1 \right\rvert}

\begin{document}
\title{Notes}
\date{January 14, 2015}
\maketitle
\section*{homework:}
no scruffy edges

name: top right

state the problem

solution, grammar, words, not just notation

repeat

finally: list resources, exclude book and this instructor. include other instructors, classmates, internet, etc.

\section*{course outline}
chapter 8, 10, 11, 16, not exhaustively

skipped things will be a good source for projects

first couple weeks will not be in the book, we cover real life integration in the small, then start chapter 8.

measure theory  is how to do integrals, they are done by approximating functions

then polynomial approximation. these are standard analysis

last bit is fun stuff of his choice

\section*{review integration}
$f:[a,b]\to \mathbb{R}$ we partition etc, randomly choose a point in the partition, sum rectangles

integrable if we can always find a partition and get arbitrarily close

\subsection*{lebesgue metaphor}
take a pile of money, coins of different values. to calculate money, you add each coin up. think of each coin as one of the riemann rectangles.

now for the lebesgue integral you sort the change, add the things of the same size together first, then add these sums.

\subsection*{riemann to lebesgue}
reimann is chopping up  domain and measuring the height,
lebesgue is chopping up range and, complication is that ``rectangles'' are a little weirder, the function goes in and out of the rectangle. or multiple riemann rectangles for each rectangle height. to make this work, we need to be able to measure the size of a set

\section*{measure theory}
\subsection*{measure the size of a set}
\begin{enumerate}
\item
what properties do we want? disjoint etc?
\item
what measures? not going here, start with the classic type of measure
\end{enumerate}
\subsection*{Lindelöf's theorem}
if $E\in \mathbb{R}$ there is a countable collection of intervals $\{(a_i,b_i)\}_{i=1}^\infty$ such thhat $E\in \bigcup_{i=1}^\infty(a_i,b_i)$

this is obvious because of $\mathbb{R}\subseteq\mathbb{R}$ check subcovers, etc

these sets
\begin{enumerate}
\item
not unique
\item
may overlap
\end{enumerate}

\subsubsection*{example}
$[0,1]\subseteq\bigcap_{n=1}^\infty(-\frac{1}{n},1+\frac{1}{n})$

$[0,1]\subseteq\bigcap_{n=1}^\infty(-n,n)$

$[0,1]\subseteq(\frac{1}{2},\frac{3}{2})\cup(\frac{1}{4},\frac{3}{4})\cup(\frac{1}{8},\frac{3}{8})\cup\dots(-\frac{1}{2},\frac{1}{2})\cup$

hello goodbye
\end{document}
