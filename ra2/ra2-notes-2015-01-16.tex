\documentclass[letterpaper]{article}

%\usepackage{fullpage}
%\usepackage{nopageno}
\usepackage{amsmath}
\usepackage{amssymb}
\allowdisplaybreaks

\newcommand{\abs}[1]{\left\lvert #1 \right\rvert}

\begin{document}
\title{Notes}
\date{January 16, 2015}
\maketitle
irrationals $\subseteq\bigcup_{i=1}^\infty(-i,i)=\mathbb{R}$

$\{0\}\subseteq$

\section*{outer measure of a sum}
$E\subseteq\mathbb{R}$ then $m*(E)=\inf\{\sum\limits_{i=1}^\infty{b_i=a_i}:E\subseteq\bigcup_{i=1}^{\infty}(a_i,b_i)\}$

\subsection*{example}
$m*(\{x\})=0$

$\{x\}\subseteq(x-\epsilon,x+\varepsilon$ for any $\varepsilon>0$, $m*(t)\le 2\varepsilon$ for any $\epsilon>0\to 0$

\begin{enumerate}
\item
note that $m*(E)\subseteq[0,\infty]$, no negative 
\item
the empty set has measure zero

$m*(\emptyset)=0$

$\emptyset\subseteq\bigcup_{i=1}^\infty(a_i,b_i)$

\item
$A\subseteq B\Rightarrow m*(A)\le m*(B)$
\subsubsection*{proof}
if $B\subseteq\bigcup_{i=1}^\infty(a_i,b_i)$ then $A\subseteq\bigcup_{i=1}^\infty(a_i,b_i)$

$\inf\{\sum b_i-a_i | A\subseteq\bigcup(a_i,b_i\}\le \inf\{\sum c_i-d_i |B\subseteq \bigcup(c_i,d_i\}$

cant do this with contradiction, that's the ``actual way''
\subsubsection*{example}
$m*([a,b])=b-a$?

$\inf\{\sum(b_i-a_i):[a,b]\subseteq\bigcup(a_i,b_i\}$

{\bfseries proof}

\begin{align*}
  [a,b]&\subseteq(a-\epsilon,b+\epsilon)\text{ for all }\epsilon>0\\
  m*([a,b])&\subseteq(b+\epsilon)-(a-\epsilon)\text{ for all }\epsilon>0\\
  &\subseteq(b-a)+2\epsilon
\end{align*}

so $m*([a,b])\subseteq b-a$

heine-borel theorem (HBT) compact set is closed and bounded

compact:every open cover has finite subcovers? check this

if $[a,b]\subseteq\bigcup(a_i,b_i)$

wlog $[a,b]\subseteq\bigcup^n(a_i,b_i)$

\begin{align*}
  a_1&<a\\
  a_2&<b_1\\
  a_3&<b_2\\
  a_4&<b_3\\
  &\vdots
  a_n&<b_{n-1}\\
  b&<b_n\\
\end{align*}
these are overlapping covers, they make a sequence, it is important that they overlap

sum to infinity is bigger than sum to $n$. 
$\sum(b_1-a_i)+(b_2-a_2)+\dots+(b_n-a_n)\ge(a_2-a_1)+(a_3-a_2)\dots(a_n-a_n-1)+(b_n-a_n)=b_n-a_1\ge b-a$

and so $m*([a,b])\ge b-a$ and we already have less than so it's equal
\subsubsection*{example 2}
$m*((a,b))=b-a$ second part of limhof? thm, uniqueness

notice that the different sets are the same size
\subsubsection*{example 3}
$m*([a,\infty))=\infty$ notice that the measure of any subset is greater than or equal and so $[a,a+k]\subseteq[a,\infty)$ for all $k$ and so $k=m*[a,a+k]\le m*[a,\infty)\forall k$

\subsubsection*{example 4,5}
$m*\mathbb{Q},m*\mathbb{C}-\mathbb{R}$
\item
$m*(\bigcup_{i=1}^\infty A_i)\le\sum\limits_{i=1}^\infty{m*(A_i)}$

{\bfseries proof}

let $\epsilon>0$ then for all $i$ there is $\{(a_i^j,b_i^j\}_{i=1}^\infty$ such that $m*A_i\le\sum\limits{b_i^j-a_i^j}\le m*A_j+\frac{\epsilon}{2^j}$

notice that $\{\{a_i^j,b_i^j\}_{i=1}^\infty\}_{j=1}^\infty$ is countable collection of intervals with $\bigcup A_j\subseteq \bigcup_{j=1}(\bigcup_{i=1}(a_i^j,b_i^j))$
$m* A_j\le \sum_{j=1}(\sum_{i=1}(a_i^j,b_i^j))$

this is called countable subadditivity
\end{enumerate}

\end{document}
