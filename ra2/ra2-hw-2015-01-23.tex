%shell-escape
\documentclass[letterpaper]{article}

\usepackage{amsmath}
\usepackage{amssymb}
\usepackage{gnuplottex}
\usepackage{fancyhdr}
\pagestyle{fancy}
\lhead{January 23, 2015}
\chead{Real Analysis II Homework}
\rhead{Jon Allen}
\allowdisplaybreaks

\newcommand{\abs}[1]{\left\lvert #1 \right\rvert}

\begin{document}
\begin{enumerate}
\item
Prove that if $A\subseteq \mathbb{R}$ and for $A\in \mathbb{R}$ we have $A+\lambda=\{a+\lambda:a\in A\}$ then $m*(A)=m*(A+\lambda)$
\subsubsection*{proof}
First we notice that because
$A\subseteq \mathbb{R}$
and $\lambda\in \mathbb{R}$
then $A+\lambda\in \mathbb{R}$.
Now using the definition of the outer measure we have
\[m*(A+\lambda)=\inf\left\{\sum\limits_{i=1}^\infty{b_i-a_i:A+\lambda\subseteq\bigcap\limits_{i=1}^\infty(a_i,b_i)}\right\}\]
Now we can rewrite this a little bit based on the definition of $A+\lambda$ to get
\[m*(A+\lambda)=\inf\left\{\sum\limits_{i=1}^\infty{(b_i+\lambda)-(a_i+\lambda):A\subseteq\bigcap\limits_{i=1}^\infty(a_i,b_i)}\right\}\]
Our lambdas cancel, so we are just left with
\[m*(A+\lambda)=\inf\left\{\sum\limits_{i=1}^\infty{b_i-a_i:A\subseteq\bigcap\limits_{i=1}^\infty(a_i,b_i)}\right\}\]
But $m*(A)$ is defined to be 
\[m*(A)=\inf\left\{\sum\limits_{i=1}^\infty{b_i-a_i:A\subseteq\bigcap\limits_{i=1}^\infty(a_i,b_i)}\right\}\]
And so we have $m*(A)=m*(A+\lambda)$ as desired.
$\Box$
\item
Prove that if $m*(A)=0$ then $m*(A\cup B)=m*(B)$ for any set $B\subseteq\mathbb{R}$
\subsubsection*{proof}
In class we showed that $m*\left(\bigcup\limits_{i=1}^\infty A_i\right)\le  \sum\limits_{i=1}^\infty{m*(A_i)}$ which means that $m*(A\cup B)\le m*(A)+m*(B)=0+m*(B)=m*(B)$.
Obviously $B\subseteq A\cup B$ and so we know from lecture that $m*(B)\le m*(A\cup B)$. Put them together and we have $m*(B)\le m*(A\cup B)\le m*(B)$ which means $m*(B)=m*(A\cup B)\Box$
\end{enumerate}
\end{document}
