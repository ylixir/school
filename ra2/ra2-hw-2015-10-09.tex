%shell-escape
\documentclass[letterpaper]{article}

\usepackage{amsmath}
\usepackage[utf8x]{luainputenc}
\usepackage{amssymb}
\usepackage{gnuplottex}
\usepackage{fancyhdr}
\pagestyle{fancy}
\chead{Real Analysis II Homework}
\rhead{Jon Allen}
\allowdisplaybreaks

\newcommand{\abs}[1]{\left\lvert #1 \right\rvert}

\begin{document}

\renewcommand{\labelenumi}{\Alph{enumi}.}
%\renewcommand{\labelenumii}{\arabic{enumii}.}
\renewcommand{\labelenumii}{(\alph{enumii})}
\section*{8.4}
%defhj
\begin{enumerate}
\setcounter{enumi}{3}
\item
%8.4 D
Does $\sum\limits_{n=1}^\infty{\frac{1}{x^2+n^2}}$ converge uniformly on the whole real line?

We know $0\le\sum\limits_{n=1}^\infty{\frac{1}{x^2+n^2}} <\sum\limits_{n=1}^\infty{\frac{1}{n^2}}$.
Because $\frac{1}{n^2}$ is convergent, then $\frac{1}{x^2+n^2}$ must also be convergent.

This also gives us uniform convergence, because  for every $\varepsilon>0$ there exists an $N$ such that $0\le||\sum\limits_{i=k+1}^l{\frac{1}{x^2+i^2}}\le ||\sum\limits_{i=k+1}^l{\frac{1}{i^2}}\le \varepsilon$ for every $l>k\ge N$ regardless of our choice of $x$.
\item
%8.4 E
Show that if $\sum\limits_{n=1}^\infty{\lvert a_n\rvert}<\infty$, then $\sum\limits_{n=1}^\infty{a_n\cos nx}$ converges uniformly on $\mathbb{R}$.

Because $0\le |\cos nx|\le 1$ then $|a_n\cos nx|\le |a_n|$. Now we know that $|a_n|$ converges and so then for any $\varepsilon>0$ there exists an $N$ such that $\sum\limits_{i=k+1}^l{|a_n|}<\varepsilon$ for any $l>k\ge N$. But $\sum\limits_{i=k+1}^l{|a_n\cos nx|}\le\sum\limits_{i=k+1}^l{|a_n|}<\varepsilon$ regardless of our choice of $x$. And since $|a_n\cos nx|$ converges uniformly, then we get $a_n\cos nx$ converging uniformly for free.
\item
%8.4 F
  \begin{enumerate}
  \item
  Let $f_n(x)=\frac{x^2}{(1+x^2)^n}$ for $x\in \mathbb{R}$. Evaluate the sum $S(x)=\sum\limits_{n=0}^\infty{f_n(x)}$.

  At $x=0$ the sum is $0$. At all other values we have a geometric series which converges to $\frac{x^2}{1-(\frac{1}{1+x^2})}=\frac{x^2}{\frac{x^2}{1+x^2}}=1+x^2$
  \item
  Is this convergence uniform? For which values $a<b$ does this series converge uniformly on $[a,b]$?

  The convergence is not uniform. Our series converges to a discontinuous function ($0<1<1+x^2$), and so it is not uniformly continuous, by theorem 8.4.4.

  We take the derivative 
  \begin{align*}
    \frac{\partial }{\partial x}\frac{x^2}{(1+x^2)^n}
    &=2x(1+x^2)^{-n}-nx^2(1+x^2)^{-n-1}2x\\
    &=\frac{2x(1+x^2)}{(1+x^2)(1+x^2)^{n}}-\frac{2nx^3}{(1+x^2)(1+x^2)^{n}}\\
    &=\frac{2x(1+x^2-nx^2)}{(1+x^2)(1+x^2)^{n}}\\
    &=\frac{2x(1+x^2(1-n))}{(1+x^2)(1+x^2)^{n}}\\
  \end{align*}
  So the denominator of our derivative has no zeros, and our numerator has zeros at $x=0$ at $x=\pm\frac{1}{\sqrt{n-1}}$. Zero is obviously a minimum because the function has no negative terms. And $\frac{1}{\sqrt{n-1}}$ is less than $1$ for all $n>2$. So if comparing $x=\frac{1}{\sqrt{n-1}}$ and $x=1$ when $n=3$ we see that
  \begin{align*}
    \frac{\frac{1}{n-1}}{(1+\frac{1}{n-1})^3}&?\frac{1}{(1+1)^3}\\
    \frac{\frac{1}{n-1}}{(\frac{n}{n-1})^3}&?\frac{1}{2^3}\\
    \frac{1}{n-1}\left(\frac{n-1}{n}\right)^3&?\frac{1}{8}\\
    \frac{(n-1)^2}{n^3}&?\frac{1}{8}\\
    \frac{2^2}{3^3}&?\frac{1}{8}\\
    0.\overline{148}&>.125\\
  \end{align*}
  And so $\frac{1}{\sqrt{n-1}}$ is a maximum. Observe that
  \begin{align*}
    \frac{\frac{1}{n-1}}{\left(1+\frac{1}{n-1}\right)^n}&=\frac{1}{n-1}\left(\frac{n-1}{n}\right)^n=\frac{(n-1)^{n-1}}{n^n}
  \end{align*}
  We have a higher degree on the bottom, so this will converge to zero. And so we have uniform convergence on $[a,\infty)$ for all $a>0$. And of course $(-\infty,-a]$ or any subinterval of these.
  \end{enumerate}
\setcounter{enumi}{7}
\item
%8.4 H
Suppose that $a_k(x)$ are continuous functions on $[0,1]$, and define $s_n(x)=\sum\limits_{k=1}^n{a_k(x)}$. Show that if $(s_n)$ converges uniformly on $[0,1]$, then $(a_n)$ converges uniformly to $0$.

If we assume that $(a_n)$ does not converge uniformly to 0.
We know that $(a_n)$ converges to zero at least pointwise, else $(s_n)$ would not converge for some $x$. And so we assume that $(a_n)$ converges but not uniformly. Now $(s_n)$ must be uniformly Cauchy and so given any $\varepsilon>0$ there exists some $N$ large enough that $||\sum\limits_{i=k+1}^l{a_i(x)}||_\infty\le \varepsilon$ for all $l>k\ge N$ We take $l=k+1$ and obtain $||a_l(x)||_\infty\le \varepsilon\forall l\ge N$. But we are assuming that $(a_n)$ does not converge uniformly. Therefore $\lim\limits_{k\to\infty}||a_k||_\infty=L$ for some $L>0$. If we choose $\varepsilon=\frac{L}{2}$, then we have $||a_l||_\infty > \varepsilon$ for some $N$ and all $l>N$. Thus we have a contradiction, and $(a_n)$ must converge uniformly.
\setcounter{enumi}{9}
\item
%8.4 J
Let $(f_n)$ be a sequence of functions defined on $\mathbb{N}$ such that $\lim\limits_{k\to\infty}f_n(k)=L_n$ exists for each $n\ge 0$. Suppose that $||f_n||_\infty\le M_n$, where $\sum\limits_{n=0}^\infty{M_n}<\infty$. Define a function $F(k)=\sum\limits_{n=0}^\infty{f_n(k)}$. Prove that $\lim\limits_{k\to\infty}F(k)=\sum\limits_{n=0}^\infty{L_n}$.
{\scshape Hint:} Think of $f_n$ as a function $g_n$ on $\{\frac{1}{k}:k\ge 1\}\cup{0}$. How will you define $g_n(0)$?

We define $g_n(x)=f_n(\frac{1}{x})$. Because $\lim\limits_{k\to\infty}\frac{1}{k}=0$ and $\lim\limits_{k\to\infty}f_n(k)=L_n$ then it makes sense to define $g_n(0)=L_n$. Further, the Weierstrass M-Test tells us that the series converges uniformly and so $G(x)$ is continuous. Thus $g_n(x)$ is defined for all $x\in[0,1]$.

Now we define $G(x)=\sum\limits_{n=0}^\infty{g_n(x)}$ and $\lim\limits_{k\to\infty}F(k)=G(0)=\sum\limits_{n=0}^\infty{L_n}$
\end{enumerate}
\section*{8.5}
\begin{enumerate}
\item
Determine the interval of convergence of the following power series:
  \begin{enumerate}
  \item
  $\sum\limits_{n=0}^\infty{n^3x^n}$ We have $\lim\limits_{n\to\infty}|\frac{(n+1)^3}{n^3}|=1$. Obviously $\sum\limits_{n=0}^\infty{n^3}$ and $\sum\limits_{n=0}^\infty{(-1)^nn^3}$ diverge, and so our interval of convergence is $(-1,1)$
  \item
  $\sum\limits_{n=1}^\infty{\frac{(-1)^n}{n^2}x^n}$. So $\lim\limits_{n\to\infty}\left\lvert\frac{(-1)^{n+1}n^2}{(-1)^n(n+1)^2}\right\rvert=1$. But $\sum\limits{(-1)^n/n^2}$ and $\sum\limits_{n=0}^\infty{1/n^2}$ both converge, and so our interval of convergence is $[-1,1]$.
  \item
  $\sum\limits_{n=0}^\infty{\frac{n^2}{2^n}x^n}$

  The limit of $\left\lvert\frac{(n+1)^22^n}{2^{n+1}n^2}\right\rvert$ as $n\to\infty$ is $1/2$.
  Now $\sum\limits_{n=0}^\infty{\frac{n^2}{2^n}(\pm2)^n}=\sum\limits_{n=0}^\infty{(-1)^nn^2}\text{ or }\sum\limits_{n=0}^\infty{n^2}$ and $\lim\limits_{n\to\infty}$ which obvious diverge, so our interval of convergence is $(-2,2)$
  \item
  $\sum\limits_{n=0}^\infty{\sqrt{n}x^n}$.
  Now $\lim\limits_{n\to\infty}\frac{\sqrt{n+1}}{\sqrt{n}}=\sqrt{1}+\lim\limits_{n\to\infty}\frac{1}{\sqrt{n}}=1$. Of course $\sum\limits_{n=0}^\infty{\sqrt{n}}$ diverges along with $\sum\limits_{n=0}^\infty{(-1)^n\sqrt{n}}$ and so our interval is $(-1,1)$
  \item
  $\sum\limits_{n=0}^\infty{(-1)^n\frac{x^{2n}}{(2n)!}}$.
So we define $a_{2k+1}=0$ and $a_{2k}=(-1)^k/(2k)!$. And $\lim\limits_{n\to\infty}\left\lvert\frac{(2k)!}{(2k+2)!}\right\rvert=\lim\limits_{n\to\infty}\frac{1}{(2k+2)(2k+1)}=0$. And so our sum only for any  interval in $\mathbb{R}$.
  \item
  $\sum\limits_{n=0}^\infty{x^{n!}}$.

  Let us say $g(x)=x^{n!}$ and $g'(x)=n!x^{n!-1}$. Then $xg'(x)=n!g(x)$ and $g(0)=0$. Now we assume there is a power series $f(x)=\sum\limits_{n=0}^\infty{a_nx^n}$ that satisfies this DE. Then $x\sum\limits_{n=1}^\infty{na_nx^{n-1}}=n!\sum\limits_{n=0}^\infty{a_nx^n}$.
  \item
  $\sum\limits_{n=0}^\infty{\frac{n!}{n^n}x^n}$
  Now $\lim\limits_{n\to\infty}\frac{(n+1)!n^n}{n!(n+1)^{n+1}}=\lim\limits_{n\to\infty}\frac{(n+1)n^n}{(n+1)(n+1)^n}=\lim\limits_{n\to\infty}\left(\frac{n+1}{n}\right)^{-n}=\frac{1}{e}$. And $\lim\left\lvert(-1)^n\frac{n!}{e/n}^n\right\rvert\ne 0$ because factorials grow faster than exponentials. And so our interval it $(-e,e)$
  \item
  $\sum\limits_{n=0}^\infty{\frac{(n!)^2}{(2n)!}x^n}$

  $\lim\frac{((n+1)!)^2(2n)!}{(2n+2)!(n!)^2}=\lim\frac{(n+1)^2}{(2n+2)(2n+1)}=\frac{1}{4}$. And $\sum\frac{(n!)^2}{(2n)!}4^n=\sum\frac{n!4^n}{(n+1)\dots(n+n)}$. The computer claims this diverges, and so our interval of convergence is $(-4,4)$
  \item
  $\sum\limits_{n=0}^\infty{\frac{1}{n}x^n}$
  
  $\lim\frac{n}{n+1}=1$ but  $\frac{1}{n}$ diverges while $\frac{(-1)^n}{n}$ converges, so our interval is $[-1,1)$
  \end{enumerate}
\item
\end{enumerate}
\end{document}
