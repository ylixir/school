%shell-escape
\documentclass[letterpaper]{article}

\usepackage{amsmath}
\usepackage[utf8x]{luainputenc}
\usepackage{amssymb}
\usepackage{gnuplottex}
\usepackage{fancyhdr}
\pagestyle{fancy}
\lhead{October 9, 2015}
\chead{Real Analysis II Homework}
\rhead{Jon Allen}
\allowdisplaybreaks

\newcommand{\abs}[1]{\left\lvert #1 \right\rvert}

\begin{document}

\renewcommand{\labelenumi}{\Alph{enumi}.}
%\renewcommand{\labelenumii}{\arabic{enumii}.}
\renewcommand{\labelenumii}{(\alph{enumii})}
\section*{8.4}
%defhj
\begin{enumerate}
\setcounter{enumi}{3}
\item
%8.4 D
Does $\sum\limits_{n=1}^\infty{\frac{1}{x^2+n^2}}$ converge uniformly on the whole real line?

We know $0\le\sum\limits_{n=1}^\infty{\frac{1}{x^2+n^2}} <\sum\limits_{n=1}^\infty{\frac{1}{n^2}}$.
Because $\frac{1}{n^2}$ is convergent, then $\frac{1}{x^2+n^2}$ must also be convergent.

This also gives us uniform convergence, because  for every $\varepsilon>0$ there exists an $N$ such that $0\le||\sum\limits_{i=k+1}^l{\frac{1}{x^2+i^2}}\le ||\sum\limits_{i=k+1}^l{\frac{1}{i^2}}\le \varepsilon$ for every $l>k\ge N$ regardless of our choice of $x$.
\item
%8.4 E
Show that if $\sum\limits_{n=1}^\infty{\lvert a_n\rvert}<\infty$, then $\sum\limits_{n=1}^\infty{a_n\cos nx}$ converges uniformly on $\mathbb{R}$.

Because $0\le |\cos nx|\le 1$ then $|a_n\cos nx|\le |a_n|$. Now we know that $|a_n|$ converges and so then for any $\varepsilon>0$ there exists an $N$ such that $\sum\limits_{i=k+1}^l{|a_n|}<\varepsilon$ for any $l>k\ge N$. But $\sum\limits_{i=k+1}^l{|a_n\cos nx|}\le\sum\limits_{i=k+1}^l{|a_n|}<\varepsilon$ regardless of our choice of $x$. And since $|a_n\cos nx|$ converges uniformly, then we get $a_n\cos nx$ converging uniformly for free.
\item
%8.4 F
  \begin{enumerate}
  \item
  Let $f_n(x)=\frac{x^2}{(1+x^2)^n}$ for $x\in \mathbb{R}$. Evaluate the sum $S(x)=\sum\limits_{n=0}^\infty{f_n(x)}$.

  At $x=0$ the sum is $0$. At all other values we have a geometric series which converges to $\frac{x^2}{1-(\frac{1}{1+x^2})}=\frac{x^2}{\frac{x^2}{1+x^2}}=1+x^2$
  \item
  Is this convergence uniform? For which values $a<b$ does this series converge uniformly on $[a,b]$?

  The convergence is not uniform. Our series converges to a discontinuous function ($0<1+x^2$), and so it is not uniformly continuous, by theorem 8.4.4.

  We take the derivative 
  \begin{align*}
    \frac{\partial }{\partial x}\frac{x^2}{(1+x^2)^n}
    &=2x(1+x^2)^{-n}-nx^2(1+x^2)^{-n-1}2x\\
    &=\frac{2x(1+x^2)}{(1+x^2)(1+x^2)^{n}}-\frac{2nx^3}{(1+x^2)(1+x^2)^{n}}\\
    &=\frac{2x(1+x^2-nx^2)}{(1+x^2)(1+x^2)^{n}}\\
    &=\frac{2x(1+x^2(1-n))}{(1+x^2)(1+x^2)^{n}}\\
  \end{align*}
  So the denominator of our derivative has no zeros, and our numerator has zeros at $x=0$ at $x=\pm\frac{1}{\sqrt{n-1}}$. Zero is obviously a minimum because the function has no negative terms. And $\frac{1}{\sqrt{n-1}}$ is less than $1$ for all $n>2$. So if comparing $x=\frac{1}{\sqrt{n-1}}$ and $x=1$ when $n=3$ we see that
  \begin{align*}
    \frac{\frac{1}{n-1}}{(1+\frac{1}{n-1})^3}&?\frac{1}{(1+1)^3}\\
    \frac{\frac{1}{n-1}}{(\frac{n}{n-1})^3}&?\frac{1}{2^3}\\
    \frac{1}{n-1}\left(\frac{n-1}{n}\right)^3&?\frac{1}{8}\\
    \frac{(n-1)^2}{n^3}&?\frac{1}{8}\\
    \frac{2^2}{3^3}&?\frac{1}{8}\\
    0.\overline{148}&>.125\\
  \end{align*}
  And so $\frac{1}{\sqrt{n-1}}$ is a maximum. Observe that
  \begin{align*}
    \frac{\frac{1}{n-1}}{\left(1+\frac{1}{n-1}\right)^n}&=\frac{1}{n-1}\left(\frac{n-1}{n}\right)^n=\frac{(n-1)^{n-1}}{n^n}
  \end{align*}
  We have a higher degree on the bottom, so this will converge to zero. And so we have uniform convergence on $[a,\infty)$ for all $a>0$. And of course $(-\infty,-a]$ or any subinterval of these.
  \end{enumerate}
\setcounter{enumi}{7}
\item
%8.4 H
Suppose that $a_k(x)$ are continuous functions on $[0,1]$, and define $s_n(x)=\sum\limits_{k=1}^n{a_k(x)}$. Show that if $(s_n)$ converges uniformly on $[0,1]$, then $(a_n)$ converges uniformly to $0$.

If we assume that $(a_n)$ does not converge uniformly to 0. Then either it does not converge at all or it converges to some non-zero value for some $x$. Let us assume that $\lim\limits_{n\to\infty}a_n(b)=c$ for some $c\ne 0$. Then $\lim\limits_{n\to\infty}s_n(b)=\infty$
\setcounter{enumi}{9}
\item
%8.4 J
Let $(f_n)$ be a sequence of functions defined on $\mathbb{N}$ such that $\lim\limits_{k\to\infty}f_n(k)=L_n$ exists for each $n\ge 0$. Suppose that $||f_n||_\infty\le M_n$, where $\sum\limits_{n=0}^\infty{M_n}<\infty$. Define a function $F(k)=\sum\limits_{n=0}^\infty{f_n(k)}$. Prove that $\lim\limits_{k\to\infty}F(k)=\sum\limits_{n=0}^\infty{L_n}$.
{\scshape Hint:} Think of $f_n$ as a function $g_n$ on $\{\frac{1}{k}:k\ge 1\}\cup{0}$. How will you define $g_n(0)$?
\end{enumerate}
\section*{8.5}
\begin{enumerate}
\item
\item
\end{enumerate}
\end{document}
