\documentclass[letterpaper]{article}

\usepackage[utf8]{inputenc}
\usepackage{fullpage}
\usepackage{nopageno}
\usepackage{amsmath}
\usepackage{amssymb}
\allowdisplaybreaks

\newcommand{\abs}[1]{\left\lvert #1 \right\rvert}

\begin{document}
\title{Notes}
\date{2 fevrier, 2015}
\maketitle
\section*{quiz}
for a simple function, the range needs to be finite and the inverse image needs to be measurable for all ranges

$\chi_E$ continuous except on $\partial E=E^C\setminus E^\circ$

$E$ is nonmeasurable if $\chi_E$ is not simple.


\section*{les notes}
if $\varphi$ is simple then r a n $\varphi=\{\alpha_1,\dots,\alpha_n\}$, $E_i=\varphi^{-1}(\{\alpha_i\})$ then $\varphi=\sum\limits_{i=1}^n{\alpha_i\chi_{E_i}}$

example
$\sum\limits_{n=1}^5{\chi_{[n,n+1]}}$

one on $[1,6]$ everywhere but $\{2,3,4,5\}$ where it is 2.

this is not in canonical form. rewritten in cannonical form is

$E_1=(-\infty,1)\cup(6,\infty)$, $E_2=[1,2)\cup(2,3)\cup(3,4)\cup(4,5)\cup(5,6]$, $E_3=\{2,3,4,5\}$

$\varphi=0\chi_{E_1}+1\chi_{E_2}+2\chi_{E_3}$

now $\sum\limits_{i=1}^n{\beta_i\chi_{E_i}}$ and $B_1=E_1\cap\dots\cap E_n$ to $(\sum\limits_{i=1}^n{\beta_1})\chi B_1+\dots$ and on with every single set getting thrown away and on with every possible combination of two sets getting thrown away and 3 and so on

$E_1,E_2,E_3$ look at $B_1=E_1\cap E_2\cap E_3$, $B_2=E_2\cap E_3)\setminus (E_1\cap E_2\cap E_3)$, $B_3=E_1\cap E_3)\setminus (E_1\cap E_2\cap E_3)$, $B_4=E_1\cap E_2)\setminus (E_1\cap E_2\cap E_3)$, and so on

note that $B_i\cap B_j=\emptyset$

$\varphi=\sum\limits_{i=1}^n{\alpha_i\chi_{E_i}}$

\section*{definition of integral}
only applies to simple functions
$\sum\limits{\varphi\mathrm{d}m}\sum\limits_{i=1}^n{\alpha_i m*(E_i)}$

now $\int{\chi_{[0,\infty]}\mathrm{d}m}=0m*(-\infty,0)+m*[0,\infty]$

tweak:
any function $f$ is $0$ outside some bounded interval $E$. $\int_E{f\mathrm{d}m}$

\subsection*{propositions}
if $\varphi,\psi$ are simple then $\int_E{(\alpha\varphi+\beta\psi)\mathrm{d}m}=\alpha\int_E{\varphi\mathrm{d}m}+\beta\int_E{\psi\mathrm{d}m}$

\subsubsection*{proof}
\[\int{\underbrace{\alpha\sum\limits{\alpha_i\chi_{E_i}}}_\varphi+\underbrace{\beta\sum\limits{\beta_i\chi_{F_i}}}_\psi}\]

move alphas and betas into sum, merge sums by changing $\chi_{E_i}$ and $\chi_{F_i}$ to $\chi_{E_i\cap F_j}$

\subsection*{proposition}
if $\varphi, \psi$ 
are simple with $\varphi\le \psi$ then $\int_E\varphi\le \int_E\psi$

proof, $(\psi-\varphi)\ge 0$. find canonical. $\sum\limits_{i=1}^n{\alpha_i\chi_{E_i}}$ notice $\alpha_i\ge 0$ and $\int_E{\psi-\varphi\mathrm{d}m}=\sum\limits_{i=1}^n{\alpha_im*(E_i)}\ge 0$ and so $\int_{E}\psi-\int_E\varphi=\int_E(\psi-\varphi)\mathrm{d}m\ge 0$ and so $\int_E{\psi}\ge\int_E{\phi}$

\subsubsection*{break}
if $\varphi,\psi$ is simple and $m*\{x:\varphi(x)\ne\psi(x)\}=0$ then $\int_E{\varphi}=\int_E\psi$

$(\varphi-\psi)=0\chi_{[\{x:\varphi(x)=\psi(x)\}]}+(mess)\chi_{\{x:\varphi(x)\ne\psi(x)\}}$

\subsection*{almost everywhere}
$f=g$ almost everywhere if $m*(\{x:f(x)\ne g(x)\})=0$

\subsubsection*{definition}
is measurable if for any $a\in \mathbb{R}$ we have $\{x:f(x)\ge a\}$ is measurable


\end{document}
