\documentclass[letterpaper]{article}

\usepackage[utf8]{inputenc}
\usepackage{fullpage}
\usepackage{nopageno}
\usepackage{amsmath}
\usepackage{amssymb}
\allowdisplaybreaks

\newcommand{\abs}[1]{\left\lvert #1 \right\rvert}

\begin{document}
\title{Notes}
\date{January 26, 2015}
\maketitle
\section*{non measurable set example}
$x,y\in [0,1)$

$x\oplus y:\begin{cases}x+y&\text{if }x+y<1\\x+y-1&\text{if }x+y\ge 1\end{cases}$

\subsection*{facts}
\begin{enumerate}
\item
$x\oplus y=y\oplus x$
\item
$(x\oplus y)\oplus z=x\oplus(y\oplus z)$
\end{enumerate}

if $E\subseteq[0,1)$ and $x\in [0,1)$ then define $E\oplus x:\{z:y\oplus x=z \text{ with }y\in E\}$

\subsection*{lemma}
if $E\subseteq[0,1)$ is measurable and $x\in[0,1)$ then $E\oplus x$ is measurable and $m*(E\oplus x)=m*(E)$
\subsubsection*{proof}
$E_1=E\cap[0,1-x), E_2=E\cap[1-x,1)$ where $E_1$ is regular addition, and $E_2$ is overflow.

notice $E_1$ and $E_2$ are measurable. Now $E_1\cup E_2=E$ and because they are disjoint and measurable then $m*(E_1)+m*(E_2)=m*(E)$

$E_1\oplus x=E_1+ x$. Now $m*(E_1\oplus x)=m*(E_1+x)=m*(E_1)$ (from homework)

on the other hand $E_2\oplus x=E_2+x-1$ and so $m*(E_2\oplus x)=m*(E_2+x-1)=m*(E_2)$.

\begin{align*}
  m*(E)=m*(E_1)+m*(E_2)&=m*(E_1\oplus x)+m*(E_2\oplus x)\\
  &=m*((E_1\oplus x)\cup(E_2\oplus x))\\
  y&\in E_1\oplus x\\
  y&\in E_2\oplus x\\
  y&=x+z_1, z_1\in [0,1-x)\\
  y&=x+z_2-1, z_2\in [1-x,1)\\
  z_1&=z_2-1\\
  (E_1\oplus x)\cap (E_2\oplus x)&=\emptyset
  &=m*(E\oplus x)
\end{align*}

take $x,y\in [0,1)$ where $x\sim y$ if $x=y+q$ where $q$ is rational. claim is that this is equivalence relation.
\begin{enumerate}
\item
symmetry

$x=y+q\to y=x+(-q)$
\item
reflexivity

$x=x+0$ and $0\in \mathbb{Q}$
\item
transitivity

$x=y+q_1, y=z+q_2\to x=(z+q_2)+q_1=z+(q_1+q_2)$
\end{enumerate}

$[0,1)/\sim=$ equivalence classes (space modulo equivalence classes)

$[0,1)=\cup[x]_\sim\leftarrow$ disjoint

\section*{axiom of choice}
if you have a collection of sets, then you can choose something from each set.

now for each equivalence class, choose one element. the collection of these choices is $P$

every rational number is equivalent to every other rational and so $P$ contains one and only one rational number.

one can show that for example $\pi/4$ will create it's own equivalence class. and $e/3$ will also have it's own class. It is difficult to show that $e$ and $\pi$ are in their own equivalence classes. $P$ has at least 3 elements. Actually it has an uncountable number of elements.

$\{r_i\}_{i=0}^\infty$ is an enumeration of $\mathbb{Q}\cap[0,1)$ with $r_0=0$.This list is the rationals in any order, but $0$ is at the beginning

let $P_i=P\oplus r_i$
\begin{enumerate}
\item
$m*(P_i)=m*(P)$
\item
$P_0=P$
\item
$P_i\cap P_j=\emptyset$ if $i\ne j$.

if $x\in P_i\cap P_j$ then $x=x_1\oplus r_i=x_2\oplus r_j$ where $x_1,x_2\in P$.

$x_1+r_1=x_2+r_2\to x_1=x_2 +(r_2-r_1)$

$x_1+r_1-1=x_2+r_2\to x_1=x_2+(r_2-r_1+1)$

$x_1+r_1=x_2+r_2-1\to x_1=x_2+(r_2-1-r_1)$

$x_1+r_1-1=x_2+r_2-1\to x_1=x_2+(r_2-r_1)$

and so $x_1\sim x_2$ and so $x$ is not in both and we have a contradiction
\item
$\bigcap_{i=1}^\infty P_i=[0,1)$

$x\in [0,1)$ and $x\sim y$ with $y\in P$ and $x=y+q$ with $q\in \mathbb{Q}$

$q\in \mathbb{Q}$ implies $q=r_j$ for some $j$ so $x\in P_j$.
\end{enumerate}

if $P$ is measurable then so is $P_i$ for all $i$ and so $1=m*([0,1))=m*(\bigcup_{i=1}^infty P_i=\sum\limits_{i=1}^\infty{m*(P_i)}=\sum\limits_{i=1}^\infty{m*(P)}$. Now either the sum is 0 and we can't get to one, or it's not 0 and we can get bigger than one. contradiction


\end{document}
