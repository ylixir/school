%shell-escape
\documentclass[letterpaper]{article}

\usepackage{amsmath}
\usepackage[utf8x]{luainputenc}
\usepackage{amssymb}
\usepackage{gnuplottex}
\usepackage{fancyhdr}
\pagestyle{fancy}
%\chead{Real Analysis II Homework}
\rhead{Jon Allen}
\allowdisplaybreaks

\newcommand{\abs}[1]{\left\lvert #1 \right\rvert}

\begin{document}

\renewcommand{\labelenumi}{\Alph{enumi}.}
%\renewcommand{\labelenumii}{\arabic{enumii}.}
\renewcommand{\labelenumii}{(\alph{enumii})}
\section*{8.5}
\begin{enumerate}
\item
Determine the interval of convergence of the following power series:
  \begin{enumerate}
  \setcounter{enumii}{5}
  \item
  $\sum\limits_{n=0}^\infty{x^{n!}}$.

  We first compare $x^n$ to $x^{n!}$. If $|x|<1$ then $|x^{n!}|<|x^n|$ and if $|x|>1$ then $|x^{n!}|>|x^n|$. Of course if $|x|=1$ then $|x^n|=1=|x^{n!}|$.

  Now examining $\sum\limits_{n=0}^\infty{x^n}$ we see that $\lim\limits_{n\to\infty}|1|^{1/n}=1$ and so our radius of convergence is $1$.

  Now $\sum\limits_{n=0}^\infty{x^n}$ is a geometric series, and so it converges only if $|x|<1$. And so by comparison $\sum\limits_{n=0}^\infty{x^{n!}}$ has an interval of convergence of $(-1,1)$
  \end{enumerate}
\item
Find a power series $\sum\limits_{n=0}^\infty{a_nx^n}$ that has a different \emph{interval} of convergence than $\sum\limits_{n=0}^\infty{na_nx^{n-1}}$.

We choose $a_n=\frac{1}{n+1}$ and $\lim\limits_{n\to\infty}\frac{n+1}{n+2}=1$. Our radius of convergence then is $1$. $\sum\limits_{n=0}^\infty{\frac{x^n}{n+1}}$ converges at $-1$ by the alternating series test. Now $\sum\limits_{n=0}^\infty{\frac{1}{2n}}<\sum\limits_{n=0}^\infty{\frac{1}{n+1}}$. But $\frac{1}{2}\sum\limits_{n=0}^\infty{\frac{1}{n}}$ diverges and so $\sum\limits_{n=0}^\infty{\frac{1}{n+1}}$ diverges. And so our interval of convergence is $[-1,1)$. Now $\sum\limits_{n=0}^\infty{\frac{n}{n+1}x^{n-1}}$ has the same radius of convergence. Now $\sum\limits_{n=0}^\infty{\frac{n}{n+1}}=\sum\limits_{n=0}^\infty{1-\frac{1}{n+1}}$. But $\lim\limits_{n\to\infty}1-\frac{1}{n+1}=1$ and so this series diverges at $1$. And similarly $(-1)^{n-1}-\frac{(-1)^{n-1}}{n+1}$ alternately approaches $1$ and $-1$ as $n$ goes to infinity. And so because $(-1)^{n-1}\frac{n}{n+1}$ has no limit, the series can not converge. Thus our interval of convergence is $(-1,1)$
\end{enumerate}
\section*{10.1}
\begin{enumerate}
\setcounter{enumi}{2}
\item
%10.1 #C
Let $f$ satisfy the hypotheses of Taylor's Theorem at $x=a$.
  \begin{enumerate}
  \item
  Show that $\lim\limits_{x\to a}\frac{f(x)-P_n(x)}{(x-a)^n}=0$.

  \begin{align*}
    \lim\limits_{x\to a}\left\lvert\frac{f(x)-P_n(x)}{(x-a)^n}\right\rvert
    &=\lim\limits_{x\to a}\left\lvert\frac{R_n(x)}{(x-a)^n}\right\rvert\\
    &\le \lim\limits_{x\to a}\left\lvert\frac{M(x-a)^{n+1}}{(n+1)!(x-a)^n}\right\rvert\\
    &= \frac{M}{(n+1)!}\lim\limits_{x\to a}\left\lvert(x-a)\right\rvert\\
    &= \frac{M}{(n+1)!}0=0\\
  \end{align*}
  \item
  If $Q(x)\in \mathbb{P}_n$ and $\lim\limits_{x\to a}\frac{f(x)-Q(x)}{(x-a)^n}=0$, prove that $Q=P_n$.

  Because $\lim\limits_{x\to a}\frac{f(x)-Q(x)}{(x-a)^n}=0$
  and $\lim\limits_{x\to a}\frac{f(x)-P_n(x)}{(x-a)^n}=0$ it follows that
  \begin{align*}
    \lim\limits_{x\to a}\frac{f(x)-Q(x)}{(x-a)^n}-\lim\limits_{x\to a}\frac{f(x)-P_n(x)}{(x-a)^n}&=0\\
    \lim\limits_{x\to a}\frac{f(x)-Q(x)-(f(x)-P_n(x))}{(x-a)^n}&=0\\
    \lim\limits_{x\to a}\frac{P_n(x)-Q(x)}{(x-a)^n}&=0\\
    \intertext{Recalling that $P_n(X),Q(x)\in\mathbb{P}_n$}
    \lim\limits_{x\to a}\frac{P_n(x)-Q(x)}{(x-a)^n}
    &=\lim\limits_{x\to a}\sum\limits_{i=0}^n{\frac{a_ix^i}{(x-a)^n}}\\
    \lim\limits_{x\to a}\sum\limits_{i=0}^n{\frac{a_ix^i}{(x-a)^n}}
    &=\sum\limits_{i=0}^n{\lim\limits_{x\to a}\frac{a_ix^i}{(x-a)^n}}\\
  \end{align*}
  Now if we assume $P_n(x)\ne Q(x)$ then there exists some $a_i\ne 0$. $\frac{a_ix^i}{(x-a)^n}$ does not converge as $x\to a$, and so neither does $\frac{P_n(x)-Q(x)}{(x-a)^n}$, which is contrary to our assumption.
  \end{enumerate}
\setcounter{enumi}{5}
\item
%10.1 #F
Let $f(x)=\log x$.
  \begin{enumerate}
  \item
  Find the Taylor series of $f$ about $x=1$.
  \begin{align*}
    f'(x)&=\frac{1}{x}&
    f''(x)&=-\frac{1}{x^2}\\
    f^{(3)}(x)&=\frac{2}{x^3}&
    f^{(k)}(x)&=\frac{(-1)^{k+1}(k-1)!}{x^k}\\
    P_n(x)&=\sum\limits_{k=1}^n{\frac{(-1)^{k+1}(k-1)!}{1^kk!}(x-1)^k}&
    P_n(x)&=\sum\limits_{k=1}^n{\frac{(-1)^{k+1}(x-1)^k}{k}}
  \end{align*}
  \item
  What is the radius of convergence of this series?
  \[\lim\limits_{k\to\infty}\left\lvert\frac{(-1)^{k+2}k}{(-1)^{k+1}(k+1)}\right\rvert
  =\lim\limits_{k\to\infty}\frac{k}{(k+1)}=1=R \]
  \item
  What happens at the two endpoints of the interval of convergence? Hence find a series converging to $\log 2$.
  \begin{align*}
    \sum\limits_{k=1}^\infty{\frac{(-1)^{k+1}(-1)^k}{k}}
    &=\sum\limits_{k=1}^\infty{\frac{-1}{k}}=\infty\\
    \sum\limits_{k=1}^\infty{\frac{(-1)^{k+1}(2-1)^k}{k}}
    &=-\sum\limits_{k=1}^\infty{\frac{(-1)^k}{k}}
  \end{align*}
  So the series does not converge at $0$, but it does at $2$, and the series is above.
  \item
  By observing that $\log 2=\log 4/3-\log 2/3$, find another series converging to $\log 2$. Why is this series more useful?
  \begin{align*}
    \sum\limits_{k=1}^\infty{\frac{(-1)^{k+1}(\frac{4}{3}-1)^k}{k}}
    -\sum\limits_{k=1}^\infty{\frac{(-1)^{k+1}(\frac{2}{3}-1)^k}{k}}\\
    \sum\limits_{k=1}^\infty{\frac{(-1)^{k+1}}{3^kk}}
    +\sum\limits_{k=1}^\infty{\frac{1}{3^kk}}\\
  \end{align*}

  We know that our error ($R_n(x)$) is not more than
  $\frac{M|x-1|^{n+1}}{(n+1)!}$
  where $M\ge |f^{(n+1)}(x)|=\left\lvert\frac{(-1)^{k+2}k!}{x^{k+1}}\right\rvert$. And swapping out $M$ we have
  \begin{align*}
    R_n(x)
    &\le \left\lvert\frac{(-1)^{k+2}k!}{x^{k+1}}\right\rvert\cdot\frac{|x-1|^{k+1}}{(k+1)!}\\
    &= \frac{|x-1|^{k+1}}{x^{k+1}(k+1)}\\
    &\simeq \frac{|x-1|^{k}}{kx^{k}}
  \end{align*}
  And so $R_n(2)\simeq\frac{1}{k2^k}$ and $R_n(4/3)\simeq\frac{1}{3^kk\frac{4}{3}^k}=\frac{1}{k4^k}$ and $R_n(2/3)\simeq\frac{1}{3^kk\frac{2}{3}^k}=\frac{1}{k2^k}$. So we are using the $\log 4/3$ term to improve the accuracy of our estimate because $R_n(4/3)\le R_n(2)$.
  \end{enumerate}
\setcounter{enumi}{8}
\item
%10.1 #I
Let $f(x)=(1+x)^{-1/2}$
  \begin{enumerate}
  \item
  Find a formula for $f^{(k)}(x)$. Hence show that
  \[f^{(k)}(0)={-\frac{1}{2} \choose k}:=\frac{-\frac{1}{2}(-\frac{1}{2}-1)\cdots(-\frac{1}{2}+1-k)}{k!}=\frac{(-1)^k(2k)!}{2^{2k}(k!)^2}=\left(\frac{-1}{4}\right)^k{2k\choose k}.\]

  \begin{align*}
    f^{(k)}&=(1+x)^{-1/2-k}\prod\limits_{i=1}^k{\frac{1}{2}-i}
  \end{align*}
  \item
  Show that the Taylor series for $f$ about $x=0$ is $\sum\limits_{k=0}^\infty{{2k\choose k}\left(\frac{-x}{4}\right)^k}$, and compute the radius of convergence.
  \item
  Show that $\sqrt{2}=1.4f(-0.02)$.
  Hence compute $\sqrt{2}$ to $8$ decimal places.
  \item
  Express $\sqrt{2}=1.415f(\varepsilon)$, where $\varepsilon$ is expressed as a fraction in lowest terms. Use this to obtain an alternating series for $\sqrt{2}$. How many terms are needed to estimate $\sqrt{2}$ to 100 decimal places?
  \end{enumerate}
\end{enumerate}
\section*{10.2}
\begin{enumerate}
\setcounter{enumi}{3}
\item
Suppose that $f$ is a continuous function on $[0,1]$ such that $\int_0^1{f(x)x^n\;\mathrm{d}x}=0$ for all $n\ge 0$. Prove that $f=0$. {\scshape Hint:} Use the Weierstrass Theorem to show that $\int_0^1{|f(x)|^2\;\mathrm{d}x}=0$
\end{enumerate}
\end{document}
