%shell-escape
\documentclass[letterpaper]{article}

\usepackage{amsmath}
\usepackage[utf8x]{luainputenc}
\usepackage{amssymb}
\usepackage{gnuplottex}
\usepackage{fancyhdr}
\pagestyle{fancy}
\lhead{September 11, 2015}
\chead{Real Analysis II Homework}
\rhead{Jon Allen}
\allowdisplaybreaks

\newcommand{\abs}[1]{\left\lvert #1 \right\rvert}

\begin{document}

\renewcommand{\labelenumi}{\Alph{enumi}.}
%\renewcommand{\labelenumii}{\arabic{enumii}.}
\renewcommand{\labelenumii}{(\alph{enumii})}
\section*{8.2}
%bdfgi
\begin{enumerate}
\setcounter{enumi}{1}
%8.2 B
\item
Show that $h_n(x)=\frac{n+x}{4n+x}$ converges uniformly on $[0,N]$ for any $N>\infty$ but not uniformly on $[0,\infty)$.

First we note that $\displaystyle \lim_{n\to\infty}h_n$ is $h(x)=\frac{1}{4}$.

Our discontinuity at $4n+x=0$ can not happen when $n\ge 1$ and $x\ge 0$. This will not affect us then.

Lets find $||h_n-h||_\infty$.
The partial derivative of $h_n-h$ with respect to $x$ is $\displaystyle \frac{3n}{(4n+x)^2}$ with a second derivative of $\displaystyle -\frac{6n}{(4n+x)^3}$.
Notice that $3n>0, 6n>0,$ and $4n+x>0$ when $n\ge 1$ and $x\ge0$. And so we have no critical points and our function is concave over the entire domain.
So then $||h_n-\frac{1}{4}||_\infty=\max(||h_n(0)-\frac{1}{4}||,||h_n(N)-\frac{1}{4}||)$.
Because both $h_n(0)$ and $h_n(N)$ both converge to $\frac{1}{4}$ then $||h_n-\frac{1}{4}||_\infty$ must converge to $\frac{1}{4}$.
It then follows that $\displaystyle \lim_{k\to\infty}||h_k-h||_\infty=0$ on $[0,N]$.

Now let us broaden our view a little and look at what happens when $x\in[0,\infty)$. Let's choose $\varepsilon=\frac{1}{2}$ and $n$ arbitrarily large and see what happens when $x\ge 8n$. Keep in mind that we never have to deal with negative numbers.
\begin{align*}
  x&\ge 8n\\
  \frac{1}{4}x&\ge 2n\\
  n+x&\ge3n+\frac{3}{4}x\\
  \frac{n+x}{4n+x}&\ge\frac{3n+\frac{3}{4}x}{4n+x}=\frac{3}{4}\cdot\frac{4n+x}{4n+x}\\
  \frac{n+x}{4n+x}-\frac{1}{4}&\ge\frac{3}{4}-\frac{1}{4}=\varepsilon
\end{align*}
And so regardless of our choice of $n$ we have $||h_n-h||\ge \varepsilon$ for all $x\ge 8n$. This breaks the definition of uniform convergence, thus this function does not converge uniformly over $[0,\infty)$
%However, notice that the limit of $h_n$ as $x$ goes to infinity is $1$.
%This means that if we choose $n\in \mathbb{N}$ arbitrarily large we can always find some $x$ in $[0,\infty)$ were $h_n(x)$ is greater than any number in $(\frac{1}{4},1)$. and so in this case $\lim_{k\to\infty}||f_k-f||_\infty=\frac{3}{4}$ and so this thing doesn't converge uniformly over all of $x\ge 0$.
\setcounter{enumi}{3}
%8.2 D
\item
Let $(f_n)$ and $(g_n)$ be sequences of continuous functions on $[a,b]$. Suppose that $(f_n)$ converges uniformly to $f$ and $(g_n)$ converges uniformly to $g$ on $[a,b]$. Prove that $(f_ng_n)$ converges uniformly to $fg$ on $[a,b]$

First we observe that because $f_n$ and $g_n$ are both continuous and both converge uniformly, then $f$ and $g$ must also be continuous. From the extreme value theorem we know then that all of $f_n,g_n,f,$ and $g$ are all bounded. First we need to find a bound for all the $g_n$s.

We know that for all $\varepsilon>0$ there exists some $N\in \mathbb{N}$ such that for all $n\ge N$ we have $||g_n-g||_\infty<\varepsilon$.
If we choose $\varepsilon=1$ then we can find some $N\in \mathbb{N}$ such that $||g_n-g||_\infty<\varepsilon$ for all $n\ge\mathbb{N}$.
Lets set $M_g=\max(||g_1||_\infty,\dots,||g_N||_\infty,||g||_\infty)+1$.
Now obviously $M_g>g_n$ when $n<N$ and if $n\ge N$ then for all $x\in[a,b]$ we have $||g_n(x)||=||g_n(x)-g(x)+g(x)||\le ||g_n(x)-g(x)||+||g(x)||\le ||g_n(x)-g(x)||+(M_g-1)\le\varepsilon+M_g-1$.
Remembering that $\varepsilon=1$ we have $||g_n(x)||\le M_g$ for all $x$ and so $||g_n||_\infty\le M_g$.

Okay now we have found our bound of $M_g$ for $g_n$.
Keeping in mind that $f$ is also bounded, we can find some $M$ that bounds both $g_n$ and $f$.
Lets say $M=\max(M_g,||f||_\infty)$.
Now we choose some arbitrary $\varepsilon>0$. We know that we can find some $N\in \mathbb{N}$ large enough that $||g_n-g||_\infty\le\frac{\varepsilon}{2M}$ and $||f_n-f||_\infty<\frac{\varepsilon}{2M}$ for all $n>N$.

Back to where we started, we can do some algebra on $||f_n(x)g_n(x)-f(x)g(x)||$ to make things a little more manageable. Lets to that. The following is true for all $x\in [a,b]$ so will leave the $x$s out of the notation.
\begin{align*}
  ||f_ng_n-fg||&=||f_ng_n-fg_n+fg_n-fg||\\
  &\le||f_ng_n-fg_n||+||fg_n-fg||\\
  &=||g_n||\cdot||f_n-f||+||f||\cdot||g_n-g||\\
\end{align*}
If we remember that $||g_n||$ and $||f||$ are both no bigger than $M$ while $||g_n-g||$ and $||f_n-f||$ are smaller than $\frac{\varepsilon}{2M}$ then we have $||f_ng_n-fg||< M\frac{\varepsilon}{2M}+M\frac{\varepsilon}{2M}=\varepsilon$. And because this is true for every $x\in[a,b]$ then we have $||f_ng_n-fg||_\infty<\varepsilon$. Thus $\displaystyle \lim_{n\to\infty}||f_ng_n-fg||_\infty=0$ as required.$\Box$
\setcounter{enumi}{5}
%8.2 F
\item
Let $f_n(x)=\arctan(nx)/\sqrt{n}$.
  \begin{enumerate}
  \item
  Find $\displaystyle f(x)=\lim_{n\to\infty}f_n(x)$, and show that $(f_n)$ converges uniformly to $f$ on $\mathbb{R}$

  Our function is bounded by $\left[-\frac{\pi}{2\sqrt{n}},\frac{\pi}{2\sqrt{n}}\right]$.
  The limit of $\pm\frac{\pi}{2\sqrt{n}}$ as $n$ goes to infinity is 0 and this function must converge uniformly to $f(x)=0$
  \item
  Compute $\displaystyle \lim_{n\to\infty}f_n'(x)$, and compare this with $f'(x)$.

  The derivative is $f_n'(x)=\frac{\sqrt{n}}{n^2x^2+1}$ and when we take the limit as $n$ goes to infinity of that thing we get
  \begin{align*}
    \displaystyle \lim_{n\to\infty}f_n'
    &=\begin{cases}\lim_{n\to\infty}\sqrt{n}&x=0\\\frac{1}{x^2}\lim_{n\to\infty}\frac{\sqrt{n}}{n^2+1/x^2}&x\ne 0\end{cases}\\
    &=\begin{cases}\infty&x=0\\\frac{1}{x^2}\lim_{n\to\infty}\frac{\sqrt{n}}{n^2+1/x^2}&x\ne 0\end{cases}\\
    &=\begin{cases}\infty&x=0\\\frac{1}{x^2}\lim_{n\to\infty}\frac{1}{n^{3/2}+1/(\sqrt{n}x^2)}&x\ne 0\end{cases}\\
    &=\begin{cases}\infty&x=0\\0&x\ne 0\end{cases}
  \end{align*}
  And $f'=0$ so the derivative of the limit and the limit of the derivative are the same everywhere except $x=0$
  \item
  Where is the convergence of $f_n'$ uniform? Prove your answer.

  Look at $f_n''(x)=\frac{\mathrm{d}}{\mathrm{d}x}\left(\frac{\sqrt{n}}{n^2x^2+1}\right)=\frac{\mathrm{d}}{\mathrm{d}x}\left(n^{3/2}x^2+n^{-1/2}\right)^{-1}=-(n^{3/2}x^2+n^{-1/2})^{-2}2n^{3/2}x$. We see immediately that because $n\ge 1$ then our only zero term is $x=0$. If we think about the graph of arctan, this matches our intuition. The slope increases as we approach zero from either direction and decreases as we get farther away. Compressing the graph vertically and horizontally by any factor will not change this. Since for any $[a,\pm\infty)$ with $|a|>0$ we have an upper bound at $f_n'(a)$ which goes to zero, then we know that we converge uniformly in either of these domains. However, taking the contrapositive of theorem 8.2.1 we find that when we include zero in our domain, then our limit is discontinuous, and so either our sequence is not continuous, or we do not converge uniformly. We know our sequence is continuous, and so we must not converge uniformly.
  \end{enumerate}
%8.2 G
\item
Suppose that functions $f_k$ defined on $\mathbb{R}^n$ converge uniformly to a function $f$. Suppose that each $f_n$ is bounded, say by $A_k$. Prove that $f$ is bounded.

We know that for any given $\varepsilon>0$ we can find some $N$ such that if $k\ge N$ then $||f_k-f||_\infty<\varepsilon$. If we pick $1=\varepsilon$ then we get $1>||f_k-f||_\infty\ge |||f_k|||_\infty-||f||_\infty|=|A_k-||f||_\infty|$ and so $-1<A_k-||f||_\infty<1$ or $A_k+1>||f||_\infty>A_k-1$ and so $f$ is bounded.
\setcounter{enumi}{8}
%8.2 I
\item
Give an example of a sequence of discontinuous functions $f_n$ that converges uniformly to a continuous function.

\[f_n(x)=\begin{cases}0&x=0\\\frac{|x|}{xn}&x\ne 0\end{cases}\]

This function is $\pm\frac{1}{n}$ when $x\ne 0$ and $0$ when $x=0$. Obviously this is discontinuous, but $\pm\frac{1}{n}$ approaches zero as $n$ goes to infinity and so it must converge uniformly since we can find $\frac{1}{n}<\varepsilon$ for any $\varepsilon>0$
\end{enumerate}
\end{document}
