\documentclass[letterpaper]{article}

\usepackage[utf8]{inputenc}
\usepackage{fullpage}
\usepackage{nopageno}
\usepackage{amsmath}
\usepackage{amssymb}
\allowdisplaybreaks

\newcommand{\abs}[1]{\left\lvert #1 \right\rvert}

\begin{document}
\title{Notes}
\date{January 30, 2015}
\maketitle
$f$ is Riemann integrable if and only if $m*(D_f)=0$ (we did this with contrapositive last time).

\section*{proof of converse}
$D_r=\{x:\omega_f(x)\ge \frac{1}{n}\}$

$D_f=\bigcup_{n\in \mathbb{N}}J_\frac{1}{n}$
because $m*(D_f)=0$ we know that $m*(J_{\frac{1}{n}})=0$

Let $P$ be a partition with $\#(P)<S$ (mesh $P$ is max $\Delta_i$).

if $\epsilon>0$ with $w_f(x)<\epsilon$ for all $x\in[a,b]$ there exists $S>0$ such that $\Omega_f(T)<\epsilon$. If $T$ is any closed interval with $m*(T)<S$.

?So if $m*(T)<S$ then $\Omega_f(T)<\frac{1}{n}$

\subsubsection*{break}
$U(P,f)-L(P,f)=\sum\limits_{S_1}{(M_i-m_i)\Delta_i}+\sum\limits_{S_2}{(M_i-m_i)\Delta_i}$

$S_1=\{[x_i,x_{i+1}]:J_{1/n}\cap(x_i,x_{i+1})\ne \emptyset\}$
$S_2=\{[x_i,x_{i+1}]:J_{1/n}\cap(x_i,x_{i+1})= \emptyset\}$

oscillation on $S_1$ is small, maybe big on $S_2$

on $S_2$ we have $M_i-m_i<\frac{1}{n}$. and $\sum\limits_{S_2}{(M_i-m_i)\Delta_i}< \frac{1}{n}\sum\limits_{i=1}^n{\Delta_i}\le \frac{1}{n}(b-a)$

function is bounded and so $M_i\le M$ where $M$ is upper bound for $f$ and $m_i>m$ is lower bound 

so on $S_1$ we havve $\sum\limits_{S_1}{(M_i-m_i)\Delta_i}\le (M-m)\sum\limits_{S_1}{\Delta_i}\le (M-m)\frac{1}{n}$

these intervals cover $J_{1/n}$. $m*(J_{1/n})\le\sum\limits_{S_1}{(x_i-x_{i-1})}$.

Any cover $U(a_i,b_i)\subseteq J_{1/n}$ with $|b_i-a_i|\subset S\to m*(J_{1/n})\le \sum\limits{b_i-a_i}\le m*(J_{1/n}))+\frac{1}{n}$

we choose a partition so that the subpartition reflects above. and then go through calculations and ge

$U(Pf)-L(P,f)\le \frac{(M-m)+(b-a)}{n}$

\section*{facts}
\begin{enumerate}
\item
if $f$ is piecewise continuous on $[a,b]$ then $f$ is Riemann integrable
\item
$\chi_s(x)=\begin{cases}1&x\in S\\0&x\not\in S\end{cases}$

where $\chi_s$ is discontinuous for any point on $\partial S$ and continuous everywhere else

$\partial S=\overline{S}/S^\circ$

$\chi_C=\overline{C}=C\setminus\emptyset$

$\partial (\chi_C)=C$

and now $\chi_{\mathbb{Q}}$ and so boundary of rationals $\partial \mathbb{Q}=\mathbb{R}$

note that $\int{\chi_s\mathrm{d}m}=m*(S)$
\end{enumerate}


$f$ is simple if
\begin{enumerate}
\item
range of $f=\{\alpha_1,\alpha_2,\dots,\alpha_k\}$ is finite
\item
$E_k=\{x:\varphi(x)=\alpha_k\}$ is measurable.

\end{enumerate}

notice that $\varphi(x)=\sum\limits_{i=1}^k{\alpha_k\chi_{E_i}(x)}$

this is the canonical representation of $\varphi$ and is unique.

$\sum\limits_{i=1}^3{\frac{1}{i}\chi_{E_i}}$ with $E_1=[0,\frac{1}{3}]$ $E_2$ and $E_3$ are other two thirds. no zeros in function, pairwise disjoint sets means canonical
\end{document}
