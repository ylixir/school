%shell-escape
\documentclass[letterpaper]{article}

\usepackage{amsmath}
\usepackage{amssymb}
\usepackage{gnuplottex}
\usepackage{fancyhdr}
\pagestyle{fancy}
\lhead{February, 2015}
\chead{Real Analysis II Homework}
\rhead{Jon Allen}
\allowdisplaybreaks

\newcommand{\abs}[1]{\left\lvert #1 \right\rvert}

\begin{document}
\begin{enumerate}
\item
Let $g_n=n\chi_{[0,\frac{1}{n}]}$. Prove that for every $x$ and $\epsilon>0$ there is an $N\ge 0$ such that $|g_n(x)|<\varepsilon$ for all $n\ge N$. Prove that
\[\int_{[0,1]}{g_n\;\mathrm{d}m}=1\]
for all $n$.

First we look at what happens when $N=n=0$. Then we have $g_0=0\chi_{[0,\frac{1}{0}]}$. We could just define $g_0=0$ which is fine I guess, but then observe that $g_1=1$ and so we never really want to set $N=0$ and so we will just say $N>0$. Notice that $g_n(0)=n$ for all $n>0$. So clearly, no matter our choice of $\varepsilon$ we can find some $n\ge N>\varepsilon$. Let us just be clear and define $x>0$. So, now that I've changed the problem to what I want it to be, lets restate:

Let $g_n(x)=n\chi_{[0,\frac{1}{n}]}(x)$.
\begin{enumerate}
\item
Prove that for every $x>0$ and $\varepsilon>0$ there is an $N> 0$ such that $|g_n(x)|<\varepsilon$ for all $n\ge N$.
\subsubsection*{proof}
First we observe that for any $x>0$ we can find $N>0$ such that $\frac{1}{N}<x$. Of course then for any $n\ge N$ we have $\frac{1}{n}\le \frac{1}{N}<x$. And because $\frac{1}{n}<x$ then $x\not\in [0,\frac{1}{n}]$ and so $g_n(x)=0<\varepsilon$ and we are done. $\Box$
\item
Prove that
\[\int_{[0,1]}{g_n\;\mathrm{d}m}=1\]
for all $n$.
\subsubsection*{proof}
By definition we have that
\begin{align*}
\int_{[0,1]}{g_n\;\mathrm{d}m}&=n\cdot m*\left[0,\frac{1}{n}\right]+0\cdot m*\left(\frac{1}{n},1\right]\\
&=n\cdot \frac{1}{n}+0\cdot\left(1-\frac{1}{n}\right)\\
&=1
\end{align*}
A little light on words but not so heavy on math either, so maybe it's okay. $\Box$
\end{enumerate}
\item
Prove that $\psi$ is simple if and only if $a\psi$ is simple for every $a\in \mathbb{R}$
\subsubsection*{proof}
Let's assume that $\psi$ is simple. Then by definition
$\psi=\{\alpha_1,\dots,\alpha_n\}$.
We say that $E_i=\psi^{-1}(\{a_i\})$
and then $\psi=\sum\limits_{i=1}^n{\alpha_i\chi_{E_i}}$.
We note that all our $E_i$'s are disjoint ($E_i\cap E_j=\emptyset\forall i\ne j$). 

Now then $a\psi=a\sum\limits_{i=1}^n{\alpha_i\chi_{E_i}}=\sum\limits_{i=1}^n{a\alpha_i\chi_{E_i}}$.

Furthermore, because our $E_i$'s are disjoint, then $a\psi(E_i)=a\alpha_i$. And so $a\psi=\{a\alpha_1,\dots,a\alpha_n\}$. Which is the definition of simple.

Now if we assume that $a\psi$ is simple, then we have a nearly identical argument.
Let $a\psi=\{\alpha_1,\dots,\alpha_n\}$
and for $E_i=(a\psi)^{-1}(\{a_i\})$
then $a\psi=\sum\limits_{i=1}^n{\alpha_i\chi_{E_i}}$.
Now $\psi=\frac{1}{a}a\psi=\frac{1}{a}\sum\limits_{i=1}^n{\alpha_i\chi_{E_i}}=\sum\limits_{i=1}^n{\frac{1}{a}\alpha_i\chi_{E_i}}$.
Again we note that our $E_i$'s are disjoint and so $\psi=\{\frac{1}{a}\alpha_1,\dots,a\alpha_n\}$. Then $\psi$ fits the definition of simple, and we are done. $\Box$
\end{enumerate}
\subsubsection*{References}
None
\end{document}
