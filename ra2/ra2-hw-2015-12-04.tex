%shell-escape
\documentclass[letterpaper]{article}

\usepackage{amsmath}
\usepackage[utf8]{luainputenc}
\usepackage{amssymb}
\usepackage{gnuplottex}
\usepackage{fancyhdr}
\pagestyle{fancy}
%\chead{Real Analysis II Homework}
\rhead{Jon Allen}
\lhead{Real Analysis 2}
\allowdisplaybreaks

\newcommand{\abs}[1]{\left\lvert #1 \right\rvert}

\begin{document}

\renewcommand{\labelenumi}{\Alph{enumi}.}
%\renewcommand{\labelenumii}{\arabic{enumii}.}
\renewcommand{\labelenumii}{(\alph{enumii})}
\subsubsection*{7.3}
\begin{enumerate}
\setcounter{enumi}{5}
\item
%7.3 F
Let $\mathbb{R}^n$ have the max morm $||\mathbf{x}||_\infty=\max\{|x_i|:1\le  i\le n\}$. Let $K$ be the unit ball of $V$ and let $v=(2,0,\dots,0)$. Find all closest points to $v$ in $K$. 

We need $\min\{||\mathbf{v}-\mathbf{x}||:\mathbf{x}\in K\}$. We know that $|v_i-x_i|=x_i$ if $i\ne 1$ so let us look at $v_1$. We just need $\min|v_1-x_1|$. Since $-1\le x_1\le 1$ then we can't do better than $x_1=1$. So the closes points to $v$ in $K$ are $\{(1,x_2,\dots,x_n):|x_i|\le 1\}$
\end{enumerate}
\subsubsection*{7.4}
\begin{enumerate}
\setcounter{enumi}{1}
\item
%7.4 B
Show that every iiner product space stisfies the parallelogram law:
\[||x+y||^2+||x-y||^2=2||x||^2+2||y||^2\text{ for all } x,y\in V\]

  \begin{align*}
    ||x+y||^2&=\langle x+y,x+y\rangle=\langle x,x\rangle+2\langle x,y\rangle+\langle y,y\rangle\\
    ||x-y||^2&=\langle x-y,x-y\rangle=\langle x,x\rangle-2\langle x,y\rangle+\langle y,y\rangle\\
    ||x+y||^2+||x-y||^2&=2\langle x,x\rangle+2\langle y,y\rangle=2(||x||^2+||y||^2)\\
    ||x+y||^2+||x-y||^2&=2||x||^2+2||y||^2
%    ||x+y||^2&=2(||x||^2+||y||^2)-||x-y||^2
  \end{align*}


\item
%7.4 C
Minimize the quantity $||x||^2-2t\langle x,y\rangle +t^2||y||^2$ over $t\in \mathbb{R}$. You will see why we chose $t$ as we did in th proof of the Cauchy-Schwarz inequality.

First we need to find any critical points.
\begin{align*}
  \frac{\mathrm{d}}{\mathrm{d}x}\left(||x||^2-2t\langle x,y\rangle +t^2||y||^2\right)&=-2\langle x,y\rangle+2t||y||^2\\
\end{align*}
Notice that the derivative has no asymptotes or discontinuities. It's only zero is at $t=\frac{\langle x,y\rangle}{||y||^2}$. The second derivative is $2||y||^2\ge 0$ and so the expression is convex with a minimum at $\frac{\langle x,y\rangle}{||y||^2}$
\setcounter{enumi}{6}
\item
%7.4 G
A normed vector space is {\bf strictly convex} if $||u||=||v||=||u+v)/2||=1$ for vectors $u,v\in V$ implies that $u=v$
\begin{enumerate}
\item
Show that an inner product space is always strictly convex.

We assume that $||u||=||v||=||(u+v)||/2||=1$. Then
\begin{align*}
  1&=||(u+v)/2||\\
  1^2&=(\frac{1}{2}||(u+v)||)^2\\
  1&=\frac{1}{4}||(u+v)||)^2\\
  4=2+2&=||u+v||^2=2||u||^2+2||v||^2-||u-v||^2\\
  0&=||y-v||^2=\langle u-v,u-v\rangle=\langle0,0\rangle
\end{align*}
Thus $u=v$
\item
Show that $\mathbb{R}^2$ with the norm $||(x,y)||_\infty=\max\{|x|,|y|\}$ is not strictly convex.

We take $(1,1)$ and $(1,0)$. Then $||(1,1)||=||(1,0)||=||(2,1)/2||=1$ but $(1,1)\ne (1,0)$.
\end{enumerate}
\end{enumerate}
\end{document}
