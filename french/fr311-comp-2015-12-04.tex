\documentclass[letterpaper]{article}
% vim:spell:spelllang=fr

%\usepackage{fullpage}
%\usepackage{nopageno}
\usepackage{amsmath}
\usepackage{amssymb}
\usepackage[utf8]{luainputenc}
%\usepackage[utf8]{inputenc}
\usepackage{aeguill}
\usepackage{setspace}
\usepackage{fancyhdr}
\pagestyle{fancy}
\lhead{Jon Allen\\Français 311 Automne 2015\\4 décembre, 2015}
\rhead{Les choix sont fictives}
%\chead{Français 311 Automne 2015}
%\rhead{Jon Allen}
\allowdisplaybreaks

\begin{document}
\doublespacing
Il est dit que «On choisit ses amis, on ne choisit pas sa famille». C'est vrai ou faux? La question semble simple, mais il est difficile de répondre.

Quand j'étais jeune, je pourrais dire «Je n'ai pas choisi ma famille».
Je suis né dans ma famille.
Je n'ai pas choisi la couleur de mes yeux.
Je n'ai pas choisi ma mère.
Je n'ai pas choisi la couleur de mes cheveux.
Je n'ai pas choisi mon père.
Et certainement, je n'ai pas choisi d'avoir seulement des sœurs!

À l'école, j'ai trouvé mon meilleur ami. C'était mon ami. C'était mon choix. Puis ma famille, nous avons déménagé à Minot. Ce n'est pas mon choix. J'ai perdu mon meilleur ami. Mais j'ai fait connaissance d'autres copains. J'ai fait des copains à l'église et à l'école.

Quand j'étais un jeune adulte, je suis devenu athée. Je n'ai pas vu mes copains de l'église. Mes parents étaient très religieux. Je ne m'identifiais pas à avec mon père. Puis, il a déménagé en Floride. Il n'était pas mon parent favori. Nous ne nous sommes pas appelés. Nous ne nous sommes pas écrit.

Plus tard, ma femme et moi, nous nous sommes rencontrés. Puis, j'ai rencontré ma fille éventuel. J'ai choisi de se marier avec ma femme, et j'ai choisi d'adopter ma fille. Nous avons choisi d'ajouter un fils à notre famille.

Quand mon père est mort, je ne lui parlais pas depuis des années. Il n'a jamais rencontré mes enfants. Était-il encore dans ma famille? Je ne pense pas. À mon regret, j'ai choisi de retirer mon père de ma famille.

Et mes amis? Ai-je choisi mes copains de l'église? Quand j'ai eu le choix, je ne suis pas allé a l'église et je les ai quittés. Étaient-ils mes amis par mon choix? Et mes amis de l'école? J'ai quitté mon ami d'enfance. Mais je parle encore avec mes amis du lycée.

Je regarde ma vie. Je vois des amis de choix et des copains de circonstance. Je vois ma famille que j'ai choisie. Je vois ma famille où je suis né. Je vois la famille que j'ai créée. Je vois la famille que j'ai abandonnée. Il est dit que 
«On choisit ses amis, on ne choisit pas sa famille». Je dis «Je choisis ma famille, je choisis mes amis». Mais il est également vrai de dire «Je ne choisis pas ma famille. Je ne choisis pas mes amis». Le question n'a pas de mauvaise réponse.
\end{document}
