\documentclass[letterpaper]{article}

\usepackage{fullpage}
\usepackage{nopageno}
\usepackage{amsmath}
\usepackage{amssymb}
\usepackage[utf8]{inputenc}
\usepackage{aeguill}
\allowdisplaybreaks

\newcommand{\abs}[1]{\left\lvert #1 \right\rvert}

\begin{document}
\title{Les devoirs}
\date{Octobre 31, 2014}
\author{Jon Allen}
\maketitle
L'exercice 3 à la page 123
\begin{enumerate}
\item
Un Gaulois est un Français. Ils ne sont pas musulmans d'habitude.
\item
Mon beau-frère et ma belle-sœur sont un couple mixte. Ils sont probablement victimes de discimination parce que beaucoup de gens sont racistes.
\item
Abandonner sa famille est habituellement mal. Par fois on est une victime. Par fois on a besoin de s'échapper.
\item
J'ai été trahi par ma femme. Elle m'a raconté une plaisanterie. J'entendais pour la cas où elle dirait une chose, mais elle a dit autre chose. La plaisanterie était drôle parce que elle a trahi mes attentes. On ne peut jamais une telle trahison. Il est trop horrible.
\item
À mon avis la mondialisation contribue à améliorer la tolérance et le respect des autres. Je crois que la familiarité engendre la compréhension. La mondialisation nous donne la familiarité avec d'autre.
\end{enumerate}
L'exercice 5 à la page 123

«L'amour n'a pas de frontières», c'est un dicton banal. Je dis que la haine n'a pas de frontières. Par fois l'amour s'échappe une frontière. Si «l'amour n'a pas de frontières», il n'y a pas de frontières.
\end{document}
