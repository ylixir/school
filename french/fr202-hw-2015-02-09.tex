\documentclass[letterpaper]{article}

\usepackage{fullpage}
\usepackage{nopageno}
\usepackage{amsmath}
\usepackage{amssymb}
\usepackage[utf8]{luainputenc}
%\usepackage{aeguill}
\usepackage{setspace}
\allowdisplaybreaks

\newcommand{\abs}[1]{\left\lvert #1 \right\rvert}

\begin{document}
\title{les devoirs}
\date{9 février, 2015}
\author{Jon Allen}
\maketitle
\doublespacing
\subsubsection*{E.}
Traduisez les phrases suivantes en français.
\begin{enumerate}
\item
She is as stressed as me.  (...as stressed as I am.)

Elle est aussi stressée que moi
\item
She has as much stress as me.  (...as I have.)

Elle a autant de stress
\end{enumerate}

Le Superlatif.  Traduisez les phrases suivantes en français.

\begin{enumerate}
\item
Marie-Agnès is the best student in the class.

Elle est la meilleure étudiante de la classe
\item
She is (C’est…) the most interesting student in the world.

C'est la étudiante la plus interessante de la monde
\item
She is (C’est …) the tallest student in the class.

C'est la plus grande étudiante de la classe
\item
She writes (Elle écrit…) the best.

Elle écrit le mieux
\item
She has (Elle a…) the most questions of all the students.

Elle a le plus de questions de tous les étudiants.
\end{enumerate}
\subsubsection*{A.}
Voici ce qui arrive.  Complétez les phrases suivantes.  
\begin{enumerate}
\item
Plus j’étudie, moins je dors.
\item
Plus on vieillit, plus on est sage.
\item
Moins on fait d’exercice physique plus on est faible.
\item
Quelque chose que j’aime de plus en plus, c’est faire d'exercice physique.
\item
Ce que je fais le moins possible, c’est les devoirs.
\item
Je comprends de mieux en mieux pourquoi les cheveux grisonnent.
\end{enumerate}
\end{document}
