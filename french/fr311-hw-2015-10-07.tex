\documentclass[letterpaper]{article}

%\usepackage{fullpage}
%\usepackage{nopageno}
\usepackage{amsmath}
\usepackage{amssymb}
\usepackage[utf8x]{luainputenc}
%\usepackage[utf8]{inputenc}
\usepackage{aeguill}
\usepackage{setspace}
\usepackage{fancyhdr}
\pagestyle{fancy}
\lhead{
Jon Allen\\
Français 311 Automne 2015\\
7 octobre, 2015}
%\chead{Français 311 Automne 2015}
%\rhead{Jon Allen}
\allowdisplaybreaks

\begin{document}
\doublespacing
\begin{enumerate}
\item accro

Quand on ne peut pas arrêter de fumer, on est accro à les cigarettes.
\item bavarder

Quand nous nous parlons des choses inconsequential, nous nous bavardons.
\item un cornichon

Ma fille définit les cornichons sucrés que les bonbons au vinaigre. Les cornichons sont les petits concombres.

\item un(e) gamin(e)

On peut dire «un gamin» afin de faire référence un enfant.

\item la timidité

La timidité est quand on a peur de parler à des étrangers.
\end{enumerate}
\end{document}
