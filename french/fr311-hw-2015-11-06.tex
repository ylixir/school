\documentclass[letterpaper]{article}

%\usepackage{fullpage}
%\usepackage{nopageno}
\usepackage{amsmath}
\usepackage{amssymb}
\usepackage[utf8x]{luainputenc}
%\usepackage[utf8]{inputenc}
\usepackage{aeguill}
\usepackage{setspace}
\usepackage{fancyhdr}
\pagestyle{fancy}
\lhead{
Jon Allen\\
Français 311 Automne 2015\\
%2 novembre, 2015
}
%\chead{Français 311 Automne 2015}
%\rhead{Jon Allen}
\allowdisplaybreaks

\begin{document}
\doublespacing
\subsection*{page 72}
Cinq mots
\begin{enumerate}
\item Salam Aleikum

Quand on est dans un pays arabe on dit «Salam Aleikum», que veut dire «Bonjour».
\item un mensonge

Quand on dit un chose que n'est pas vrais, alors on a dit un mensonge.
\item septant-neuf (Suisse)

Si on est dans la Suisse, on dit septante-neuf, mais si on est dans la France, on dit soixante-dix-neuf.
\item des somnifères

Si on a sommeille et on ne dort pas, des pilules appelaient des somnifères aidera à dormir.
\item un singe

Un singe est un animal avec des pouces opposables qui ressemble à les gens.
\end{enumerate}
\end{document}
