\documentclass[letterpaper]{article}

\usepackage{fullpage}
\usepackage{nopageno}
\usepackage{amsmath}
\usepackage{amssymb}
\usepackage[utf8]{luainputenc}
%\usepackage{aeguill}
\usepackage{setspace}
\allowdisplaybreaks

\newcommand{\abs}[1]{\left\lvert #1 \right\rvert}

\begin{document}
\title{les devoirs}
\date{6 février, 2015}
\author{Jon Allen}
\maketitle
\doublespacing
\begin{enumerate}
\item
Combien de temps par jour passez-vous en utilisant la technologie numérique (digital)?  (à regarder la télé, à utiliser votre portable ou votre ordinateur, à écouter de la musique, etc.)

Je passe environ huit heures par jour en utilisant la technologie numérique. Je passe le plus de temps sur mon portable, mais je passe beaucoup de temps en utilisant mon ordiphone.
\item
Comment restez-vous en contact avec vos amis?  avec vos parents?  avec vos grands-parents?   Pourquoi préférez-vous ce mode de communication  - ou pourquoi est-ce qu'ils le préfèrent?


Je ne préfère pas Facebook ou sms ou le téléphone. Mes amis est bons avec les ordinateurs et les téléphones portables donc je reste en contact avec mes amis par Facebook ou sms ou le téléphone. Ma mère est Luddite alors je reste en contac avec elle par le téléphone.
Malheureusement, mes grand-parents sont morts alors je ne reste pas en contact avec eux.
\item
Quels sont, d'après vous, les bienfaits de la technologie?  Et quels en sont les dangers?  Donnez le pour et le contre.

La technologie améliore notre cerveaux et notre corps. Mais il est compliqué alors il fait compliquer notre vies et notre société.
\item
Racontez une situation où la technologie a eu des conséquences néfastes sur vos relations avec les autres ou sur votre vie en général.  Si elle n'a pas eu de conséquences néfastes, expliquez.

Je suis devenu accro à World of Warcraft. Je laisse tous les amis. Je laisse ma maison. Je déménage à une boîte en carton. Je m'enivre avoir chaud. Mais j'ai encore mon portable alors je joue encore!
\end{enumerate}
\end{document}
