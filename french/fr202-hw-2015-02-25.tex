\documentclass[letterpaper]{article}

\usepackage{fullpage}
\usepackage{nopageno}
\usepackage{amsmath}
\usepackage{amssymb}
\usepackage[utf8]{luainputenc}
%\usepackage{aeguill}
\usepackage{setspace}
\allowdisplaybreaks

\newcommand{\abs}[1]{\left\lvert #1 \right\rvert}

\begin{document}
\title{les devoirs}
\date{25 février, 2015}
\author{Jon Allen}
\maketitle
\doublespacing
\begin{enumerate}
\setcounter{enumi}{3}
\item
Peut-on faire des études et une activité ou un sport sans sacrifier l'un ou l'autre?

Oui, on peut faire des études et une activité ou un sport sans sacrifier l'un ou l'autre. Mais, il y a vingt-quatre heures dans un jour. Il faut savoir qui est le plus important. On peut faire des études et une ou deux activités. Peut-être on a les temps pour un sport aussi. Mais je n'ai pas les temps pour toutes les choses que je veux faire.
\item
Les parents doivent-ils soutenir leurs enfants coûte que coûte (at all costs) dans leurs activités?  Ou vaut-il mieux qu'ils (les parents) soient réalistes et les encouragent (encouragent les enfants) à choisir une autre voie?

Les parents doivent protéger leurs enfants et le
\item
Parfois les parents cherchent à vivre un rêve par l'intermédiaire de leurs enfants.  Que pensez-vous de cette attitude?
\end{enumerate}
\end{document}
