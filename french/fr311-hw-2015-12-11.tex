\documentclass[letterpaper]{article}

%\usepackage{fullpage}
%\usepackage{nopageno}
\usepackage{amsmath}
\usepackage{amssymb}
\usepackage[utf8x]{luainputenc}
%\usepackage[utf8]{inputenc}
\usepackage{aeguill}
\usepackage{setspace}
\usepackage{fancyhdr}
\pagestyle{fancy}
\lhead{
Jon Allen\\
Français 311 Automne 2015\\
%2 novembre, 2015
}
%\chead{Français 311 Automne 2015}
%\rhead{Jon Allen}
\allowdisplaybreaks

\begin{document}
\doublespacing
\subsection*{CFG Ex C p131-132}
\begin{enumerate}
\setcounter{enumi}{1}
\item
Je suis content que vous soyez ici.
\setcounter{enumi}{3}
\item
Je veux que vous partiez.
\setcounter{enumi}{5}
\item
Il est bon que vous dormiez.
\setcounter{enumi}{9}
\item
Il va avant qu'elle vienne.
\item
Je ne pense pas qu'il soit américain.
\item
Je me doute qu'il est français.
\item
Je doute qu'elle puisse finir.
\item
J'espère qu'elle peut finir.
\item
Il est possible que nous sachions la réponse.
\setcounter{enumi}{17}
\item
Bien que j'aie fait les courses, je ne pense pas que vous veniez. 
\end{enumerate}

\subsection*{la mél}
Bonjour mon ami.

Je souhaite que tu  aies un bon voyage à Fargo. J'ai peur que tu trouves le temps formidable. Si tu oubliez ton manteau, je serai triste que tu aies froid. Mais, je serai surpris d'apprendre que vous soyez mort du froid. Il semble que Fargo soit très différent de Paris. Cependant, je doute que tu ne sois pas content. Je serai content que tu viennes me rendre visite.Je veut que tu te promenes dans le centre-ville. Il faut que tu voies le Théâtre Fargo! À la fin du semestre, tu préféres peut-être que Fargo aie été en France.

Au revoir, mon ami
\end{document}
