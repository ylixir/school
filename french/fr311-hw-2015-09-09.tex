\documentclass[letterpaper]{article}

%\usepackage{fullpage}
%\usepackage{nopageno}
\usepackage{amsmath}
\usepackage{amssymb}
\usepackage[utf8x]{luainputenc}
%\usepackage[utf8]{inputenc}
\usepackage{aeguill}
\usepackage{setspace}
\usepackage{fancyhdr}
\pagestyle{fancy}
\lhead{9 septembre, 2015}
\chead{Les Devoirs pour français 311}
\rhead{Jon Allen}
\allowdisplaybreaks

\begin{document}
\doublespacing
\begin{enumerate}
\item
  Quand on déguste le sucre ou les bon bons, on a un goûte très doux.
\item
  On peut effacer un dessin au crayon, mais on ne peut pas effacer un dessin au stylo.
\item
  Il n'y a pas un lieu où on ne trouve pas une chose alors cette chose est partout. Elle est par toutes autres choses.
\item
  On n'a plus un chose cher, alors on a une perte.
\item
  On prend quelque chose qui ne leur appartient pas quand on dérobe.
\end{enumerate}
\end{document}
