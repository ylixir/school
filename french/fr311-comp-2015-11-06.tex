\documentclass[letterpaper]{article}

%\usepackage{fullpage}
%\usepackage{nopageno}
\usepackage{amsmath}
\usepackage{amssymb}
\usepackage[utf8x]{luainputenc}
%\usepackage[utf8]{inputenc}
\usepackage{aeguill}
\usepackage{setspace}
\usepackage{fancyhdr}
\pagestyle{fancy}
\lhead{Jon Allen\\Français 311 Automne 2015\\14 octobre, 2015}
\rhead{Le pesant de l'amitié}
%\chead{Français 311 Automne 2015}
%\rhead{Jon Allen}
\allowdisplaybreaks

\begin{document}
\doublespacing
Catherine Deneuve a dit «L'amitié, comme l'amour, demande beaucoup d'efforts, d'attention, de constance, elle exige surtout de savoir offrir ce que l'on a de plus cher dans la vie: du temps!»
Elle n'a pas menti.
L'amitié exige beaucoup de temps, mais je ne suis pas d'accord avec elle.

Quand j'étais jeune, j'avais un ami qui s'appelait Schuyler.
Il était mon meilleur ami.
Un jour, j'ai déménagé.
J'ai envoyé une lettre à lui.
Il a répondu.
Mais, je n'ai pas pris le temps d'envoyer une autre lettre.
Il y a deux ans, nous sommes devenus amis sur Facebook, qui était notre dernière communication.
Je souhaiterais que l'amitié exigeasse plus.

Je aurais fait de nouveaux amis.
Jake est un bon ami.
Nous étions amis pendant 15 ans.
Après avoir obtenu son baccalauréat, il a déménagé à la côte est.
Nous avons gardé contact.
Nous nous sommes rendu visite.
Après être devenu père, il est retourné au Dakota du Nord.
Maintenant, j'ai une famille aussi et nous n'avons pas beaucoup de temps pour se voir.
Mais le mois dernier, nous sommes sortis et nous avons pris des bières.
Cette amitié n'exige pas beaucoup de temps, mais elle est forte.

Ma meilleure amie est ma femme.
J'adore passer du temps avec elle.
Nous n'avons jamais assez de temps pour s'amuser.
Notre amitié n'exige pas de temps.
Le temps n'est pas un devoir.
Le temps que je passe avec elle est un privilège.

Nous voyons qu'une amitié ne peut pas réussir sans un peu de temps.
Mais une forte amitié peut rester forte sans beaucoup de temps et une très forte amitié demande toujours plus de temps.
Dans les mots de Tahar Ben Jelloun «Toujours présente, jamais pesante, telle devrait être la devise de toute amitié.»
\end{document}
