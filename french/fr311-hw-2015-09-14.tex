\documentclass[letterpaper]{article}

%\usepackage{fullpage}
%\usepackage{nopageno}
\usepackage{amsmath}
\usepackage{amssymb}
\usepackage[utf8x]{luainputenc}
%\usepackage[utf8]{inputenc}
\usepackage{aeguill}
\usepackage{setspace}
\usepackage{fancyhdr}
\pagestyle{fancy}
\lhead{14 septembre, 2015}
\chead{Les Devoirs pour français 311}
\rhead{Jon Allen}
\allowdisplaybreaks

\begin{document}
\doublespacing
\begin{enumerate}
\item
Cette année elle est allée à la mer avec ses copines.
\item
Elles se promenaient sur la plage, quand elles a vu un garçon qui jouait de la guitare.
\item
Aline a demandé au garçon comment il s'appelait.
\item
Il a répondu qu'il s'appelait Lucas.
\item
Aline et Lucas sont devenus copains et s'entendaient très bien en général.
\item
Mais un jour, Lucas a voulu faire de la planche à voile et Aline, qui avait peur, n'a pas voulu.
\item
Alors, Ils se sont disputés et ont rompu.
\item
Heureusement, ses amies étaient là et elles l'ont consolée.
\end{enumerate}
\end{document}
