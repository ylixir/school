\documentclass[letterpaper]{article}

\usepackage{fullpage}
\usepackage{nopageno}
\usepackage{amsmath}
\usepackage{amssymb}
\usepackage[utf8]{inputenc}
\usepackage{aeguill}
\usepackage{setspace}
\allowdisplaybreaks

\newcommand{\abs}[1]{\left\lvert #1 \right\rvert}

\begin{document}
\title{Les devoirs}
\date{1 décembre, 2014}
\author{Jon Allen}
\maketitle
Écrivez les exercices 2 et 3 à la page 174
\section*{2}
\begin{spacing}{3.0}
\begin{enumerate}
\item
As-tu de l'argent dans ton sac?
\item
Nous demandons du respect de leur part.
\item
J'ai acheté des olives pour la salade de ce soir.
\item
Il n'y a plus d'asperges dans le frigo.
\item
Ces personnes donnent de la valeur à cet objet.
\item
Vous n'avez jamais eu de courage dans votre vie!
\end{enumerate}
\end{spacing}
\section*{3}
\begin{spacing}{3.0}
\begin{enumerate}
\item
Ce pays a beaucoup de gens.
\item
Je ne veux plus manger de pommes de terre.
\item
Je sais que la moitié des tout choses.
\item
Notre peuple a peu de l'argent.
\item
Veux-tu que je donne de la cadeaux.
\item
Mes amis ont manqué quelques examens.
\item
La population de notre État a trop de gens blancs.
\item
Nous sommes sortis pour acheter une boîte de vin.
\end{enumerate}
\end{spacing}
\end{document}
