\documentclass[letterpaper]{article}

\usepackage{fullpage}
\usepackage{nopageno}
\usepackage{amsmath}
\allowdisplaybreaks

\newcommand{\abs}[1]{\left\lvert #1 \right\rvert}

\begin{document}
\title{Notes}
\date{August 28, 2014}
\maketitle
session id 1561

\section*{on the docket}
\begin{enumerate}
\item
define health and wellness
\item
identify and explore the different dimensions of wellness

\end{enumerate}
\section*{review}
emails: include t/th in subject heading so he knows which session to look at

clicker question: what team is ndsu playing? 
textbook is? an invitation to health

\section*{what does health and wellness look like?}
rhetorical, stupid waste of time slides...

\subsection*{health}
who definition:

health is a state of complete physical, mental and social wellbeing and not merely the absence of

book:
a state of complete well-being including physical, psychological, spiritual social intellectual and enviromental dimensions

\subsection*{difference between health and wellness}
state vs a trait

health is the outcome

wellness is a verb

health:
\begin{itemize}
\item
sense of control
\item
energy and freedom from serious illness
\item
personal satisfaction
\item
fitness
\item
other?
\end{itemize}
wellness:
\begin{itemize}
\item
intentional
\item
lifestyle
\item
process
\item
integrates body, mind, spirit
\end{itemize}
\section*{elevator discussion}
dimensions of health

what do you do regularly

\section*{dimensions of health}
as per ndsu:

emotional, environmental, spiritual, physical, social, occupational, intellectual

\emph{test}
as per book:

physical, psychological, spiritual, social, intellectual, environmental

the national wellness institute also has a list of dimensions, but we are rolling with book

\subsubsection*{physical}
diet and exercise, self-care, sleep, risky behaviors
\subsubsection*{psychological}
emotional awareness, self-efficacy, coping, stress
\subsubsection*{spiritual}
purpose and meaning, perspective, giving and receiving
\subsubsection*{social}
relationships, roles, identity, community involvement
\subsubsection*{intellectual}
problem solving, learning, goal setting, resilience

key concept is resilience, confidence in problem solving ability, handling out of control situations
\subsubsection*{environmental}
harmful elements, occupational health, natural resource stewardship

\section*{'murica}
key areas of disparity  in u.s.: race and ethnicity, gender, socio-economic status

clicker: highest rate of diabetes: american indian

clicker: greater risk for chronic disease such as arthritis or an autoimmune disease: women

clicker: adults in us with highest bmi in any developped country is 20-34

\subsection*{health disparity in u.s.}
\subsubsection*{race and ethnicity}
cancer mortality, diabetes, cvd (cerebro vascular disease) and stroke
\subsubsection*{gender/sex}
overweight, mental health, mortality, life expectancy
\subsubsection*{socioeconomic status}
obesity, tobacco use/alcohol abuse, overall life expectancy

\subsection*{u.s. lags behind where?}
\begin{itemize}
\item
birth outcomes
\item
life expectancy
\item
injuries and homicides
\item
teen  pregnancy and sexually transmitted infections
\item
hiv and aids
\item
mortality and morbidity of metabolic conditions and cvd
\end{itemize}
places where we are going up or down. this is not improving, but literally up or down, aka weight is increasing over time.
\begin{itemize}
\item
up: fitness, people are getting more fit but weight is going up?
\item
down: weight
\item
down: overall health
\item
down: medical conditions
\item
down: health care
\item
up mortality
\end{itemize}

top causes of death in young adults in u.s.?

1)accidents, 2)violence(homicide/assaultt), 3)suicide

\section*{american college health association}
healthy people 2020

health on campus
\subsection*{impediments to academic performance}
stress, sleep, anxiety, infectious disease, work
\subsection*{other objectives}
campus safety, mental health, sti prevention, reduced risky behavior, pa(physical activity) and nutrition
\subsection*{healthy campus 2020}
\section*{making a change}
modifiable vs. non-modifiable.
\subsection*{cant control}
age, sex, race/ethnicity, family history, genetics
\subsection*{can control}
diet, physical activity, inactivity, risky behaviors
\subsubsection*{risky behaviors}
binge drinking, smoking, unprotected sex, poor diet, lack of exercise, lack of sleep
\subsubsection*{life extending behaviors}
not smoking, eating lots of fruits and vegetables (whole foods), exercising regularly, drinking alcohol in moderation
\subsection*{predisposing, reinforcing, enabling factors}
\subsubsection*{predisposing}
knowledge, attitude, beliefs, values, persceptions
\subsubsection*{reinforcing}
praise from others, rewards, encouragement, recognition, sense of achievement
\subsubsection*{enabling}
skills, resources, accessible facilities, physical capabilities, mental capabilities
\subsection*{behavioral theories of change}
health belief model, self-determination model, transtheoretical model
\subsubsection*{health belief model}
people believe they can make a change if...

1) feel they can avoid a negative consequence (locus of control)
2)expect a positive outcome (optimism)
3)believe that they can successfully take action (self-efficacy)

\subsubsection*{locus of control}
continuum from stuff happens to me to master of my own fate
\end{document}
