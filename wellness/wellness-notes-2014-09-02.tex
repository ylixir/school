\documentclass[letterpaper]{article}

\usepackage{fullpage}
\usepackage{nopageno}
\usepackage{amsmath}
\allowdisplaybreaks

\newcommand{\abs}[1]{\left\lvert #1 \right\rvert}

\begin{document}
\title{Notes}
\date{September 2, 2014}
\maketitle

\section*{health belief model}
people believe that they can make a change if
\begin{enumerate}
\item
feel the can avoid a negative consequence \emph{locus of control}
\item
expect a positive outcome 
\item
self efficacy
\end{enumerate}
\section*{self determination model}
\begin{enumerate}
\item
amotivated

does not value a behavior and/or does not believe it will lead to a desirable outcome
\item
externally motivated

engage in behavior to gain positive outcome or avoid negative outcome
\item
intrinsically motivated

engage in behavior because it is fun; most likely to succeed at behavior
\end{enumerate}
\section*{transtheoretical model}
most discussed in book
\begin{enumerate}
\item
stages of change

six stages which are not linear in sequence--often repeat and cycle through
\item
processes of change

cognitive and behavioral activities that facilitate change
\item
self-efficacy

belief in one's ability to accomplish a goal or change a behavior
\end{enumerate}

\subsection*{example:physical activity}

\begin{enumerate}
\item
precontemplative stage

weighing cost-benefit, etc

\item
contemplation
\item
preparation

procrastination, etc
\item
action

behavior begins
\item
maintenance
\item
termination
\end{enumerate}

\section*{wellness in action}
\subsection*{locus of control}

\section*{think tank}

\section*{end of review}
sesion id 1864

\subsubsection*{clicker}
which is not a separate dimension of wellness? financial

psychological, social, environmental all are

\subsubsection*{clicker}
a student wants to exercise more frequently and starts using the bicycle

pre

\subsubsection*{clicker}
what percentage of americans currently  take anti depressant medicine?

1/11 (10\%)
\subsubsection*{clicker}
heart disease and depression increase likelyhood of eachother

true


\subsubsection*{clicker}
strongest predictor of depression in teenagers

smoking cigarettes

\section*{the brain: the last frontier}
main sections:cerebellum (little ball in the back), cerebrum (main big part of brain)

\subsection*{hindbrain}
includes cerebellum, medulla, pons.

called ``little brain''
\subsection*{midbrain}
top part of brain stem
coordination, regulation of movement
\subsection*{front brain}
most advanced part of brain

emotion, reasoning, vision, hearing, memory, though, voluntary movement

perietal love: sensory

occipital love: vision

temporal lobe: hearing and some memory

cerebrum: cerebral hemispheres(lobes)

\subsection*{inner brain}
emotional state/fight or flight

most advanced part of the brain

hypothalamus, thalamus, hippocampus, basal ganglia

\subsection*{neurons}
four main parts

nucleus (part of soma)

axon

axon terminal

dendrite


\subsubsection*{what they do}
chemical and electrical impulses are used to communicate

basic working unit of the brain

transmits information incoming outgoing

independent from other neurons

may  form thousands of connections
\subsubsection*{parts of a neuron}
\begin{enumerate}
\item
cell body

contains nucleus
\item
axon

long fiber that carries electircal signal
\item
axon terminal

where the axon ends and its signal is transferred to the dendrite of a different neuron
\item
dendrite

shorter fiber that receives signal from the axon terminal
\end{enumerate}
\subsubsection*{how the action happens}
where: synapses (synaptic cleft): area between the axon terminal of one neuron and the dendrite of another neuron

how: electrical$\rightarrow$chemical (signal propagation)

neurotransmitters: packets of chemicals released from the axon terminal of the neuron sending the signal to the dendrites of neighboring neuron

\subsubsection*{neurotransmitters}
common examples:seratonin, dopamine, acetylcholine, GABA, etc

malfunction: production, release, binding, reuptake

some drugs that affect neurotransmission: caffeine, nicotine, alcohol, and cocaine

shorter fiber that receives signal from the axon terminal

\subsubsection*{summary}
part of the central nervous system

differences between male and female:
size and neural networking, sensory perception, emotional response, ``intelligence'' equal

consists of 3 main sections

neurons send/receive messages

neurotransmitters enable brain signals

continues developing into young adulthood

\section*{understanding mental health}
a mentally healthy individual:

establishes and maintains close relationships

carries out responsibilities

values himself/herself

pursues work that suits talents and training

accepts own limitations and possibilities

feels a sense of fulfillment in daily living

perceives reality as it is
\subsection*{apa/government definition of mental disorder}
``clinically significant behavioral or psychological syndrome or pattern that is associated with presesent...''

book: `behavioral or psychological syndrome associated with distress or disability or with a significantly increase risk of suffering death, pain, disability, loss of freedom


\subsection*{depressive disorders}
most common mental health disorder

ke contributors in collega-age:
\begin{itemize}
\item
stress
\item
substance abuse
\item
sleep loss
\end{itemize}

gender
\begin{itemize}
\item
more ``common'' in females
\item
``under'' disease in males
\item
video
\end{itemize}
\subsubsection*{groups more likely to experience depression}
women, racial and ethnic minorities, those without...


\subsubsection*{disorders}
major depressive disorder v. dysthymia: duration, serverity of symptoms

80\% recurrence

symptoms listed in book
\begin{itemize}
\item
ferwer or extreme feeling

depressed, helpess, hopless, restless or slow, no interest in pleasurable activities, physical symptoms
\item
alteration in thinking
\item
\end{itemize}
\subsubsection*{anxiety disorders}
phobies, panic attacks, generalized anxiety disorder

generalized anxiety disorder (GAD), an anxiety disorder characterized as chronic distress, worry is blogal, irrational and constant, physical symptoms: restlessness, fatigue, stress responses, etc.

\subsubsection*{ocd}
obsessive-compulsive disorders

an anxiety disorder characterized by osbsessions and or compulsions that impair one's ability to function and form relationships

obsession+compulsion

examples: hand washing, checking somthing (door shut....)

\subsubsection*{attentional disorders}
attention deficit hyperactivity disorder

autism: repetitive patterns of thoughts and behavior, no verbal communication before age 3, four times more likely to occur in boys than girls.  treatment: behavior therapy, speech-language therapy, physical therapy, school-based educational programs.

schizophrenia:

affects every aspect of psychological functioning: hallucinations, and delusions

suicide:

book talks about it as not a mental health disorder, but cdc and dsm categorizes as such, act is not disorder, but mental state is

highest rate of attempt and completion in oung adults (18-25)

over 1000 annually

attempt rate higher in females, success rate higher in males

factors that are related: previous attempts, substance use/abuse, combat stress, family history, ``life events'', physical and mental health, access to guns

sex differences:

females attempt, males commit, more attempts under 35, more success under 20 or over 60

friends are often the first to notice signs of concern or even suicidal signs in their friends, in one study, 80\% of teen suicidal fatalitiies had peers that knew their friend was suicidal, but chose not to seek out older adults for help

reasons for silenc: not taking threats seriously, thinking their  frinend will get in trouble, afraid their friend will be angry


ndsu cares project, website, blah blah
\end{document}
