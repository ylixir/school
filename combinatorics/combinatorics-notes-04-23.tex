\documentclass{article}
\usepackage{fullpage}
\usepackage{nopageno} 
\usepackage{amsmath}
\usepackage{amssymb}
\allowdisplaybreaks

\newcommand{\abs}[1]{\left\lvert #1 \right\rvert}

\begin{document}
\title{Notes}
\date{April 26, 2014}
\maketitle
homework questions
catalan problems (1,2,36), could prove formula directly but that can be hard, easier is write bijection from objects to catalan numbers.
1)b to d, 2)a to d, 36)c to d

\subsubsection*{worksheet}

prove \# length 2n ballot seq$=\frac{1}{n+1}\binom{2n}{n}$

a)$\binom{2n}{n}$

b)$A_n+U_n=\binom{2n}{n}$

c)$a_k=-1$, $a_1+a_2+\dots+a_{k-1}=0$

d)is $k$ odd or even? odd

e)one more one, one less negative one. this process gives all possible sequences ofn+1 1's and n-1 -1's. it is really a bijection from $A_n+U_n$ to n+1 1's and n-1 -1's. so $\abs{U_n}=\binom{2n}{n+1}$, $A_n=\binom{2n}{n}-\binom{2n}{n+1}$

this leads to a catalan number.


f)

\section*{sterling numbers}
\subsubsection*{warm up}
how many ways are there to partition $\{1,2,\dots,p\}$ into $k$ indistinguishable boxes?

$k$ choices for each object in set. $k^p$. but the boxes here are distinguisheable. divide by the number of permutations to eliminate distinguishability. $\frac{k^p}{k!}$.

\subsubsection*{question}
what if no box may be left empty? (choose k elements, then distribute)
\subsubsection*{answer}
use inclusion exclusion. let $A_i$=number of distributions where box i is empty.
\begin{align*}
  
\end{align*}
the bell number$B_p$ counts \# partitions of $\{1,\dots,p\}$ into nonempty indistinguishable boxes. This is sum of stirling numbers, sum of pth row of the triangle.
\end{document}
