\documentclass{article}
\usepackage{fullpage}
\usepackage{nopageno} 
\usepackage{amsmath}
\usepackage{amssymb}
\usepackage[normalem]{ulem}
\allowdisplaybreaks

\newcommand{\abs}[1]{\left\lvert #1 \right\rvert}

\begin{document}
Jon Allen

February 12, 2014

\section*{14}
Generate the 6-tuples of 0s and 1s by using the base 2 arithmetic generating scheme and identify them with subsets of the set $\{x_5,x_3,x_2,x_1,x_0\}$.

000000$\to\emptyset$ 000001$\to\{x_0\}$
000010$\to\{x_1\}$ 000011$\to\{x_1,x_0\}$
000100$\to\{x_2\}$ 000101$\to\{x_2,x_0\}$
000110$\to\{x_2,x_1\}$ 000111$\to\{x_2,x_1,x_0\}$
001000$\to\{x_3\}$ 001001$\to\{x_3,x_0\}$
001010$\to\{x_3,x_1\}$ 001011$\to\{x_3,x_1,x_0\}$
001100$\to\{x_3,x_2\}$ 001101$\to\{x_3,x_2,x_0\}$
001110$\to\{x_3,x_2,x_1\}$ 001111$\to\{x_3,x_2,x_1,x_0\}$
010000$\to\{x_4\}$ 010001$\to\{x_4,x_0\}$
010010$\to\{x_4,x_1\}$ 010011$\to\{x_4,x_1,x_0\}$
010100$\to\{x_4,x_2\}$ 010101$\to\{x_4,x_2,x_0\}$
010110$\to\{x_4,x_2,x_1\}$ 010111$\to\{x_4,x_2,x_1,x_0\}$
011000$\to\{x_4,x_3\}$ 011001$\to\{x_4,x_3,x_0\}$
011010$\to\{x_4,x_3,x_1\}$ 011011$\to\{x_4,x_3,x_1,x_0\}$
011100$\to\{x_4,x_3,x_2\}$ 011101$\to\{x_4,x_3,x_2,x_0\}$
011110$\to\{x_4,x_3,x_2,x_1\}$ 011111$\to\{x_4,x_3,x_2,x_1,x_0\}$
100000$\to\{x_5\}$ 100001$\to\{x_5,x_0\}$
100010$\to\{x_5,x_1\}$ 100011$\to\{x_5,x_1,x_0\}$
100100$\to\{x_5,x_2\}$ 100101$\to\{x_5,x_2,x_0\}$
100110$\to\{x_5,x_2,x_1\}$ 100111$\to\{x_5,x_2,x_1,x_0\}$
101000$\to\{x_5,x_3\}$ 101001$\to\{x_5,x_3,x_0\}$
101010$\to\{x_5,x_3,x_1\}$ 101011$\to\{x_5,x_3,x_1,x_0\}$
101100$\to\{x_5,x_3,x_2\}$ 101101$\to\{x_5,x_3,x_2,x_0\}$
101110$\to\{x_5,x_3,x_2,x_1\}$ 101111$\to\{x_5,x_3,x_2,x_1,x_0\}$
110000$\to\{x_5,x_4\}$ 110001$\to\{x_5,x_4,x_0\}$
110010$\to\{x_5,x_4,x_1\}$ 110011$\to\{x_5,x_4,x_1,x_0\}$
110100$\to\{x_5,x_4,x_2\}$ 110101$\to\{x_5,x_4,x_2,x_0\}$
110110$\to\{x_5,x_4,x_2x_1\}$ 110111$\to\{x_5,x_4,x_2x_1,x_0\}$
111000$\to\{x_5,x_4,x_3\}$ 111001$\to\{x_5,x_4,x_3,x_0\}$
111010$\to\{x_5,x_4,x_3,x_1\}$ 111011$\to\{x_5,x_4,x_3,x_1,x_0\}$
111100$\to\{x_5,x_4,x_3,x_2\}$ 111101$\to\{x_5,x_4,x_3,x_2,x_0\}$
111110$\to\{x_5,x_4,x_3,x_2,x_1\}$ 111111$\to\{x_5,x_4,x_3,x_2,x_1,x_0\}$
\section*{16}
For each of the subsets (a), (b), (c), and (d) in the preceding exercise, determine the subset that immediately \emph{precedes} it in the base 2 arithmetic generating scheme.
\subsection*{(a)}
$\{x_4,x_1,x_0\}=00010011\gets00010010$, or $\{x_4,x_1\}$.
\subsection*{(b)}
$\{x_7,x_5,x_3\}=10101000\gets10100111$, or $\{x_7,x_5,x_2,x_1,x_0\}$
\subsection*{(c)}
$\{x_7,x_5,x_4,x_3,x_2,x_1,x_0\}=10111111\gets10111110$, or $\{x_7,x_5,x_4,x_3,x_2,x_1\}$
\subsection*{(d)}
$\{x_0\}=00000001\gets00000000$, or $\emptyset$
\section*{17}
Which subset of $\{x_7,x_6,\dots,x_1,x_0\}$ is 150th on the list of subsets of $S$ when the base 2 arithmetic generating scheme is used? 200th? 250th? (As in Section 4.3, the places on the list are numbered beginning with 0.)
\begin{align*}
  2^0&=1&
  2^1&=2&
  2^2&=4&
  2^3&=8&
  2^4&=16&
  2^5&=32&
  2^6&=64&
  2^7&=128
\end{align*}
\begin{align*}
  150&=128+16+4+2\to\{x_7,x_4,x_2,x_1\}&
  200&=128+64+8\to\{x_7,x_6,x_3\}\\
  250&=128+64+32+16+8+2\to\{x_7,x_6,x_5,x_4,x_3,x_1\}
\end{align*}
\section*{22}
Determine the reflected Gray code of order 6.

000000 000001 000011 000010 000110 000111 000101 000100
001100 001101 001111 001110 001010 001011 001001 001000
011000 011001 011011 011010 011110 011111 011101 011100
010100 010101 010111 010110 010010 010011 010001 010000
110000 110001 110011 110010 110110 110111 110101 110100
111100 111101 111111 111110 111010 111011 111001 111000
101000 101001 101011 101010 101110 101111 101101 101100
100100 100101 100111 100110 100010 100011 100001 100000

\section*{24}
Determine the predecessors of each of the 9-tuples in Exercise 23 in the reflected Gray code of order 9.
\subsection*{(a)}
$010100110\gets010100010$
\subsection*{(b)}
$110001100\gets110000100$
\subsection*{(c)}
$111111111\gets111111110$
\section*{27}
Generate the 2-subsets of $\{1,2,3,4,5,6\}$ in lexicographic order by using the algorithm described in Section 4.4.
\[\begin{array}{ccc}
  12&23&35\\
  13&24&36\\
  14&25&45\\
  15&26&46\\
  16&34&56\\
\end{array}\]
\section*{29}
Determine the 7-subset of $\{1,2,\dots,15\}$ that immediately follows 1,2,4,6,8,14,15 in the lexicographic order. Then determine the 7-subset that immediately precedes 1,2,4,6,8,14,15.

Since 14 and 15 are as high as we can go we increment the 8 and start counting from there.
\[1,2,4,6,8,14,15\text{ is followed by }1,2,4,6,9,10,11\]

Since we can't decrement the 15 to 14 because we already have a 14 we decrement the 14 to 13 and leave the 15 since it is the max and we want it to roll over on the next count up.
\[1,2,4,6,8,14,15\text{is preceded by}1,2,4,6,8,13,15\]

\section*{31}
Generate the 3-permutations of $\{1,2,3,4,5\}$
\begin{align*}
  \begin{array}{cccccccccc}
    123&124&125&134&135&145&234&235&245&345\\
    132&142&152&143&153&154&243&253&254&354\\
    312&412&512&413&513&514&423&523&524&534\\
    321&421&521&431&531&541&432&532&542&543\\
    231&241&251&341&351&451&342&352&452&453\\
    213&214&215&314&315&415&324&325&425&435\\
  \end{array}
\end{align*}
\section*{33}
In which position does the subset 2489 occur in the lexicographic order of the 4-subsets of $\{1,2,3,4,5,6,7,8,9\}$?

Using theorem 4.4.2 we have:
\[\binom{9}{4}-\binom{7}{4}-\binom{5}{3}-\binom{1}{2}-\binom{0}{1}=\frac{9!}{4!(9-4)!}-\frac{7!}{4!(7-4)!}-\frac{5!}{4!(5-3)!}-0-0=81\]

\section*{34}
Consider the r-subsets of $\{1,2,\dots,n\}$ in lexicographic order.
\subsection*{(a)}
What are the first $(n-r+1)$ $r$-subsets?
\subsection*{(b)}
What are the last $(r+1)$ $r$-subsets?
\section*{35}
The \emph{complement} $\bar{A}$ of an $r$-subset $A$ of $\{1,2,\dots,n\}$ is the $((n-r)$-subset of $\{1,2,\dots,n\}$, consisting of all those elements that do not belong to $A$. Let $M=\binom{n}{r}$, the number of $r$-subsets and, at the same time, the number of $(n-r)$-subsets of $\{1,2,\dots,n\}$. Prove that, if
\[A_1,A_2,A_3,\dots,A_M\]
are the $r$-subsets in lexicographic order, then
\[\overline{A_M},\dots,\overline{A_3},\;\overline{A_2},\;\overline{A_1}\]
are the $(n-r)$-subsets in lexicographic order.
\end{document}
