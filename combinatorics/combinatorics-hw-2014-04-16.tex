\documentclass{article}
%\usepackage{fullpage}
%\usepackage{nopageno} 
\usepackage[margin=1.5in]{geometry}
\usepackage{amsmath}
\usepackage{amssymb}
\usepackage[normalem]{ulem}
\usepackage{fancyhdr}
\usepackage{cancel}
%\renewcommand\headheight{12pt}
\pagestyle{fancy}
\lhead{April 16, 2014}
\rhead{Jon Allen}
\allowdisplaybreaks

\newcommand{\abs}[1]{\left\lvert #1 \right\rvert}

\begin{document}
Read Sections 7.2 and 7.3 and do the following problems from Chapter 7: \#16, 17, 18, 19, 22, 24(a)(b)(c) (be sure to write your answer in closed form), 26, 28, 50 (Grad: 49).
\begin{enumerate}
\setcounter{enumi}{15}
\item
Formulate a combinatorial problem for which the generating function is
\[(1+x+x^2)(1+x^2+x^4+x^6)(1+x^2+x^4+\cdots)(x+x^2+x^3+\cdots)\]

How many combinations of red, yellow, green and blue colored pencils are there in which there are at most 2 red pencils, an even number of at most six yellow pencils, an even number of green pencils and at least one blue pencil?
\item
Determine the generating function  for the number $h_n$ of bags of fruit of apples, oranges, bananas, and pears in which there are an even number of apples, at most two oranges, a multiple of three number of bananas, and at most one pear. Then find a formula for $h_n$ from the generating function.

\begin{align*}
  g(x)&=\underbrace{\left(\sum\limits_{n=0}^\infty{x^{2n}}\right)}_{\text{even apples}}\underbrace{\left(\sum\limits_{n=0}^2{x^n}\right)}_{\text{oranges}\le 2}\underbrace{\left(\sum\limits_{n=0}^\infty{x^{3n}}\right)}_{3\mid \text{bananas}}\underbrace{\left(\sum\limits_{n=0}^1{x^{n}}\right)}_{\text{pears}\le 1}\\
  &=\frac{1}{1-x^2}\left[\left(\sum\limits_{n=0}^\infty{x^n}\right)-\left(\sum\limits_{n=3}^\infty{x^n}\right)\right]\frac{1}{1-x^3}\left[\left(\sum\limits_{n=0}^\infty{x^n}\right)-\left(\sum\limits_{n=2}^\infty{x^n}\right)\right]\\
  &=\frac{1}{1-x^2}\left[\left(1-x^3\right)\left(\sum\limits_{n=0}^\infty{x^n}\right)\right]\frac{1}{1-x^3}\left[\left(1-x^2\right)\left(\sum\limits_{n=0}^\infty{x^n}\right)\right]\\
  &=\left(\frac{1}{1-x^2}\right)\left(\frac{1-x^3}{1-x}\right)\left(\frac{1}{1-x^3}\right)\left(\frac{1-x^2}{1-x}\right)\\
  &=\frac{1}{(1-x)^2}
  =\frac{\mathrm{d}}{\mathrm{d}x}\left(\frac{1}{1-x}\right)
  =\frac{\mathrm{d}}{\mathrm{d}x}\sum\limits_{n=0}^\infty{x^n}\\
  &=\sum\limits_{n=0}^\infty{nx^{n-1}}
  =\sum\limits_{n=-1}^\infty{(n+1)x^{n}}
  =0+\sum\limits_{n=0}^\infty{(n+1)x^{n}}\\
  h_n&=n+1
\end{align*}
\item
Determine the generating function for the number $h_n$ of nonnegative integral solutions of
\[2e_1+5e_2+e_3+7e_4=n\]
This is equivalent to saying how many ways can you pick $n$ items from the multiset $\{\infty\cdot e_1,\infty\cdot e_2,\infty\cdot e_3,\infty\cdot e_4\}$ where you have an even number of $e_1$'s, a multiple of 5 $e_2$'s and a multiple of 7 $e_4$'s.
\begin{align*}
  g(x)&=\left(\sum\limits_{n=0}^\infty{x^{2n}}\right)\left(\sum\limits_{n=0}^\infty{x^{5n}}\right)\left(\sum\limits_{n=0}^\infty{x^{n}}\right)\left(\sum\limits_{n=0}^\infty{x^{7n}}\right)\\
  &=\left(\frac{1}{1-x^2}\right)\left(\frac{1}{1-x^5}\right)\left(\frac{1}{1-x}\right)\left(\frac{1}{1-x^7}\right)
\end{align*}
\item
Let $h_0,h_1,h_2,\dots,h_n,\dots$ be the sequence defined by $h_n=\binom{n}{2},(n\ge0)$. Determine the generating function for the sequence.
\begin{align*}
  g(x)&=\sum\limits_{n=0}^\infty{\binom{n}{2}x^n}\\
  &=\sum\limits_{n=0}^\infty{\frac{n!}{2!(n-2)!}x^n}\\
  &=\frac{1}{2}\sum\limits_{n=0}^\infty{n(n-1)x^n}\\
  h_n&=\frac{1}{2}(n^2-n)
\end{align*}
\setcounter{enumi}{21}
\item
Determine the exponential generating function for the sequence of factorials: $0!,1!,2!,3!,\dots,n!,\dots$
\begin{align*}
  g^{(e)}(x)&=\sum\limits_{n=0}^\infty{n!\cdot\frac{x^n}{n!}}\\
  &=\sum\limits_{n=0}^\infty{x^n}\\
  &=\frac{1}{1-x}
\end{align*}
\setcounter{enumi}{23}
\item
Let $S$ denote the multiset $\{\infty\cdot e_1,\infty\cdot e_2,\dots,\infty\cdot e_k\}$. Determine the exponential generating function for the sequence $h_0,h_1,h_2,\dots,h_n,\dots$, where $h_0=1$ and, for $n\ge 1$
\begin{enumerate}
\item
$h_n$ equals the number of $n$-permutations of $S$ in which each object occurs an odd number of times.
\begin{align*}
  g^{(e)}(x)&=\left(x+\frac{x^3}{3!}+\frac{x^5}{5!}+\dots\right)^k\\
  &=\left[\left(1+x+\frac{x^2}{2!}+\frac{x^3}{3!}+\dots\right)+\left(-1+x-\frac{x^2}{2!}+\frac{x^3}{3!}+\dots\right)\right]^k\frac{1}{2^k}\\
  &=\left(\frac{1}{2}\sum\limits_{n=0}^\infty{\frac{x^n+(-x)^n}{n!}}\right)^k\\
  &=\left(\frac{e^x+e^{-x}}{2}\right)^k
\end{align*}
\item
$h_n$ equals the number of $n$-permutations of $S$ in which each object occurs at least four times.
\begin{align*}
  g^{(e)}(x)&=\left(\frac{x^4}{4!}+\frac{x^5}{5!}+\frac{x^6}{6!}+\dots\right)^k\\
  &=\left(\sum\limits_{n=4}^\infty{\frac{x^n}{n!}}\right)^k=\left(-\frac{x^3}{3!}+\sum\limits_{n=3}^\infty{\frac{x^n}{n!}}\right)^k\\
  &=\left(-1-x-\frac{x^2}{2!}-\frac{x^3}{3!}+\sum\limits_{n=3}^\infty{\frac{x^n}{n!}}\right)^k\\
  &=\left(e^x-1-x-\frac{x^2}{2!}-\frac{x^3}{3!}\right)^k
\end{align*}
\item
$h_n$ equals the number of $n$-permutations of $S$ in which $e_1$ occurs at least once, $e_2$ occurs at least twice, $\dots,e_k$ occurs at least $k$ times.
\begin{align*}
  g^{(e)}(x)&=\left(x+\frac{x^2}{2!}+\dots\right)\left(\frac{x^2}{2!}+\frac{x^3}{3!}+\dots\right)\dots\left(\frac{x^{k-1}}{(k-1)!}+\frac{x^k}{k!}+\dots\right)\\
  &=\prod_{i=1}^k{\left(\sum\limits_{n=i}^\infty{\frac{x^n}{n!}}\right)}\\
  &=\prod_{i=1}^k{\left(\sum\limits_{n=0}^\infty{\frac{x^n}{n!}}-\sum\limits_{n=0}^{i-1}{\frac{x^n}{n!}}\right)}\\
  &=\prod_{i=1}^k{\left(e^x-\sum\limits_{n=0}^{i-1}{\frac{x^n}{n!}}\right)}\\
\end{align*}
\end{enumerate}
\setcounter{enumi}{25}
\item
Determine the number of ways to color the squares of a 1-by-$n$ chessboard, using the colors red, blue, green, and orange if an even number of squares is to be colored red and an even number is to be colored green.
\begin{align*}
  g^{(e)}(x)&=\underbrace{\left(\sum\limits_{n=0}^\infty{\frac{x^{2n}}{(2n)!}}\right)^2}_{\text{red and green are even}}\cdot\overbrace{\left(\sum\limits_{n=0}^\infty{\frac{x^{n}}{n!}}\right)^2}^{\text{blue and orange are unrestricted}}\\
  &=\left(\frac{1}{2}\sum\limits_{n=0}^\infty{\frac{x^n+(-x)^n}{n!}}\right)\left(e^x\right)^2\\
  &=\frac{e^{2x}}{2}\left(\sum\limits_{n=0}^\infty{\frac{x^n}{n!}}+\sum\limits_{n=0}^\infty{\frac{(-x)^n}{n!}}\right)\\
  &=\frac{e^{2x}}{2}\left(e^x+e^{-x}\right)=\frac{e^{3x}+e^x}{2}\\
  &=\frac{1}{2}\sum\limits_{n=0}^\infty{\frac{(3x)^n+x^n}{n!}}=\sum\limits_{n=0}^\infty{\frac{3^n+1}{2}\cdot\frac{x^n}{n!}}
\end{align*}
So we can color the board $\frac{3^n+1}{2}$ ways
\setcounter{enumi}{27}
\item
Determine the number of $n$-digit numbers with all digits at least 4, such that 4 and 6 each occur an even number of times, and 5 and 7 each occur at least once, there being no restriction on the digits 8 and 9.
\begin{align*}
  g^{(e)}(x)&=\underbrace{\left(\sum\limits_{n=0}^\infty{\frac{x^{2n}}{(2n)!}}\right)^2}_{\text{4,6 are even}}\cdot\overbrace{\left(\sum\limits_{n=1}^\infty{\frac{x^n}{n!}}\right)^2}^{\text{5,7 occur at least once}}\cdot\underbrace{\left(\sum\limits_{n=0}^\infty{\frac{x^n}{n!}}\right)^2}_{\text{8,9 are unrestricted}}\\
  &=\left(\frac{1}{2}\sum\limits_{n=0}^\infty{\frac{x^n+(-x)^n}{n!}}\right)^2\left(-1+\sum\limits_{n=0}^\infty{\frac{x^n}{n!}}\right)^2\left(\sum\limits_{n=0}^\infty{\frac{x^n}{n!}}\right)^2\\
  &=\frac{e^{2x}}{4}\left(e^x+e^{-x}\right)^2(e^x-1)^2\\
  &=\frac{e^{2x}}{4}\left(e^{2x}+2+e^{-2x}\right)(e^{2x}-2e^x+1)\\
  &=\frac{e^{2x}}{4}\left(e^{4x}-2e^{3x}+e^{2x}+2e^{2x}-4e^x+2+1-2e^{-x}+e^{-2x}\right)\\
  &=\frac{e^{2x}}{4}\left(e^{4x}-2e^{3x}+3e^{2x}-4e^x+3-2e^{-x}+e^{-2x}\right)\\
  &=\frac{1}{4}\left(e^{6x}-2e^{5x}+3e^{4x}-4e^{3x}+3e^{2x}-2e^{x}+1\right)\\
  &=\frac{1}{4}+\frac{1}{4}\sum\limits_{n=0}^\infty{\frac{(6x)^n-2(5x)^n+3(4x)^n-4(3x)^n+3(2x)^n-2x^n}{n!}}\\
  &=\frac{1}{4}+\frac{1}{4}\sum\limits_{n=0}^\infty{\left[6^n-2(5^n)+3(4^n)-4(3^n)+3(2^n)-2\right]\frac{x^n}{n!}}
\end{align*}
So the number of $n$-digit numbers that meet the criteria is 0 when $n=0$ and $\frac{1}{4}\left[6^n-2(5^n)+3(4^n)-4(3^n)+3(2^n)-2\right]$ when $n\ge1$
\setcounter{enumi}{49}
\item
Call a subset of $S$ of the integers $\{1,2,\dots,n\}$ \emph{extraordinary} provided its smallest integer equals its size:
\[\min\{x:x\in S\}=\lvert S\rvert.\]
For example, $S=\{3,7,8\}$ is extraordinary. Let $g_n$ be the number of extraordinary subsets of $\{1,2,\dots,n\}$. Prove that
\[g_n=g_{n-1}+g_{n-2},\quad(n\ge3),\]
with $g_1=1$ and $g_2=1$.
\subsubsection*{proof}
The number of extraordinary subsets of $\{1,2,\dots,n\}$ is the sum of the number of extraordinary subsets of each size from 1 to $n$. A $k$ sized extraordinary subset of $\{1,2,\dots,n\}$ contains $k$ and $k-1$ elements of the set $\{k+1,k+2,\dots,n\}$. We can choose these $k-1$ elements in $\binom{n-(k+1)+1}{k-1}$ ways. Lets simplify a little:
\begin{align*}
  \binom{n-(k+1)+1}{k-1}&=\binom{n-k}{k-1}\\
  g_n&=\sum\limits_{k=1}^n{\binom{n-k}{k-1}}\\
  \binom{n+1}{k+1}&=\binom{n}{k}+\binom{n}{k+1}\\
  g_n&=\sum\limits_{k=1}^n{\binom{n-k-1}{k-1-1}}+\sum\limits_{k=1}^n{\binom{n-k-1}{k-1}}\\
  &=\binom{n-n-1}{n-2}+\sum\limits_{k=1}^{n-1}{\binom{n-k-1}{k-1-1}}+\binom{n-n-1}{n-1}+\sum\limits_{k=1}^{n-1}{\binom{n-k-1}{k-1}}\\
  &=\frac{(n-n-1)!}{(n-n-1-(n-2))!(n-2)!}+\frac{(n-n-1)!}{(n-n-1-(n-1))!(n-1)!}\\&\quad+\sum\limits_{k=1}^{n-1}{\binom{n-k-1}{k-1-1}}+\sum\limits_{k=1}^{n-1}{\binom{(n-1)-k}{k-1}}\\
  &=\frac{(-1)!}{(1-n)!(n-2)!}+\frac{(-1)!}{(-n)!(n-1)!}\\&\quad+\sum\limits_{k=1}^{n-1}{\binom{n-(k-1)}{(k-1)-1}}+g_{n-1}\\
  &=\frac{(-1)!(n-1)}{(1-n)!(n-1)!}+\frac{(-1)!(1-n)}{(1-n)!(n-1)!}\\&\quad+\sum\limits_{k=0}^{n-2}{\binom{n-k}{k-1}}+g_{n-1}\\
  &=\frac{(-1)!n-(-1)!+(-1)!-(-1)!n}{(1-n)!(n-1)!}\\&\quad+\binom{n}{-1}+\sum\limits_{k=1}^{n-2}{\binom{n-k}{k-1}}+g_{n-1}\\
  &=0+\sum\limits_{k=1}^{n}{\binom{n-k}{k-1}}-\binom{n-(n-1)}{(n-1)-1}-\binom{n-n}{n-1}+g_{n-1}\\
  &=\sum\limits_{k=1}^{n}{\binom{n-k}{k-1}}-\binom{1}{n-2}-\binom{0}{n-1}+g_{n-1}\\
%  &=\sum\limits_{k=1}^{n-1}{\binom{n-k}{k-1}}+\binom{n-n}{n-1}
\end{align*}
%Now if we have $g_n$, then what is $g_{n-1}$? We have 
%\begin{align*}
%  g_{n-1}&=\sum\limits_{k=1}^{n-1}{\binom{n-k-1}{k-1}}\\
%\end{align*}
%Now lets see what happens if we add $g_{n}$ and $g_{n+1}$
%\begin{align*}
%  \sum\limits_{k=1}^{n}{\binom{n-k}{k-1}}+\sum\limits_{k=1}^{n+1}{\binom{(n+1)-k}{k-1}}&=\sum\limits_{k=1}^{n}{\binom{n-k}{k-1}}+\sum\limits_{k=1}^{n}{\binom{(n+1)-k}{k-1}}+\binom{(n+1)-(n+1)}{(n+1)-1}
%\end{align*}
%lets make an assumption and see if it holds
%\begin{align*}
%  \sum\limits_{k=1}^{n+2}{\binom{(n+2)-k}{k-1}}&=\sum\limits_{k=1}^{n}{\binom{n-k}{k-1}}+\sum\limits_{k=1}^{n}{\binom{(n+1)-k}{k-1}}+\binom{0}{n}\\
%  \sum\limits_{k=1}^{n}{\binom{(n+2)-k}{k-1}}+\binom{0}{n+1}+\binom{1}{n}-\sum\limits_{k=1}^{n}{\binom{(n+1)-k}{k-1}}&=\sum\limits_{k=1}^{n}{\binom{n-k}{k-1}}\\
%  \sum\limits_{k=1}^{n}{\binom{(n+2)-k}{k-1}}-\sum\limits_{k=1}^{n}{\binom{(n+1)-k}{k-1}}&=\sum\limits_{k=1}^{n}{\binom{n-k}{k-1}}\\
%  \sum\limits_{k=1}^{n}{\frac{((n+2)-k)!}{(k-1)!((n+2)-k-(k-1))!}}\qquad&\\
%  -\sum\limits_{k=1}^{n}{\frac{((n+1)-k)!}{(k-1)!((n+1)-k-(k-1))!}}&=\sum\limits_{k=1}^{n}{\frac{(n-k)!}{(k-1)!(n-k(k-1))!}}\\
%  \sum\limits_{k=1}^{n}{\frac{(n-k+2)!}{(k-1)!(n-2k+3)!}}\qquad&\\
%  -\sum\limits_{k=1}^{n}{\frac{(n-k+1)!}{(k-1)!(n-2k+2))!}}&=\sum\limits_{k=1}^{n}{\frac{(n-k)!}{(k-1)!(n-2k+1))!}}\\
%  \sum\limits_{k=1}^{n}{\frac{(n-k+2)(n-k+1)(n-k)!}{(k-1)!(n-2k+3)(n-2k+2)(n-2k+1)!}}\qquad&\\
%  -\sum\limits_{k=1}^{n}{\frac{(n-k+1)(n-k)!}{(k-1)!(n-2k+2)(n-2k+1)!}}&=\sum\limits_{k=1}^{n}{\frac{(n-k)!}{(k-1)!(n-2k+1))!}}\\
%  \sum\limits_{k=1}^{n}{\left[\begin{aligned}\frac{(n-k+2)(n-k+1)}{(n-2k+3)(n-2k+2)}\\-\frac{(n-k+1)}{(n-2k+2)}\end{aligned}\right]\frac{(n-k)!}{(k-1)!(n-2k+1))!}}&=\sum\limits_{k=1}^{n}{\frac{(n-k)!}{(k-1)!(n-2k+1))!}}\\
%  \sum\limits_{k=1}^{n}{\left[\frac{\begin{aligned}(n-k+2)(n-k+1)\\-(n-k+1)(n-2k+3)\end{aligned}}{(n-2k+3)(n-2k+2)}\right]\frac{(n-k)!}{(k-1)!(n-2k+1))!}}&=\sum\limits_{k=1}^{n}{\frac{(n-k)!}{(k-1)!(n-2k+1))!}}\\
%  \sum\limits_{k=1}^{n}{\left[\frac{(n-k+1)(n-k+2-n+2k-3)}{(n-2k+3)(n-2k+2)}\right]\frac{(n-k)!}{(k-1)!(n-2k+1))!}}&=\sum\limits_{k=1}^{n}{\frac{(n-k)!}{(k-1)!(n-2k+1))!}}\\
%  \sum\limits_{k=1}^{n}{\left[\frac{(n-k+1)(k-1)}{(n-2k+3)(n-2k+2)}\right]\frac{(n-k)!}{(k-1)!(n-2k+1))!}}&=\sum\limits_{k=1}^{n}{\frac{(n-k)!}{(k-1)!(n-2k+1))!}}\\
%  \sum\limits_{k=1}^{n}{\left[\frac{1}{(n-2k+3)(n-2k+2)}\right]\frac{(n-k+1)!}{k!(n-2k+1))!}}&=\sum\limits_{k=1}^{n}{\frac{(n-k)!}{(k-1)!(n-2k+1))!}}\\
%  \sum\limits_{k=1}^{n}{\left[\frac{1}{(n-2k+3)(n-2k+2)}\right]\binom{n-k+1}{k}}&=\sum\limits_{k=1}^{n}{\binom{n-k}{k-1}}\\
%  \sum\limits_{k=2}^{n+1}{\left[\frac{\binom{(n-1)-(k-1)+1}{k-1}}{((n-1)-2(k-1)+3)((n-1)-2(k-1)+2)}\right]}&=\sum\limits_{k=1}^{n}{\binom{n-k}{k-1}}\\
%  \sum\limits_{k=2}^{n+1}{\left[\frac{\binom{n-k+1}{k-1}}{(n-2k+4)(n-2k+3)}\right]}&=\sum\limits_{k=1}^{n}{\binom{n-k}{k-1}}\\
%  \sum\limits_{k=1}^{n}{\left[\frac{\frac{n-k+1}{k}\binom{n-k}{k-1}}{(n-2k+3)(n-2k+2)}\right]}&=\sum\limits_{k=1}^{n}{\binom{n-k}{k-1}}\\
%  &=\sum\limits_{k=1}^{n}{\binom{n-k}{k-1}}+\sum\limits_{k=1}^{n}{\frac{(n+1-k)!}{(k-1)!(n-k+1-k+1)!}}+\binom{0}{n}\\
%  &=\sum\limits_{k=1}^{n}{\binom{n-k}{k-1}}+\sum\limits_{k=1}^{n}{\frac{(n+1-k)(n-k)!}{(k-1)!(n-k+1-k+1)(n-k+1-k)!}}\\
%  &=\sum\limits_{k=1}^{n}{\binom{n-k}{k-1}}+\sum\limits_{k=1}^{n}{\frac{(n+1-k)}{(n+2-2k)}\binom{n-k}{k-1}}\\
%  &=\sum\limits_{k=1}^{n}{\binom{n-k}{k-1}}+\sum\limits_{k=1}^{n}{\frac{(n-k+1)}{(n-2k+2)}\binom{n-k}{n-k+1}}\\
%\end{align*}
\end{enumerate}
\end{document}
