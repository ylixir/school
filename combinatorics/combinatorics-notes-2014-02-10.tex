\documentclass{article}
\usepackage{fullpage}
\usepackage{nopageno} 
\usepackage{amsmath}
\usepackage{amssymb}
\allowdisplaybreaks

\newcommand{\abs}[1]{\left\lvert #1 \right\rvert}

\begin{document}
\title{Notes}
\date{February 10, 2014}
\maketitle
\section*{homework}
for number 35 write down subsets in lexicographic order, and complements (this will be reverse lexicographic order)

number 33 has a theorem in the book
\section*{pigeonhole principle}
this probably won't be on the test, but "it will blow my mind"

so if we have 11 cups and 12 markers than distributing the markers we have at least one cup with 2 markers. or if you have $n$ holes and $n+1$ pigeons then you will have at least one hole with more than one pigeon.
\subsection*{aside}
we have 24 people in the class. what are the chances that anyone has the same birthday?
\subsection*{in math mathyness}
\subsubsection*{abstract pigeonhole principle}
let $x$ and $y$ be finite sets and $f:x\Rightarrow y$. If $|x|>|y|$ then $f$ is not injective. If $|x|=|y|$ then $f$ is surjective if and only if $f$ is injective (that is bijective).
\subsection*{example}
given a sequence$a_1,\dots,a_m,a_i\in\mathbb{N}\quad \forall i$. $\exists$ consecutive $a_i$'s whose sum is divisible by $m$.
\subsubsection*{proof}
consider $a_1,a_1+a_2,a_1+a_2+a_3,\dots,a_1+a_2+\dots+a_m$. Suppose none of these are divisible by $m$. So the remainders of each sum when divided by m are in $\{1,\dots,m-1\}$. So $\exists k<l$ such that $a_1+\dots+a_k$ and $a_1+\dots+a_l$have the same remainder $r$. So $\exists b,c\in\mathbb{N}$ such that
\begin{align*}
  a_1+\dots+a_l&=cm+r\\
  -(a_1+\dots+a_k&=bm+r)\\
  a_{k+1}+\dots+a_l&=(c-b)m
\end{align*}
So $a_{k+1}+\dots+a_l$ is divisible by $m$.
\subsubsection*{note}
what if $a_i$'s$\in\mathbb{Z}$?
\subsection*{back to the birthday problem}
use the subtraction principle. The number of ways for 24 people to have a pair with the same birthday is the total number of birthday distributions minus the number of ways to have distinct birthdays. So $365^{24}$ is the total number of was for people to have birthdays. And $\frac{365!}{(365-24)!}$ is the number of ways to have distinct birthdays. And the number of ways for 24 people to have a pair of the same birthday is $365^{24}-\frac{365!}{(365-24)!}$.
\end{document}
