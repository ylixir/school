\documentclass{article}
\usepackage{fullpage}
\usepackage{nopageno} 
\usepackage{amsmath}
\usepackage{amssymb}
\allowdisplaybreaks

\newcommand{\abs}[1]{\left\lvert #1 \right\rvert}

\begin{document}
\title{Notes}
\date{February 7, 2014}
\maketitle
\section*{gray codes}
cyclic vs noncyclic

reflected$\to$ specific. inductive definition, algorithm.

went over proof of generating algorithm. It's in the book
\section*{generating r subsets}
\[S=\{x_{n-1},x_{n-2},\dots,x_1,x_0\}\equiv \{1,2,\dots,n\}\]
subsets are equivalent to words in letters $\{1,\dots,n\}$ where the word is increasing (strictly).

\subsection*{example}
want to generate 2-subsets of $\{x_3,x_2,x_1,x_0\}=\{1,2,3,4\}$
\[0_40_31_21_1\approx\{x_1,x_0\}\approx12\]

binary order's are lexicographical, squashed...

it's better to do lexicographic order on the \emph{words} (r-subsets).
\begin{align*}
  12&&
  13&&
  14&&
  23&&
  24&&
  34
\end{align*}
\subsection*{algorithm}
begin with $a_1a_2\cdots a_r=12\dots r$. Find the farthest right position such that $a_k+1\leq n$ and $a_k+1$ is not in the word. Then the next word is $a_1\dots a_{k-1}(a_k+1)(a_k+2)(a_k+3)\dots(a_k+r-k+1)$

\subsubsection*{do 26 and 28}
12 
\subsubsection*{28}
2,3,4,6,9,10$\to$2,3,4,7,8,9
\section*{problems 31\&32}
generating r-premutations:

gnerate r-subsets

for each r-subset generate the permutations
\end{document}
