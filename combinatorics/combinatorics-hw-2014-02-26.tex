\documentclass{article}
%\usepackage{fullpage}
%\usepackage{nopageno} 
\usepackage[margin=1.5in]{geometry}
\usepackage{amsmath}
\usepackage{amssymb}
\usepackage[normalem]{ulem}
\usepackage{fancyhdr}
%\renewcommand\headheight{12pt}
\pagestyle{fancy}
\lhead{February 26, 2014}
\rhead{Jon Allen}
\allowdisplaybreaks

\newcommand{\abs}[1]{\left\lvert #1 \right\rvert}

\begin{document}
\section*{Chapter 5}
\subsection*{5.}
Expand $(2x-y)^7$ using the binomial theorem.
\subsection*{7.}
Use the binomial theorem to prove that
\begin{align*}
  3^n&=\sum\limits_{k=0}^n{\binom{n}{k}2^k}.
\end{align*}
Generalize to find the sum
\begin{align*}
  \sum\limits_{k=0}^n{\binom{n}{k}r^k}
\end{align*}
\subsection*{8.}
Use the binomialtheorem to prove that
\begin{align*}
  2^n&=\sum\limits_{k=0}^n{(-1)^k\binom{n}{k}3^{n-k}}.
\end{align*}
\subsection*{10.}
Use \emph{combinatorial} reasoning to prove the identity (5.2).
\subsection*{11.}
Use \emph{combinatorial} reasoning to prove the identity (in the form given)
\begin{align*}
  \binom{n}{k}-\binom{n-3}{k}&=\binom{n-1}{k-1}+\binom{n-2}{k-1}+\binom{n-3}{k-1}
\end{align*}
(\emph{Hint:} Let $S$ be a set with three distinguished elements $a, b,$ and $c$ and count certain $k$-subsets of $S$.)
\subsection*{14.}
\subsection*{15.}
\subsection*{16.}
\subsection*{17.}
\subsection*{21.}
\subsection*{27.}
\subsection*{28.}
\end{document}
