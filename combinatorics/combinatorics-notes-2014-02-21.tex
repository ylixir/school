\documentclass{article}
\usepackage{fullpage}
\usepackage{nopageno} 
\usepackage{amsmath}
\usepackage{amssymb}
\allowdisplaybreaks

\newcommand{\abs}[1]{\left\lvert #1 \right\rvert}

\begin{document}
\title{Notes}
\date{February 21, 2014}
\maketitle

last time, we derived Pascal's equation:
\begin{align*}
  \binom{n}{k}&=\binom{n-1}{k-1}+\binom{n-1}{k}
\end{align*}
\section*{combinatorial proof:}

$\binom{n}{k}$ counts the number of $k$-elt subsets $S$ of $[n]$. 2 cases.
\subsubsection*{case 1}
$n\in S$ then the remaining elts of $S$ are chosen from $[n-1]$ in $\binom{n-1}{k-1}$ ways
\subsubsection*{case 2}
$n\not\in S$ then $S$ is made up of $k$-elts from $[n-1]$. There are $\binom{n-1}{k}$ such subsets.

Therefore $\binom{n}{k}=\binom{n-1}{k-1}+\binom{n-1}{k}$
$\Box$

\section*{fibanacci in pascal's triangle}
neat
\section*{combinatorial proof of binomial theorem}
$\overbrace{(x+y)(x+y)\dots(x+y)}^{n\text{-factors}}=(x+1)^n$. Each term in the sum arises from picking $x$ or $y$ in each of the $n$ factors. The coefficients of $x^ky^{n-k}$ counts the number of ways to pick $k$ $x$'s. This is counted by $\binom{n}{k}$
\section*{moving on}
notice if you set $x-y=1$, you obtain:
\begin{align*}
  2^n&=\sum\limits_{k=0}^n{\binom{n}{k}}
\end{align*}
if you set $y=1$ you get
\begin{align*}
  (1+x)^n&=\sum\limits_{k=0}^n{\binom{n}{k}x^k}
\end{align*}
\subsection*{wksht 6}
$x=3, y=1$ in binomial theorem.
\begin{align*}
  (3+1)^{10}&=\sum\limits_{k=0}^{10}{\binom{10}{k}3^k1^{10-k}}
  &=4^{10}
\end{align*}
\subsection*{wksht 7}
\subsection*{wksht 8}
Vandermonde convolution:
\begin{align*}
  000&\dots0&m\text{ things}\\
  000&\dots0&n\text{ things}\\
\end{align*}
choose k things
\subsection*{wksht 9}
calculus proof
\begin{align*}
  \sum\limits_{k=0}^n{k\binom{n}{k}x^{k-1}}&=\frac{\mathrm{d}}{\mathrm{d}x}\sum\limits_{k=0}^n{\binom{n}{k}x^k}\\
  &=\frac{\mathrm{d}}{\mathrm{d}x}(1+n)^n\\
  &=n(1+x)^{n-1}
\end{align*}
set $x=1$ $\sum\limits_{k=0}^n{k\binom{n}{k}}=n(1+1)^{n-1}=n2^{n-1}$
\section*{question}
What does this sum equal:
$\binom{n}{0}^2+\binom{n}{1}^2+\binom{n}{2}^2+\dots+\binom{n}{n}^2$
\begin{align*}
  \sum\limits_{k=0}^n{\binom{n}{k}^2}&=\sum\limits_{k=0}^n{\binom{n}{k}\binom{n}{n-k}}\\
  &=\binom{2n}{n}
\end{align*}
\section*{sidebar}
\subsection*{question:}is there any meaning to something like $\binom{12/7}{3}$?
\subsection*{answer}
for and $n\in\mathbb{R},k\in\mathbb{Z}$ let $\binom{n}{k}=\begin{cases}\frac{\overbrace{n(n-1)(n-2)\dots(n-k+1)}^{k \text{ factors}}}{k!}&\text{if } k\ge1\\1&\text{if }k=0\\0&\text{if }k\le-1\end{cases}$
\begin{align*}
  \binom{12/7}{3}&=\frac{12/7(12/7-1)(12/7-2)}{3!}=\frac{-20}{7^3}
\end{align*}
Is it a good definition? yes. e.g. it is true that
\begin{align*}
  \binom{12/7}{3}&=\binom{5/7}{2}+\binom{5/7}{3}
\end{align*}
\section*{iterate pascal's identity:}
\begin{align*}
  \binom{n}{k}&=\binom{n-1}{k}+\binom{n-1}{k-1}\\
  \binom{n}{k}&=\binom{n-1}{k}+\binom{n-1}{k-1}+\binom{n-2}{k-2}\\
\end{align*}
\end{document}
