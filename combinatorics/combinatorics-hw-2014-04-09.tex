\documentclass{article}
%\usepackage{fullpage}
%\usepackage{nopageno} 
\usepackage[margin=1.5in]{geometry}
\usepackage{amsmath}
\usepackage{amssymb}
\usepackage[normalem]{ulem}
\usepackage{fancyhdr}
\usepackage{cancel}
%\renewcommand\headheight{12pt}
\pagestyle{fancy}
\lhead{April 9, 2014}
\rhead{Jon Allen}
\allowdisplaybreaks

\newcommand{\abs}[1]{\left\lvert #1 \right\rvert}

\begin{document}
\begin{enumerate}
\item
Let $f_0,f_1,f_2,\dots,f_n,\dots$ denote the Fibonacci sequence. By evaluating each of the following expressions for small values of $n$, conjecture a general formula and then prove it, using mathematical induction and the Fibonacci recurrence:
\begin{enumerate}
\item
$f_1+f_3+\dots+f_{2n-1}$
\begin{align*}
  &\cancel{0},1,\cancel{1},2,\cancel{3},5,\cancel{8},13,\cancel{21},34,\dots\\
  1&=1,\quad 1+2=3,\quad 1+2+5=8,\quad 1+2+5+13=21\\
  f_{2n}&=f_1+f_3+\dots+f_{2n-1}
\end{align*}
\subsubsection*{proof}
We already know the sum for $n=1$. Lets look at $n>1$
\begin{align*}
  f_1+f_3+f_5+\dots+f_{2n-1}&=f_1+\sum\limits_{k=2}^{n}{f_{2k-1}}\\
  f_n&=f_{n-1}+f_{n-2}\\
  f_1+f_3+f_5+\dots+f_{2n-1}&=1+\sum\limits_{k=2}^{n}{f_{2k-3}+f_{2k-2}}\\
  &=1+0+\sum\limits_{k=1}^{2n-2}{f_{k}}=1+f_0+\sum\limits_{k=1}^{2n-2}{f_{k}}\\
  s_n&=f_0+f_1+f_2+\dots+f_n=f_{n+2}-1\\
  f_1+f_3+f_5+\dots+f_{2n-1}&=1+s_{2n-2}=1+f_{2n-2+2}-1\\
  &=f_{2n}
\end{align*}
And we have our result. $\Box$
\item
$f_0+f_2+\dots+f_{2n}$
\begin{align*}
  &0,\cancel{1},1,\cancel{2},3,\cancel{5},8,\cancel{13},21,\cancel{34},55,\dots\\
  0&=0,\quad0+1=1\quad0+1+3=4\quad0+1+3+8=12,\quad0+1+3+8+21=33\\
  f_{2n+1}-1&=f_0+f_2+\dots+f_{2n}
\end{align*}
\subsubsection*{proof}
When looking for the pattern so we established that the formula is true for $n=0,1,2,3,4$ which is more than sufficient for a basis.
\begin{align*}
  f_0+f_2+\dots+f_{2n}&=\sum\limits_{k=0}^n{f_{2k}}\\
  f_n&=f_{n-1}+f_{n-2}\\
  f_0+f_2+\dots+f_{2n}&=f_0+\sum\limits_{k=1}^n{f_{2k-2}+f_{2k-1}}\\
  &=0+\sum\limits_{k=0}^{2n-1}{f_{k}}\\
  s_n&=f_0+f_1+f_2+\dots+f_n=f_{n+2}-1\\
  f_0+f_2+\dots+f_{2n}&=s_{2n-1}=f_{(2n-1)+2}-1\\
  &=f_{2n+1}-1
\end{align*}
And our result is proved. $\Box$
\item
$f_0-f_1+f_2-\dots+(-1)^nf_{n}$
\begin{align*}
  n&=1:0-1=-1,\quad n=2:0-1+1=0,\quad n=3:0-1+1-2=-2\\
  n&=4:0-1+1-2+3=1,\quad n=5:0-1+1-2+3-5=-4\\
  &f_0-f_1+(f_0+f_1)-(f_1+f_2)+(f_2+f_3)-\dots\\
  &f_0-f_1+f_0+f_1-f_1-f_2+f_2+f_3-\dots\\
  &-f_1+(-1)^nf_{n-1}
\end{align*}
So our general formula is $-1+(-1)^nf_{n-1}$
\subsubsection*{proof}
We know that the sum is 0 when $n=0$ and the sum is -1 when $n=1$. Let us look at when $n>1$
\begin{align*}
  f_0-f_1+f_2-\dots+(-1)^nf_{n}&=\sum\limits_{k=0}^n{(-1)^kf_k}\\
  &=f_0-f_1+\sum\limits_{k=2}^n{(-1)^kf_k}\\
  &=-1+\sum\limits_{k=2}^n{(-1)^k(f_{k-2}+f_{k-1})}\\
  &=-1+\sum\limits_{k=0}^{n-2}{(-1)^k(f_{k}+f_{k+1})}\\
  &=-1+\sum\limits_{k=0}^{n-2}{(-1)^kf_{k}}+\sum\limits_{k=0}^{n-2}{(-1)^kf_{k+1}}\\
  &=-1+\sum\limits_{k=0}^{n-2}{(-1)^kf_{k}}-\sum\limits_{k=1}^{n-1}{(-1)^kf_{k}}\\
  &=-1+(-1)^0f_0+\sum\limits_{k=1}^{n-2}{(-1)^kf_{k}}-\sum\limits_{k=1}^{n-2}{(-1)^kf_{k}}-(-1)^{n-1}f_{n-1}\\
  &=-1+(-1)^nf_{n-1}
\end{align*}
And we have our proof. $\Box$
%\item
%${f_0}^2+{f_1}^2+\dots+{f_n}^2$
%\begin{align*}
%  0^2&=0\\
%  0+1^2&=1\\
%  0+1^2+1^2&=2\\
%  0+1^2+1^2+2^2&=6\\
%  0+1^2+1^2+2^2+3^2&=15\\
%  0+1^2+1^2+2^2+3^2+5^2&=40\\
%  0+1^2+1^2+2^2+3^2+5^2+6^2&=76\\
%  0+1^2+1^2+2^2+3^2+5^2+6^2+7^2&=125\\
%\end{align*}
%\begin{align*}
%  {f_0}^2+{f_1}^2+\dots+{f_n}^2&=\sum\limits_{k=0}^n{{f_{k}}^2}\\
%  &={f_0}^2+{f_1}^2+\sum\limits_{k=2}^n{(f_{k-1}+f_{k-2})^2}\\
%  &={f_0}^2+{f_1}^2+\sum\limits_{k=2}^n{{f_{k-1}}^2+2f_{k-2}f_{k-1}+{f_{k-2}}^2}\\
%  &={f_0}^2+{f_1}^2+{f_{1}}^2+2f_{0}f_{1}+{f_{0}}^2+\sum\limits_{k=3}^n{{f_{k-1}}^2+2f_{k-2}f_{k-1}+{f_{k-2}}^2}\\
%  &={f_0}^2+{f_1}^2+{f_{1}}^2+2f_{0}f_{1}+{f_{0}}^2+\sum\limits_{k=3}^n{(f_{k-2}+f_{k-3})^2+2f_{k-2}f_{k-1}+{f_{k-2}}^2}\\
%  %{f_0}^2+{f_1}^2+\sum\limits_{k=2}^n{{f_{k}}^2}&={f_0}^2+{f_1}^2+\sum\limits_{k=2}^n{{f_{k-1}}^2+2f_{k-2}f_{k-1}+{f_{k-2}}^2}\\
%  %\sum\limits_{k=2}^n{{f_{k}}^2}-\sum\limits_{k=2}^n{{f_{k-1}}^2}&=\sum\limits_{k=2}^n{2f_{k-2}f_{k-1}+{f_{k-2}}^2}\\
%  %\sum\limits_{k=2}^n{{f_{k}}^2}-\sum\limits_{k=1}^{n-1}{{f_{k}}^2}&={f_n}^2+\sum\limits_{k=2}^{n-1}{{f_{k}}^2}-{f_1}^2-\sum\limits_{k=2}^{n-1}{{f_{k}}^2}
%\end{align*}
\end{enumerate}
\setcounter{enumi}{2}
\item
Prove th following about the Fibonacci numbers:
\begin{enumerate}
\setcounter{enumii}{1}
\item
$f_n$ is divisible by 3 if and only if $n$ is divisible by 4.
\subsubsection*{proof}
Let $f_n=3a_n+b_n$ where $a_n\in\mathbb{N}$ and $b_n\in\{0,1,2\}$.
Lets assume that $b_{n-1}=b_{n-2}=0$.
Then $f_n=3a_{n-1}+3a_{n-2}$ which means that for all $n$, $3\mid f_n$.
Three seconds of scratchwork shows this is clearly not true at least for low values of $n$.
Now lets assume that $3\nmid f_n$ for all $n$.
But $3\mid f_0$ and $3\mid f_4$ so clearly this isn't true.
We know now that 3 sometimes divides Fibonacci numbers, but not always.
So lets assume that $3\mid f_{n}$.
Then for $f_{n-1}=3a_{n-1}+2$ or $f_{n-1}=3a_{n-1}+1$.
Note that these cases can be seen at $f_4$ and $f_8$ respectively.
Also note that $4\mid4$ and $4\mid8$.
Finally notice that these two (four?) facts conveniently provide a basis for the following inductive proof in two cases where $3\mid f_n$ and $4\mid n$.

First we look at the case where $f_{n-1}=3a_{n-1}+2$.
Let the algebra walk the walk.
\begin{align*}
  f_{n+1}&=3a_n+3a_{n-1}+2=3(a_n+a_{n-1})+2=3a_{n+1}+2\\
  f_{n+2}&=3a_{n+1}+2+3a_n=3a_{n+2}+2\\
  f_{n+3}&=3a_{n+2}+2+3a_{n+1}+2=3a_{n+2}+3a_{n+1}+3+1=3a_{n+3}+1\\
  f_{n+4}&=3a_{n+3}+1+3a_{n+2}+2=3(a_{n+3}+a_{n+2}+1)
\end{align*}
And because $4\mid n$ then $4\mid n+4$.
Further, $3\mid f_{n+4}$ in this case.
Also notice that we have shown the other half of the if and only if.
Three does not divide any of $f_{n+1},f_{n+2},$ or $f_{n+3}$.

Now lets examine the case where $f_{n-1}=3a_{n-1}+1$. We proceed as above, with maths.
\begin{align*}
  f_{n+1}&=3a_n+3a_{n-1}+1=3a_{n+1}+1\\
  f_{n+2}&=3a_{n+1}+1+3a_{n}=3a_{n+2}+1\\
  f_{n+3}&=3a_{n+1}+1+3a_{n+2}+1=3a_{n+3}+2\\
  f_{n+4}&=3a_{n+3}+2+3a_{n+2}+1=3(a_{n+3}+a_{n+2}+1)
\end{align*}
So we see that this case meets all the conditions of the last case.

We have shown that a Fibanocci style recurrence relation is either always divisible by three or is divisible by three only every fourth number. In our case the numbers are not always divisible by three. The index of every fourth number that divides three is itself divided by four. Thus we have proven the assertion. $\Box$
\end{enumerate}
\item
Prove that the Fibonacci sequence is the solution of the recurrence relation
\[a_n=5a_{n-4}+3a_{n-5},\quad(n\ge5),\]
where $a_0=0,a_1=1,a_2=1,a_3=2,$ and $a_4=3$. Then use this formula to show that the Fibonacci numbers satisfy the condition that $f_n$ is divisible by 5 if and only if $n$ is divisible by 5.
\subsubsection*{proof}
Lets just try to wrangle the Fibanocci sequence into the relation shown.
\begin{align*}
  f_n&=f_{n-1}+f_{n-2}\\
  f_n&=(f_{n-2}+f_{n-3})+(f_{n-3}+f_{n-4})\\
  f_n&=(f_{n-3}+f_{n-4})+(f_{n-4}+f_{n-5})+(f_{n-4}+f_{n-5})+f_{n-4}\\
  f_n&=(f_{n-4}+f_{n-5})+f_{n-4}+f_{n-4}+f_{n-5}+f_{n-4}+f_{n-5}+f_{n-4}\\
  f_n&=5f_{n-4}+3f_{n-5}
\end{align*}
Since the relation is the same and the first five numbers are the same, the sequence is the same. $\Box$

The case where $5\mid f_{n-4}$ and $5\mid f_{n-5}$ is the trivial case and would mean that five divides all $f_n$ which is clearly not true, so we will just write off that case.

Let $f_n=5a_n+b_n$ where $a_n\in\mathbb{N}$ and $b_n\in\{0,1,2,3,4\}$.

Now lets assume that $5\mid f_{n-5}$ and $5\nmid f_{n-4}$. In other words $b_{n-5}=0$ and $b_{n-4}\ne0$
\begin{align*}
  f_n&=5(5a_{n-4}+b_{n-4})+3(5a_{n-5})\\
  f_n&=5(5a_{n-4}+b_{n-4}+3a_{n-5})
\end{align*}
So $b_n=0$ and therefore $5\mid f_n$ if $5\mid f_{n-5}$ and $5\nmid f_{n-4}$.

Now lets check on $5\mid f_{n-4}$ and $5\nmid f_{n-5}$
\begin{align*}
  f_n&=5(5a_{n-4})+3(5a_{n-5}+b_{n-5})\\
  f_n&=5(5a_{n-4}+3a_{n-5})+3b_{n-5}
\end{align*}
Since $3b_{n-5}\in\{3,6,9,12\}$ we can say that $5\nmid 3b_{n-5}$ and therefore $5\nmid f_n$
And for our last case lets find out what happens if five divides neither $f_{n-4}$ nor $f_{n-5}$.
\begin{align*}
  f_n&=5(a_{n-4}+b_{n-4})+3(5a_{n-5}+b_{n-5)}\\
  f_n&=5(a_{n-4}+b_{n-4}+3a_{n-5})+3b_{n-5}
\end{align*}
As above $5\nmid 3b_{n-5}$ so $5\nmid f_n$.

So we see 5 divides $f_n$ only every 5th $f_n$. Since $f_5=5$ we can say that $5\mid f_n$ if and only if $5\mid n$.
\setcounter{enumi}{6}
%\item
%* Let $m$ and $n$ be positive integers whose greatest common divisor is $d$. Prove that the greatest common divisor of the Fibonacci numbers $f_m$ and $f_n$ is the Fibonacci number $f_d$.
\setcounter{enumi}{10}
\item
The \emph{Lucas numbers} $l_0,l_1,l_2,\dots,l_n,\dots$ are defined using the same recurrence relation defining the Fibonacci numbers, but with differenct initial conditions:
\[l_n=l_{n-1}+l_{n-2}, (n\ge2),l_0=2,l_1=1.\]
Prove that
\begin{enumerate}
\item
$l_n=f_{n-1}+f_{n+1}$ for $n\ge1$
\subsubsection*{proof}
We can see that $l_1=1=0+1=f_0+f_2=f_{1-1}+f_{1+1}$. Also note that $l_2=l_1+l_0=1+2=f_{1}+f_{3}=f_{2-1}+f_{2+1}$. So our assumption holds for $n=1,2$. Lets assume it holds for any $n-1, n-2$ and verify that it holds for $n$.
\begin{align*}
  l_n&=l_{n-1}+l_{n-2}&\text{by definition of Lucas number}\\
  &=(f_{n-2}+f_{n})+(f_{n-3}+f_{n-1})&\text{by assumption}\\
  &=(f_{n}+f_{n-1})+(f_{n-2}+f_{n-3})\\
  &=f_{n+1}+f_{n-1}&\text{by definition of Fibanocci number}
\end{align*}
So the assumption holds for $n\ge3$ and we have proved our result by induction. $\Box$
\item
${l_0}^2+{l_1}^2+\dots+{l_n}^2=l_nl_{n+1}+2$ for $n\ge0$
\subsubsection*{proof}
So we see that ${l_0}^2=2^2=4=2+2=(2)(1)+2=l_0l_1$ and ${l_0}^2+{l_1}^2=2^2+1^2=5=3+2=(1)(3)+2=l_1l_2+2$. Now we know that our idea holds for $n=0,1$. Let us assume that our idea holds for all $n$. Lets see if it holds for $n+1$.
\begin{align*}
  {l_0}^2+{l_1}^2+\dots+{l_n}^2+{l_{n+1}}^2&=l_nl_{n+1}+2+{l_{n+1}}^2\\
  &=l_nl_{n+1}+2+l_{n+1}(l_{n}+l_{n-1})\\
  &=l_nl_{n+1}+2+l_{n+1}l_{n}+l_{n+1}l_{n-1}\\
  &=l_{n+1}(l_n+l_{n}+l_{n-1})+2\\
  &=l_{n+1}(l_n+l_{n+1})+2\\
  &=l_{n+1}l_{n+2}+2\\
\end{align*}
Well, it looks like it holds for $n+1$ and therefore by induction it holds for all $n$. $\Box$
\end{enumerate}
\item
Let $h_0,h_1,h_2,\dots,h_n,\dots$ be the sequence defined by
\[h_n=n^3,(n\ge0).\]
Show that $h_n=h_{n-1}+3n^2-3n+1$ is the recurrence relation for the sequence.
\begin{align*}
  h_n&=n^3\\
  &=(n-1+1)^3\\
  &=(n-1)^3+3(n-1)^2+3(n-1)+1\\
  &=h_{n-1}+3(n^2-2n+1)+3n-3+1\\
  &=h_{n-1}+3n^2-6n+3+3n-3+1\\
  &=h_{n-1}+3n^2-3n+1\\
\end{align*}
\item
Determine the generating function for each of the following sequences:
\begin{enumerate}
\setcounter{enumii}{2}
\item
$\binom{\alpha}{0},-\binom{\alpha}{1},\binom{\alpha}{2},\dots,(-1)^n\binom{\alpha}{n},\dots,$ ($\alpha$ is a real number)
\begin{align*}
  \binom{\alpha}{0}-\binom{\alpha}{1}x+\binom{\alpha}{2}x^2-\dots+(-1)^n\binom{\alpha}{n}x^n+\dots&=\sum\limits_{n=0}^\infty{(-1)^n\binom{\alpha}{n}x^n}\\
  \sum\limits_{n=0}^\infty{(-1)^n\binom{\alpha}{n}x^n}&=\sum\limits_{n=0}^\infty{\binom{\alpha}{n}(-x)^n}\\
  \intertext{By newton's generalised binomial theorem}
  &=(1-x)^\alpha\\
\end{align*}
\setcounter{enumii}{4}
\item
$1,-\frac{1}{1!},\frac{1}{2!},\dots,(-1)^n\frac{1}{n!},\dots$
\end{enumerate}
\begin{align*}
  1-\frac{1}{1!}x+\frac{1}{2!}x^2-\dots+(-1)^n\frac{1}{n!}x^n+\dots&=\sum\limits_{n=0}^\infty{(-1)^n\frac{1}{n!}x^n}\\
  &=\sum\limits_{n=0}^\infty{\frac{(-x)^n}{n!}}\\
  &=e^{-x}
\end{align*}
\item
Let $S$ be the multiset $\{\infty\cdot e_1,\infty\cdot e_2,\infty\cdot e_3,\infty\cdot e_4\}$. Determine the generating function for the sequence $h_0,h_1,h_2,\dots,h_n,\dots,$ where $h_n$ is the number of $n$-combinations of $S$ with the following added restrictions:
\begin{enumerate}
\setcounter{enumii}{2}
\item
The element $e_1$ does not occur, and $e_2$ occurs at most once.
\begin{align*}
  1&=(1+x^2+x^3+\dots)-(x+x^2+x^3\dots)\\
  &=\frac{1}{1-x}-\frac{x}{1-x}=\frac{1-x}{1-x}\\
  1+x&=(1+x^2+x^3+\dots)-(x^2+x^3+x^4\dots)\\
  &=\frac{1}{1-x}-\frac{x^2}{1-x}=\frac{1-x^2}{1-x}\\
  (1+x+x^2+x^3+\dots)^2&=\frac{1}{(1-x)^2}\\
  g(x)&=\frac{1-x^2}{(1-x)^3}
\end{align*}
\item
The element $e_1$ occurs 1,3,or 11 times, and the element $e_2$ occurs 2,4, or 5 times.
\begin{align*}
  x^n&=1+x^2+x^3+\dots-1-x-x^2-\dots-x^{n-1}-x^{n+1}-x^{n+2}\dots\\
  &=\frac{1}{1-x}-\frac{1-x^{n-1}}{1-x}-\frac{x^{n+1}}{1-x}=\frac{x^{n-1}-x^{n+1}}{1-x}\\
  x^1+x^3+x^{11}&=\frac{x^0-x^2}{1-x}+\frac{x^2-x^4}{1-x}+\frac{x^{10}-x^{12}}{1-x}\\
  &=\frac{1-x^4+x^{10}-x^{12}}{1-x}\\
  x^2+x^4+x^5&=\frac{x^1-x^3}{1-x}+\frac{x^3-x^5}{1-x}+\frac{x^4-x^6}{1-x}\\
  &=\frac{x+x^4-x^5-x^6}{1-x}\\
  g(x)&=\frac{(1-x^4+x^{10}-x^{12})(x+x^4-x^5-x^6)}{(1-x)^2}
\end{align*}
\item
Each $e_i$ occurs at least 10 times.
\begin{align*}
  (x^{10}+x^{11}+x^{12}+\dots)^4&=x^{40}(1+x+x^2+\dots)^4\\
  &=\frac{x^{40}}{(1-x)^4}
\end{align*}
\end{enumerate}
\item
Determine the generating function for the sequence of cubes
\[0,1,8,\dots,n^3,\dots\]
\begin{align*}
  0x^0+1x^1+8x^2+\dots&=\sum\limits_{n=0}^{\infty}{n^3x^n}\\
  \frac{1}{1-x}&=\sum\limits_{n=0}^\infty{x^n}\\
  \frac{\mathrm{d}}{\mathrm{d}x}\left(\frac{1}{1-x}\right)&=\frac{\mathrm{d}}{\mathrm{d}x}\left(\sum\limits_{n=0}^\infty{x^n}\right)\\
  \frac{1}{(1-x)^2}&=\sum\limits_{n=0}^\infty{nx^{n-1}}\\
  \frac{x}{(1-x)^2}&=\sum\limits_{n=0}^\infty{nx^n}\\
  \frac{\mathrm{d}}{\mathrm{d}x}\left(\frac{x}{(1-x)^2}\right)&=\frac{\mathrm{d}}{\mathrm{d}x}\left(\sum\limits_{n=0}^\infty{nx^n}\right)\\
  \frac{1}{(1-x)^2}+\frac{2x}{(1-x)^3}&=\sum\limits_{n=0}^\infty{n^2x^{n-1}}\\
  \frac{x}{(1-x)^2}+\frac{2x^2}{(1-x)^3}=\frac{x-x^2+2x^2}{(1-x)^3}=\frac{x+x^2}{(1-x)^3}&=\sum\limits_{n=0}^\infty{n^2x^n}\\
  \frac{\mathrm{d}}{\mathrm{d}x}\left(\frac{x+x^2}{(1-x)^3}\right)&=\frac{\mathrm{d}}{\mathrm{d}x}\left(\sum\limits_{n=0}^\infty{n^2x^n}\right)\\
  \frac{1+2x}{(1-x)^3}+\frac{3x+3x^2}{(1-x)^4}&=\sum\limits_{n=0}^\infty{n^3x^{n+1}}\\
  \frac{1-x+2x-2x^2+3x+3x^2}{(1-x)^4}=\frac{1+4x+x^2}{(1-x)^4}&=\sum\limits_{n=0}^\infty{n^3x^{n+1}}\\
  \frac{x+4x^2+x^3}{(1-x)^4}&=\sum\limits_{n=0}^\infty{n^3x^n}\\
  g(x)&=\frac{x+4x^2+x^3}{(1-x)^4}\\
%  \frac{\mathrm{d}}{\mathrm{d}x}\left(\frac{x}{(1-x)^2}\left(1+\frac{2x}{1-x}\right)\right)&=\frac{\mathrm{d}}{\mathrm{d}x}\left(\sum\limits_{n=0}^\infty{n^2x^n}\right)\\
%  \left(\frac{1}{(1-x)^2}+\frac{2x}{(1-x)^3}\right)\left(1+\frac{2x}{1-x}\right)+\frac{x}{(1-x)^2}\left(\frac{2}{1-x}+\frac{2x}{(1-x)^2}\right)&=\sum\limits_{n=0}^\infty{n^3x^{n-1}}\\
%  x\left(\frac{1}{(1-x)^2}+\frac{2x}{(1-x)^3}\right)\left(1+\frac{2x}{1-x}\right)+x\frac{x}{(1-x)^2}\left(\frac{2}{1-x}+\frac{2x}{(1-x)^2}\right)&=\sum\limits_{n=0}^\infty{n^3x^n}\\
%  x\left(\frac{1}{(1-x)^2}+\frac{2x}{(1-x)^3}+\frac{2x}{(1-x)^3}+\frac{4x^2}{(1-x)^4}\right)+x\left(\frac{2x}{(1-x)^2}+\frac{2x^2}{(1-x)^4}\right)&=\sum\limits_{n=0}^\infty{n^3x^n}\\
%  \frac{x}{(1-x)^2}+\frac{4x^2}{(1-x)^3}+\frac{4x^3}{(1-x)^4}+\frac{2x^2}{(1-x)^2}+\frac{2x^3}{(1-x)^4}&=\sum\limits_{n=0}^\infty{n^3x^n}\\
%  \frac{x+2x^2}{(1-x)^2}+\frac{4x^2}{(1-x)^3}+\frac{6x^3}{(1-x)^4}&=\sum\limits_{n=0}^\infty{n^3x^n}\\
%  \frac{(x+2x^2)(1-x)^2}{(1-x)^4}+\frac{4x^2(1-x)}{(1-x)^4}+\frac{6x^3}{(1-x)^4}&=\sum\limits_{n=0}^\infty{n^3x^n}\\
%  \frac{(x+2x^2)(1-2x+x^2)+4x^2-4x^3+6x^3}{(1-x)^4}&=\sum\limits_{n=0}^\infty{n^3x^n}\\
%  \frac{x-2x^2+x^3+2x^2-4x^3+2x^4+4x^2+2x^3}{(1-x)^4}&=\sum\limits_{n=0}^\infty{n^3x^n}\\
%  \frac{x-x^3+2x^4+4x^2}{(1-x)^4}&=\sum\limits_{n=0}^\infty{n^3x^n}\\
%  \frac{x+4x^2-x^3+2x^4}{(1-x)^4}&=\sum\limits_{n=0}^\infty{n^3x^n}\\
\end{align*}
\end{enumerate}
\end{document}
