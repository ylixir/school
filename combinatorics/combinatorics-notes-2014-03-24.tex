\documentclass{article}
\usepackage{fullpage}
\usepackage{nopageno} 
\usepackage{amsmath}
\usepackage{amssymb}
\allowdisplaybreaks

\newcommand{\abs}[1]{\left\lvert #1 \right\rvert}

\begin{document}
\title{Notes}
\date{March 24, 2014}
\maketitle
\section*{homework}
\subsection*{\#10}
looking for solutions to $x_1+x_2+\dots+x_k=r, 0\le x_i\le n_i$. If there is a solution to this equation that satisfies this, then you should be able to use in/ex principle to show that the intersection is $\emptyset$. $A_i=$solutions to plynomial such that $x_i>n_i$. Show that $A_1\cap A_2\cap\dots\cap A_k=\emptyset$ 
\subsection*{\#24a}
find \# of ways to place 6 nonattacking rooks on the board shown. 
%\begin{table}[|c|c|c|c|c|c|]
%x&x\\
%\end{table}
$=6!-r_15!+r_24!-r_33!+r_42!-r_51!+r_60!$
\begin{align*}
  r_i&=&\text{\# of ways to place $i$ rooks on forbidden positions}\\
  r_1&=6&\text{number of forbidden positions}\\
  r_2&=\binom{3}{2}\cdot2^2&\text{pick 2 rows, pick one of 2 spots in each row}\\
  r_3&=2^3\\
  r_4&=0\\
  r_5&=0\\
  r_6&=0
\end{align*}
\section*{project fun time}
inverse of a permutation. 
\begin{align*}
  321547896\\
  123456789\\
  321549678
\end{align*}
use combinatorial collection template to edit.

\subsection*{resources}
sage reference manual, wolfram mathworld, wikipedia, arxiv.org

links in statistic thingie to oeis

link from wiki to statistics page.

dyck paths and permutations are most well developed examples. example: [[St000013]]
\end{document}
