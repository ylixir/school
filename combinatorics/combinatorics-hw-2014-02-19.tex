\documentclass{article}
\usepackage{fullpage}
\usepackage{nopageno} 
\usepackage{amsmath}
\usepackage{amssymb}
\usepackage[normalem]{ulem}
\allowdisplaybreaks

\newcommand{\abs}[1]{\left\lvert #1 \right\rvert}

\begin{document}
Jon Allen

February 19, 2014

\section*{Chapter 3}
\subsection*{4.}
Show that if $n+1$ integers are chosen from the set $\{1,2,\dots,2n\}$, then there are always two which differ by 1.
\subsection*{proof}
Since we want all the integers to differ by more than one, we can only pickevery other integer from $\{1,2,\dots,2n\}$. This gives us a maximum of $n$ integers. Since we are choosing $n+1$ integers, we know that at least two of them must differ by only one.$\Box$
\subsection*{5.}
Show that if $n+1$ distinct integers are chosen from the set $\{1,2,\dots,3n\}$, then there are always two which differ by at most 2.
\subsection*{proof}
Since we want all the integers to differ by more than two, we can only pickevery third integer from $\{1,2,\dots,3n\}$. This gives us a maximum of $n$ integers. Since we are choosing $n+1$ integers, we know that at least two of them must differ by two or less.$\Box$
\subsection*{6.}
Generalize Exercises 4 and 5.
\subsection*{hypothesis}
If $n+1$ distinct integers are chosen from the set $\{1,2,\dots,mn\}$ where $m$ is a positive integer then there are always two which differ by at most $m-1$.
\subsection*{proof}
We can select at most $n$ integers which have a difference of $m$ or more. Since we are selecting $n+1$ integers then we must have at least two which differ by $m-1$ or less.$\Box$
\subsection*{8.}
Use the pigeonhole principle to prove that the decimal expansion of a rational number $m/n$ eventually is repeating. For example,
\begin{align*}
  \frac{34,478}{99,900}&=0.345125125125\cdots.
\end{align*}
\subsection*{12.}
Show by example that the conclusion of the Chinese rmainder theorem (Application 6) need not hold when $m$ and $n$ are not relatively prime.

Take 3 and 9 for $m$ and $n$. Take $2$ and $4$ for $a$ and $b$. Then we should be able to find an $x$ such that:
\begin{align*}
  x&=3p+2\\
  x&=9q+4\\
  3p+2&=9q+4\\
  3p&=9q+2\\
  3&\mid 3p\\
  3&\nmid 9q+2\\
  3p&\neq9q+2\\
\end{align*}
So we see that $x$ does not exist
\end{document}
