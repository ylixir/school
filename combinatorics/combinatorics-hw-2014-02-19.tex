\documentclass{article}
\usepackage{fullpage}
\usepackage{nopageno} 
\usepackage{amsmath}
\usepackage{amssymb}
\usepackage[normalem]{ulem}
\allowdisplaybreaks

\newcommand{\abs}[1]{\left\lvert #1 \right\rvert}

\begin{document}
Jon Allen

February 19, 2014

\section*{Chapter 3}
\subsection*{4.}
Show that if $n+1$ integers are chosen from the set $\{1,2,\dots,2n\}$, then there are always two which differ by 1.
\subsection*{5.}
Show that if $n+1$ distinct integers are chosen from the set $\{1,2,\dots,3n\}$, then there are always two which differ by at most 2.
\subsection*{6.}
Generalize Exercises 4 and 5.
\subsection*{8.}
Use the pigeonhole principle to prove that the decimal expansion of a rational number $m/n$ eventually is repeating. For example,
\begin{align*}
  \frac{34,478}{99,900}&=0.345125125125\cdots.
\end{align*}
\subsection*{12.}
Show by example that the conclusion of the Chinese rmainder theorem (Application 6) need not hold when $m$ and $n$ are not relatively prime.
\end{document}
