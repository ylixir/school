\documentclass{article}
%\usepackage{fullpage}
%\usepackage{nopageno} 
\usepackage[margin=1.5in]{geometry}
\usepackage{amsmath}
\usepackage{amssymb}
\usepackage[normalem]{ulem}
\usepackage{fancyhdr}
%\renewcommand\headheight{12pt}
\pagestyle{fancy}
\lhead{February 19, 2014}
\rhead{Jon Allen}
\allowdisplaybreaks

\newcommand{\abs}[1]{\left\lvert #1 \right\rvert}

\begin{document}
\section*{Chapter 3}
\subsection*{4.}
Show that if $n+1$ integers are chosen from the set $\{1,2,\dots,2n\}$, then there are always two which differ by 1.
\subsection*{proof}
Since we want all the integers to differ by more than one, we can only pickevery other integer from $\{1,2,\dots,2n\}$. This gives us a maximum of $n$ integers. Since we are choosing $n+1$ integers, we know that at least two of them must differ by only one.$\Box$
\subsection*{5.}
Show that if $n+1$ distinct integers are chosen from the set $\{1,2,\dots,3n\}$, then there are always two which differ by at most 2.
\subsection*{proof}
Since we want all the integers to differ by more than two, we can only pickevery third integer from $\{1,2,\dots,3n\}$. This gives us a maximum of $n$ integers. Since we are choosing $n+1$ integers, we know that at least two of them must differ by two or less.$\Box$
\subsection*{6.}
Generalize Exercises 4 and 5.
\subsection*{hypothesis}
If $n+1$ distinct integers are chosen from the set $\{1,2,\dots,mn\}$ where $m$ is a positive integer then there are always two which differ by at most $m-1$.
\subsection*{proof}
We can select at most $n$ integers which have a difference of $m$ or more. Since we are selecting $n+1$ integers then we must have at least two which differ by $m-1$ or less.$\Box$
\subsection*{8.}
Use the pigeonhole principle to prove that the decimal expansion of a rational number $m/n$ eventually is repeating. For example,
\begin{align*}
  \frac{34,478}{99,900}&=0.345125125125\cdots.
\end{align*}
\subsection*{proof}
We assume the $n>0$ because if $n=0$ we don't really have a rational number and if $n<0$ we can simply multiply by $\frac{-1}{-1}$ to make $n$ positive.

Now we will start building our decimal representation of our number by dividing.
\begin{align*}
  m&=q_0n+r_0,\qquad 0\le r_0\le n-1
\end{align*}
Now we have an integer part $q_0$ and our fractional decimal part $\frac{r_0}{n}$. We expand our fractional decimal digits by multiplying successive remainders by 10 and dividing by $n$ repeatedly. So for our $\frac{1}{10^i}$ place we have:
\begin{align*}
  r_{i-1}\cdot10&=q_in+r_i
\end{align*}
This will give us $q_i$ which is the digit in the $\frac{1}{10^i}$th spot. Notice that because we are dividing by $n$ our remainders will always satisfy $0\le r_i\le n-1$. Now lets take some sequence of $n$ remainders from our fractional expansion. Say $r_i,\dots,r_{i+n-1}$. Because we have $n$ remainders which can have $n-1$ possible values, we know from the pigeon hole principle that at least two of these remainders are the same. Lets pick $r_j$ from the sequence $r_i,\dots,r_{i+n-1}$ such that $r_j=r_i$. Applying the division algorithm to obtain:
\begin{align*}
  r_{i}\cdot10&=q_{i+1}n+r_{i+1}=r_{j}\cdot10\\
  r_{j}\cdot10&=q_{j+1}n+r_{j+1}=r_{i}\cdot10
\end{align*}
We know that because the remainders and quotients of division are unique $q_{i+1}=q_{j+1}$ and $r_{i+1}=r_{j+1}$. Given this we can say from induction that for any integer $1\le k$
\begin{align*}
  q_{i+k}&=q_{j+k}\\
  r_{i+k}&=r_{j+k}
\end{align*}
And in fact $q_{i+k}=q_{j+k}=q_{i+2(j-i)+k}=q_{i+3(j-i)+k}=q_{i+4(j-i)+k}=\dots$ and so we see that fractional part of the decimal representation of the ratio will at some point begin to repeat.$\Box$
\subsection*{12.}
Show by example that the conclusion of the Chinese rmainder theorem (Application 6) need not hold when $m$ and $n$ are not relatively prime.

Take 3 and 9 for $m$ and $n$. Take $2$ and $4$ for $a$ and $b$. Then we should be able to find an $x$ such that:
\begin{align*}
  x&=3p+2\\
  x&=9q+4\\
  3p+2&=9q+4\\
  3p&=9q+2\\
  3&\mid 3p\\
  3&\nmid 9q+2\\
  3p&\neq9q+2\\
\end{align*}
So we see that $x$ does not exist
\end{document}
