\documentclass{article}
%\usepackage{fullpage}
%\usepackage{nopageno} 
\usepackage[margin=1.5in]{geometry}
\usepackage{amsmath}
\usepackage{amssymb}
\usepackage[normalem]{ulem}
\usepackage{fancyhdr}
%\renewcommand\headheight{12pt}
\pagestyle{fancy}
\lhead{March 12, 2014}
\rhead{Jon Allen}
\allowdisplaybreaks

\newcommand{\abs}[1]{\left\lvert #1 \right\rvert}

\begin{document}
\section*{Project Prep}
I chose binary trees. I found out you can map Dyck paths to binary trees and back. I also never really thought about how many ways you could symmetrise the treees. The statistic finder is meh, just generates trees. The database is interesting in that the statistics in there are kind of random from where I'm sitting. Like for example the first statistic is ``The number of left oriented leafs except the first one.''
\section*{Chapter 5}
\begin{enumerate}
\setcounter{enumi}{29}\item
Prove that the only antichain of $S=\{1,2,3,4\}$ of size 6 is the antichain of all 2-subsets of $S$

We can easily see that the only antichain that contains $\emptyset$ is $\{\emptyset\}$, similarly the only antichain containing $S$ is $\{S\}$.
Both of these are obviously of size one.
We see then that any antichain that is to be bigger than 6 must contain a subset of size 1,2 or 3.
Lets pick a subset of size one to be in our antichain.
Without loss of generality we say that subset is $\{1\}$.
Since any other subsets in our antichain must not be a superset of $\{1\}$ we can form the remaining subsets in our antichain from $\{2,3,4\}$.
We know from theorem 5.3.3 that we can form an antichain of maximum size $\binom{3}{\left\lfloor \frac{3}{2}\right\rfloor}=3$ from this set of size 3.
The maximum size then of an antichain containing a subset of size 1 is then $3+1=4$ which is less than 6.
The antichain we are looking for then must be made up entirely of subsets of size 2 and 3.
Since the antichain we are looking for is not all subsets of size two, we must have at least one of size 3.

Let us assume our antichain contains a subset of size 3.
Canditates for the antichain consist of any subsets of $A$ with the last element from $S$ added in, or the last element of $S$ only.
Without loss of generality we choose a subset $A$ of size 3 from $S$ to be in our antichain.
We'll say $A=\{1,2,3\}$.
Then an antichain with $A$ can also contain the subset the remaining element, $\{4\}$ or any two combinations from $S$ which contain the remaining element and one element from $A$, eg, $\{1,4\}$, or three combinations of the remaining element and two elements from $A$ eg, $\{1,2,4\}$. Any subsets not containing the last element are subsets of $A$ and therefore not candidates for inclusion in the antichain.
There is one possible combination consisting solely of the last element, 3 combinations consisting of the remaining element and one of the elements from $A$ and 3 combinations consisting of the remaining element and two of the 3 elements in $A$.
So we have taken one of our 6 possible subsets with $A$. Leaving us with 5 slots to fill with 7 candidates.
If we choose the remaining element as a subset of our antichain, then none of the other candidates will fit, as they all also contain the remaining element. This gives us an antichain of size 2.
Likewise if we choose one of the 2 element subsets consisting of an element in $A$ and the remaining element of $S$ then we have chosen a subset that is also a subset of two of the 3 element subsets. This gives us an antichain of size 2 and a pool of size 3. This gives us a maximum of 5 subsets in our antichain at this point.
So, no matter how we choose from our pool of 7 candidates, we reduce the available remaining candidates to the point where we can not build an antichain of size 6.$\Box$
%Let us assume our antichain contains a subset of size 2. Say without loss of generality  $\{1,2\}$. Then the biggest possible antichain containing this set is given by adding the sizes of the largest possible antichains of the original set without each element of $\{1,2\}$ and subtracting the possible antichains of the original set without either element. So again without loss of genrality we have $\{2,3,4\}$ and $\{1,3,4\}$ minus largest antichain of $\{3,4\}$ which is $2\cdot\binom{3}{\left\lfloor\frac{3}{2}\right\rfloor}-\binom{2}{\left\lfloor\frac{2}{2}\right\rfloor}=2*3-1=5$. And adding back in the original set we chose, we have a maximum antichain size of 6.
\setcounter{enumi}{32}\item
Construct a partition of the subsets of $\{1,2,3,4,5\}$ into symmetric chains.
\begin{align*}
  \emptyset\subset\{1\}\subset\{1,2\}\subset\{1,2,3\}\subset\{1,2,3,4\}\subset\{1,2,3,4,5\}\\
  \{5\}\subset\{1,5\}\subset\{1,2,5\}\subset\{1,2,3,5\}\\
  \{4\}\subset\{1,4\}\subset\{1,2,4\}\subset\{1,2,4,5\}\\
  \{4,5\}\subset\{1,4,5\}\\
  \{2\}\subset\{2,3\}\subset\{2,3,4\}\subset\{2,3,4,5\}\\
  \{2,5\}\subset\{2,3,5\}\\
  \{2,4\}\subset\{2,4,5\}\\
  \{3\}\subset\{1,3\}\subset\{1,3,4\}\subset\{1,3,4,5\}\\
  \{3,5\}\subset\{1,3,5\}\\
  \{3,4\}\subset\{3,4,5\}\\
\end{align*}
%34 is for graduate students
%\item
%In a partition of the subsets of $\{1,2,\dots,n\}$ into symmetric chains, how many chains have only one subset in them? two subsets? $k$ subsets?
\setcounter{enumi}{36}\item
Use the multinomial theorem to show that, for positive integers $n$ and $t$,
\begin{align*}
  t^n&=\sum{\binom{n}{n_1n_2\dots n_t}},
\end{align*}
where the summation extends over all nonnegative integral solutions $n_1,n_2,\dots,n_t$ of $n_1+n_2+\dots+n_t=n$
\begin{align*}
  t^n&=(1_1+1_2+\dots1_t)^n\\
  &=\sum{\binom{n}{n_1n_2\cdots n_t}{1_1}^{n_1}{1_2}^{n_2}\cdots{1_t}^{n_t}}\\
  &=\sum{\binom{n}{n_1n_2\cdots n_t}}
\end{align*}
\item
Use the multinomial theorem to expand $(x_1+x_2+x_3)^4$
\begin{align*}
  (x_1+x_2+x_3)^4&=\sum\limits_{n_1+n_2+n_3=4}{\binom{4}{n_1n_2n_3}{x_1}^{n_1}{x_2}^{n_2}{x_3}^{n_3}}\\
  &=\binom{4}{4\;0\;0}{x_1}^4{x_2}^0{x_3}^0
  +\binom{4}{0\;4\;0}{x_1}^0{x_2}^4{x_3}^0
  +\binom{4}{0\;0\;4}{x_1}^0{x_2}^0{x_3}^4\\
  &\quad+\binom{4}{3\;1\;0}{x_1}^3{x_2}^1{x_3}^0
  +\binom{4}{3\;0\;1}{x_1}^3{x_2}^0{x_3}^1
  +\binom{4}{0\;3\;1}{x_1}^0{x_2}^3{x_3}^1\\
  &\quad+\binom{4}{1\;3\;0}{x_1}^1{x_2}^3{x_3}^0
  +\binom{4}{0\;1\;3}{x_1}^0{x_2}^1{x_3}^3
  +\binom{4}{1\;0\;3}{x_1}^1{x_2}^0{x_3}^3\\
  &\quad+\binom{4}{2\;0\;2}{x_1}^2{x_2}^0{x_3}^2
  +\binom{4}{2\;2\;0}{x_1}^2{x_2}^2{x_3}^0
  +\binom{4}{0\;2\;2}{x_1}^0{x_2}^2{x_3}^2\\
  &\quad+\binom{4}{2\;1\;1}{x_1}^2{x_2}^1{x_3}^1
  +\binom{4}{1\;2\;1}{x_1}^1{x_2}^2{x_3}^1
  +\binom{4}{1\;1\;2}{x_1}^1{x_2}^1{x_3}^2\\
  &={x_1}^4+{x_2}^4+{x_3}^4\\
  &\quad+4{x_1}^3{x_2}+4{x_1}^3{x_3}+4{x_2}^3{x_3}\\
  &\quad+4{x_1}{x_2}^3+4{x_2}{x_3}^3+4{x_1}{x_3}^3\\
  &\quad+6{x_1}^2{x_3}^2+6{x_1}^2{x_2}^2+6{x_2}^2{x_3}^2\\
  &\quad+12{x_1}^2{x_2}{x_3}+12{x_1}{x_2}^2{x_3}+12{x_1}{x_2}{x_3}^2\\
\end{align*}
\setcounter{enumi}{39}\item
What is the coefficient of ${x_1}^3{x_2}^3x_3{x_4}^2$ in the expansion of
\[(x_1-x_2+2x_3-2x_4)^{9}?\]

\[\binom{9}{3\;3\,1\;2}\cdot1^3\cdot(-1)^3\cdot2^1\cdot(-2)^2=5040\cdot -1\cdot2\cdot4=-40320\]

\setcounter{enumi}{41}\item
Prove the identity (5.21) by a combinatorial argument. (\emph{Hint:} Consider the permutations of a multiset of objects of $t$ different types with repetition numbers $n_1,n_2,\dots,n_t$, respectively. Partition these permutations according to what type of object is in the first position.)

We know that $\binom{n}{n_1n_2\cdots n_t}$ gives us the number of permutations of a multiset of objects of $t$ different types with repetition numbers $n_1,n_2,\dots,n_t$, respectively. Now lets calculate the number of permutations of the same multiset which start with an object of type $i$. The number of objects we are choosing from is reduced by one to $n-1$ because the first object is chosen already. Likewise we have $n_i-1$ objects of type $i$ to choose from. So we have $\binom{n-1}{n_1,\dots,n_i-1,\dots,n_t}$ permutations which start with the $i$ object type. Now if we want to find the total number of permutations, we just add together all the permutations possible possible which start with every type.
\begin{align*}
  \sum\limits_{i=1}^t{\binom{n-1}{n_1\dots n_i\dots n_t}}
\end{align*}
Which gives us our identity.$\Box$
\end{enumerate}
\section*{Chapter 6}
\begin{enumerate}
\item
Find the number of integers between 1 and 10,000 inclusive that are not divisible by 4,5, or 6.

I'll start by just saying that all the math I do will be integer math, e.g. $\frac{10}{2}=2$. So we can just subtract from 10,000 the number of integers that are divisable by 4,5, or 6 to get the number of integers that are not. The number of integers divisable by 4 is $10000/4=2500$. Similarly for 5 we have $10000/5=2000$ and for 6 we have $10000/6=1666$. Since 20 is the least common multiple of 5 and 4 the number of integers that can be obtained by dividing 10000 by 4 or 5 is $10000/20=500$ and similarly for 4 and 6 we have $10000/12=833$. And for 5 and 6 $10000/30=333$. Finally the number of integers that can be obtained by dividing by 4,5, or 6 is $10000/60=166$. Putting it all together with the inclusion-exclusion principle (inverted since we are subtracting from the total) we have $10000-2500-2000-1666+500+833+333-166=5334$ integers.
\item
Find the number of integers between 1 and 10,000 inclusive that are not divisible by 4,6,7, or 10.

As in problem 1, all math will be integer math, not real. Logic is also the same as in problem 1.
\begin{align*}
  \text{lcm}(4,6)&=12&\text{lcm}(4,7)&=28&\text{lcm}(10,4)&=20\\
  \text{lcm}(6,7)&=42&\text{lcm}(6,10)&=30&\text{lcm}(7,10)&=70\\
  \text{lcm}(4,6,7)&=84&\text{lcm}(4,6,10)&=60&\text{lcm}(6,7,10)&=210&\text{lcm}(4,7,10)&=140\\
  \text{lcm}(4,6,7,10)&=420\\
  10000/4&=2500&10000/6&=1666&10000/7&=1428&10000/10&=1000\\
  10000/12&=833&10000/28&=357&10000/20&=500\\
  10000/42&=238&10000/30&=333&10000/70&=142\\
  10000/84&=119&10000/60&=166&10000/210&=47&10000/140=71\\
  10000/420&=23
\end{align*}
So we have $10000-2500-1666-1428-1000+833+357+500+238+333+142-119-166-47-71+23=5429$ integers.
\item
Find the number of integers between 1 and 10,000 that are neither perfect squares nor perfect cubes.

Because $\sqrt{10000}=100$ the perfect squares between 1 and 10000 inclusive are $\{1^2,2^2,\dots,100^2\}$. There are then 100 perfect squares. Similarly because $21^3<10000<22^3$ there are 21 perfect cubes which are $\{1^3,2^3,\dots,21^3\}$. Now if something is a perfect cube \emph{and} a perfect square, than it is a perfect sixth root. Because $4^6<10000<5^6$ there are 4 numbers which are perfect roots and perfect squares between 1 and 10000. Putting it all together we have $10000-100-21+4=9883$ integers that are neither perfect cubes, nor perfect squares.
\end{enumerate}

\end{document}
