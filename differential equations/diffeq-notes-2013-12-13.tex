\documentclass{article}
\usepackage{fullpage}
\usepackage{nopageno}
\usepackage{amsmath}
\allowdisplaybreaks

\newcommand{\abs}[1]{\left\lvert #1 \right\rvert}

\begin{document}
\title{Notes}
\date{December 13, 2013}
\maketitle
\section*{final exam}
thursday dec 19. see blackboard for details

unit step function is zero before a and 1 after. denoted U(t-a)
\begin{align*}
  \mathcal{L}\{U(t-a)\}&=\frac{e^{-as}}{s}\\
  \mathcal{L}\{f(t-a)U(t-a)\}&=e^{-as}F(s)
\end{align*}
for the exam you should be able to write a problem in terms of the unit step function. for example function that starts as 10 and switches to 0 at 2 is
\begin{align*}
  f(t)&=10-10U(t-a)
  \mathcal{L}\{f(t)\}&=\frac{10}{s}-\frac{10}{s}e^{-2s}
\end{align*}
that is one possibility for exam, another is suppose you are given function. compute inverse laplace
\begin{align*}
  \frac{e^{-\frac{\pi s}{r6s}}}{s^3}\\
  \mathcal{L}^{-1}\{\frac{e^{-\frac{\pi s}{r6s}}}{s^3}\}\\
  \mathcal{L}^{-1}\{e^{-as}F(s)\}&=f(t-a)U(t-a)\\
  a&=\frac{\pi}{6}\\
  F(s)&-\frac{2}{2}\frac{1}{s^3}\\
  f(t)&=\frac{1}{2}t^2\\
  \mathcal{L}^{-1}\{\frac{e^{-\frac{\pi s}{r6s}}}{s^3}\}&=\frac{1}{2}(t-\frac{\pi}{6})U(t-\frac{\pi}{6})
\end{align*}
\subsection*{announcement}
he'll be in office from 1-3 on mw and in mathin tomorrow from 11-1
\subsection*{example}
\begin{align*}
  x\frac{\mathrm{d}y}{\mathrm{d}x}+y&=e^{-x}\\
  \frac{\mathrm{d}y}{\mathrm{d}x}+\frac{1}{x}y&=\frac{e^{-x}}{x}\\
  \intertext{it is linear, so compute integration factor}
  e^{int{\frac{1}{x}\,\mathrm{d}x}}=e^{\ln x}&=x\\
  \intertext{multiply out by integration factor, it's already in exact form}
  x\frac{\mathrm{d}y}{\mathrm{d}x}+y&=e^{-x}\\
  \frac{\mathrm{d}}{\mathrm{d}x}\left(y*IF\right)&=e^{-x}\\
  \frac{\mathrm{d}}{\mathrm{d}x}\left(yx\right)&=e^{-x}\\
  \int{d(yx)}&=int{e^{-x}\,\mathrm{d}x}
\end{align*}
in the exam there is an integral differentiation problem like this
\begin{align*}
  \frac{\mathrm{d}y}{\mathrm{d}x}-4y+4\int_0^t{y(u)\,\mathrm{d}u}&=t^4e^{2t}\\
  y(0)&=0\\
  \mathcal{L}\{y(t)\}&=Y(s)\\
  \mathcal{L}\{\frac{\mathrm{d}y}{\mathrm{d}t}\}&=sY(s)-y(0)\\
  \mathcal{L}\{-4y\}&=-4Y(s)\\
  \mathcal{L}\{4\int_0^t{y(u)\,\mathrm{d}u}\}&=\frac{4}{s}Y(s)\\
  \mathcal{L}\{t^4\}&=\frac{4!}{s^5}=\frac{24}{s^5}\\
  \mathcal{L}\{t^4e^{2t}\}&=\frac{24}{(s-2)^5}\\
  (s-4+\frac{4}{s})Y(s)&=\frac{24}{(s-2)^5}\\
  (s^2-4s+4)Y(s)&=\frac{24s}{(s-2)^5}\\
  (s-2)^2Y(s)&=\frac{24s}{(s-2)^5}\\
  Y(s)&=\frac{24s}{(s-2)^7}\\
  &=\frac{24(s-2)+48}{(s-2)^7}\\
  &=\frac{24}{(s-2)^6}+\frac{48}{(s-2)^7}\\
  y(t)&=\frac{1}{5}t^5e^{2t}+\frac{1}{15}t^6e^{2t}
\end{align*}
\begin{align*}
  y^2\mathrm{d}t+2ty\mathrm{d}y&=0 \intertext{check for exactness}\\
  \frac{\partial M}{\partial y}&=\frac{\partial N}{\partial t}=2y\\
  \frac{\partial f}{\partial t}&=y^2\\
  \frac{\partial f}{\partial y}&=2ty\\
  \int{\mathrm{d}f}&=\int{y^2\mathrm{d}t}=y^2t+\phi(y)\\
  \frac{\partial f}{\partial y}&=2yt+\phi'(y)\\
  2ty&=2yt+\phi'(y)\\
  \phi'(y)&=0\\
  \phi'(y)&=C\\
  y^2t+C&=\text{constant}\\
\end{align*}
\end{document}
