\documentclass{article}
\usepackage{fullpage}
\usepackage{nopageno}
\usepackage{amsmath}
\allowdisplaybreaks

\newcommand{\abs}[1]{\left\lvert #1 \right\rvert}
\newcommand{\degree}{\ensuremath{^\circ}}

\begin{document}
Jon Allen

November 13, 2013

\section*{8.1}
compute the Laplace transforms
\subsection*{\#5}
\[f(t)=2\sin 2t\]

\subsubsection*{solution}
\begin{align*}
	\mathcal{L}\{2\sin 2t\}&
	=2\int_0^\infty{e^{-st}\sin 2t\,\mathrm{d}t} \qquad \text{use maxima to do indefinite integral}\\
	&=2\cdot\lim_{M\to \infty}\left[\frac{1}{s^2+4}e^{-st}(-s\cdot\sin 2t - 2 \cos 2t)\right]_{t=0}^M\\
	&=\frac{2}{s^2+4}\left[0-(-s\cdot\sin 0-2\cos 0)\right]=\frac{4}{s^2+4}
\end{align*}

\subsection*{\#6}
\[
	f(t)=\left\{
	\begin{aligned}
		1&, & &\text{if} & 0\leq&t\leq2\\
		0&, & &\text{if} & &t>2
	\end{aligned}
	\right.
\]

\subsubsection*{solution}
\begin{align*}
	\mathcal{L}\{f(t)\}&
	=\int_0^\infty{e^{-st}f(t)\,\mathrm{d}t}
	=\int_0^2{e^{-st}\,\mathrm{d}t}+\int_2^\infty{0\,\mathrm{d}t}
	=\left[-\frac{1}{s}e^{-st}\right]_{t=0}^2+0
	=-\frac{1}{s}(e^{-2s}-1)
	=\frac{1}{s}-\frac{1}{se^{2s}}
\end{align*}

\subsection*{\#7}
\[
	f(t)=\left\{
	\begin{aligned}
		0&, & &\text{if} & 0\leq&t\leq 1\\
		1&, & &\text{if} & &t>1
	\end{aligned}
	\right.
\]

\subsubsection*{solution}
\begin{align*}
	\mathcal{L}\{f(t)\}&
	=\int_0^\infty{e^{-st}f(t)\,\mathrm{d}t}
	=\int_0^1{0\,\mathrm{d}t}+\int_1^\infty{e^{-st}\,\mathrm{d}t}
	=0+\lim_{M \to \infty}\left[-\frac{1}{s}e^{-st} \right]_{t=1}^M
	=-\frac{1}{s}(0-e^{-s})
	=\frac{1}{se^s}
\end{align*}

\subsection*{\#16}
\[f(t)=\cos kt\]
\subsubsection*{solution}
\begin{align*}
	\mathcal{L}\{\cos kt\}&=\int^\infty_0{e^{-st}\cos kt\,\mathrm{d}t} \qquad \text{use maxima to calculate indefinite integral}\\
	&=\lim_{M \to \infty}\left[\frac{1}{s^2+k^2}e^{-st}(k\cdot\sin kt-s\cdot\cos kt)\right]_{t=0}^M
	=\frac{1}{s^2+k^2}\left[0-(k\cdot\sin0-s\cdot\cos 0)\right]=\frac{s}{s^2+k^2}
\end{align*}

\subsection*{\#23}
\[\mathcal{L}\{e^{-t}\sin 5t\}\]
\subsubsection*{solution}
\begin{align*}
	\mathcal{L}\{\sin 5t\}&=\frac{5}{s^2+25}\\
	\mathcal{L}\{e^{-t}f(t)\}&=F(s+1)=\frac{5}{(s+1)^2+25}=\frac{5}{s^2+2s+26}
\end{align*}

\section*{8.2}

\subsection*{\#10}
\[F(s)=\frac{1}{s^2+12s+61}\]
\subsubsection*{solution}
\begin{align*}
	\frac{1}{s^2+12s+61}&=\frac{1}{(s+6)^2+25}\\
	\mathcal{L}\{e^{-6t}f(t)\}&=F(s+6) & \mathcal{L}\{\sin 5t\}&=\frac{5}{s^2+25}\\
	\mathcal{L}\{\frac{1}{5}e^{-6t}\sin 5t\}&=\frac{1}{(s+6)^2+25} & f(t)&=\frac{1}{5}e^{-6t}\sin 5t
\end{align*}
\subsection*{\#11}
\[F(s)=\frac{s}{s^2-5s-14}\]
\subsubsection*{solution}
\begin{align*}
	\frac{s}{s^2-5s-14}&=\frac{s}{(s-7)(s+2)} &
	\frac{1}{(s-7)(s+2)}&=\frac{A}{s-7}+\frac{B}{s+2}\\
	A(s+2)+B(s-7)&=1=s(A+B)+2A-7B & A=-B &\quad 2A+7A=1\\
	F(s)=\frac{1}{9}\left(\frac{s}{s-7}-\frac{s}{s+2}\right)&=\frac{1}{9}\left(\frac{s-7+7}{s-7}-\frac{s+2-2}{s+2}\right) \\
	=\frac{1}{9}\left(1+7\frac{1}{s-7}-1+2 \frac{1}{s+2}\right)&=\frac{7}{9}\left(\frac{1}{s-7}\right)+\frac{2}{9}\left(\frac{1}{s+2}\right) & f(t)&=\frac{7}{9}e^{7t}+\frac{2}{9}e^{-2t}
\end{align*}

\end{document}
