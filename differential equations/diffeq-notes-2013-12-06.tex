\documentclass{article}
\usepackage{fullpage}
\usepackage{nopageno}
\usepackage{amsmath}
\allowdisplaybreaks

\newcommand{\abs}[1]{\left\lvert #1 \right\rvert}

\begin{document}
\title{Notes}
\date{December 6, 2013}
\maketitle
quiz in 6.2-6.4 fri?

\section*{section 6.3}
\begin{align*}
  x'&=3x+3y\\
  y'&=3x+8y \intertext{both x and y are functions of t}
\end{align*}
\subsection*{example}
show that $\left(\begin{array}{c}-2e^{2t}\\e^t\end{array}\right)$ is a solution of $\mathbb{X}'\left(\begin{array}{cc}1&-2\\2&6 \end{array}\right)\mathbb{x}$
\begin{align*}
  \mathbb{X}'&=\left(\begin{array}{c}
      -4e^{2t}\\2e^{2t}
      \end{array}\right)
  &=\left(\begin{array}{cc}
      1&-2\\2&6
      \end{array}\right)
  \left(\begin{array}{c}
      -2e^{2t}\\e^{2t}
      \end{array}\right)
  &=\left(\begin{array}{c}
      -2e^{2t}-2e^{2t}\\
      -4e^{2t}+6e^{2t}
      \end{array}\right)
  &=\left(\begin{array}{c}
      -4e^{2t}\\2e^{2t}
      \end{array}\right)
\end{align*}
\begin{align*}
  W&=\left\lvert\begin{array}{cc}y_1&y_2\\y_1'&y_2'\end{array}\right\rvert\\
  \phi_1&=\left(\begin{array}{c}\alpha_1(t)\\\alpha_2(t)\end{array}\right)\\
  \phi_2&=\left(\begin{array}{c}\beta_1(t)\\\beta_2(t)\end{array}\right)\\
  W(\phi_1,\phi_2)&=\left(\begin{array}{cc}\alpha_1&\beta_1\\ \alpha_2&\beta_2 \end{array}\right)
\end{align*}
\subsection*{example}
check if $\phi$'s are independant solutions
\begin{align*}
  \mathbb{X}'&=\left(\begin{array}{cc}4&-3\\-2&-1 \end{array}\right)\mathbb{X}\\
  \phi_1&=\left(\begin{array}{c}e^{-2t}\\2e^{-2t} \end{array}\right)\\
  \phi_2&=\left(\begin{array}{c}3e^{5t}\\e^{5t} \end{array}\right)\\
  W(\phi_1,\phi_2)&=e^{3t}+6e^{3t}=7e^{3t}\neq0
\end{align*}
they are independant

\subsection*{eigenvalues}
\begin{align*}
  A&=\left(\begin{array}{cc}2&3\\-4&-5\end{array}\right)
\end{align*}
find the eigenvalue of A
\begin{align*}
  \left\lvert\righ\left(\begin{array}{cc}2-\lambda&3\\-4&-5-\lambda\end{array}\right)t\rvert&=0\\
    (2-\lambda)(-5-\lambda)+12&=0\\
    -10-2\lambda+5\lambda+\lambda^2+12&=0\\
    \lambda^2+3\lambda+2&=0\\
    (\lambda+1)(\lambda+2)&=0\\
    \lambda&=-1,-2
\end{align*}
eigenvector corresponding to $\lambda=-1$
\begin{align*}
  A\left(\begin{array}{c}x_1\\x_2\end{array}\right)&=-1\left(\begin{array}{c}x_1&x_2\end{array}\right)\\
  \left(\begin{array}{cc}2&3\\-4&-5\end{array}\right)\left(\begin{array}{c}x_1&x_2\end{array}\right)&=-\left(\begin{array}{c}x_1&x_2\end{array}\right)\\
  \left(\begin{array}{c}2x_1+3x_2\\-4x_1-5x_2\end{array}\right)=-\left(\begin{array}{c}x_1&x_2\end{array}\right)\\
  2x_1+3x_2&=-x_1\\
  -4x_1-5x_2&=x_2\\
  3x_1+3x_2&=0\\
  -4x_1-4x_2&=0\\
  x_1+x_2&=0\\
  x_1&=x_2 \intertext{choose}
  x_1&=1,x_2=-1\\
  \text{eigenvector}&=\left(\begin{array}{c}1\\-1\end{array}\right)
\end{align*}
eigenvector corresponding to $\lambda=-2$ is $\left(\begin{array}{c}3\\-4\end{array}\right)$
  
quiz show that eigenvector
\end{document}
