%% LyX 2.0.6 created this file.  For more info, see http://www.lyx.org/.
%% Do not edit unless you really know what you are doing.
\documentclass[letterpaper,english]{article}
\usepackage[T1]{fontenc}
\usepackage[latin9]{inputenc}
\usepackage{amsmath}
\usepackage{esint}

\makeatletter

%%%%%%%%%%%%%%%%%%%%%%%%%%%%%% LyX specific LaTeX commands.
\pdfpageheight\paperheight
\pdfpagewidth\paperwidth


\makeatother

\usepackage{babel}
\begin{document}
Jon Allen

Homework 4

September 25


\section*{2.5}


\subsection*{29}

Solve $(t^{2}-y^{2})\mathrm{d}y+(y^{2}+ty)\mathrm{d}t=0$.

$(x^{2}t^{2}-x^{2}y^{2})=x^{2}(t^{2}-y^{2})=(x^{2}y^{2}+xtxy)$

The equation is homogeneous so we substitute $t=vy$

\begin{align*}
(v^{2}y^{2}-y^{2})\,\mathrm{d}y+(y^{2}+vy^{2})(y\,\mathrm{d}v+v\,\mathrm{d}y) & =0\\
(v^{2}y^{2}-y^{2})\,\mathrm{d}y+y^{2}y\,\mathrm{d}v+vy^{2}y\,\mathrm{d}v+y^{2}v\,\mathrm{d}y+vy^{2}v\,\mathrm{d}y & =0\\
(v^{2}y^{2}-y^{2}+y^{2}v+v^{2}y^{2})\,\mathrm{d}y+(y^{3}+vy^{3})\,\mathrm{d}v & =0\\
y^{2}(v^{2}-1+v+v^{2})\,\mathrm{d}y+y^{3}(1+v)\,\mathrm{d}v & =0\\
y^{2}(2v^{2}+v-1)\,\mathrm{d}y+y^{3}(1+v)\,\mathrm{d}v & =0\\
y^{2}(2v^{2}+v-1)\,\mathrm{d}y & =-y^{3}(1+v)\,\mathrm{d}v\\
\frac{1}{y}\,\mathrm{d}y & =-\frac{1+v}{2v^{2}+v-1}\,\mathrm{d}v\\
\frac{1}{y}\,\mathrm{d}y & =-\frac{1+v}{(v+1)(2v-1)}\,\mathrm{d}v\\
\frac{1}{y}\,\mathrm{d}y & =-\frac{1}{2v-1}\,\mathrm{d}v & u=2v-1\\
\int\frac{1}{y}\,\mathrm{d}y & =-\frac{1}{2}\int\frac{1}{u}\,\mathrm{d}u & \mathrm{d}u=2\mathrm{d}v\\
\ln y & =\ln\frac{1}{\sqrt{2v-1}}+C_{0}\\
y & =e^{C_{0}}\frac{1}{\sqrt{2v-1}} & e^{C_{0}}=C\\
y & =C\frac{1}{\sqrt{2\frac{t}{y}-1}}
\end{align*}



\subsection*{36}

Solve $(t+y)\,\mathrm{d}t-t\,\mathrm{d}y=0,y(1)=1$
\begin{align*}
(tx+yx)-tx & =xM(x,y)+xN(x,y)
\end{align*}
The equation is homogeneous and $N(x,y)$is simpler so set $y=wt$and
$\mathrm{d}y=t\mathrm{\, d}w+w\,\mathrm{d}t$
\begin{align*}
(t+wt)\,\mathrm{d}t-t(t\mathrm{\, d}w+w\,\mathrm{d}t) & =0\\
t\,\mathrm{d}t-t^{2}\,\mathrm{d}w & =0\\
\frac{1}{t}\,\mathrm{d}t & =\mathrm{d}w\\
\int\frac{1}{t}\,\mathrm{d}t & =\int\mathrm{d}w\\
\ln t & =\frac{y}{t}+C_{0} & C=-C_{0}\\
y & =t(\ln t+C)\\
1 & =1(0+C)=C\\
y & =t\ln t+t
\end{align*}



\section*{2.6}

Do only the case for h=0.1


\subsection*{2}

Approximate the solution to the IVP $y'=4x-y+1,y(0)=0$at $x=1$for
$h=0.1$
\begin{align*}
y_{0} & =0\\
y_{1} & =0.1*(4*0-0+1)+0=0.1\\
y_{2} & =0.1*(4*0.1-0.1+1)+0.1=0.23\\
y_{3} & =0.1*(4*0.2-0.23+1)+0.23=0.387\\
y_{4} & =0.1*(4*0.3-0.387+1)+0.387=0.5683\\
y_{5} & =0.1*(4*0.4-0.5683+1)+0.5683=0.77147\\
y_{6} & =0.1*(4*0.5-0.77147+1)+0.77147=.9943230000000001\\
y_{7} & =0.1*(4*0.6-.994323+1)+.994323=1.2348907\\
y_{8} & =0.1*(4*0.7-1.2348907+1)+1.2348907=1.49140163\\
y_{9} & =0.1*(4*0.8-1.49140163+1)+1.49140163=1.762261467\\
y_{10} & =0.1*(4*0.9-1.762261467+1)+1.762261467=2.0460353203\\
y(1) & \approx2.0460353203
\end{align*}



\subsection*{3}

Approximate the solution to the IVP $y'-x=y^{2}-1,y(0)=1$at $x=1$for
$h=0.1$
\begin{align*}
y_{0} & =1\\
y_{1} & =0.1*(y_{0}^{2}-1+0)+y_{0}=1\\
y_{2} & =0.1*(y_{1}^{2}-1+0.1)+y_{1}=1.01\\
y_{3} & =0.1*(y_{2}^{2}-1+0.2)+y_{2}=1.03201\\
y_{4} & =0.1*(y_{3}^{2}-1+0.3)+y_{3}=1.06851446401\\
y_{5} & =0.1*(y_{4}^{2}-1+0.4)+y_{4}=1.122686779989858\\
y_{6} & =0.1*(y_{5}^{2}-1+0.5)+y_{5}=1.198729340586257\\
y_{7} & =0.1*(y_{6}^{2}-1+0.6)+y_{6}=1.302424543784494\\
y_{8} & =0.1*(y_{7}^{2}-1+0.7)+y_{7}=1.442055513009718\\
y_{9} & =0.1*(y_{8}^{2}-1+0.8)+y_{8}=1.630007923269891\\
y_{10} & =0.1*(y_{9}^{2}-1+0.9)+y_{9}=1.885700506262153\\
y(1) & \approx1.885700506262153
\end{align*}



\section*{3.1}


\subsection*{5}

Suppose that the half-life of an element is 1000 h. If there are initially
100 g, how much remains after 1 h? How much remains after 500 h?
\begin{align*}
y(0) & =100\\
y(1000) & =50\\
y(t) & =100e^{kt}\\
50 & =100e^{1000m}\\
\left(\frac{5}{10}\right)^{1/1000} & =e^{m}\\
\frac{\ln\frac{1}{2}}{1000} & =m\\
y(1) & =100e^{m}=99.93070929904525\mathrm{g}\\
y(500) & =100e^{500m}=70.71067811865476\mathrm{g}
\end{align*}



\subsection*{6}

Suppose that the population of a small town is initially 5000. Due
to the construction of an interstate highway, the population doubles
over the next year. If the rate of growth is proportional to the current
population, when will the population reach 25,000? What is the population
after 5 years?
\begin{align*}
y(0) & =5000\\
y(1) & =10000\\
y(t) & =y_{0}e^{mt}\\
10000 & =5000e^{m}\\
\ln2 & =m\\
25000 & =5000e^{t\ln2}\\
\ln5 & =t\ln2\\
t & =\frac{\ln5}{\ln2}\approx2.3\mathrm{years}\\
y(5) & =5000e^{5\ln2}=5000*2^{5}=160000\mathrm{people}
\end{align*}

\end{document}
