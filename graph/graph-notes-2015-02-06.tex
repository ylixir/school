\documentclass[letterpaper]{article}

\usepackage{fullpage}
\usepackage{nopageno}
\usepackage{amsmath}
\usepackage{amssymb}
\usepackage{tikz}
\usepackage[utf8]{inputenc}
\usepackage{aeguill}
\usepackage{setspace}

\usetikzlibrary{graphs,graphdrawing}
\usegdlibrary{trees}

\allowdisplaybreaks

\newcommand{\abs}[1]{\left\lvert #1 \right\rvert}

\begin{document}
\title{Notes}
\date{6 février, 2015}
\maketitle
wednesday was eulerian graphs (bridges of königsberg).

cycles are not circuits and trails are not paths

\section*{3.2 hamiltonian paths}
Euler is E, edges is E. Hamiltonian graphs is H, vertices is...not

a {\bfseries Hamiltonian path} or {\bfseries cycle} is a path or cycle that meets every vertex of $G$ exactly once.

a graph with hamiltonian cycle is called {\bfseries hamiltonian}
\subsection*{example}
\begin{enumerate}
\item
Hamiltonian:

\begin{tikzpicture}[main_node/.style={circle,draw,text=black,inner sep=1pt,outer sep=0pt]}]
  \node[main_node] (1) at (-1,-1) {};
  \node[main_node] (2) at (1,-1) {};
  \node[main_node] (3) at (1,1) {};
  \node[main_node] (4) at (-1,1) {};
  \draw (1) -- (2) -- (3) -- (4)--(1);
\end{tikzpicture}
\item
Not hamiltonian

\tikz\path [graphs/.cd, nodes={shape=circle, draw, text=black,inner sep=1pt,outer sep=0pt}]
  graph [tree layout] { 1 -- {2 -- 3};2--4 }
  [shift=(0:1)];
\item
is hamiltonian
\tikz\path [graphs/.cd, nodes={shape=circle, draw, text=black,inner sep=1pt,outer sep=0pt}]
  graph [tree layout] { 1 -- 2;1--3;1--4;1--5;1--6;2--3--4--5--6 }
  [shift=(0:1)];
\end{enumerate}

\subsection*{theorem (Ore)}
if $G$ is a graph of order $n\ge 3$ and $\forall u,v$ vertices $\deg(u)+\deg(v)\ge n$ then $G$ is hamiltonian

{\scshape note:} completely nonconstructive proof

\subsubsection*{proof}
assume for a contradiction that for all $u,v\in V(G)$, $\deg(u)+\deg(v)\ge n$ but $G$ is not hamiltonian.

without loss of generality, we can assume that $G$ is ``maximal'' with this property. why?

$G$ is finite therefore $G$ is a subgraph  of some complete graph $G\le K_n$. But $K_n$ is hamiltonian. Since $G$ is not hamiltonian and $K_n$ is then. somewhere added enough edges to $G$ to make it hamiltonian.

Add edge $xy$ to $G$. The $G\cup \{xy\}$ is hamiltonian, so there is an $x-y$ path in $G$. We have to use the $xy$ edge in the hamiltonian cycle, else the graph would already be hamiltonian. In fact the $x-y$ path is hamiltonian.

let the $x-y$ path be $x=v_1,\dots,v_n=y$. If $x$ is adjacent to $v_i$ then $y$ cannot be adjacent to $v_{i-1}$. why? because then you would have a hamiltonian cycle. $v_1v_i\dots v_nv_{i-1}v_1$

Therefore, for every neighbor of $x$ we can eliminate a possible neighbor of $y$. This means that $\deg(y)\le(n-1)-\deg(x)$. because $n-1$ is max possible deg and $\deg(x)$ are the things that can't be adjacent to $y$. Now $\deg(y)+\deg(x)\le n-1$ which is a contradiction because $\deg(y)+\deg(x)\ge n$

\subsection*{corollary}
if $G$ of order $n\ge 3$ has the property that for all $v\in V(G)$ then $\deg(v)\ge \frac{n}{2}$, then $G$ is hamiltonian.
\subsubsection*{proof}
$\forall u,v\in V(G)$ then $\deg(u)+\deg(v)\ge n$

\section*{independent sets}
a subset of $V(G)$ is called {\bfseries independent} if no vertices are adjacent to one another.

the maximal cardinality of independent sets is called the {\bfseries independent number}. it is denoted $\alpha(G)$

what is $\alpha(K_n)$? 1


\section*{theorem chvátal-erdös}
let $G$ be a graph of order $n\ge 3$. if connectedness$\kappa(G)\ge \alpha(G)$, then $G$ is Hamiltonian.

\section*{homework}
1,7,14
\end{document}
