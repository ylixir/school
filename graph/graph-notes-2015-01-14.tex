\documentclass[letterpaper]{article}

%\usepackage{fullpage}
%\usepackage{nopageno}
\usepackage{amsmath}
\usepackage{amssymb}
\allowdisplaybreaks

\newcommand{\abs}[1]{\left\lvert #1 \right\rvert}

\begin{document}
\title{Notes}
\date{January 14, 2015}
\maketitle
\section*{graph theory birthday}
leibniz and the geometry of postion (maybe called topology now)

200 years before algebraists, at this time it's all analysis

he says maybe you only care about relations, not distance

\subsection*{next}
mid 18th century, bridges of königsberg.

königsberg has river around it, with bridges throughout. check drawing online

game is to cross every bridge exactly once and return to starting place, enter euler

``very  little relationship to mathematics'' but changes his mind later. then solves every bridge problem ever. 21 paragraphs.

he also solved the optimal number of sails to have on a sailing vessel.

anachronistic to say that he invented graph theory. however the bridges problem is considered the original graph theory problem.

in modern theory/notation each land bit is a point and each bridge is an edge.

\subsection*{definition}
a graph is written $G=G(V,E)$ where $V, E$ are a pair of sets such that $E$ is a subset of $V\times V$ where $E$ can have repetitions.

\subsection*{abstract examples}
$V=\{v_1,v_2,v_3,v_4\}$, $E=\{(v_1,v_2),(v_1,v_2), (v_1,v_1), \dots\}$

\section*{definitions}
\subsection*{simple graph}
a graph is simple if $E$ has no repetitions and excludes $(v_i,v_i)$
\subsection*{multigraph}
a graph that is not simple is called a multigraph

königsberg is an example of this
\subsection*{incident/adjacent}
two vertices $v_i$ and $v_j$ are called incident or adjacent if $(v_i,v_j)$ is in $E$
\subsection*{neighborhood of $v_i$}
if $v_i\in V$ then $N_G(v_i)=\{(v_i,v_j)\in E\}$
\subsection*{order}
$|V(G)|$ is the order of G
\subsection*{size}
$|E(G)|$ is the size of G
\subsection*{path}
a path is a graph like $.\_.\_.\_.$ (a line with discrete points on it. this one is order 4, size 3)

a path on $n$ vertices is denoted $P_n$ but instructor will often call $P_{n-1}=P_n$ because $P_n$ had $P_{n-1}$ edges

\subsection*{trivial graph}
if $|V(G)|$ (order) is one and $|E(G)|$ (size) is zero then we have a trivial graph

this is opposed to not having any vertices or having multiple vertices and no edges

\subsection*{empty}
if $|E(G)|$ size is zero then we have an empty  graph
\subsection*{complete graph}
$E(G)=\{V\times V-(v_i,v_i)\}$ then $G$ is a complete graph
\subsubsection*{example}
$V=\{(v_1,v_2,v_3\}$ with a triangle graph.

the trivial graph is a complete graph

\subsubsection*{notation}
$K_n$ is complete graph  on $n$ vertices. 
$K_1$ is a point, $K_2$ is a line, $K_3$ is a triangle, $K_4$ is a square with an x

\subsection*{order of neighborhood}
$|N_G(v_i)|$ is the degree of a vertex (valence)

\section*{theorem}
if $G$ is finite and simple, then 
\[\sum\limits_{v_i\in V}{|N_G(v_i)|}=2|E|\]
\subsection*{proof}
adding each degree counts one end of each edge

each two vertices to each edge

\section*{homework}
define: path, cycle, isomorphism, subgraph, regular graph, bipartate graph, complement

numbers: 2,3,4,11,13,18
\end{document}
