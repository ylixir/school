\documentclass[letterpaper]{article}

\usepackage[utf8]{inputenc}
\usepackage{fullpage}
\usepackage{nopageno}
\usepackage{amsmath}
\usepackage{amssymb}
\usepackage{tikz}

\usetikzlibrary{graphs,graphdrawing}
\usegdlibrary{trees}

\allowdisplaybreaks

\newcommand{\abs}[1]{\left\lvert #1 \right\rvert}

\begin{document}
\title{Notes}
\date{January 23, 2015}
\maketitle
skipping 1.4

a graph is connected if for all $u,v$ in $E(G)$ there is a $u-v$ path

how can you tell if a graph is connected with the adjacency matrix? raise it to powers until it hits a nonzero entry. but you have to raise to infinity to prove non connectedness. this is inefficient at best.

there is a very easy way to do this

what happens when graph is disconnected?
\tikz\path [graphs/.cd, nodes={shape=circle, draw, text=black,inner sep=1pt,outer sep=0pt}]
  graph [tree layout] { 1 -- {2 -- 3} -- 1; 4--5 }
  [shift=(0:1)];


\[A(G)=\left[\begin{array}{ccccc}
0&1&1&0&0\\
1&0&1&0&0\\
1&1&0&0&0\\
0&0&0&0&1\\
0&0&0&1&0\\
\end{array}\right]\]

notice blocks of zeros on upper right and lower left

this depends on vertex labelling, note homework problem on isomorphic graphs and adjacency matrices

reduced row echelon form? determine negates when you switch columns or rows? same eigenvalues (?unsure). spectrum of graph

\section*{definition}
$k(G)$ is the number of components of a graph

\subsection*{example}
from above, $k(G)$ is 2

if $G$ is connected $k(G)=1$

\begin{description}
\item[cut vertex]
vertex $v$ such that $k(G)<k(G-v)$
\subsubsection*{example}
\tikz\path [graphs/.cd, nodes={shape=circle, draw, text=black,inner sep=1pt,outer sep=0pt}]
  graph [tree layout] { {1 -- 9}--8--7--6--5;1--{8--2}--1;8--3--6;{7--3};6--{4--5} }
  [shift=(0:1)];

8 and 6 are cut vertices
\item[nonseparable]
a connected graph with no cut vertices
\end{description}
\section*{theorem}
if $G$ is nonseparable and is simple and $|G|\ge 3$ then every pair of vertices lies on a cycle.

another way of thinking of this is that there are two paths between any two vertices.

\subsubsection*{proof}
picture:

\tikz\path [graphs/.cd, nodes={shape=circle, draw, text=black,inner sep=1pt,outer sep=0pt}]
  graph [tree layout] {  }
  [shift=(0:1)];
\begin{tikzpicture}[main_node/.style={circle,draw,text=black,inner sep=1pt,outer sep=0pt]}]

  \node[main_node] (1) at (-1,-1) {u};
  \node[main_node] (2) at (1,-1) {v};
  \draw (1) to[out=45,in=135] (2);
  \draw (1) to[out=90,in=215] (2);
  \draw (1) to[out=-45,in=90] (2);
\end{tikzpicture}

$G$ has no cut vertices and $|G|\ge 3$ since $G$ is connected, there is a $u-v$ path $P$. Let $\mathcal{P}=\{p_1,\dots,p_r\}$ be all the $u-v$ paths different from $P$. why is $\mathcal{P}\ne \emptyset$? nonseperability

\begin{tikzpicture}[main_node/.style={circle,draw,text=black,inner sep=1pt,outer sep=0pt]}]

  \node[main_node] (1) at (-1,-1) {u};
  \node[main_node] (2) at (1,-1) {w};
  \node[main_node] (3) at (3,-1) {v};
  \draw (1) to[out=45,in=135] (2);
  \draw (1) to[out=90,in=90] (2);
  \draw (1) to[out=0,in=180] (2);
  \draw (1) to[out=-90,in=-90] (2);
  \draw (1) to[out=-45,in=-135] (2);
  \draw (2) to[out=45,in=135] (3);
  \draw (2) to[out=90,in=90] (3);
  \draw (2) to[out=0,in=180] (3);
  \draw (2) to[out=-90,in=-90] (3);
  \draw (2) to[out=-45,in=-135] (3);
\end{tikzpicture}

if $P_i\cap P=\{u,v\}$ for some $i$, then we are done. if $P_i\cap P\ni w\forall i,$ then $w$ is a cut vertex. so
$\not\exists w\in P_i\cap P\forall i$.
let $v_i\in P_i\cap P\not\ni v_j\in P_j\cap P\not\ni v_i$ 

$\qquad
\quad
P_1
\qquad
\qquad
\qquad
\qquad
\qquad
P_2$

\begin{tikzpicture}[main_node/.style={circle,draw,text=black,inner sep=1pt,outer sep=0pt]}]
  \node[main_node] (1) at (-1,-1) {u};
  \node[main_node] (2) at (1,-1) {$v_i$};
  \node[main_node] (3) at (3,-1) {$v_j$};
  \node[main_node] (4) at (5,-1) {v};
  \draw (1) to[out=45,in=135] (2);
  \draw (2) to[out=-45,in=215] (4);
  \draw (1) to[out=-45,in=215] (3);
  \draw (3) to[out=45,in=135] (4);
\end{tikzpicture}

{\bfseries claim:} we can choose $v_i,v_j$ such that $\{v_i,v_j\}\le N_G(u)$
{\bfseries proof:} by nonsep $\deg(u)\ge 2$. if every $u-v$ path used edge $uw$ then w is a cut verte, therefore there are at least two $u-v$ paths starting from $u$ going through  distinct neighbors.

fromthe claim, let $v_i,v_j$ be as before. then we have
\tikz\path [graphs/.cd, nodes={shape=circle, draw, text=black,inner sep=1pt,outer sep=0pt}]
  graph [tree layout] { 1 -- {2 -- 3} -- 4 -- 1; 5 }
  [shift=(0:1)];

$vi=1, v_j=3, u=2, v=3$

repeat for $v_i, v_j$ to find two disjoint $u-v$paths making a cycle

\section*{corollary 1}
a connected graph $G$ of order 3 or more is nonsep  iff every pair of vertices has two internally disjoint paths between them

\section*{corollary 2}
let $u,v$be two distinct vertices in a nonseparable graph $G$. if $H$ is obtained by  adding a vertex to $G$ and connecting it to $u$ and $v$ $G$ then $H$ is nonsep

\section*{corollary 3}
if $G$ is nonsep and order 4 or more and $u,v\subseteq V(G)$ with $|U|=|V|=2$ and $U\cap V=\emptyset$ then $G$ contains two internally disjoint paths from vertices in $U$ to vertices in $V$.

\subsection*{internally disjoint:}
two$u-v$ paths $P_1$ and $P_2$ are internally disjoint if $P_1\cap P_2=\{u,v\}$
\end{document}


