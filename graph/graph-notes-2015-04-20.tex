\documentclass[letterpaper]{article}

\usepackage{fullpage}
\usepackage{nopageno}
\usepackage{amsmath}
\usepackage{amssymb}
\usepackage{tikz}
\usepackage[utf8]{luainputenc}
\usepackage{aeguill}
\usepackage{setspace}

\tikzstyle{edge} = [fill,opacity=.5,fill opacity=.5,line cap=round, line join=round, line width=50pt]
\usetikzlibrary{graphs,graphdrawing}
\usegdlibrary{trees}

\pgfdeclarelayer{background}
\pgfsetlayers{background,main}

\allowdisplaybreaks

\newcommand{\abs}[1]{\left\lvert #1 \right\rvert}

\begin{document}
\title{Notes}
\date{20 avril, 2015}
\maketitle
\section*{Heawood Map-coloring theorem} 
Sylvester converted maps into planar graphs

we are biased towards planar graphs because we live on a sphere

suppose we lived on a toroidal planet in addition to strange gravity we would color things differently

\subsubsection*{question?}
is there a four color theorem for toroidal graphs?

heawood was interested in the ``chromatic number of a surface''.

\subsection*{definition}
let $S_k$ be the surface of genus $k$ (ie the $k$-holed torus) $\chi(S_k)$ is the largest chromatic number of any graph embeddable on the surface $S_k$.
\subsubsection*{note}
we are not assuming 2-cell embedding, in general we don't care about the embedding (it will happen naturally that we focus on more complicated embeddings)

\subsubsection*{example}
four color theorem: 
$\chi(S_0)=4$

\subsubsection*{task}
lower bounds on $\chi(S_1)$, ie find graphs of various chromatic numbers embeddable on $S_1$

\section*{thrm 7.9?}
$\tau(k_n)=\lceil\frac{(n-3)(n-4)}{12}\rceil$

we have that $K_7$ embeds on $S_1$ but $k_8$ does not. this shows $\chi(S_1)\ge 7$ but does not show $\chi(S_1)\le 8$.

\section*{thm}
$\chi(S_1)=7$

\subsubsection*{proof}
$k_7$ embeds in $S_1$ so $\chi(S_1)\ge 7$. we will use $\chi(G)\le 1+\delta(G)$ (1 plus minimal degree) to show that $\chi(S_1)\le 7$

we need to show that for every graph embeddable on the torus has a degree of no more than six.

let $G$ be a toroidal graph and $\delta(G)$ is maximal among all toroidal graphs.

if $|G|\le 7$ then $\delta(G)\le 6$. this finishes the claim, so we have $|G|>7$.

since $\tau(G)\le 1$ we have $1\ge \tau(G)\ge \frac{m}{6}-\frac{n}{2}+1$

solving for $m$ we have $m\le 3n$. 

further back $\sum\limits{\deg_G(v_i)}=2m$ (twice the edges)

now $\sum\limits{\deg_G(v_i)}=2m\le 6n$ therefore average degree is 6 and minimum degree is less than or equal to average degree.

\section*{another theorem}
for $k>0$ $\chi(S_k)\le\left\lfloor\frac{7+\sqrt{1+48k}}{2}\right\rfloor$

what about equality?

recall thm 7.9: $\tau(k_n)=\lceil\frac{(n-3)(n-4)}{12}\rceil$

$\tau(k_n)\ge \frac{(n-3)(n-4)}{12}$

let $n=\left\lfloor\frac{7+\sqrt{1+48k}}{2}\right\rfloor$ so $n\le\frac{7+\sqrt{1+48k}}{2}$

solving we have $\frac{(2n-7)^2-1}{48}\le k$

$\frac{4n^2-28n+48}{48}=\frac{n^2-7n+12}{12}\le k$

$\frac{(n-3)(n-4)}{12}\le k$

$\tau(k_n)=\lceil\frac{(n-3)(n-4)}{12}\rceil\le k$

hence the chromatic number of the complete graph on $\chi(k_{\tau(k_n)})\le \chi(S_k)$

since $k_{\tau (k_n)}$ is trivially embeddable (one vertex for each hole) on $S_{\tau(k_n)}$

so it follow that $\chi(S_k)\ge n=$ 1+48 root mess

\section*{homework}
1,2,3
\end{document}
