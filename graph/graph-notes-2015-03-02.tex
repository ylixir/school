\documentclass[letterpaper]{article}

\usepackage{fullpage}
\usepackage{nopageno}
\usepackage{amsmath}
\usepackage{amssymb}
\usepackage{tikz}
\usepackage[utf8]{luainputenc}
\usepackage{aeguill}
\usepackage{setspace}

\tikzstyle{edge} = [fill,opacity=.5,fill opacity=.5,line cap=round, line join=round, line width=50pt]
\usetikzlibrary{graphs,graphdrawing}
\usegdlibrary{trees}

\pgfdeclarelayer{background}
\pgfsetlayers{background,main}

\allowdisplaybreaks

\newcommand{\abs}[1]{\left\lvert #1 \right\rvert}

\begin{document}
\title{Notes}
\date{2 mars, 2015}
\maketitle
\section*{edge ideals}
consider $\mathbb{R}$ and polynomials in $X$ with coefficients in $\mathbb{R}$.

\subsection*{example}
\begin{align*}
  x^2+1\\
  x-2\\
  \pi\\
  3x^3-5x+7
\end{align*}

we denote the set of all of these as $\mathbb{R}[x]$.

some polynomials ``do fun things''

\begin{enumerate}
\item
factor
\begin{align*}
  x^2-1&=(x+1)(x-1)\\
  x^2-2x+1&=(x-1)^2\\
  x^3-x^2+4x-4\\
  etc
\end{align*}
we say all plynomials in $\mathbb{R}[x]$ with  the property that $x-1$ divides it, this entire set is an ideal.

this is denoted $\langle x-1\rangle\subseteq \mathbb{R}[x]$ 

if we want all polynomials divisable by several things, then $\langle x-1,x^2+1,x^4-2\rangle$. This is ``or'' or union. the ``and'' or intersection would be formed by just multiplying the divisors.

in general $\langle f_1,\dots,f_r\rangle=\{\sum\limits_{i=1}^r{p_if_i}|p_i\in \mathbb{R}[x]\}$

we can do this in many variables like $\mathbb{R}[x,y,z]$ or $\mathbb{R}[x_1,\dots,x_n]$ or even $\mathbb{R}[x_1,\dots]$

all ideals have many invariants, eg, dimesnion, projective dimension, injective dimension, height, resolutions, betti numbers, etc.

\end{enumerate}

let $G$ be a graph and label the vertices $x_1,\dots,x_n$. then create an ideal in $\mathbb{R}[x_1,\dots,x_n]$. $I=\langle x_ix_j|x_j$

\subsubsection*{example}
\begin{tikzpicture}[main_node/.style={node distance=1cm,circle,draw,text=black,inner sep=1pt,outer sep=0pt]}]
  \node[main_node] (1)  {1};
  \node[main_node] (2) [below left of=1] {1};
\end{tikzpicture}
these ideals create a dictionary between graph theory properties and algebraic properties
\end{document}
