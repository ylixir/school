\documentclass[letterpaper]{article}

\usepackage{fullpage}
\usepackage{nopageno}
\usepackage{amsmath}
\usepackage{amssymb}
\usepackage{tikz}
\usepackage[utf8]{inputenc}
\usepackage{aeguill}
\usepackage{setspace}

\usetikzlibrary{graphs,graphdrawing}
\usegdlibrary{trees}

\allowdisplaybreaks

\newcommand{\abs}[1]{\left\lvert #1 \right\rvert}

\begin{document}
\title{Notes}
\date{January 30, 2015}
\maketitle
\section*{2.3 homework}
1,2,15,17,19

\section*{recap/finish}

\begin{description}
\item[spanning subgraph]
$H$ of $G$ such that $V(H)=G(H)$
\item[weighted graph]
one in which the edges are given a value
\item[greedy algorithm]
gives a minimal spanning tree
\end{description}

\subsection*{example}
\begin{tikzpicture}[main_node/.style={circle,draw,text=black,inner sep=1pt,outer sep=0pt]}]
  \node[main_node] (1) at (0,0) {};
  \node[main_node] (2) at (1,-1) {};
  \node[main_node] (3) at (1,1) {};
  \node[main_node] (4) at (2,-1) {};
  \node[main_node] (5) at (2,1) {};
  \node[main_node] (6) at (2,0) {};
  \node[main_node] (7) at (3,0) {};
  \draw (1) -- (2) -- (3) -- (1);
  \draw (3) --(5)--(2)--(4)--(6)--(5)--(7)--(6);
  \node[draw=white] at (1/2,1) {4};
  \node[draw=white] at (1/2,-1) {3};
  \node[draw=white] at (3/4,0) {$\pi$};
  \node[draw=white] at (3/2,3/2) {5};
  \node[draw=white] at (3/2,1/4) {1};
  \node[draw=white] at (3/2,-3/2) {-4};
  \node[draw=white] at (15/8,1/2) {7};
  \node[draw=white] at (15/8,-1/2) {5};
\end{tikzpicture}

\subsubsection*{algorithm}
choose an edge of least weight, choose another edge of least weight not creating a cycle, repeat, stop when $H$ is a spanning tree

\begin{tikzpicture}[main_node/.style={circle,draw,text=black,inner sep=1pt,outer sep=0pt]}]
  \node[main_node] (1) at (0,0) {};
  \node[main_node] (2) at (1,-1) {};
  \node[main_node] (3) at (1,1) {};
  \node[main_node] (4) at (2,-1) {};
  \node[main_node] (5) at (2,1) {};
  \node[main_node] (6) at (2,0) {};
  \node[main_node] (7) at (3,0) {};
  \draw (1)--(2)--(3);
  \node[draw=white] at (1/2,1) {4};
  \node[draw=white] at (1/2,-1) {3};
  \node[draw=white] at (3/4,0) {$\pi$};
  \node[draw=white] at (3/2,3/2) {5};
  \node[draw=white] at (3/2,1/4) {1};
  \node[draw=white] at (3/2,-3/2) {-4};
  \node[draw=white] at (15/8,1/2) {7};
  \node[draw=white] at (15/8,-1/2) {5};
\end{tikzpicture}

a vertex cut is a subset $S\subseteq V(G)$ such that $k(G-S)>k(G)$

if $S$ is of minimal cadinality, we say $G$ is $|S|$-connected. if we have a cut vertex the graph is $1$-connected. if $G$ is disconnected, then it is $0$-connected. nonseparable means more than $2$-connected 

\begin{tikzpicture}[main_node/.style={circle,draw,text=black,inner sep=1pt,outer sep=0pt]}]
  \node[main_node] (1) at (-1,-1) {1};
  \node[main_node] (2) at (-1,1) {1};
  \node[main_node] (3) at (0,0) {1};
  \node[main_node] (4) at (1,0) {1};
  \node[main_node] (5) at (2,-1) {1};
  \node[main_node] (6) at (2,1) {1};
  \draw (1) -- (2) -- (3) -- (1);
  \draw (4) -- (5) -- (6) -- (4);
  \draw (1)--(5);
  \draw (2)--(6);
  \draw (3)--(4);
\end{tikzpicture}

the notation for connectedness is $\kappa(G)$

question: what is $\kappa(k_n)$? (connectedness of a complete graph order $n$)
removing any vertex gives $k_{n-1}$ and so on. so we just define it to be $\kappa(k_n)$ is $n-1$. analog is $0!=1$.

edge connectedness: an edge cut is a set $S\subseteq E(G)$ such that $k(G-S)>k(G)$.

if $S$ is of minimal cardinality, then we say $G$ is $|S|$-edge-connected, denots $\lambda(G)=|S|$

example
\begin{tikzpicture}[main_node/.style={circle,draw,text=black,inner sep=1pt,outer sep=0pt]}]
  \node[main_node] (1) at (-1,-1) {1};
  \node[main_node] (2) at (1,-1) {2};
  \node[main_node] (3) at (-1,0) {3};
  \node[main_node] (4) at (1,0) {4};
  \node[main_node] (5) at (0,1) {5};
  \node[main_node] (6) at (-1,2) {6};
  \node[main_node] (7) at (1,2) {7};
  \draw (1) -- (2) -- (4) -- (5)--(7)--(6)--(5)--(3)--(1);
\end{tikzpicture}
$\lambda(G)=2$
\section*{theorem}
for every simple graph $\kappa(G)\le\lambda(G)\le\delta(G)$

\subsubsection*{proof}
case 1: $G$ is disconnectd, then $\kappa(G)=\lambda(G)=0$ and $\delta(G)=0$

case 2: $G$ is the complete graph. what is the $\lambda(k_n)$? it is $n-1$. we can remove $n-1$ to disconnect a single vertex. removing less leaves something connected because $\delta(G)=n-1$. and so $\kappa(k_n)=n-1=\lambda(k_n)=\delta(k_n)$

case 3: every thing else. in this case $\delta(G)\le n-2$ because it's not complete, and so there is a vertex of degree less than $n-1$. pick a vertex $v$ such that $\deg(v)=\delta(G)$. remove all incident edges to disconnect the graph, and so $\lambda(G)\le\deg(G)=\delta(G)\le n-2$. showing the last inequality is harder.

\section*{corollary}
if $G$ is a graph such that the order of $G$ is $n$ and the size is $m$ $m\ge n-1$ then $\kappa(G)\le \left\lfloor\frac{2m}{n}\right\rfloor$
\subsubsection*{proof}
$\sum\limits{d(v)}=2m$ so the average degree is $\frac{2m}{n}$. By theorem $\kappa(G)\le\delta(G)$ and since $\frac{2m}{n}$ is average $\delta(G)\le\lfloor 2m/n\rfloor$

\section*{2.4 homework}
1,2,9,13,14
\end{document}

