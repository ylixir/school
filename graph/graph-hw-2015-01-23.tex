%shell-escape
\documentclass[letterpaper]{article}

\usepackage[utf8]{inputenc}
\usepackage{fullpage}
\usepackage{nopageno}
\usepackage{amsmath}
\usepackage{amssymb}
\usepackage{gnuplottex}
\usepackage{tikz}

\usetikzlibrary{graphs,graphdrawing}
\usegdlibrary{trees}

\allowdisplaybreaks

\newcommand{\abs}[1]{\left\lvert #1 \right\rvert}

\begin{document}
\title{Graph Theory Homework}
\date{January 23, 2015}
\author{Jon Allen}
\maketitle
\renewcommand{\labelenumi}{1.\arabic{enumi}}
\renewcommand{\labelenumii}{\arabic{enumii}.}
%\renewcommand{\labelenumii}{\Alph{enumii}.}
\renewcommand{\labelenumiii}{(\alph{enumiii})}
\section*{Definitions}
\begin{description}
\item[path]
A graph of order $n$ and size $n-1$ whose vertices can be labeled by $v_1,v_2,\dots,v_n$ and whose edges are $v_1v_{i+1}$ for $i=1,2,\dots,n-1$.
\item[cycle]
A graph of order $n$ and size $n$ whose vertices can be labeled by $v_1,v_2,\dots,v_n$ and whose edges are $v_1v_n$ and $v_1v_{i+1}$ for $i=1,2,\dots,n-1$.
\item[isomorphism]
If $G$ and $H$ are graphs and $\phi:V(G)\to V(H)$ is a bijective function such that two vertices $u$ and $v$ are adjacent in $G$ if and only if $\phi(u)$ and $\phi(v)$ are adjacent in $H$. The function $\phi$ is an isomorphism.
\item[subgraph]
Let $G$ and $H$ be graphs. Then if $V(H)\subseteq V(G)$ and $E(H)\subseteq E(G)$ then $H$ is a subgraph of $G$. That is to say, $H$ is a subgraph of $G$ if $G$ contains all the vertices and edges of $H$.
\item[regular graph]
A graph whose vertices all have the same degree.
\item[bipartate graph]
A graph whose vertices can be partitioned into two sets in such a way that every edge of the graph joins vertices from both sets.
\item[complement]
A complement of a graph $G$ is the graph $\overline{G}$ which has the same vertex set as $G$ and where any two vertices are adjacent if and only if these vertices are not adjacent in $G$.
\end{description}
\section*{Exercises}
\begin{enumerate}
\item
  \begin{enumerate}
  \setcounter{enumii}{1}
  %1.1 #2
  \item
  A graph $G=(V,E)$ of order 8 has the power set of the set $S=\{1,2,3\}$ as its vertex set, that is $V$ is the set of subsets of $S$. Two vertices $A$ and $B$ of $V$ are adjacent if $A\cap B=\emptyset$. Draw the graph $G$, determine the degree of each vertex of $G$ and determine the size of $G$.

  \begin{tikzpicture}[main_node/.style={circle,draw,text=black,inner sep=1pt,outer sep=0pt]}]
    \node[main_node] (1) at (1,3) {$\emptyset$};
    \node[main_node] (2) at (0,2) {$\{1\}$};
    \node[main_node] (3) at (2,2) {$\{2\}$};
    \node[main_node] (4) at (4,2) {$\{3\}$};
    \node[main_node] (5) at (2,4) {$\{1,2\}$};
    \node[main_node] (6) at (4,4) {$\{1,3\}$};
    \node[main_node] (7) at (0,4) {$\{2,3\}$};
    \node[main_node] (8) at (5,3) {$\{1,2,3\}$};
    \draw (1) -- (2) -- (3) -- (4) -- (1);
    \draw (8) -- (1) -- (3) -- (6) -- (1);
    \draw (2) -- (7) -- (1) -- (5) -- (4) to[out=-135,in=-45] (2);
  \end{tikzpicture}
  \begin{align*}
    \deg \emptyset &= 7 & \deg \{1\}&=\deg \{2\}=\deg\{3\}=3\\
    \deg \{1,2,3\} &= 1 & \deg \{1,2\}&=\deg \{2,3\}=\deg\{1,3\}=2
  \end{align*}
  The size $|E(G)|$ of $G$ is $7+1+9+6=23$
  %1.1 #3
  \item
  A graph $G$ of order 26 and size 58 has 5 vertices of degree 4, 6 vertices of degree 5 and 7 vertices of degree 6. The remaining vertices of $G$ all have the same degree. What is this degree?

  \begin{align*}
    26-5-6-7&=8\\
    116-5\cdot4-6\cdot5-7\cdot6&=24\\
    24\div 8&=3
  \end{align*}
  The remaining 8 vertices have degree 3.
  %1.1 #4
  \item
  A graph of $G$ has order $n=3k+3$ for some positive integer $k$. Every vertex of $G$ has degree $k+1, k+2$ or $k+3$. Prove that $G$ has at least $k+3$ vertices of degree $k+1$ or at least $k+1$ vertices of degree $k+2$ or at least $k+2$ vertices of degree $k+3$.
  \setcounter{enumii}{10}
  %1.1 #11
  \item
  Prove for every graph $G$ and every integer $r\ge\Delta(G)$ that there exists an $r$-regular graph containing $G$ as an induced subgraph.
  \setcounter{enumii}{12}
  %1.1 #13
  \item
  Determine all bipartite graphs $G$ such that $\overline{G}$ is bipartite
  \setcounter{enumii}{17}
  %1.1 #18
  \item
  Let $G$ be a self-complementary graph of order $n$, where $n\equiv 1\mod 4$. Prove that $G$ contains an odd number of vertices of degree $(n-1)/2$.
  \end{enumerate}
\item
  \begin{enumerate}
  \setcounter{enumii}{5}
  \item
  Let $G$ and $H$ be two graphs that are neither empty nor complete. The graph $H$ is said to be obtained from $G$ by an {\bfseries edge rotation} if $G$ contains three vertices $u,v,$ and $w$ where $uv\in E(G)$ and $uw\not\in E(G)$ and $H\cong G-uv+uw$.

    $G_1:$\begin{tikzpicture}[main_node/.style={circle,draw,text=black,inner sep=1pt,outer sep=0pt]}]
      \node[main_node] (1) at (1,0) {};
      \node[main_node] (2) at (0,1) {};
      \node[main_node] (3) at (0,2) {};
      \node[main_node] (4) at (1,3) {};
      \node[main_node] (5) at (2,3) {};
      \node[main_node] (6) at (3,2) {};
      \node[main_node] (7) at (3,1) {};
      \node[main_node] (8) at (2,0) {};
      \draw (1) -- (2) -- (3) -- (4) -- (5) -- (6) -- (7) -- (8);
      \draw (1) -- (7);
      \draw (2) -- (7);
      \draw (2) -- (6);
      \draw (2) -- (5);
    \end{tikzpicture}
    $G_2:$
    \begin{tikzpicture}[main_node/.style={circle,draw,text=black,inner sep=1pt,outer sep=0pt]}]
      \node[main_node] (1) at (0,0) {};
      \node[main_node] (2) at (0,1) {};
      \node[main_node] (3) at (0,2) {};
      \node[main_node] (4) at (0,3) {};
      \node[main_node] (5) at (1,3) {};
      \node[main_node] (6) at (1,2) {};
      \node[main_node] (7) at (1,1) {};
      \node[main_node] (8) at (1,0) {};
      \draw (1) -- (2) -- (3) -- (4) -- (5) -- (6) -- (7) -- (8) -- (1);
      \draw (1) -- (6) -- (2);
      \draw (3) -- (6);
    \end{tikzpicture}
    $G_3:$
    \begin{tikzpicture}[main_node/.style={circle,draw,text=black,inner sep=1pt,outer sep=0pt]}]
      \node[main_node] (1) at (1,0) {};
      \node[main_node] (2) at (0,1) {};
      \node[main_node] (3) at (1,2) {};
      \node[main_node] (4) at (0,3) {};
      \node[main_node] (5) at (3,3) {};
      \node[main_node] (6) at (2,2) {};
      \node[main_node] (7) at (3,1) {};
      \node[main_node] (8) at (2,0) {};
      \draw (1) -- (2) -- (3) -- (4) -- (5) -- (6) -- (7) -- (8) -- (1);
      \draw (2) -- (8);
      \draw (5) -- (3) -- (6);
    \end{tikzpicture}
    \begin{enumerate}
    %1.2 #6a
    \item
    Show that the graph $G_2$of figure 1.33 is obtained from $G_1$ by an edge rotation.
    %1.2 #6b
    \item
    Show that $G_3$ of figure 1.33 cannot be obtained from $G_1$ by an edge rotation.
    \end{enumerate}
  %1.2 #7
  \item
  Determine whether the following sequences are graphical. If so, construct a graph with the appropriate degree sequence.
  \begin{enumerate}
  \item
  4,4,3,2,1
  \item
  3,3,2,2,2,2,1,1
  \item
  7,7,6,5,4,4,3,2
  \item
  7,6,6,5,4,3,2,1
  \item
  7,4,3,3,2,2,2,1,1,1
  \end{enumerate}
  \setcounter{enumii}{9}
  %1.2 #10
  \item
  For which integers $x (0\le x\le 7)$, if any, is the sequence $7,6,4,3,2,1,x$ graphical?
  \setcounter{enumii}{14}
  %1.2 #15
  \item
  Two finite sequences $s_1$ and $s_2$ of nonnegative integers are called {\bfseries bigraphical} if there exists a bipartite graph $G$ with partite sets $V_1$ and $V_2$ such that $s_i$ lists the degrees of the vertices of $G$ in $V_i$ for $i=1,2$. Prove that the sequences $s_1:a_1,a_2,\dots,a_r$ and $s_2:b_1,b_2,\dots,b_t$ of nonnegative integers with $r\ge 2, a_1\ge a_2\ge\dots\ge a_r, b_1\ge b_2\ge\dots\ge b_t, 0<a_1\le t$ and $0<b_1\le r$ are bigraphical if and only if the sequences $s_1':a_2,a_3,\dots,a_r$ and $s_2':b_1-1,b_2-1,\dots,b_{a_1}-1,b_{a_1+1},\dots,b_t$ are bigraphical
  \end{enumerate}
\end{enumerate}
\end{document}
