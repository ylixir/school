%shell-escape
\documentclass[letterpaper]{article}

\usepackage[utf8]{inputenc}
\usepackage{fullpage}
\usepackage{nopageno}
\usepackage{amsmath}
\usepackage{amssymb}
\usepackage{gnuplottex}
\usepackage{tikz}

\usetikzlibrary{graphs,graphdrawing}
\usegdlibrary{trees}

\allowdisplaybreaks

\newcommand{\abs}[1]{\left\lvert #1 \right\rvert}

\begin{document}
\title{Graph Theory Homework}
\date{February 6, 2015}
\author{Jon Allen}
\maketitle
The homework for 2/6 is from sections 2.2, 2.3 and 2.4, and the problems are 1,2,3,10,14,17; 1,2,15,17,19; and 1,2,9,13,14, respectively.
\renewcommand{\labelenumi}{2.\arabic{enumi}}
\renewcommand{\labelenumii}{\arabic{enumii}.}
%\renewcommand{\labelenumii}{\Alph{enumii}.}
\renewcommand{\labelenumiii}{(\alph{enumiii})}
\begin{enumerate}
\setcounter{enumi}{1}
%2.2
\item
  \begin{enumerate}
  %2.2 #1
  \item
  %2.2 #2
  \item
    For number 2, assume G is connected.
  %2.2 #3
  \item
  \setcounter{enumii}{9}
  %2.2 #10
  \item
    The forward direction of 10 is probably the hardest part from this section.
  \setcounter{enumii}{13}
  %2.2 #14
  \item
    For number 14, use an edge counting argument to limit your possibilities.
  \setcounter{enumii}{16}
  %2.2 #17
  \item
    For number 17, remove one of the vertices, and ask yourself what happens to the order and size.  Then repeat it for the other vertex.
  \end{enumerate}
%2.3
\item

The reverse direction of number 15 is probably the hardest part of this section.

For number 17, the number of edges of a complete graph is n(n-1)/2.

%2.4
\item

For number 1, a k-partite graph is like a bipartite graph, but with k different partitions.  It is complete when all vertices of each partition are adjacent to all other vertices outside its own partition.

Number 9 is tricky, but the hint in the book is a good one.
\end{enumerate}
\end{document}
