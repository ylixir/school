\documentclass[letterpaper]{article}

\usepackage{fullpage}
\usepackage{nopageno}
\usepackage{amsmath}
\usepackage{amssymb}
\usepackage{tikz}
\usepackage[utf8]{inputenc}
\usepackage{aeguill}
\usepackage{setspace}

\usetikzlibrary{graphs,graphdrawing}
\usegdlibrary{trees}

\allowdisplaybreaks

\newcommand{\abs}[1]{\left\lvert #1 \right\rvert}

\begin{document}
\title{Notes}
\date{2 fevrier, 2015}
\maketitle
\section*{from the grader}
\begin{enumerate}
\item
no frillies
\item
no paperclips
\item
no folding corners
\item
staple
\end{enumerate}

\subsection*{grading scheme}
10 available points, 5 for completeness, 5 for 1 or 2 graded problems. 80\% is an A

\section*{2.5 Menger's Theorem}
a $u-v$ {\bfseries separating set} is a set $S\subseteq V(G)$ such that $S$ separates $G$ with  $u$ and $v$ in different components.

\subsubsection*{exmple}
\begin{tikzpicture}[main_node/.style={circle,draw,text=black,inner sep=1pt,outer sep=0pt]}]
  \node[main_node] (1) at (0,-1) {c};
  \node[main_node] (2) at (-1,0) {u};
  \node[main_node] (3) at (1,0) {d};
  \node[main_node] (4) at (-1,1) {a};
  \node[main_node] (5) at (1,1) {v};
  \node[main_node] (6) at (0,2) {b};
  \draw (1) -- (2) -- (4) -- (6) --(5)--(1)--(3)--(4);
\end{tikzpicture}
example $S$ are $\{a,c\}, \{b,c\}, \{a,b,c\}$.

\begin{tikzpicture}[main_node/.style={circle,draw,text=black,inner sep=1pt,outer sep=0pt]}]
  \node[main_node] (1) at (-1,-1) {};
  \node[main_node] (2) at (-1,1) {u};
  \node[main_node] (3) at (0,-1) {b};
  \node[main_node] (4) at (0,1) {a};
  \node[main_node] (5) at (1,0) {c};
  \node[main_node] (6) at (2,-1) {};
  \node[main_node] (7) at (2,1) {v};
  \draw (1) -- (2) -- (4) -- (3) --(1);
  \draw (4) -- (5) -- (6) -- (7) --(5)--(3);
\end{tikzpicture}
$\{a,b\}$ minimally separates $u$ and $v$ in the sense that no subset of $\{a,b\}$ separates $u$ and $v$ but $\{c\}$ is a minimum $u-v$ separating set

\subsubsection*{theorem}
let $u$ and $v$ be non-adjacent vertices in $G$. the size of a minimum $u-v$ separating set is equal to the number of internally disjoint $u-v$ paths.

\subsubsection*{thm/corollary (whitney)}
a non-trivial graph is $k$-connected $k\ge 2$ if and only if every pair of vertices has at least $k$ internally disjoint paths between them.

{\scshape note:} adjacency isn't in the second theorem.

{\scshape proof:} forward direction:

let $k\ge 2$ and let $S$ be a minimal vertex cut. Take any two different points $u,v$. any $u-v$ separating has at least size $k$ (it is $S$). By menger, there are at least $k$ internally disjoint paths from $u$ to $v$. but menger's theorem doesn't apply if $u,v$ are adjacent. if $u$ and $v$ are adjacent, remove $uv$ (the edge). this reduces connectivity by up to $1$. {\bfseries (check this for homework)}. now repeat the argument on $G-\{uv\}$

this results in at least $k-1$ internally disjoint paths by menger. of course add in the edge we removed and we have $k$ internally disjoint paths.

reverse direction:
any two vertices $u,v$ have $k$ internally  disjoint paths between them. let $S$ be a minimal vertex cut. them $G-S=G_1\cup G_2$ where $G_1\cap G_2=\emptyset$. union with dot is disjoint union.

pick $u\in G_1$ and $v\in G_2$. There are at least $k$ internally disjoint $u-v$ paths. so $S$ must contain at least one element of eachpath,hence $G$ is at least $|S|$-connected. $S$ is minimal, so $|S|=k$

issue: check complete graph. if $G$ were complete then our proof fails. $|V(G)|\ge k+1$ {\bfseries why? second part of homework}. so $\kappa(G)\ge k+1-1=k$ because $k(k_n)=n-1$ 

$\Box$

aside:
mengers theorem is often referred to as the max-flow min-cut theorem outside of graph theory.

\section*{homework}
prove check this and why, also 2.5 \#1-5
\end{document}

