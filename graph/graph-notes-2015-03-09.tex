\documentclass[letterpaper]{article}

\usepackage{fullpage}
\usepackage{nopageno}
\usepackage{amsmath}
\usepackage{amssymb}
\usepackage{tikz}
\usepackage[utf8]{luainputenc}
\usepackage{aeguill}
\usepackage{setspace}

\tikzstyle{edge} = [fill,opacity=.5,fill opacity=.5,line cap=round, line join=round, line width=50pt]
\usetikzlibrary{graphs,graphdrawing}
\usegdlibrary{trees}

\pgfdeclarelayer{background}
\pgfsetlayers{background,main}

\allowdisplaybreaks

\newcommand{\abs}[1]{\left\lvert #1 \right\rvert}

\begin{document}
\title{Notes}
\date{9 mars, 2015}
\maketitle
\section*{7.1 genus of a graph}
last chapter we discussed ``how far'' from planar a graph was. we used the crossing \#

in the same vein, but more useful is the genus of a graph. we know that $K_{3,3}$ cannot be embedded on the plane. what about on a donut (torus)?

recall that $\text{cr}(K_5)=1$.

Now we can ``mold'' the torus into figure 7.5 on page 271.

so handles can get around crossings.

a sphere with $k$ handles is called a surface with genus $k$. the book calls it $S_k$. think a $k$ holed torus. 

an easier way:

think of the torus as a plane rolled up into a tube, with edges connected. now we associated  opposite edges. use arrows or something to show this

with this interpretation

\subsection*{thrm}
if $G$ is connected with $|G|=n, |E(G)|=m$ and $G$ is embedded minimally with $r$ regions, then we have the $n-m+2=2-2\gamma(G)$ where $\gamma(G)=$minimal genus

like before we get a bound right away:

if $G$ is a connected graph with $|G|\ge 3$ then $\gamma(G)\ge \frac{m}{6}-\frac{n}{2}+1$

\section*{Homework}
2,8,11
\end{document}
