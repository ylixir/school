%shell-escape
\documentclass[letterpaper]{article}

\usepackage{fullpage}
\usepackage{nopageno}
\usepackage{amsmath}
\usepackage{amssymb}
\usepackage{gnuplottex}
\usepackage{tikz}
\usepackage[utf8]{luainputenc}

\tikzstyle{edge} = [fill,opacity=.5,fill opacity=.5,line cap=round, line join=round, line width=50pt]
\usetikzlibrary{graphs,graphdrawing}
\usegdlibrary{trees}

\pgfdeclarelayer{background}
\pgfsetlayers{background,main}

\allowdisplaybreaks

\newcommand{\abs}[1]{\left\lvert #1 \right\rvert}

\begin{document}
\title{Graph Theory Homework}
\date{February 20, 2015}
\author{Jon Allen}
\maketitle
\renewcommand{\labelenumi}{2.\arabic{enumi}}
\renewcommand{\labelenumii}{\arabic{enumii}.}
%\renewcommand{\labelenumii}{\Alph{enumii}.}
\renewcommand{\labelenumiii}{(\alph{enumiii})}
\begin{description}
\item[Worksheet]$\quad$
\renewcommand{\labelenumi}{\arabic{enumi}.)}
\renewcommand{\labelenumii}{(\alph{enumii})}
\begin{enumerate}
\item
Create a 3-regular, 3-uniform hypergraph. Is it possible to create an $k$-regular, $k$-uniform (simple) hypergraph? Prove or disprove.
\item
Let $H$ be the pictured hypergraph.

\begin{tikzpicture}[main_node/.style={circle,draw,text=black,inner sep=1pt,outer sep=0pt]}]
  %nodes here
  \node[main_node] (1) at (-.5,6) {A};
  \node[main_node] (2) at (6,6.5) {B};
  \node[main_node] (3) at (8,3) {C};
  \node[main_node] (4) at (7,4.5) {D};
  \node[main_node] (5) at (0,0) {E};
  \node[main_node] (6) at (-2.5,4) {F};
  \node[main_node] (7) at (-2,2) {T};
  \node[main_node] (8) at (-.5,2.25) {S};
  \node[main_node] (9) at (1.5,3) {U};
  \node[main_node] (11) at (.5,4.75) {$e_1$};
  \node[main_node] (12) at (3.25,6.25) {$e_2$};
  \node[main_node] (13) at (9,4) {$e_3$};
  \node[main_node] (14) at (7.5,5) {$e_4$};
  \node[main_node] (15) at (3,1) {$e_5$};
  \node[main_node] (16) at (-1.5,4) {$e_6$};
  \node[main_node] (17) at (-1.25,2) {$e_7$};
  \begin{pgfonlayer}{background}
    \begin{scope}[transparency group,opacity=.5]
      %draws here
      \draw[edge,color=red] (1)--(2);
      \draw[edge,color=blue] (2)--(4)--(3)--(13)--(2);
      \draw[edge,color=red] (4)--(14);
      \draw[edge,color=red] (3)--(5);
      \draw[edge,color=red] (7)--(8)--(9);
      \draw[edge,color=yellow] (5)--(7)--(5)--(8)--(5)--(9)--(6)--(7)--(6)--(8);
      \draw[edge,color=blue] (1)--(9);
    \end{scope}
  \end{pgfonlayer}
\end{tikzpicture}
  \begin{enumerate}
  \item
  Create the associated bipartite graph to $H$.

  I've labelled the edges, so $e_1=\{A,U\}$, $e_2=\{A,B\}$, $e_3=\{B,D,C\}$ and so on.

  \begin{tikzpicture}[main_node/.style={circle,draw,text=black,inner sep=1pt,outer sep=0pt]}]
    \node[main_node] (1) at (0,0) {A};
    \node[main_node] (2) at (1,0) {B};
    \node[main_node] (3) at (2,0) {C};
    \node[main_node] (4) at (3,0) {D};
    \node[main_node] (5) at (4,0) {E};
    \node[main_node] (6) at (5,0) {F};
    \node[main_node] (7) at (6,0) {T};
    \node[main_node] (8) at (7,0) {S};
    \node[main_node] (9) at (8,0) {U};
    \draw (1);
  \end{tikzpicture}
  \item
  What is the adjacency matrix of $H$?
  \item
  What is $H*$?
  \end{enumerate}
\end{enumerate}
\item[3.3]
\item[4.1]
\end{description}
\end{document}
