\documentclass[letterpaper]{article}

\usepackage{fullpage}
\usepackage{nopageno}
\usepackage{amsmath}
\usepackage{amssymb}
\usepackage{tikz}
\usepackage[utf8]{luainputenc}
\usepackage{aeguill}
\usepackage{setspace}

\tikzstyle{edge} = [fill,opacity=.5,fill opacity=.5,line cap=round, line join=round, line width=50pt]
\usetikzlibrary{graphs,graphdrawing}
\usegdlibrary{trees}

\pgfdeclarelayer{background}
\pgfsetlayers{background,main}

\allowdisplaybreaks

\newcommand{\abs}[1]{\left\lvert #1 \right\rvert}

\begin{document}
\title{Notes}
\date{11 février, 2015}
\maketitle
\section*{3.3 line graphs}
a {\bfseries line graph} is the result of an operation on another graph

if $G$ is a simple grpah, the line graph $L(G)$ is such that $V(L(G))=E(G)$. Edges in $L(G)$ occur between vertices that represent edges in $G$

edges that share a vertice in $G$ are vertices that share an edge in $L(G)$
\begin{tikzpicture}[main_node/.style={circle,draw,text=black,inner sep=1pt,outer sep=0pt]}]
  \node[main_node] (5) at (0,0) {5};
  \node[main_node] (4) at (1,0) {4};
  \node[main_node] (3) at (2,0) {3};
  \node[main_node] (1) at (1/2,1) {1};
  \node[main_node] (2) at (3/2,1) {2};
  \draw (5)--(4)--(3)--(2)--(1)--(5);
  \draw (1)--(4)--(2);
\end{tikzpicture}

if $S_n$ is the star graph, what is the line graph of $S_n$? $K_n$.
small case:
\begin{tikzpicture}[main_node/.style={circle,draw,text=black,inner sep=1pt,outer sep=0pt]}]
  \node[main_node] (1) at (0,0) {1};
  \node[main_node] (2) at (0,1) {2};
  \node[main_node] (3) at (1,0) {3};
  \node[main_node] (4) at (0,-1) {4};
  \node[main_node] (5) at (-1,0) {5};
  \draw (2)--(1)--(3);
  \draw (4)--(1)--(5);
\end{tikzpicture}

if $|G|=n$ and $|E(G)|=m$ then $|L(G)|=m$ and $|E(L(G))|=?$

each edge in $L(G)$ corresponds to a choice of two edges from $G$, in particular, these two edges must be adjacent (share a vertex). let that vertex be $v_i$ and have degree $d_i$. then we have $\binom{d_i}{2}$ edges corresponding to $v_i$. $|E(L(G))|=\sum\limits_{i=1}^n{\binom{d_i}{2}}$

every two edges coming of the original vertice make an edge in the new vertice. so we can choose two edges $\binom{d_i}{2}$ different ways

\section*{converting line graphs}
can we obtain $G$ from $L(G)$? given a graph $L$ can we say $L=L(G)$ for some $G$?

assume $L$ is connected and non-trivial

\subsection*{theorem}
if $G_1$ and $G_2$ are connected simple graphs with $L(G_1)\cong L(G_2)$ then $G_1\cong G_2$ so long as $G_1\not\cong K_3$ or $G_1\not\cong K_1,3=S_3$

there is a huge theorem called {\bfseries kuratowskis} theorem that we will cover later. it's so popular that it has spawned a whole class of theorems called kuratowski type theorems. this one is one of them
\subsection*{theorem}
a graph is a line graph of some other graph iff it is not an induced subgraph of $9$ specific graphs. see page $142$

a kuratowski theorem is a generalized structure theorem.  something is true as long as it's not true about a finite number of graphs.

theory of excluded minors (boring).

a generalization of hamiltonicity is a watchmans tour. don't have to visit every ``hallway'', you just have to be able to see down every hallway. a watchman's tour is a tree so that every edge in $G$ is in the tree or adjacent to the tree.

definition in book:
{\bfseries dominating circuit} is a circuit $C$ of $G$ such that every edge of $G$ is in $C$ or adjacent to $C$.

\subsubsection*{example}
peterson graph. not hamiltonian. does have dominating circuit, not hamiltonian

\subsubsection*{theorem}
let $G$ be a connected graph, then $L(G)$ is hamiltonian iff $G$ has a diminating circuit.

\section*{homework}
10,11,12
\end{document}
 
