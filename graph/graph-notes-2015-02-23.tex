\documentclass[letterpaper]{article}

\usepackage{fullpage}
\usepackage{nopageno}
\usepackage{amsmath}
\usepackage{amssymb}
\usepackage{tikz}
\usepackage[utf8]{luainputenc}
\usepackage{aeguill}
\usepackage{setspace}

\tikzstyle{edge} = [fill,opacity=.5,fill opacity=.5,line cap=round, line join=round, line width=50pt]
\usetikzlibrary{graphs,graphdrawing}
\usegdlibrary{trees}

\pgfdeclarelayer{background}
\pgfsetlayers{background,main}

\allowdisplaybreaks

\newcommand{\abs}[1]{\left\lvert #1 \right\rvert}

\begin{document}
\title{Notes}
\date{23 février, 2015}
\maketitle
no class 3/4, 3/6
\section*{6.1 planarity}
if a graph can be drawn on a plane without any crossings, it is called {\bfseries planar}

note that if it is planar, you can draw it without crossings with all straight lines.
\subsection*{examples}
$K_3$, $K_4$

\subsubsection*{drawing:}
geogebra.org

\subsection*{more}
we often think of planar graphs as collections of polygons (remember we can draw all planar graphs in straight lines)

picture on board is pentagon glued to one side of a square, and a triangle glued to another
when we have polygons glued together, then we can count things. things like
\begin{enumerate}
\item
faces, eg the polygons

for faces we have several bounded, and one unbounded

4
\item
edges

10
\item
vertices

8
\end{enumerate}

note that $V-E+F=2$

this generalizes off the plane, but the idea of ``faces'' kind of breaks down and we have to looks at cycles and such.

\section*{theorem}

if a graph is planar then $v-e+f=2$

\subsubsection*{proof}
by induction on edges. if you add an edge, then you are adding a vertex or a face

if $E=0$ and $G$ is connected then $G\cong K_1$. $1-0+1=2$ and so check. assume this is true for $e=k$. suppose $G$ is a tree with $e=k+1$. Now remove an edge that is part of a cycle and we have $e=k$ and number of faces is reduced by 1. Now $v-(e-1)+(f-1)=2$ by inductive hypothesis. adding the edge back in and we have $v-e+f=2$. Removing an edge not in a cycle reduced the number of vertices and $v-e+f=2$ similar to above.

\section*{theorem}
if $G$ is planar and $|G|\ge  4$ then $E\le 3V-6$. Proof crux: every face has at least 3 edges on it's boundary

contrapositive:
if $E>3V-6$ and $|G|\ge 4$ then $G$ is not planar.

\section*{homework}
6.1 numbers 1,2,5
\end{document}
 
