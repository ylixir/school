\documentclass[letterpaper]{article}

\usepackage{fullpage}
\usepackage{nopageno}
\usepackage{amsmath}
\usepackage{amssymb}
\usepackage{tikz}
\usepackage[utf8]{luainputenc}
\usepackage{aeguill}
\usepackage{setspace}

\tikzstyle{edge} = [fill,opacity=.5,fill opacity=.5,line cap=round, line join=round, line width=50pt]
\usetikzlibrary{graphs,graphdrawing}
\usegdlibrary{trees}

\pgfdeclarelayer{background}
\pgfsetlayers{background,main}

\allowdisplaybreaks

\newcommand{\abs}[1]{\left\lvert #1 \right\rvert}

\begin{document}
\title{Notes}
\date{13 février, 2015}
\maketitle
\section*{1.4 directed graphs}
{\bfseries digraphs} are graphs with orientations (arrows) on the edges

every term from graphs has a directed version for digraphs.

one notable lexicographical difference: edges are called {\bfseries arcs}.

\subsection*{examples}
\tikz\path [graphs/.cd, nodes={shape=circle, draw, text=black,inner sep=1pt,outer sep=0pt}]
  graph [tree layout] { 1<-2, 1->3, 1<-4->5,2->3,{2,3}<-5}
  [shift=(0:1)];

a digraph is {\bfseries weakly connected} if the underlying graph is connected.

a digraph is {\bfseries strongly connected} if for every $u,v\in V(G)$, there exists a directed $u-v$ path and a directed $v-u$ path.

\subsection*{thrm}
a digraph is strongly connected iff it contains a closed spanning directed walk.

\subsection*{notes}
in proving things about digraphs, the degree of a vertex is nuanced. eg arcs in and arcs out might be different numbers. so we say: let $D$ be a digraph and $v\in V(D)$ and we say {\bfseries id$(v)$} is the number of incoming arcs to $v$ and {\bfseries od$(v)$} is the number of outgoing arcs

\section*{4.1 directed graphs}
if $D$ is a simple digraph of size $m$ then $\sum\limits{\text{od}(v)}=\sum\limits{\text{id}(v)}=m$

\section*{eulerian digraphs}
eulerian circuit that is directed
\subsection*{thm}
a digraph $D$ has an eulerian circuit iff od$(v)=$id$(v)$ for all $v\in V(D)$
\subsubsection*{proof}
basically identical to undirected case

need everything to be even so we can go in and out, now we need to be able to go in and out similarly

\subsection*{thm}
a digraph $D$ has an eulerian path iff id$(v)=$od$(v)$ for all vertices but two called $u,v$ and od$(u)=$id$(u)+1$ and id$(v)=$od$(v)+1$
\section*{Homework}
2,7
\end{document}
 
