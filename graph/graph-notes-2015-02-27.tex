\documentclass[letterpaper]{article}

\usepackage{fullpage}
\usepackage{nopageno}
\usepackage{amsmath}
\usepackage{amssymb}
\usepackage{tikz}
\usepackage[utf8]{luainputenc}
\usepackage{aeguill}
\usepackage{setspace}

\tikzstyle{edge} = [fill,opacity=.5,fill opacity=.5,line cap=round, line join=round, line width=50pt]
\usetikzlibrary{graphs,graphdrawing}
\usegdlibrary{trees}

\pgfdeclarelayer{background}
\pgfsetlayers{background,main}

\allowdisplaybreaks

\newcommand{\abs}[1]{\left\lvert #1 \right\rvert}

\begin{document}
\title{Notes}
\date{27 février, 2015}
\maketitle
\section*{6.3 how non-planar are you?}
if we draw a nonplanar graph in a plane, what must occur? we have crossing edges.

the number of such crossings indicates how nonplanar you are

when drawing a graph, we will use 5 conventions.

\begin{enumerate}
\item
no edge will cross itself
\item
at any crossing, only two edges are involved
\item
no edge goes through a vertex
\item
adjacent edges never cross
\item
edges cross at most once
\end{enumerate}

definition: using these conventions to draw a graph $G$, the minumal number of crossings is called {\bfseries the crossing number} and is denoted $\text{cr}(G)$

minimizing the crossing number is challenging to find, but it exists.

\subsection*{thm}
if $G$ is a graph with $|V(G)|=n\ge 3$ and $|E(G)|=m$, then $\text{cr}(G)\ge m-3n+6$. 
\subsubsection*{proof}
Draw $G$ in the plane with minimal crossings.

if $\text{cr}(G)>0$ then we have crossings. draw a vertex at each crossing. Call this graph $G'$ and note that it is planar.

$|V(G')|=|V(G)|+\text{cr}(G)$ and $|E(G')|=|E(G)|+2\text{cr}(G)$. Now we know that $|E(G')|\le 3|V(G')|-6$ (thm 6.3) and $\text{cr}(G)\ge m-3n+6$

for complete graphs, we have other bounds

\subsection*{thm}
$\text{cr}(K_n)\ge\frac{1}{5}\binom{n}{4}$ and $n\ge 5$

\subsection*{thm}
$\text{cr}(K_n)\le \frac{1}{4}\lfloor\frac{n}{2}\rfloor\lfloor\frac{n-1}{2}\rfloor\lfloor\frac{n-2}{2}\rfloor\lfloor\frac{n-3}{2}\rfloor$

\subsection*{thm}
previous theorem is equal (sharp) for $1\le n\le 12$

\subsection*{sixth convention}
all edges are straight (rectilineary embeddings)

\subsubsection*{fact}
most bounds are now shot with these rectilinear embeddings

\subsection*{thm}
for planar graphs, the crossing number of the rectilinear embedding is also $0$.

\subsubsection*{notation}
rectilinear crossing number is $\overline{\text{cr}}(G)$
\end{document}
