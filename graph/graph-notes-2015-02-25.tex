\documentclass[letterpaper]{article}

\usepackage{fullpage}
\usepackage{nopageno}
\usepackage{amsmath}
\usepackage{amssymb}
\usepackage{tikz}
\usepackage[utf8]{luainputenc}
\usepackage{aeguill}
\usepackage{setspace}

\tikzstyle{edge} = [fill,opacity=.5,fill opacity=.5,line cap=round, line join=round, line width=50pt]
\usetikzlibrary{graphs,graphdrawing}
\usegdlibrary{trees}

\pgfdeclarelayer{background}
\pgfsetlayers{background,main}

\allowdisplaybreaks

\newcommand{\abs}[1]{\left\lvert #1 \right\rvert}

\begin{document}
\title{Notes}
\date{25 février, 2015}
\maketitle
\section*{6.2 planarity vs nonplanarity}

how to tell if a graph is planar without drawing it in a plane?

on monday we saw some options for finding non-planarity.
\begin{enumerate}
\item
$e> 3v-6$
\item
can't have $K_{3,3}$ or $K_5$ as a subgraph
\item
didn't get to this last time, but all vertices have degree 6 or more (this follows from one of the above)
\end{enumerate}

if $G$ is a graph then $H$ is a {\bfseries subdivision} of $G$ if $H$ is obtained from $G$ by adding a vertex on any edge of $G$.

The process can be iterated.

\subsubsection*{example}
\begin{tikzpicture}[main_node/.style={circle,draw,text=black,inner sep=1pt,outer sep=0pt]}]
  \node[main_node] (1) at (0,0) {};
  \node[main_node] (2) at (0,1) {};
  \node[main_node] (3) at (1,1) {};
  \node[main_node] (4) at (1,0) {};
  \draw (1)--(2)--(3)--(4)--(1);
\end{tikzpicture}
\begin{tikzpicture}[main_node/.style={circle,draw,text=black,inner sep=1pt,outer sep=0pt]}]
  \node[main_node] (1) at (0,0) {};
  \node[main_node] (2) at (0,1) {};
  \node[main_node] (3) at (1,1) {};
  \node[main_node] (4) at (1,0) {};
  \node[main_node] (5) at (0.5,0) {};
  \draw (1)--(2)--(3)--(4)--(5)--(1);
\end{tikzpicture}

\begin{enumerate}
\item
$H$ is a subdivision of $G$ implies that $H$ and $G$ have the same number of cycles (faces if planar)
\item
$H$ is a subdivision of $G$ implies that $|G|=m$ and $|E(G)|=m$ implies that $|H|=n+c$ and $|E(H)|=m+c$.
\end{enumerate}

\subsection*{kuratowski's theorem}
a graph $G$ is planar if and only if $G$ contains no subdivisions of $K_{3,3}$ or $K_5$ as subgraphs.

\subsubsection*{remark}
there is an inverse operation to subdivision called {\bfseries edge contraction}

an {\bfseries edge contraction} of $uv\in E(G)$ is when $uv$ is replaced by vertex $w$. The neighbors of $w$ are the neighbors of $u$ and $v$ with duplicates deleted. Then $u$ and $v$ are removed.

\subsubsection*{example}
\begin{tikzpicture}[main_node/.style={circle,draw,text=black,inner sep=1pt,outer sep=0pt]}]
  \node[main_node] (1) at (0,0) {};
  \node[main_node] (2) at (0,1) {};
  \node[main_node] (3) at (1,1) {u};
  \node[main_node] (4) at (1,0) {v};
  \draw (1)--(2)--(3)--(4)--(1)--(3);
\end{tikzpicture}
\begin{tikzpicture}[main_node/.style={circle,draw,text=black,inner sep=1pt,outer sep=0pt]}]
  \node[main_node] (1) at (0,0) {};
  \node[main_node] (2) at (0,1) {};
  \node[main_node] (3) at (1,1) {w};
  \draw (1)--(2)--(3)--(1);
\end{tikzpicture}

\begin{tikzpicture}[main_node/.style={circle,draw,text=black,inner sep=1pt,outer sep=0pt]}]
  \node[main_node] (1) at (0,1) {a};
  \node[main_node] (2) at (1,0) {b};
  \node[main_node] (3) at (2,0) {c};
  \node[main_node] (4) at (3,1) {z};
  \node[main_node] (5) at (1,2) {x};
  \node[main_node] (6) at (2,2) {y};
  \draw (1)--(2)--(3)--(4)--(6)--(5)--(1);
  \draw (2)--(5)--(3)--(6);
\end{tikzpicture}
contract $bc$
\begin{tikzpicture}[main_node/.style={circle,draw,text=black,inner sep=1pt,outer sep=0pt]}]
  \node[main_node] (1) at (0,1) {a};
  \node[main_node] (3) at (2,0) {bc};
  \node[main_node] (4) at (3,1) {z};
  \node[main_node] (5) at (1,2) {x};
  \node[main_node] (6) at (2,2) {y};
  \draw (1)--(3)--(4)--(6)--(5)--(1);
  \draw (6)--(3)--(5);
\end{tikzpicture}

a {\bfseries minor} of a graph is any graph obtained from edge contraction, edge deletion or vertex deletion

all subgraphs are minors. add in edge contraction to subgraphs and you get a minor.

no connection to minor of a matrix

\section*{wagner/kuratowski thrm}
a graph is planar if and only if it contains no minors isomorphic to $K_{3,3}$ or $K_5$.

\section*{overarching thrm}
if property $P$ is maintained under taking minors then there exists a finite list of excluded/impossible minors.

ie for planarity, the list is $K_{3,3}, K_5$.

what are some minor closed properties? planarity, forest (excluded minor is $K_3$)

\section*{homework}
2,3,7,10
\end{document}
