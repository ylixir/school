\documentclass[letterpaper]{article}

\usepackage{fullpage}
\usepackage{nopageno}
\usepackage{amsmath}
\usepackage{amssymb}
\usepackage{tikz}
\usepackage[utf8]{luainputenc}
\usepackage{aeguill}
\usepackage{setspace}

\tikzstyle{edge} = [fill,opacity=.5,fill opacity=.5,line cap=round, line join=round, line width=50pt]
\usetikzlibrary{graphs,graphdrawing}
\usegdlibrary{trees}

\pgfdeclarelayer{background}
\pgfsetlayers{background,main}

\allowdisplaybreaks

\newcommand{\abs}[1]{\left\lvert #1 \right\rvert}

\begin{document}
\title{Notes}
\date{17 avril, 2015}
\maketitle
9.4 chromatic plynomials

the chromatic poly counts the number of \lambda-colorings of a graph $G$

notation $P(G,\lambda)$

polynomial starts at zero and then becomes positive. once positive it's positive forever. discrete poly (integer), not continuous.

leading coefficient is positive from end ending positive

zeros are factors

two colorings $C, C'$ are distinct if $\exists v\in V(G)$ such that $C(v)\ne C'(v)$

example:

\tikz\path [graphs/.cd, nodes={shape=circle, draw, text=black,inner sep=1pt,outer sep=0pt}]
  graph [tree layout] { 1 -- {2 -- 3} -- 1 --4--3}
  [shift=(0:1)];

let $\lambda=6$. how many ways can we color $G$ with 6 colors? 1 has 6, 3 has 5, 4 choices for 2 and 4. so 480.

we have shown:

$P(G,6)=480$ 

the enumeration of choices above, we can get $P(G,\lambda)=\lambda(\lambda-1)(\lambda-2)^2$

\subsubsection*{properties}
that should make sense
\begin{enumerate}
\item
chromatic number: $\chi(G)=3$
\item
what is the smallest $\lambda$ such that $P(G,\lambda)>0$? 3o
\item
convention: $P(G,0)=0\forall G$
\end{enumerate}

formally, if $P(G,\lambda)$ is the chrompoly of $G$ then $\chi(G)=\min\limits_\lambda(P(G,\lambda)>0)$

\subsubsection*{exercise}
find the $P(K_n,\lambda)$

$P(K_3,\lambda)=\lambda(\lambda-1)(\lambda-2)$

$P(K_4,\lambda)=\lambda(\lambda-1)(\lambda-2)(\lambda-3)$

$P(K_n,\lambda)=\prod\limits_{i=0}^{n-1}(\lambda-i)=\frac{\lambda!}{(\lambda-n)!}$


if $E_n$ is the empty graph (no edges) on $n$ vertices $P(E_n,\lambda)=\lambda^n$

\subsubsection*{exercise}
$P(C_4,\lambda)$
\tikz\path [graphs/.cd, nodes={shape=circle, draw, text=black,inner sep=1pt,outer sep=0pt}]
  graph [tree layout] { 1 -- 2 -- 3 -- 4--1 }
  [shift=(0:1)];

  $P(C_4,\lambda)=\lambda(\lambda-1)^2(\lambda-2)$ unless the two adjacent to the starting count are the same then $\lambda(\lambda-1)^3$
  
  this leads to sums in $P(G,\lambda)$ and so we have graph theoretically two options
\begin{enumerate}
\item
if they are the same edge contraction
\item
if they are different then we can insert an edge with no change
\end{enumerate}

either choice leads to  a complete graph

\section*{theorem}
if $uv\not\in E(G)$ and $H$ is the graph $G+uv$ with $uv$ contracted, then $P(G,\lambda)=P(G+uv,\lambda)+P(H,\lambda)$

we are going to draw graphs instead of using this notation

\subsubsection*{example}
\tikz\path [graphs/.cd, nodes={shape=circle, draw, text=black,inner sep=1pt,outer sep=0pt}]
  graph { 1;2--3 --1--2;4--5;2--4;5--3};
  =
\tikz\path [graphs/.cd, nodes={shape=circle, draw, text=black,inner sep=1pt,outer sep=0pt}]
  graph { 1;2--3 --1--2;4--5;2--4;5--3;5--1};
  +
\tikz\path [graphs/.cd, nodes={shape=circle, draw, text=black,inner sep=1pt,outer sep=0pt}]
  graph { 2--3;4--5;2--4;5--3;5--2};

=
\tikz\path [graphs/.cd, nodes={shape=circle, draw, text=black,inner sep=1pt,outer sep=0pt}]
  graph { 1;2--3 --1--2;4--5;2--4;5--3;5--1};
  +
\tikz\path [graphs/.cd, nodes={shape=circle, draw, text=black,inner sep=1pt,outer sep=0pt}]
  graph { 2--3;4--5;2--4;5--3;5--2};
+
\tikz\path [graphs/.cd, nodes={shape=circle, draw, text=black,inner sep=1pt,outer sep=0pt}]
  graph { 2--3;4--5;2--4;5--3;5--2;4--3};
+
\tikz\path [graphs/.cd, nodes={shape=circle, draw, text=black,inner sep=1pt,outer sep=0pt}]
  graph { 3;4--3--5--4};

$\vdots$

=
\tikz\path [graphs/.cd, nodes={shape=circle, draw, text=black,inner sep=1pt,outer sep=0pt}]
  graph { 1;2--3 --1--2;4--5;2--4;5--3;5--1;4--3;5--2};
  +
\tikz\path [graphs/.cd, nodes={shape=circle, draw, text=black,inner sep=1pt,outer sep=0pt}]
  graph { 2--3;4--5;2--4;5--3;5--2;4--3};
+
$3\cdot($
\tikz\path [graphs/.cd, nodes={shape=circle, draw, text=black,inner sep=1pt,outer sep=0pt}]
  graph { 2--3;4--5;2--4;5--3;5--2;4--3};
+
\tikz\path [graphs/.cd, nodes={shape=circle, draw, text=black,inner sep=1pt,outer sep=0pt}]
  graph { 3;4--3--5--4};
$)$

homework is $1,6,7$
\end{document}
