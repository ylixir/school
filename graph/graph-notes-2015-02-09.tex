\documentclass[letterpaper]{article}

\usepackage{fullpage}
\usepackage{nopageno}
\usepackage{amsmath}
\usepackage{amssymb}
\usepackage{tikz}
\usepackage[utf8]{luainputenc}
\usepackage{aeguill}
\usepackage{setspace}

\tikzstyle{edge} = [fill,opacity=.5,fill opacity=.5,line cap=round, line join=round, line width=50pt]
\usetikzlibrary{graphs,graphdrawing}
\usegdlibrary{trees}

\pgfdeclarelayer{background}
\pgfsetlayers{background,main}

\allowdisplaybreaks

\newcommand{\abs}[1]{\left\lvert #1 \right\rvert}

\begin{document}
\title{Notes}
\date{9 février, 2015}
\maketitle
{\bfseries hypergraph} $H$ is a set $V$ called the vertex set together with nonempty subsets of $V$ called edges.
\subsubsection*{example}
\begin{enumerate}
\item
every  graph is a hypergraph. in a graph all edges are size $2$. 
\item
$H: V=\{1,2,3,4\}, E=\{\{1,2,4\},\{2,3,4\},\{3,4,5\}\}$
\item
\begin{tikzpicture}[main_node/.style={circle,draw,text=black,inner sep=1pt,outer sep=0pt]}]
\node[main_node] (1) at (-1,-1) {};
\node[main_node] (2) at (0,-1) {};
\node[main_node] (3) at (1,-1) {};
\node[main_node] (4) at (0,1) {};
\node[main_node] (5) at (1,1) {};
\begin{pgfonlayer}{background}
\draw[edge,color=yellow] (1) -- (2) -- (3);
\begin{scope}[transparency group,opacity=.5]
\draw[edge,opacity=1,color=green] (2) -- (3) -- (5);
%\fill[edge,opacity=1,color=green] (v3.center) -- (v5.center) -- (v6.center) -- (v3.center);
\end{scope}
\draw[edge,color=red,line width=40pt] (4) -- (5);
\end{pgfonlayer}
\end{tikzpicture}
\end{enumerate}
if each edge has the same size then they are called {\bfseries $k$-uniform}. degree still makes sense,as does multi vs simple

we can associate a bipartite graph to any hypergraph


the bipartite graph of $H$ has partites $V$ and $E$. it has edges from $v_i$ to $e_j$ if $v_i\in e_j$
\begin{tikzpicture}[main_node/.style={circle,draw,text=black,inner sep=1pt,outer sep=0pt]}]
\node[main_node] (5) at (-1,-1) {5};
\node[main_node] (6) at (0.5,1) {a};
\node[main_node] (4) at (0,-1) {4};
\node[main_node] (3) at (1,-1) {3};
\node[main_node] (1) at (0,1) {1};
\node[main_node] (2) at (1,1) {2};
\node[main_node] (7) at (-1.5,-1) {d};
\node[main_node] (8) at (0.5,-1) {c};
\node[main_node] (9) at (.5,0) {b};
\begin{pgfonlayer}{background}
\draw[edge,color=yellow] (5) -- (4) -- (3);
\begin{scope}[transparency group,opacity=.5]
\draw[edge,opacity=1,color=green] (2) -- (3) -- (4);
%\fill[edge,opacity=1,color=green] (v3.center) -- (v5.center) -- (v6.center) -- (v3.center);
\end{scope}
\draw[edge,color=red,line width=40pt] (1) -- (2);
\draw[edge,color=purple,line width=40pt] (5)--(5);
\end{pgfonlayer}
\end{tikzpicture}
\tikz\path [graphs/.cd, nodes={shape=circle, draw, text=black,inner sep=1pt,outer sep=0pt}]
%  graph [tree layout] { 1 -- a,a--2,2--b,b--3,3--c,b---4,4--c,5--d,c--5 }
  graph [tree layout] {1--a--2--b--3--c--5--d, b--4--c}
  [shift=(0:1)];
\subsubsection*{question}
can we construct a hypergraph out of a bipartite graph? no in general.

restriction: cannot have an isolated vertex on the edge side of hypergraph.

$\{\text{hpergraphs}\}\leftrightarrow\{\text{hypergraphs w/ no isolated right side vertices}\}$

\section*{adjacency matrices}
they still exist

a hypergraph $H=(V,E)$ is an $|V|\times|E|$ matrix $A(H)$ such that $a_{ij}=1$ if $v_i\in e_j$ or else $a_{ij}=0$

example:
\begin{tikzpicture}[main_node/.style={circle,draw,text=black,inner sep=1pt,outer sep=0pt]}]
\node[main_node] (5) at (-2,-2) {5};
\node[main_node] (6) at (1,2.3) {b};
\node[main_node] (4) at (0,-2) {4};
\node[main_node] (3) at (2,-2) {3};
\node[main_node] (3) at (2.5,-2) {e};
\node[main_node] (1) at (0,2) {1};
\node[main_node] (2) at (2,2) {2};
\node[main_node] (7) at (-1,0) {c};
\node[main_node] (8) at (1,-2) {d};
\node[main_node] (9) at (1.3,-.) {a};
\begin{pgfonlayer}{background}
\draw[edge,color=yellow] (5) -- (4) -- (3);
\begin{scope}[transparency group,opacity=.5]
\draw[edge,opacity=1,color=green] (2) -- (5) -- (4);
%\fill[edge,opacity=1,color=green] (v3.center) -- (v5.center) -- (v6.center) -- (v3.center);
\end{scope}
\draw[edge,color=red,line width=40pt] (1) -- (2);
\draw[edge,color=purple,line width=40pt] (1)--(3);
\draw[edge,color=blue,line width=40pt] (3)--(3);
\end{pgfonlayer}
\end{tikzpicture}

the transpose of $A(h)$ has another hypergraph associated to it called the dual of $H$. Denoted $H^*$. note $H^{**}=H$.

\subsection*{question:} how do hypergraph adjacency matrices compare to ``regular'' adjacency matrices?

hypergraph:
no symmetry, $A^T(H)\ne A(H)$
and $H^*\ne H$

graph:
$A^T(G)=A(G)$, $G^*=G$

it turns out that the matrices are \emph{very}  different. 
\end{document}
 
