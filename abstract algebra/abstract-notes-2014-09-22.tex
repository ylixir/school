\documentclass[letterpaper]{article}

\usepackage{fullpage}
\usepackage{nopageno}
\usepackage{amsmath}
\usepackage{amssymb}
\allowdisplaybreaks

\newcommand{\abs}[1]{\left\lvert #1 \right\rvert}

\begin{document}
\title{Notes}
\date{September 22, 2014}
\maketitle
\section*{assignment}
Section 3.1: \#2, 7, 9, 11, 20.
\begin{enumerate}
\setcounter{enumi}{1}
\item
\begin{enumerate}
\item
no, not all elements have inverses
\item
no, no identity
\item
not associative
\item
no identity
\item
yes
\item
yes
\end{enumerate}
\setcounter{enumi}{6}
\item
this only includes things that are relatively prime to the item we are making the table for. of course 7 is prime, 
\setcounter{enumi}{8}
\item
identity is $e$
\begin{align*}
  e=a^x\\
  \ln e=\ln a^x\\
  x\ln a=1\\
  x=\frac{1}{\ln a}
\end{align*}
\setcounter{enumi}{19}
\item
note that s is finite. we need to prove that an identity exists and every element is invertible.

$S=\{a_1,\dots,a_n\}, a\in S$. $\alpha:S\to S$ where $\alpha(x)=a\star x$. Claim, $\alpha$ is injective.$\alpha(x)=\alpha(y)=a\star x=a\star y\to x=y$. Since $S$ is finite and $\alpha$ is injective then $\alpha$ must be bijective and then surjective.

so $a\in S$ so there exists $e\in S$ st $\alpha(e)=a\star e=a$.

now we need to show that $e$ works for all elements, not just the $a$ that we chose

there exists $c\in S$ st $b=ca$. we have $\beta(x)=xa$ similar to above

so $b*e=(c*a)*e=c*(a*e)=c*a=b$

so there exists $b*e=b$ for all $b$ and similarly $e'*b=b$ for all $b$ in $S$.

claim $e=e'$. shown by $e'*e=e'$ and $e'*e=e$

so we have our identity.

and so on

proof doesn't work if $S$ is infinite because surjectivity isn't implied from injectivity

counterexample would be $(\mathbb{Z}^*,\cdot)$

note that this is proved more generally in 3.1.8
\end{enumerate}

\section*{3.2 subgroups}
let $(G,*)$ be a group and $H\subseteq G$ not equal to $\emptyset$. we say that $H$ is a subgroup of $G$ if with respect to the same operation, $H$ is a group.

\subsubsection*{example}
$(\mathbb{R},+), \mathbb{Q}\subseteq\mathbb{R}$, $\mathbb{Q}$ is a subgroup of $R$

checking associativity is a waste of time, it's given by super group. identity must also be in subgroup, inverse must exist for all elements of subgroup. and operation must be binary. eg it must be closed.

\subsection*{proposition 3.2.2}
\begin{itemize}
\item
$ab\in H$ for all $a,b\in H$
\item
$e\in H$
\item
$a^{-1}\in H$ for all $a\in H$
\end{itemize}
\subsubsection*{proof}
the operation being closed is clear

e being unique is done by observing $e'e'=e'$ where $e'$ is inverse in $H$. now in $G$ $e'e=e'$ and so $e'e'=e'e\to(e')^{-1}e'=(e')^{-1}e'e\to e'=e$

so the inverse is unique

then $e=ab$ but $e=a^{-1}$ so $ab=aa^{-1}\to a^{-1}ab=a^{-1}aa^{-1}$ so $b=a^{-1}$ if $b\in H$


closure brings identity and inverse automatically if $G$ is finite

trivial subgroups are the group itself and it's identity element
\section*{examples}

\end{document}

