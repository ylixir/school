\documentclass[letterpaper]{article}

\usepackage{fullpage}
\usepackage{nopageno}
\usepackage{amsmath}
\usepackage{amssymb}
\allowdisplaybreaks

\newcommand{\abs}[1]{\left\lvert #1 \right\rvert}

\begin{document}
\title{Notes}
\date{September 5, 2014}
\maketitle
\section*{7}
$680=2^3\cdot5\cdot17, 2^3\cdot5+17=57, m,n=40,17$
\section*{8}
\begin{align*}
  (h,k)=m\\
  m|dh\rightarrow m|a\\
  m|dk\rightarrow m|b\\
\end{align*}
\section*{12}
\begin{align*}
  (a,b)=1\\
  (a,c)=1\\
  \Leftrightarrow\\
  (a,[b,c])=1
\end{align*}
\section*{19}
p,q are twin primes, provethat pq+1 is square iff p,q are twin primes

\begin{align*}
  q=p+2\\
  pq+1=p(p+2)+1=p^2+2p+1=(p+1)^2\\
  m^2=pq+1\\
  mm-1=pq
  (m+1)(m-1)=pq\\
  (a+1)=pq, or 1 or p
  (a-1)=1, or pq, or q
\end{align*}
\section*{23}
$x^m-1=(x-1)(x^{m-1}+x^{m-2}+...+x+1)$

$x^{2k+1}+1=(x+1)(x^{2k}-x^{2k-1}+x^{2k-2}-...+x^2-x+1)$

$2^n+1$ is prime is given. n is a power of two iff prime factorization of n is $2^m$. prove by contradiction. assume there exists $p=2k+1$ that divides n. $n=(2k+1)\cdot q$.
  
\begin{align*}
  2^n+1=2^{q(2k+1)}\\
  =(2^{q})^{2k+1}=(2^q+1)(2^{q2k}-2^{q(2k-1)}+...+1)
\end{align*}
now $2^n+1$ is not prime unless $p=1$ and $p$ is prime
\section*{last time}
$a,n\in\mathbb{Z}, n>1$ the equation $ax\equiv 1 \mod n$ has a solution iff $(a,n)=1$.

\section*{thm}
$a,b,n\in\mathbb{Z},n>1$
\begin{enumerate}
\item
the only eq $ax\equiv b\mod n$ has a solution iff $d|b$ where $d=\gcd(a,n)$.
\item
assume that $d|b$ then the integer solutions of the equation are of the form $...x-\frac{2n}{d},x-\frac{n}{d},x,x+\frac{n}{d},x+\frac{2n}{d},...$, in particular modulo n, there exist exactly d distinct solutions,namely $x,x+\frac{n}{d},x+\frac{2n}{d},...,x+\frac{(d-1)n}{d}$
\end{enumerate}
\subsection*{proof}
assume that $ax\equiv  b\mod n$ has a solutionn. then there exist $\alpha,q\in\mathbb{Z}$ such that $a\alpha-b=nq$. this implies that $b=a\alpha-nq\rightarrow d|b$ because $d|a\alpha$ and $d|nq$


assume $d|b$. then $b=d\beta$ for some $\beta\in\mathbb{Z}$
\begin{align*}
  b=(as+nt)\beta, s,t\in\mathbb{Z}\\
  as\beta\equiv b\mod n\rightarrow s\beta\text{ is a solution}
\end{align*}

assume $d|b$, let $m=\frac{n}{d}$. claim $\alpha$ solution $\rightarrow\alpha+km$ solution for all $k\in\mathbb{Z}$.
\subsubsection*{proof of claim}
$\alpha$ solution$\Rightarrow a\alpha\equiv b\mod n$ but $a(\alpha+km)=a\alpha+akm$ and $akm=ak\frac{n}{d}=n\frac{a}{d}k\in\mathbb{Z}$ so $akm\equiv a\alpha\equiv b \mod n$

to finish we need to prove the following:

if $\alpha,\beta$  are solutions then $\beta-\alpha$ is a multiple of m.
\begin{align*}
  a\alpha\equiv b\mod n\\
  a\beta\equiv b\mod n\\
  a\alpha \equiv a\beta\mod n\\
  n|a(\beta-\alpha)\\
  n=md\\
  md|a(\beta-\alpha)\\
  a=a'd\\
  md|a'd(\beta-\alpha)\\
  m|a'(\beta-\alpha)
\end{align*}
if we know that gcd of $m$ and $a'$ is one then $m|(\beta-\alpha)$. we know it is because $md=n$  and $a=a'd$ and d is gcd of $a,n$ so if there were another divisor then d wouldn't be the gcd, it would be pd.

\section*{chinese remainder theorem}
$m,n\in\mathbb{Z}^+$ then the system $x\equiv a\mod n, x\equiv b\mod m$ has an integer solution iff m and n are relatively prime. moreover, any two solutions are congruent modulo mn.

\subsection*{proof}
m,n are relatively prime, write $m\alpha+\beta n=1$, let $x=a\alpha m+b\beta n$ then $x\equiv a\alpha m\equiv a\mod n$ because $\alpha m$ is congruent to 1. and $x\equiv b\beta n\equiv b\mod m$
\section*{exercises}
second part of chinese remainder theorem
\end{document}
