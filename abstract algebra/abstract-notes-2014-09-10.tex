\documentclass[letterpaper]{article}

\usepackage{fullpage}
\usepackage{nopageno}
\usepackage{amsmath}
\usepackage{amssymb}
\allowdisplaybreaks

\newcommand{\abs}[1]{\left\lvert #1 \right\rvert}

\begin{document}
\title{Notes}
\date{September 10, 2014}
\maketitle
\section*{assignment}
\subsection*{2.2 \#16}
prove f is onto iff there exists $g:B\to A$ such that $f\circ g=f(g(x))=1_B$
\subsubsection*{proof}
Assume $f$ is onto then $\forall b\in B$ $\exists x_b\in A$ such  that $f(x_b)=b$.

For every $b\in B$ choose $x_b\in A$

define $g:B\to A, g(b)=x_b$, then $(f\circ f)(b)=f(g(b))=f(x_b)=b$

other way

prove that for every $b\in B$ there exists $x\in A$ such that $f(x)=b$

we have $b=f(g(b))$. $g(b)=x$. almost a tautology. done
\subsection*{2.2 \#18}
\subsubsection*{LOOK HERE}
Let $A$ be a 

\section*{last time}
equivalence relations
\subsection*{example}
let $f:S\to T$. on $S$ we define the equivalence relation as follows: $x,y\in S$ then $x\sim_f y$ iff $f(x)=f(y)$. 

two elements are related iff they have the same image. if $f$ is injective then $[f]=f$. constant function has one equivalence class.

\subsection*{proposition}
there exists a one to one (bijection) from the set of equivalence classes $S/\sim_f$ and $f(s)$. $\bar f:S/\sim_f\to f(S)$. Namelly $[x]\to f(x)$.

question? does $[x]=[y]$ imply that $f(x)=f(y)$? in this case, $[x]=[y]\Rightarrow x\sim_f y\Rightarrow f(x)=f(y)$ so f is a well defined function. (well defined is redundant, but places proper emphasis)

surjectivity of $\bar f$ is clear. what about injectivity? if $f(x)=f(y)\to x\sim_f y\to [x]=[y]$
\section*{1.4 integers modulo n}
$n>1,n\in\mathbb{Z}$. on $\mathbb{Z}$ we define the equiv relation $\equiv$ as $a\equiv b \mod n$ if and only if $n|(a-b)$. 

$\mathbb{Z}_n=\mathbb{Z}/\equiv\to$ set of equiv classes. $\mathbb{Z}_n=\{[0],[1],\dots,[n-1]\}$

alternate notation is $\mathbb{Z}_n=\{\bar0,\bar1,\dots,\bar{n-1}\}$.

define operations on $\mathbb{Z}_n$ as follows:

addition: $[a]+[b]=[a+b]$

multiplication: $[a]\cdot[b]=[a\cdot b]$

$[a]=[a'],[b]=[b']\to[a+b]=[a'+b']$.

$n|(a-a'),n|(b-b')\to n|((a+b)-(a'+b'))$

similarly for multiplication.

$[a]=[a'],[b]=[b']\to[a\cdot b]=[a'\cdot b']$.

\subsection*{proposition}
these operation satisfy associative, commutative, distributive

definition:
Let $[a]_n\in\mathbb{Z}_n$. If there exists $[b]_n\in\mathbb{Z}$ such that $[b]\ne 0, [a][b]=[0]$. We say that $[a]$ is a zero-divisor in $\mathbb{Z}_n$

\subsubsection*{example}
$\mathbb{Z}_6=\{[0],[1],[2],[3],[4],[5],[6]\}$. zero divisors are $\{[0],[2],[3],[4]\}$

\subsection*{multiplicative inverse}
if $[a][b]=1$ for some $[b]\in\mathbb{Z}_n$ we say that $[a]$ is an invertible element of $\mathbb{Z}_n$ and $[b]$ is a multiplicative inverse of $[a]$.

$\mathbb{Z}_6$ invertible elements:$\{[1][5]\}$ because $[5][5]=1$ (25 mod 6 is 1)

lets take $[a]\in \mathbb{Z}$ then $[a]$ is invertible iff $\gcd(a,n)=1$.
\subsubsection*{proof}
assume $(a,n)=1$ then $a\alpha+n\beta=1$ with $\alpha,\beta\in\mathbb{Z}$.Then $[1]=[a][\alpha]+[n][\beta]$ $[n]=[0]$

assume $[a]$ is invertible. then there exists $[b]$ such that $[a][b]=1, n|(ab-1)$. $ab-1=nk, (a,n)=1$
\end{document}
