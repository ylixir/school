\documentclass[letterpaper]{article}

\usepackage{fullpage}
\usepackage{nopageno}
\usepackage{amsmath}
\usepackage{amssymb}
\allowdisplaybreaks

\newcommand{\abs}[1]{\left\lvert #1 \right\rvert}

\begin{document}
\title{Homework}
\date{November 26, 2014}
\author{Jon Allen}
\maketitle
Section 4.3: 9, 12
Section 4.4: 4 (d), 17
\renewcommand{\labelenumi}{4.\arabic{enumi}}
\renewcommand{\labelenumii}{\arabic{enumii}.}
\renewcommand{\labelenumiii}{(\alph{enumiii})}
\begin{enumerate}
\setcounter{enumi}{2}
\item
  \begin{enumerate}
  \setcounter{enumii}{8}
  \item
  If we multiply two elements $[a+bx],[c+dx]\mathbb{R}[x]/\langle x^2+x+1\rangle$ then we get $ac+(ad+bc)x+bdx^2$. Note that $x^2+x+1\equiv0\mod x^2+x+1$ and so $x^2\equiv-x-1\mod x^2+x+1$ which means $ac+(ad+bc)x+bdx^2=ac-bd+(ad+bc-bd)x$

  We first need to construct $\phi:\mathbb{C}\to \mathbb{R}[x]/\langle x^2+x+1\rangle$.
  We know that $\phi(1)=1$  so we have half done.
  Because $i^2=-1$ we need to find $\phi(i)^2=-1$. Now $(a+bx)^2=a^2+2abx+b^2x^2=a^2-b^2+(2ab-b^2)x=-1$ and so
  \begin{align*}
    2ab-b^2&=0\\
    a^2-b^2&=-1\\
    2a&=b\\
    a^2-4a^2&=a^2(1-4)=-1\\
    a^2=\frac{1}{3}
  \end{align*}
  We can choose $a=\frac{1}{\sqrt{3}}$ or $a=-\frac{1}{\sqrt{3}}$. We choose $a=\frac{1}{\sqrt{3}}$ and say $\phi(a+bi)=a+b(\frac{1}{\sqrt{3}}+\frac{2}{\sqrt{3}}x)=a+bz$.
  \begin{align*}
    \phi((a+bi)+(c+di))&=\phi(a+c+(b+d)i)\\
    &=a+c+(b+d)z\\
    &=a+bz+c+dz\\
    &=\phi(a+bi)+\phi(c+di)\\
    \phi((a+bi)(c+di))&=\phi(ac-bd^2+(ad+bc)i)\\
    &=ac-bd+(ad+bc)z\\
    \phi(a+bi)\phi(c+di)&=(a+bz)(c+dz)\\
    &=ac+(ad+bc)z+bdz^2\\
    &=ac+(ad+bc)z-bd\\
    &=ac-bd+(ad+bc)z
  \end{align*}
  I claim that surjectivity is obvious because both fields are reals plus something times a real. 

  \setcounter{enumii}{11}
  \item
  We build need to construct $\phi: \mathbb{Q}(\sqrt{2})\to\mathbb{Q}[x]/\langle x^2-2\rangle$. We already have $\phi(1)=1$ and so we are half done. First lets find $\phi(\sqrt{2})^2=\phi(2)$
  Note that $x^2\equiv 2\mod x^2-2$
  \begin{align*}
    (a+bx)^2&=a^2+2abx+b^2x^2\\
    &=a^2+2b^2+2abx\\
    2&=a^2+2b^2\\
    0&=a\\
    1&=b^2
  \end{align*}
  We choose $b=1$ and so $\phi(a+b\sqrt{2})=a+bx$. Which is obvious to me now, but I already did the work, so I'm gonna leave it.
  \begin{align*}
    \phi(a+b\sqrt{2})+\phi(c+d\sqrt{2})&=a+c+(b+d)x=\phi(a+c+(b+d)\sqrt{2}\\
    \phi(a+b\sqrt{2})\cdot\phi(c+d\sqrt{2})&=ac+(ad+bc)x+bdx^2=ac+2bd+(ad+bc)x\\
    \phi((a+b\sqrt{2})\cdot(c+d\sqrt{2}))&=\phi(ac+2bd+(ad+bc)\sqrt{2})=ac+2bd+(ad+bc)x
  \end{align*}
  \end{enumerate}
\item
  \begin{enumerate}
  \setcounter{enumii}{3}
  \item
    \begin{enumerate}
    \setcounter{enumiii}{3}
    \item
      $x^2+2x-5$

      $(x-1)^2+2x-2-5=x^2-2x+1+2x-2-5=x^2-6$

      $(x+1)^2+2x+2-5=x^2+2x+1+2x+2-5=x^2+4x-2$

      So in either shift we have 2 dividing all the coefficients, but not $1$ and 4 does not divide 6 or two. So it is irreducible
    \end{enumerate}
  \setcounter{enumii}{16}
  \item
    \begin{enumerate}
    \item
    \begin{align*}
      f'(x)&=6x^5+3x^2\\
      f''(x)&=30x^4+6x\\
      f^3(x)&=120x^3+6\\
      f^4(x)&=360x^2\\
      f^5(x)&=720x\\
      f^6(x)&=720\\
      f(x+c)&=x^6+6cx^5+15c^2x^4+(20c^3+1)x^3+(15c^4+3c)x^2+(6c^5+3c^2)x+c^6+c^3+1
    \end{align*}
    We see immediately that $x-1$ doesn't work because $(-1)^6+(-1)^3+1=1$ which doesn't give us anything to divide. $x+1$ gives us a divisor of $3$ that works. $x-2$ gives us $a_0=3\cdot19$ which works for $3$ again. $x+2$ gives us $a_0=73$ which is prime, so that doesn't work.
    \end{enumerate}
  \end{enumerate}
\end{enumerate}
\end{document}
