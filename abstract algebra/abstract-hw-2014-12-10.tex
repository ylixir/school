\documentclass[letterpaper]{article}

\usepackage{fullpage}
\usepackage{nopageno}
\usepackage{amsmath}
\usepackage{amssymb}
\allowdisplaybreaks

\newcommand{\abs}[1]{\left\lvert #1 \right\rvert}

\begin{document}
\title{Homework}
\date{December 10, 2014}
\author{Jon Allen}
\maketitle
Section 5.1: 4
Section 5.2: 18
Section 5.3: 11, 17.
\renewcommand{\labelenumi}{5.\arabic{enumi}}
\renewcommand{\labelenumii}{\arabic{enumii}.}
\renewcommand{\labelenumiii}{(\alph{enumiii})}
\begin{enumerate}
\item
  \begin{enumerate}
  \setcounter{enumii}{3}
  \item
  Let $R=\{m+n\sqrt{2}:m,n\in \mathbb{Z}\}$
    \begin{enumerate}
    \item
    Show that $m+n\sqrt{2}$ is a unit in $R$ if and only if $m^2-2n^2=\pm 1.$

    {\em Hint}: Show that if $(m+n\sqrt{2})(x+y\sqrt{2})=1$ then $(m-n\sqrt{2})(x-y\sqrt{2})=1$ and multiply the two equations.

    Assume that $m+n\sqrt{2}$ is a unit in $R$. Then we can find some $x+y\sqrt{2}$ such that
    \begin{align*}
      (m+n\sqrt{2})(x+y\sqrt{2})=mx+\sqrt{2}(my+nx)+2ny&=1\\
      mx-2ny&=1
    \end{align*}
    Note that $my+nx$ has to be $0$ (the reals are an integral domain, and we have an integer [rational] result).
    \begin{align*}
      (m-n\sqrt{2})(x-y\sqrt{2})&=mx-\sqrt{2}(my+nx)+2ny\\
      &=mx+2ny=1\\
      (m-n\sqrt{2})(x-y\sqrt{2})(m+n\sqrt{2})(x+y\sqrt{2})&=1\\
      (m^2-2n^2)(x^2-2y^2)&=1\\
    \end{align*}
    And because we are dealing with integers, we know that $m^2-2n^2=x^2-2y^2=\pm 1$.
    
    Now $m^2-2n^2=\pm 1=(m+n\sqrt{2})(m-n\sqrt{2})$ and so $m+n\sqrt{2}$ has an inverse of either $m-n\sqrt{2}$ or $-m+n\sqrt{2}$.
    $\Box$
    \item
    Show that $1+2\sqrt{2}$ has infinite order in $R^\times$

    If $1+2\sqrt{2}$ doesn't  have infinite order, then there exists some $n$ such that $(1+2\sqrt{2})^n=1=(1+2\sqrt{2})(1+2\sqrt{2})^{n-1}$. I don't know what $(1+2\sqrt{2})^{n-1}$ is but if it exists, then it is the inverse of $1+2\sqrt{2}$ which would make $1+2\sqrt{2}$ a unit. And so we know $1^2-2\cdot 2^2=\pm 1$. Whoops, guess it's not a unit, and therefore has infinite order.
    \item
    Show that $1$ and $-1$ are the only  units that have finite order in $R^\times$

    We take a finite order element $(m+n\sqrt{2})$ and note that if it has finite order, then it has an inverse and so $m^2-2n^2=\pm 1$ and so $m^2=n^2+1$ or $n^2=m^2+1$. We assume that $m^2=n^2+1$. And so $n^2+1=m^2> n^2\therefore |m|\ge |n|+1>|n|$. Now $(|n|+1)^2=n^2+2|n|+1$ gives us $n^2+1\ge n^2+2|n|+1$ and $0\ge |n|$. So we know that $m^2=\pm 1$ and then $m=\pm 1$ as expected. Doing the same trick the other way around we get $0=m$ and then $-2n^2=\pm1$. Which means that $(\pm 2)^{-1}=n^2$. Obviously $\pm 2$ has no inverse in $\mathbb{Z}$ and $n^2\in \mathbb{Z}$ so the only finite order elements are $1$ and $-1$.
    \end{enumerate}
  \end{enumerate}
\item
  \begin{enumerate}
  \setcounter{enumii}{17}
  \item
  Define $\phi:\mathbb{Z}\to\mathbb{Z}_m\oplus\mathbb{Z}_n$ by $\phi(x)=([x]_m,[x]_n)$. Find the kernel and image of $\phi$. Show that $\phi$ is onto if and only if $\gcd(m,n)=1$.

  %It's fairly obvious that we have a homomorphism, and the question implies that we can take that for granted, but just to be sure, note that $[a+b]=[a]+[b]$ and $[ab]=[a][b]$.

  If $[x]_m=[0]_m$ then $m|x$ and similarly if $[x]_n=[0]_n$ then $n|x$. Then $\ker(\phi)=\{x\in \mathbb{Z}:m|x\text{ and }n|x\}$.

  Let $k=\gcd(m,n)$ and $m=ka, n=kb$. Then $[x]_k=[mq+[x]_m]_k=[kaq+[x]_m]_k=[[x]_m]_k$ and similarly $[x]_k=[[x]_n]_k$ and so $[[x]_m]_k=[[x]_n]_k$. So the image consists of $\{(x,y)\in \mathbb{Z}_m\oplus\mathbb{Z}_n:[x]_{\gcd(m,n)}=[y]_{\gcd(m,n)}\}$.
  
  Now if $\gcd(m,n)$ is one then there are no restrictions on the image and so it is onto. That is to say $[x]_1=[0]_1$ regardless of ones choice of $x$. If $\gcd(m,n)=k>1$ then $0\in\mathbb{Z}_m$ and $1\in \mathbb{Z}_n$. Now $k>1$ and so $[[0]_m]_k=[0]_k\ne[1]_k=[[1]_n]_k$. And so $([0]_m,[1]_n)\not\in\phi(\mathbb{Z})$ and then $\phi$ is not onto.
  \end{enumerate}
\item
  \begin{enumerate}
  \setcounter{enumii}{10}
  \item
  Let $R$ be a commutative ring, with $a\in R$. The {\bfseries annihilator} of $a$ is defined by
  \[\text{Ann}(a)=\{x\in R:xa=0\}\]
  Prove that $\text{Ann}(a)$ is an ideal of $R$
  \subsubsection*{proof}
  Let us take any $x,y\in \text{Ann}(a)$. Then $(x+y)a=xa+ya=0$ and so $x+y\in \text{Ann}(a)$. Similarly $(x-y)a=xa-ya=0$ and so $x\pm y\in \text{Ann}(a)$. Now if we take an arbitrary $r\in R$ and an arbitrary $x\in \text{Ann}(a)$ then $(rx)a=r(xa)=r0=0$ and so $rx\in \text{Ann}(a)$.
  $\Box$

  \setcounter{enumii}{16}
  \item
  Let $R$ be the set of all matrices $\left[\begin{array}{cc}a&b\\c&d\end{array}\right]$ over $\mathbb{Q}$ such that $a=d$ and $c=0$.

  So $\left[\begin{array}{cc}a&b\\0&a\end{array}\right]$
    \begin{enumerate}
    \item
    Verify that $R$ is a commutative ring.

    We get that addition is an abelian group for free because the elements of the matrices are in $\mathbb{Q}$ which is an abelian group.

    \begin{align*}
      \left[\begin{array}{cc}a_1&b_1\\0&a_1\end{array}\right]
      \left[\begin{array}{cc}a_2&b_2\\0&a_2\end{array}\right]
      &=\left[\begin{array}{cc}a_1a_2&a_1b_2+a_2b_1\\0&a_1a_2\end{array}\right]
    \end{align*}
    Because $\mathbb{Q}$ is commutative under multiplication and addition, it is easy to see that that above multiplication is also commutative (swapping the ones and twos doesn't change anything).
    Now changing $a_2$ to $a_2+a_3$ and likewise $b_2\Rightarrow b_2+b_3$ the above result becomes
    \begin{align*}
      \left[\begin{array}{cc}a_1(a_2+a_3)&a_1(b_2+b_3)+(a_2+a_3)b_1\\0&a_1(a_2+a_3)\end{array}\right]
      &=\left[\begin{array}{cc}a_1a_2&a_1b_2+a_2b_1\\0&a_1a_2\end{array}\right]
      +\left[\begin{array}{cc}a_1a_3&a_1b_3+a_3b_1\\0&a_1a_3\end{array}\right]
    \end{align*}
    Which give us distribution for multiplication, so we have a commutative ring.
    \item
    Let $I$ be the set of all such matrices for which $a=d=0$. Show that $I$ is an ideal of $R$.

    Obviously $\left[\begin{array}{cc}0&b_1\\0&0\end{array}\right]+\left[\begin{array}{cc}0&b_2\\0&0\end{array}\right]=\left[\begin{array}{cc}0&b_1+b_2\\0&0\end{array}\right]\in I$ and $\left[\begin{array}{cc}a&b_1\\0&a\end{array}\right]\left[\begin{array}{cc}0&b_2\\0&0\end{array}\right]=\left[\begin{array}{cc}0&ab_2\\0&0\end{array}\right]$. Because $ab_2\in \mathbb{Q}$ then $\left[\begin{array}{cc}0&ab_2\\0&0\end{array}\right]\in I$
    \item
    Use the fundamental homomorphism theorem for rings to show that $R/I\cong \mathbb{Q}$.

    $\phi:\left[\begin{array}{cc}a&b\\0&a\end{array}\right]\to a$

    \begin{align*}
      \phi\left(\left[\begin{array}{cc}1&0\\0&1\end{array}\right]\right)&=1\\
      \phi\left(\left[\begin{array}{cc}a_1&b_1\\0&a_1\end{array}\right]+\left[\begin{array}{cc}a_2&b_2\\0&a_2\end{array}\right]\right)&=a_1+a_2\\
      &=\phi\left(\left[\begin{array}{cc}a_1&b_1\\0&a_1\end{array}\right]\right)+\phi\left(\left[\begin{array}{cc}a_2&b_2\\0&a_2\end{array}\right]\right)\\
      \phi\left(\left[\begin{array}{cc}a_1&b_1\\0&a_1\end{array}\right]
      \left[\begin{array}{cc}a_2&b_2\\0&a_2\end{array}\right]\right)
      &=a_1a_2\\
      &=\phi\left(\left[\begin{array}{cc}a_1&b_1\\0&a_1\end{array}\right]\right)
      \phi\left(\left[\begin{array}{cc}a_2&b_2\\0&a_2\end{array}\right]\right)\\
      \phi\left(\begin{array}{cc}0&b\\0&0\end{array}\right)&=0
    \end{align*}

    So we see that $\phi$ is a homomorphism and $I$ is $\ker\phi$

    Clearly $\phi(R)=\mathbb{Q}$ as we have no restriction on the ``$a$'' element. And so by the fundamental homomorphism theorem for rings we have $R/I\cong\mathbb{Q}$
    \end{enumerate}
  \end{enumerate}
\end{enumerate}
\end{document}
