\documentclass[letterpaper]{article}

\usepackage{fullpage}
\usepackage{nopageno}
\usepackage{amsmath}
\usepackage{amssymb}
\allowdisplaybreaks

\newcommand{\abs}[1]{\left\lvert #1 \right\rvert}

\begin{document}
\title{Notes}
\date{September 26, 2014}
\maketitle
if $G$ is a group, and $A\subseteq G$ and $B\subseteq G$ then $AB=\{ab|a\in A, b\in B\}\subseteq G$.

\section*{proposition}

let $G$ be a group, then $H,K$ subgroups of $G$. Assume that $h^{-1}kh\in K$ for all $h\in H$, $k\in K$ then $HK$ is a subgroup of $G$ that contain s both $H$ and $K$, in fact, $HK$ is the smallest subgroup of $G$ that contains both $H$ and $K$. Assumption only important if we are not dealing with abelian groups.

\subsubsection*{proof}
$a,b\in HK$. Write $a=h_1k_1,b=h_2k_2$ with $h_i\in H,k_i\in K$ then $a\cdot b=h_1k_1h_2k_2=h_1h_2(h_2^{-1}k_1h_2)k_2\in HK$

$a=hk, a^{-1}=(hk)^{-1}=k^{-1}h^{-1}=h^{-1}(hk^{-1}h^{-1})\in HK$

\subsection*{examples}
$S_3, H=\{(1),(12)\}, K=\{(1),(123),(132)\}, (12)(123)=(23)\in HK, (12)(132)=(13)\in HK$ so $HK=G$ and is therefore contained by G

$(\mathbb{Z},+)$, $H=a\mathbb{Z}, k=b\mathbb{Z}$, let $d=(a,b)$

claim: $a\mathbb{Z}+b\mathbb{Z}=d\mathbb{Z}$. clearly $a\mathbb{Z}\subseteq d\mathbb{Z}$, $b\mathbb{Z}\subseteq d\mathbb{Z}$. 

$a\mathbb{Z}+b\mathbb{Z}$ is the smallest subgroup that contains both $a\mathbb{Z}$ and $b\mathbb{Z}$. so $a\mathbb{Z}+b\mathbb{Z}\subseteq d\mathbb{Z}$.

$d=\gcd(a,b)$ so we can write $d=ma+nb$. let $\alpha\in d\mathbb{Z}$ and write $\alpha=dt, t\in \mathbb{Z}$ then $\alpha=dt=mat+nbt\in a\mathbb{Z}+b\mathbb{Z}$. so $d\mathbb{Z}\subseteq a\mathbb{Z}+b\mathbb{Z}$ 
\end{document}

