\documentclass[letterpaper]{article}

\usepackage{fullpage}
\usepackage{nopageno}
\usepackage{amsmath}
\usepackage{amssymb}
\allowdisplaybreaks

\newcommand{\abs}[1]{\left\lvert #1 \right\rvert}

\begin{document}
\title{Notes}
\date{December 1, 2014}
\maketitle
\section*{5.2 \#13}
if $\varphi:\mathbb{Z}+\mathbb{Z}\to\mathbb{Z}$ is a ring homomorphism then for every $m,n\in\mathbb{Z}$ we have $\varphi(m,n)=\varphi((m,0)+(0,n))=\varphi(m,0)+\varphi(0,n)=\varphi(\underbrace{1+1+\dots+1}_m,0)+\varphi(0,\underbrace{1+1+\dots+1}_n)=\underbrace{\varphi(1,0)+\dots\varphi(1,0)}_m+\underbrace{\varphi(0,1)+\dots\varphi(0,1)}_n=m\varphi(1,0)+n\varphi(0,1)$

now $\alpha=\varphi(1,0)$ and $\beta=\varphi(0,1)$ and $\alpha+\beta=\varphi(1,1)=1$ and so $\alpha+\beta=1$

also $\alpha\beta=\varphi(0,0)=0$.

\begin{enumerate}
\item
$\alpha=0$ then $\beta=1$ and so $\varphi(m,n)=m\cdot0+n\cdot 1=n$ which is a ring homomorphism
\item
$\beta=0$ then $\alpha=1$ and so $\varphi(m,n)=m$ which is a ring homomorphism.
\end{enumerate}

we start with a homomorphism, check all possible outputs and check that all outputs are homomorphisms.
\section*{last time}
$\varphi:R\to S$ is a ring hom

$\ker\varphi=x\in R:\varphi(x)=0$

$R/\ker\varphi=\{[x]:x\in R\}$

$x\sim y\Leftrightarrow x-y\in \ker\varphi$

$\Leftrightarrow\varphi(x)=\varphi(y)$

\section*{5.3}
\subsection*{definition}
given a comm ring $R$ and a non-empty subset $I\subseteq R$ we say that $I$ is an {\bfseries ideal} of(in) $R$ if
\begin{enumerate}
\item
for every two elements $x,y\in I$ we have $x+y,x-y\in I$ in particular $0\in I$.
\item
for every $r\in R$ and every $x\in I$ we have $rx\in I$

\end{enumerate}

\subsection*{examples}
every commutative ring has at least two ideals (unless $0=1$)

$\{0\}$ is an ideal of $R$

$R$ is an ideal of $R$.

if these are the only possible ideals then $R$ is a field

\subsection*{thrm}
let $R$ be a commutative ring. then $R$ is a field if and only if $\{0\}$ and $R$ are the only ideals of $R$.

\subsubsection*{proof}
$\Rightarrow$ assume that $R$ is a field, and let $I$ be an ideal of $R$. if $I=\{0\}$ we are done so assome $I\ne \{0\}$. we want to prove that $I=R$. let $x\ne0\in I$. because $R$ is a field then $x^{-1}\in R$ and $x^{-1}x\in I$ because $x\in I$ and so $1\in I$ and $r=r\cdot 1\in I$ and so $I=R$

$\Leftarrow$
Take $a\ne 0\in R$. we want to prove that $a$ is invertible.
$I=\{ra:r\in R\}\subseteq R$

claim: $I$ is an ideal of $R$. Now we have an ideal different from zero because $a\in I$ therefor $I=R$ and so $1\in I$. $\exists r\in R$ such that $ra=1$ and so $r=a^{-1}$.

\subsubsection*{observation}
\begin{enumerate}
\item
given $a\in R$ then $\{ra:r\in R\}$ is an ideal of $R$ denoted $Ra$ or $(a)$ or $aR$. In fact this is the smallest ideal of $R$ that contains $a$.

we call this the ideal generated by $a$.
\item
$1\in I$ iff $I=R$.
\end{enumerate}
\subsection*{definition}
for $R$ commutative ring, we say the $R$ is a {\bfseries principle ideal} if every ideal of $R$ is principal. that is every  ideal of $R$ is of the form $(a)$

\subsection*{definition}
we say that $R$ is a {\bfseries principal ideal domain} (PID) if $R$ is an integral domain and a principal ideal ring.

\subsection*{example}
fields are always principal ideal rings generated by $1$ (and PID).

$\mathbb{Z}$ is a PID.
\subsection*{example 5.3.1}
let $I\in \mathbb{Z}$ be an ideal different from 0. Let $a\in I$ be the smallest positive integer in $I$. (note that there is an element not zero in $I$ and so if there is a negative in $I$ then it's additive inverse is in $I$.

$\supseteq$

$a\in I\to ra\in I$

$\subseteq$

pick some $x\in I$ $x=qa+r$ where $0\le  r<0$ then $r=x+-q(a)$ and because $a\in I$ then $-qa\in I$ and $r\in I$ and so $r=0$ and so $x\in (a)$

\section*{examples}
let $K$ be a field then $K[x]$ is a PID.

let $I\subseteq K[x]$ be a nonzero ideal. let $q(x)\in I$ be a non-zero polynomial of minimal degree

claim:$I=(q(x))$
\subsubsection*{proof}
$q(x)K[x]\subseteq I$ is clear

let $f(x)\in I$. $f(x)=b(x)q(x)+r(x)$ where $r(x)=0$ or $\deg r(x)<\deg q(x)$

but $r(x)=f(x)+(-b(x)q(x))\in I$ of course $q(x)$ has minimal degree and so by this choice we must have $r(x)=0$ and so $f(x)$ is a multiple of $q(x)$ and we are done.
\end{document}



