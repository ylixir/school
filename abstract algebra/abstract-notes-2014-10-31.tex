\documentclass[letterpaper]{article}

\usepackage{fullpage}
\usepackage{nopageno}
\usepackage{amsmath}
\usepackage{amssymb}
\allowdisplaybreaks

\newcommand{\abs}[1]{\left\lvert #1 \right\rvert}

\begin{document}
\title{Notes}
\date{October 31, 2014}
\maketitle
\section*{4.1 \#9,12,13}
\subsection*{9}
$(a^{-1})^{-1}=e(a^{-1})^{-1}=a(a^{-1}(a^{-1})^{-1})=ae=a$

Note that $-e\cdot-e=e$
$(-a)^{-1}=(e\cdot -a)^{-1}=-e\cdot a)^{-1}=(-e)^{-1}a^{-1}=-ea^{-1}=e\cdot-a^{-1}=-a^{-1}$

\section*{last time}
examples always keep in mind, $\mathbb{R}\mathbb{Q} \mathbb{C}\mathbb{Z}_p$

$\mathbb{Z}_n$ is a field iff $n$ is prime

$\to$ n not prime take $m|n, m<n$ then $[m]$ is not invertible so $n$ is prime 


\subsection*{def}
we say that for polynomials $g(x)|f(x)\in K[x]$ if $\exists h(x)$ such that $f(x)=g(x)h(x)$

for example $(x+1)|(x^2-1)$ in $\mathbb{R}[x]$. $(x+1)\not|(x^2+1)$ in $\mathbb{R}[x]$ but it does in $\mathbb{Z}_2[x]$: $(x+1)(x+1)=x^2+x+x+1=x^2+x(1+1)+1=x^2+x(0)+1=x^2+1$

in $\mathbb{Z}_p$ $(a+b)^p=a^p+b^p$.

\subsection*{thm}
if $K$ is a field and $c\in K$ and $f(x)\in K[x]$ then there exists a unique $g(x)\in K[x]$ such that $f(x)=g(x)(x-c)+f(c)$.
\subsubsection*{proof}
claim $x-c|f(x)-f(c)$.
$f(x)=a_mx^m+\dots+a_1x+a_0$
$f(c)=a_mc^m+\dots+a_1c+a_0$
$f(x)-f(c)=a_m(x-c^m)+\dots+a_1(x-c)$

$x^t-c^t=(x-c)(x^{t-1}+x^{t-2}c+x^{t-3}c^2+\dots+xc^{t-2}+c^{t-1}$

and so $f(x)-f(c)=g(x)(x-c)\to f(x)=g(x)(x-c)+f(c)$.

now assume we have $g'(x)$ and $g(x)$ that satisfy then $g(x)(x-c)-f(c)=g'(x)(x-c)-f(c)\to (x-c)(g(x)-g'(x))=0$. $x-c$ has a coefficient of 1 and so is not zero so $g(x)-g'(x)=0$ and is unique.

\subsection*{def}
$c\in K$ is called a root of $f(x)\in K[x]$ if $f(c)=0$.

$c$ is a root of $f(x)$ iff $x-c$ divides $f(x)$

$\to$ assume $c$ is root, then $f(c)=0$. by previous theorem $\exists q(x)\in K[x]$ such that $f(x)=q(x)(x-c)+f(c)=q(x)(x-c)$ so $(x-c)$ divides $f(x)$

$\leftarrow$ assume $(x-c)$ divides $f(x)$. then $f(x)=h(x)(x-c)\to f(c)=h(c)(c-c)=h(c)\cdot 0=0$.

\subsubsection*{corollary}
$f(x)\in K[x]$, $\deg f=n$. then $f(x)$ has at most $n$ distinct roots. (assuming non-zero polynomial)

induction  on n.

$n=0$ then $f(x)=c\ne 0$. $n=1$ then $f(x)=a_1x+a_0$, $a_1\ne 0$.

assume $c\ne d$ are solutions. then $f(x)=(x-c)q(x)$ and $f(d)=(d-c)q(d)$. note that $(d-c)\ne 0$ then $q(d)=0$ and then 

now take a polynomial of degree $n-1$ that has $n-1$. if it has no roots, we are done. 
\end{document}


