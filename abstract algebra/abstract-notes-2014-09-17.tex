\documentclass[letterpaper]{article}

\usepackage{fullpage}
\usepackage{nopageno}
\usepackage{amsmath}
\usepackage{amssymb}
\allowdisplaybreaks

\newcommand{\abs}[1]{\left\lvert #1 \right\rvert}

\begin{document}
\title{Notes}
\date{September 17, 2014}
\maketitle
\section*{assignment}
Section 1.4: \# 17, 19, 20, 23, 24, 27. 
\renewcommand{\labelenumi}{2.\arabic{enumi}}
\renewcommand{\labelenumii}{\arabic{enumii}.}
\renewcommand{\labelenumiii}{(\alph{enumiii})}
\begin{enumerate}
\setcounter{enumi}{2}
\item
  \begin{enumerate}
  \setcounter{enumii}{4}
  \item
  
  \item
  (1),(12),(13),(14),(23),(24),(34),(123),(124),(134),(234),(132),(243),(142),(143),(1234)
  \end{enumerate}
\end{enumerate}

\section*{2.3.5 theorem}
sketch of a proof
\begin{align*}
  \tau&=\left(
  \begin{aligned}
    1&\quad2&\quad3&\quad\dots& n\\
    \sigma(1)&\quad\sigma(2)&\quad\sigma(3)&\quad\dots&\sigma(n)
  \end{aligned}
  \right)
\end{align*}
\section*{example}
\begin{align*}
  \sigma&=\left(
  \begin{aligned}
    &1&&\quad2&&\quad3&&\quad4&&\quad5&&\quad6&&7\\
    &7&&\quad2&&\quad6&&\quad3&&\quad1&&\quad5&&4\\
  \end{aligned}
  \right)=(1,7,4,3,6,5)
\end{align*}
definition: let $\sigma\in S$ the least positive integer $m$ such that $\sigma^m=(1)$ is called the order of $\sigma$
\subsubsection*{example}
$\sigma=(123)(45)\to(123)^3(45)^2=\sigma^6$
\subsubsection*{observation for 2.3.8 proof}
$(a1a2a3\dots ak)^i=(1)$ and $(b1b2b3\dots bk)^i=1$ because they are disjoint and applying the a permutions don't change the b elements and vice versa
\end{document}

