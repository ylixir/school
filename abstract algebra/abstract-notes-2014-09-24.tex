\documentclass[letterpaper]{article}

\usepackage{fullpage}
\usepackage{nopageno}
\usepackage{amsmath}
\usepackage{amssymb}
\allowdisplaybreaks

\newcommand{\abs}[1]{\left\lvert #1 \right\rvert}

\begin{document}
\title{Notes}
\date{September 24, 2014}
\maketitle
\section*{assignment}
assignment 3.1 no 23, 3.2no 3,6,7

if $(G,\cdot), a\in G$ then the cyclic subgroup generated by a is $<a>=\{\dots,a^{-2},a^{-1},e,a,a^2,a^3,\dots\}\subseteq G$.

$<a>$ is a subgroup of $G$ and if $H$ is a subgroup of $G$ and $a\in H$ then $<a>\subseteq H$, hence $<a>$ is the smallest subgroup of $G$

example: $(\mathbb{Z},+)\to <n>=n\mathbb{Z}$

$<1>=<-1>=\mathbb{Z}$

we say that the group $G$ is cyclic if there exists $a\in G$ such that $<a>=G$. So $(\mathbb{Z},+)$ is cyclic because $<1>=\mathbb{Z}$

another example $(\mathbb{Z}_n,+)$, $<[1]>=\mathbb{Z}_n$.

\section*{definition}
if there exists $n>0, n\in\mathbb{Z}$ such that $a^n=e$ we say that a has finite order and $\text{ord}(a)=\min\{n|n>0,n\in\mathbb{Z},a^n=e\}$. otherwise it has infinite order 

\subsection*{proposition}
if $G$ is a finite group, $a\in G$ then $a$ has finite order. $\{e,a,a^2,a^3,\dots\}\in G$. Since $G$ is finite, we have some m,n where $a^m=a^n$ wlog $m>n$, $a^ma^-n=a^na^-n=e=a^{m-n}$
\section*{examples}
$(\mathbb{Q}^*,\cdot)$, $\text{ord}(-1)=2, \text{ord}(1)=1, \text{ord}(2)=\infty$
\subsubsection*{proof for proposition3.2.8ii}
use division algorithm, $k=ord(a)\cdot q+r$ with $0\le r<ord(a)$

then $e=a^k=a^{ord(a)q+r=[a^{ord(a)}]^{q}}a^r=a^r$

excercise, complete this proof
\end{document}

