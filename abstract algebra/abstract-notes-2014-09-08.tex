\documentclass[letterpaper]{article}

\usepackage{fullpage}
\usepackage{nopageno}
\usepackage{amsmath}
\usepackage{amssymb}
\allowdisplaybreaks

\newcommand{\abs}[1]{\left\lvert #1 \right\rvert}

\begin{document}
\title{Notes}
\date{September 8, 2014}
\maketitle
\section*{exercises}
second part of chinese remainder theorem
Section 1.3: exercises \# 4, 6, 12, 18, 20, 24.
\begin{enumerate}
\setcounter{enumi}{3}
\item
\begin{align*}
  20x\equiv12\mod 72\\
  \gcd(20,12)=4\\
  4|12\\
  ax=b+qn\\
  20x=12+q72
  20=4a_1,12=4b_1,72=4m\\
  a_1x=b_1=qm\\
  a_1x\equiv b_1\mod m\\
  5x\equiv 3\mod 18\\
  ca_1\equiv1\mod m\\
  c5\equiv1\mod 18\\
  55=18*3+1\\
\end{align*}
\setcounter{enumi}{23}
\item
claim:remainder of integer when divided by 9. 

proof:
\begin{align*}
  n_0\equiv r\mod 9\\
  n_0=10^na_n+10^{n-1}a_{n-1}+\dots+a_0\\
  a\equiv b\mod n\\
  c\equiv d\mod n\\
  ac\equiv bd\mod n\\
  a\equiv b\mod n\to a^k\equiv b^k\mod n\\
  10\equiv1\mod 9\\
  10^k\equiv 1\mod 9\\
  n_0\equiv a_n+a_{n-1}+\dots+a_0\mod 9
\end{align*}
similar to 25
\end{enumerate}
\section*{section 2.1}
$f:S\longrightarrow T$ and $S$ is domain, $T$ is codomain.

$f':S'\longrightarrow T'$

$f=f'\Leftrightarrow S=S',T=T'$ and $f(x)=f'(x)\forall x\in S$

The image of $f$ is $f(s)=\{f(t)|x\in S\}$

\section*{example}
\begin{align*}
  f:R\to R\\
  f(x)=x^2\\
  \text{Im} f=f(R)=[0,\infty)
\end{align*}
one to one (injective functions) $f:S\to T$ $f(x_1)=f(x_2)\Rightarrow x_1=x_2$

onto (surjective) $f:S\to T$ $f(S)=T$

one to one correspondences (bijective) satisfy both  injective and surjective (one-to-one and onto)

inverse function $f:S\to  T$ $f^{-1}:T\to S$. $f(f^{-1}(x))=x\forall x\in T$ and $f^{-1}(f(x))=x\forall x\in S$. defined iff $f$ is bijective
\section*{section 2.2 equivalence relations}
$S$ set

an equivalence relation is a subset $R\subseteq S\times S$ with the properties
\begin{enumerate}
\item
for all $x\in S$ we have that $(x,x)\in R$
\item
$\forall x,y\in S$ if $(x,y)\in R$ then $(y,x)\in R$
\item
$\forall x,y,z\in S$ if $(x,y)\in R$ and $(y,z)\in R$ then $(x,z)\in R$
\end{enumerate}
\subsection*{notation}
we write $a\sim{ }b$ to indicate that $a,b\in R$
\subsection*{example}
\begin{align*}
  S=\mathbb{Z}\\
  n\in\mathbb{Z}\\
  n>0\\
  \intertext{we say  that $x\sim y$ iff}
  x\equiv y\mod n
\end{align*}
\subsection*{example}
\begin{align*}
  S=\mathbb{R}
\end{align*}
$x\sim y$ iff $x+y\ge 0$. is this equivalence? no $x+x$ might be negative
\subsection*{example}
\begin{align*}
  S=[0,\infty)
\end{align*}
$x\sim y$ iff $x+y\ge 0$. is this equivalence? yes

\subsubsection*{note}
equality  is always equivalence relation, the trivial case

\section*{equivalence class}
$S$ is a set and $~$ is and equivalence relation. let $a\in S$, $[a]=\{x\in S|a\sim x\}$ where $[a]$ is equivalence class of $a$. $S/\sim$ is the set of all equivalence classes
\subsection*{example}
$S=\mathbb{Z}$ and $\sim$ is the congruence modulo n, then the set $\mathbb{Z}/\sim$ has $n$ elements: $[0],[1],\dots,[n-1]$
\subsection*{observation}
\begin{enumerate}
\item

let $\sim$ be an equivalence relation on the set $S$. take two elements $a,b\in S$ then $a\sim b\Leftrightarrow [a]=[b]$
\item
if $a\not\sim b$ then $[a]\cap[b]=\emptyset$
\item
$S=\cup_{a\in S}[a]$. each element of S belongs to exactly one equivalence class. the equivalence classes form a partition of S.
\subsection*{question}
if we have a partition of S, can we ``naturally'' define an equivalence on S? yes, two way relation $x\sim y$ iff $x,y$ belong to the same subset of the partition.
\end{enumerate}
\subsection*{observation}
let $\sim$ be an equiv relation on $S$. then we can define a function $\pi:S\to S/\sim$. $\pi(x)=[x]$. aside (call $S/\sim$ factor set from now on). is this function surjective? $S/\sim$ is the set of all possible equiv classes, so $\pi$ (the natural projection) is always surjective. it is injective iff every equiv classes has one element (itself) and is therefore the trivial equality relation.
\end{document}
