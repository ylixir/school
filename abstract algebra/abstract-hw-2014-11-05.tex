\documentclass[letterpaper]{article}

\usepackage{fullpage}
\usepackage{nopageno}
\usepackage{amsmath}
\usepackage{amssymb}
\allowdisplaybreaks

\newcommand{\abs}[1]{\left\lvert #1 \right\rvert}

\begin{document}
\title{Homework}
\date{November 5, 2014}
\author{Jon Allen}
\maketitle
Section 3.8: 7, 12, 19, 22
\renewcommand{\labelenumi}{3.\arabic{enumi}}
\renewcommand{\labelenumii}{\arabic{enumii}.}
\renewcommand{\labelenumiii}{(\alph{enumiii})}
\begin{enumerate}
\setcounter{enumi}{7}
\item
  \begin{enumerate}
  \setcounter{enumii}{6}
  %3.8 7
  \item
    Let $H$ be a subggroup of $G$, and let $a\in G$. Show that $aHa^{-1}$ is a subgroup of $G$ that is isomorphic to $H$.

    $e\in H\to aea^{-1}=e\in aHa^{-1}$. So $H$ is non-empty. Let $h,g\in H$.
    \begin{align*}
      (aga^{-1})^{-1}&=((ag)(a^{-1}))^{-1}=(a^{-1})^{-1}(ag)^{-1}=ag^{-1}a^{-1}\\
      (aha^{-1})(ag^{-1}a^{-1})&=aheg^{-1}a^{-1}=ahg^{-1}a^{-1}
    \end{align*}
    Because $g\in H$ we know that $g^{-1}\in H$ and then $hg^{-1}\in H$ so $ahg^{-1}a^{-1}\in aHa^{-1}$.
    \[aha^{-1}aga^{-1}=ahga^{-1}\]
    So it's homeomorphic. Now if we assume that $aha^{-1}=ah'a^{-1}$ cancelling left and right easily  shows that $h=h'$ and so it is injective. And obviously any $aha^{-1}\in H$ has a preimage of $h\in H$ and so we have surjectivity falling out of the definition of the group. And that is the final condition for it to be an isomorphism.
  \setcounter{enumii}{11}
  %3.8 12
  \item
    Let $H$ and $K$ be normal subgroups of $G$ such that $H\cap K=\langle e\rangle$. Show that $hk=kh$ for all $h\in H$ and $k\in K$.

    Take any $k\in K$ and $h\in H$. Because for any $g\in G$ we know that $ghg^{-1}\in H$. We also know that $gkg^{-1}\in K$.
    So because $k\in G$ we have $h=khk^{-1}$ and then $hk=khk^{-1}k=kh$
  \setcounter{enumii}{18}
  %3.8 19
  \item
    Show that $(\mathbb{Z}\times \mathbb{Z})/\langle (0,1)\rangle$ is an infinite cyclic group.

    Notice that $\langle (0,1)\rangle=\{0\}\times \mathbb{Z}$ because $\langle 1\rangle=\mathbb{Z}$ and $\langle 0\rangle=\{0\}$.
    If we take $\{0\}\times \mathbb{Z}+(a,0)$ for some $a\in \mathbb{Z}$ then we have the set $\{a\}\times \mathbb{Z}$. Also note that $\bigcup\limits_{a\in \mathbb{Z}}\{a\}\times \mathbb{Z}=\mathbb{Z}\times\mathbb{Z}$ and $\{a\}\times\mathbb{Z}\cap\{b\}\times \mathbb{Z}=\emptyset$ for all $a\ne b$. This means that $\{a\}\times\mathbb{Z}$ partitions $\mathbb{Z}\times\mathbb{Z}$ and cosets of $\langle(0,1)\rangle$ have the form $(a,0)+\langle(0,1)\rangle$. And $(a,0)\in \langle(1,0)\rangle$. And so $(\mathbb{Z}\times\mathbb{Z})/\langle(0,1)\rangle$ is generated by  $(1,0)+\langle(0,1)\rangle$. This makes the group cyclic. And obviously the number of elements in the group is the number of ways we can pick our $(a,0)$ which is the number of ways we can pick an integer. And so we have an infinite cyclic group. 
  \setcounter{enumii}{21}
  %3.8 22
  \item
    Show that $\mathbb{R}^\times/\langle-1\rangle$ is isomorphic to the group of positive real numbers under multiplication.

    Note that $\langle -1\rangle=\{-1,1\}$.
    And so the factor group has the form $\{-a,a\}$ where $a\in \mathbb{R}$.
    Notice also that if $b=-a$ then $\{-a,a\}=\{b,-b\}$ and so $\mathbb{R}^{\times}/\langle -1\rangle=\{\{-a,a\}:a\in \mathbb{R}^+\}$.
    Let us define $\phi:\mathbb{R}^{\times}/\langle -1\rangle\to\mathbb{R}$ as $\phi:\{-a,a\}\to a$.
    \[\phi(\{-a,a\}\cdot\{-b,b\})=\phi(\{-a\cdot-b,-a\cdot b,a\cdot -b,a\cdot b\})=\phi(\{-ab,ab\})=ab=\phi(\{-a,a\})\cdot\phi(\{-b,b\})\]
    Clearly we have a homeomorphism.
    I think surjectivity is clear, any $a\in \mathbb{R}^{+}$ has a preimage $\{-a,a\}$ in our factor set. Now if $\phi(\{-a,a\})=\phi(\{-b,b\})$ then $a=b$. If $a=b$ then $\{-a,a\}=\{-b,b\}$ and so we have injectivity. Thus our factor set is isomorphic to the positive integers.
  \end{enumerate}
\end{enumerate}
\end{document}
