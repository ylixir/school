\documentclass[letterpaper]{article}

\usepackage{fullpage}
\usepackage{nopageno}
\usepackage{amsmath}
\usepackage{amssymb}
\allowdisplaybreaks

\newcommand{\abs}[1]{\left\lvert #1 \right\rvert}

\begin{document}
\title{Homework}
\date{September 24, 2014}
\author{Jon Allen}
\maketitle
Section 1.4: \# 9, 20.
Section 2.2: \# 7, 9.
\renewcommand{\labelenumi}{1.\arabic{enumi}}
\renewcommand{\labelenumii}{\arabic{enumii}.}
\renewcommand{\labelenumiii}{(\alph{enumiii})}
\begin{enumerate}
\setcounter{enumi}{3}
\item
  \begin{enumerate}
  \setcounter{enumii}{8}
  %1.4 9
  \item
    Let $(a,n)=1$. The smallest positive integer $k$ such that $a^k\equiv 1 \mod n$ is called the {\bfseries multiplicative order} of $[a]$ in $\mathbb{Z}_n^\times$.
    \begin{enumerate}
    \item
      Find the multiplicative orders of $[5]$ and $[7]$ in $\mathbb{Z}_{16}^\times$.

      \begin{align*}
        [5]&=[5]&[25]&=[9]&[45]&=[13]\\
        [65]&=[1]\\
        [7]&=[7]&[49]&=[1]
      \end{align*}
      So the multiplicative order of $[5]$ in $\mathbb{Z}_{16}^\times$ is $4$.
      The multiplicative order of $[7]$ in $\mathbb{Z}_{16}^\times$ is $2$.
    \item
      Find the multiplicative orders of $[2]$ and $[5]$ in $\mathbb{Z}_{17}^\times$.
      \begin{align*}
        [2]&=[2]&[4]&=[4]&[8]&=[8]&[16]&=[16]&4\\
        [32]&=[15]&[30]&=[13]&[26]&=[9]&[18]&=[1]&8\\
        [5]&=[5]&[25]&=[8]&[40]&=[6]&[30]&=[13]&4\\
        [65]&=[14]&[70]&=[2]&[10]&=[10]&[50]&=[16]&8\\
        [80]&=[12]&[60]&=[9]&[45]&=[11]&[55]&=[4]&12\\
        [20]&=[3]&[15]&=[15]&[75]&=[7]&[35]&=[1]&16
      \end{align*}
      So the multiplicative order of $[2]$ in $\mathbb{Z}_{17}^\times$ is $8$.
      The multiplicative order of $[5]$ in $\mathbb{Z}_{17}^\times$ is $16$.
    \end{enumerate}
  \setcounter{enumii}{19}
  %1.4 20
  \item
    Show that $\varphi(1)+\varphi(p)+\dots+\varphi(p^{\alpha})=p^{\alpha}$ for any prime number $p$ and any positive integer $\alpha$
    \begin{align*}
      \varphi(1)+\varphi(p)+\dots+\varphi(p^{\alpha})
      &=\varphi(1)+\sum\limits_{n=1}^{\alpha}{\varphi(p^\alpha)}\\
      &=1+\sum\limits_{n=1}^{\alpha}{p^n\left(1-\frac{1}{p}\right)}\\
      &=1+\sum\limits_{n=1}^{\alpha}{\left(p^n-p^{n-1}\right)}\\
      &=1+\sum\limits_{n=1}^{\alpha}{p^n}-\sum\limits_{n=0}^{\alpha-1}{p^n}\\
      &=1+p^\alpha-p^0\\
      &=p^{\alpha}
    \end{align*}
  \end{enumerate}
\renewcommand{\labelenumi}{2.\arabic{enumi}}
\setcounter{enumi}{1}
\item
  \begin{enumerate}
  \setcounter{enumii}{6}
  %2.2 7
  \item
    Define an equivalence relation on the set $\mathbb{R}$ that partitions the real line into subsets of length $1$.

    We define $x\sim y$ for all $x,y\in\mathbb{R}$ if $\lfloor x\rfloor=\lfloor y\rfloor$. For all $x\in\mathbb{R}$ we define $\lfloor x\rfloor=n$ where $n\in\mathbb{Z}$ and $x-1<n\le x$. It is trivial to see that this relation satisfies reflexivity, symmetry, and transitivity. Furthermore, because the relation partitions the elements of $\mathbb{R}$ into classes that span $\mathbb{Z}$, it partitions the real line into subsets of length one (the distance between two integers is a multiple of one).
  \setcounter{enumii}{8}
  %2.2 9
  \item
    Let $S$ be a set. A subset $R\subseteq S\times S$ is called a {\bfseries circular relation} if (i) for each $a\in S, (a,a)\in R$ and (ii) for each $a,b,c\in S$, if $(a,b)\in R$ and $(b,c)\in R$, then $(c,a)\in R$. Show that any circular relation must be an equivalence relation.

  \subsubsection*{proof}
  First we note that reflexivity is given. Let's choose some $(b,a)\in R$. Because we are given reflexivity, we know that $(a,a)\in R$. So then by the definition of the circular relation we see that because we have $(b,a)\in R$ and $(a,a)\in R$ then we must have $(a,b)\in R$. And we see that symmetry is preserved in this relation. And finally, lets take $a,b,c\in S$ where $(a,b)\in R$ and $(b,c)\in R$. We are told that $(c,a)\in R$ and we have shown that symmetry holds, so we know that $(a,c)$ is also in $R$. And that gives us transitivity. Three out of three conditions met. We are done here.
  $\Box$
  \end{enumerate}
\end{enumerate}
\end{document}
