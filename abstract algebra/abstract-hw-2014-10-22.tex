\documentclass[letterpaper]{article}

\usepackage{fullpage}
\usepackage{nopageno}
\usepackage{amsmath}
\usepackage{amssymb}
\allowdisplaybreaks

\newcommand{\abs}[1]{\left\lvert #1 \right\rvert}

\begin{document}
\title{Homework}
\date{October 22, 2014}
\author{Jon Allen}
\maketitle
Section 3.5: \#5, 15, 11
Section 3.6: \#7, 20.

\renewcommand{\labelenumi}{3.\arabic{enumi}}
\renewcommand{\labelenumii}{\arabic{enumii}.}
\renewcommand{\labelenumiii}{(\alph{enumiii})}
\begin{enumerate}
\setcounter{enumi}{4}
\item
  \begin{enumerate}
  \setcounter{enumii}{4}
  %3.5 #5
  \item
    Find the cyclic subgroup of $\mathbb{C}^\times$ generated by $(\sqrt{2}+\sqrt{2}i)/2$.

    \begin{align*}
      \frac{\sqrt{2}+\sqrt{2}i}{2}&=\frac{\sqrt{2}}{2}\left(1+i\right)\\
      \left(\frac{\sqrt{2}}{2}\left(1+i\right)\right)^2&=\frac{2}{4}2i=i&
      \left(\frac{\sqrt{2}}{2}\left(1+i\right)\right)^3&=\frac{\sqrt{2}}{2}\left(1+i\right)i=\frac{\sqrt{2}}{2}\left(i-1\right)\\
      \left(\frac{\sqrt{2}}{2}\left(1+i\right)\right)^4&=i^2=-1&
      \left(\frac{\sqrt{2}}{2}\left(1+i\right)\right)^5&=-\frac{\sqrt{2}}{2}\left(1+i\right)\\
      \left(\frac{\sqrt{2}}{2}\left(1+i\right)\right)^6&=i^3=-i&
      \left(\frac{\sqrt{2}}{2}\left(1+i\right)\right)^7&=-\frac{\sqrt{2}}{2}\left(i-1\right)=\frac{\sqrt{2}}{2}\left(1-i\right)\\
      \left(\frac{\sqrt{2}}{2}\left(1+i\right)\right)^8&=(-1)^2=1&
      \left(\frac{\sqrt{2}}{2}\left(1+i\right)\right)^9&=\frac{\sqrt{2}}{2}\left(1+i\right)\\
      \intertext{And to double check}
      \left(\frac{\sqrt{2}}{2}\left(1+i\right)\right)^{-1}&=\sqrt{2}\frac{1}{1+i}&
      \sqrt{2}\frac{1}{1+i}&=\sqrt{2}\frac{1-i}{(1+i)(1-i)}=\frac{\sqrt{2}}{2}\left(1-i\right)\\
      \left(\frac{\sqrt{2}}{2}\left(1+i\right)\right)^8&=\left(\frac{\sqrt{2}}{2}\left(1+i\right)\right)^0&
      \left(\frac{\sqrt{2}}{2}\left(1+i\right)\right)^7&=\left(\frac{\sqrt{2}}{2}\left(1+i\right)\right)^{-1}
    \end{align*}
    And so the generated group is:
    \[
    \langle(\sqrt{2}+\sqrt{2}i)/2\rangle=\{1,i,-1,-i,\frac{\sqrt{2}}{2}(1+i),i\frac{\sqrt{2}}{2}(1+i),-\frac{\sqrt{2}}{2}(1+i),-i\frac{\sqrt{2}}{2}(1+i)\}
    \]

  \setcounter{enumii}{10}
  %3.5 #11
  \item
    Which of the multiplicative groups $\mathbb{Z}_7^\times,\mathbb{Z}_{10}^\times,\mathbb{Z}_{12}^\times,\mathbb{Z}_{14}^\times$ are isomorphic?

    \begin{align*}
      \mathbb{Z}_7^\times&=\{[2^{\alpha_1}3^{\alpha_2}5^{\alpha_3}]_7:\alpha_1,\alpha_2,\alpha_3\in\mathbb{Z}\}\\
      5^6&=25\cdot5^4=4\cdot5^4=20\cdot5^3=6\cdot5^3=30\cdot5^2=2\cdot5^2=10\cdot5=3\cdot 5=15=1\\
      \mathbb{Z}_7^\times&=\{[\left(5^4\right)^{\alpha_1}\left(5^5\right)^{\alpha_2}5^{\alpha_3}]_7:\alpha_1,\alpha_2,\alpha_3\in\mathbb{Z}\}=\langle5\rangle\cong\mathbb{Z}_6\\
      \mathbb{Z}_{10}^\times&=\left\{\left[3^{\alpha_1}7^{\alpha_2}9^{\alpha_3}\right]_{10}:\alpha_1,\alpha_2,\alpha_3\in\mathbb{Z}\right\}\\
      3^4&=9\cdot3^2=27\cdot3=7\cdot3=21=1\\
      \mathbb{Z}_{10}^\times&=\{[3^{\alpha_1}\left(3^3\right)^{\alpha_2}\left(3^2\right)^{\alpha_3}]_7:\alpha_1,\alpha_2,\alpha_3\in\mathbb{Z}\}=\langle3\rangle\cong\mathbb{Z}_4\\
      \mathbb{Z}_{12}^\times&=\{5^{\alpha_1}7^{\alpha_2}11^{\alpha_3}:\alpha_1,\alpha_2,\alpha_3\in\mathbb{Z}\}\\
      5^2&=25=1\quad\quad\quad\;\;\,
      7^2=49=1\quad\quad\;\;\,
      11^2=121=1\\
      5\cdot7&=35=11\quad\quad
      7\cdot11=77=5\quad\quad
      5\cdot11=55=7\\
      Z_{12}^\times&=\{1,5\}\times\{1,7\}\cong\mathbb{Z}_2\times\mathbb{Z}_2\\
      Z_{14}^\times&=\{[3^{\alpha_1}5^{\alpha_2}9^{\alpha_3}11^{\alpha_4}13^{\alpha_5}\}\\
      3^6&=9\cdot3^4=27\cdot3^3=13\cdot3^3=39\cdot3^2=11\cdot3^2=33\cdot3=5\cdot3=15=1\\
      Z_{14}^\times&=\{[3^{\alpha_1}\left(3^5\right)^{\alpha_2}\left(3^2\right)^{\alpha_3}\left(3^4\right)^{\alpha_4}\left(3^3\right)^{\alpha_5}]:\alpha_1,\alpha_2,\alpha_3,\alpha_4,\alpha_5\in\mathbb{Z}\}=\langle3\rangle\cong\mathbb{Z}_6
    \end{align*}
    So $\mathbb{Z}_7^\times$ and $\mathbb{Z}_{14}^\times$ are isomorphic.
  \setcounter{enumii}{14}
  %3.5 #15
  \item
  \end{enumerate}
\item
  \begin{enumerate}
  \setcounter{enumii}{6}
  %3.5 #7
  \item
  \setcounter{enumii}{19}
  %3.5 #20
  \item
  \end{enumerate}
\end{enumerate}
\end{document}
