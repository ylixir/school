\documentclass[letterpaper]{article}

\usepackage{fullpage}
\usepackage{nopageno}
\usepackage{amsmath}
\usepackage{amssymb}
\allowdisplaybreaks

\newcommand{\abs}[1]{\left\lvert #1 \right\rvert}

\begin{document}
\title{Homework}
\date{October 22, 2014}
\author{Jon Allen}
\maketitle
Section 3.5: \#5, 15, 11
Section 3.6: \#7, 20.

\renewcommand{\labelenumi}{3.\arabic{enumi}}
\renewcommand{\labelenumii}{\arabic{enumii}.}
\renewcommand{\labelenumiii}{(\alph{enumiii})}
\begin{enumerate}
\setcounter{enumi}{4}
\item
  \begin{enumerate}
  \setcounter{enumii}{4}
  %3.5 #5
  \item
    Find the cyclic subgroup of $\mathbb{C}^\times$ generated by $(\sqrt{2}+\sqrt{2}i)/2$.

    \begin{align*}
      \frac{\sqrt{2}+\sqrt{2}i}{2}&=\frac{\sqrt{2}}{2}\left(1+i\right)\\
      \left(\frac{\sqrt{2}}{2}\left(1+i\right)\right)^2&=\frac{2}{4}2i=i&
      \left(\frac{\sqrt{2}}{2}\left(1+i\right)\right)^3&=\frac{\sqrt{2}}{2}\left(1+i\right)i=\frac{\sqrt{2}}{2}\left(i-1\right)\\
      \left(\frac{\sqrt{2}}{2}\left(1+i\right)\right)^4&=i^2=-1&
      \left(\frac{\sqrt{2}}{2}\left(1+i\right)\right)^5&=-\frac{\sqrt{2}}{2}\left(1+i\right)\\
      \left(\frac{\sqrt{2}}{2}\left(1+i\right)\right)^6&=i^3=-i&
      \left(\frac{\sqrt{2}}{2}\left(1+i\right)\right)^7&=-\frac{\sqrt{2}}{2}\left(i-1\right)=\frac{\sqrt{2}}{2}\left(1-i\right)\\
      \left(\frac{\sqrt{2}}{2}\left(1+i\right)\right)^8&=(-1)^2=1&
      \left(\frac{\sqrt{2}}{2}\left(1+i\right)\right)^9&=\frac{\sqrt{2}}{2}\left(1+i\right)\\
      \intertext{And to double check}
      \left(\frac{\sqrt{2}}{2}\left(1+i\right)\right)^{-1}&=\sqrt{2}\frac{1}{1+i}&
      \sqrt{2}\frac{1}{1+i}&=\sqrt{2}\frac{1-i}{(1+i)(1-i)}=\frac{\sqrt{2}}{2}\left(1-i\right)\\
      \left(\frac{\sqrt{2}}{2}\left(1+i\right)\right)^8&=\left(\frac{\sqrt{2}}{2}\left(1+i\right)\right)^0&
      \left(\frac{\sqrt{2}}{2}\left(1+i\right)\right)^7&=\left(\frac{\sqrt{2}}{2}\left(1+i\right)\right)^{-1}
    \end{align*}
    And so the generated group is:
    \[
    \langle(\sqrt{2}+\sqrt{2}i)/2\rangle=\{1,i,-1,-i,\frac{\sqrt{2}}{2}(1+i),i\frac{\sqrt{2}}{2}(1+i),-\frac{\sqrt{2}}{2}(1+i),-i\frac{\sqrt{2}}{2}(1+i)\}
    \]

  \setcounter{enumii}{10}
  %3.5 #11
  \item
    Which of the multiplicative groups $\mathbb{Z}_7^\times,\mathbb{Z}_{10}^\times,\mathbb{Z}_{12}^\times,\mathbb{Z}_{14}^\times$ are isomorphic?

    The multiplicative groups consist of powers of the elements of the original group that are relatively prime to $n$. The elements that aren't relatively prime can be represented as multiples of powers of relatively prime numbers and so are redundant.
    \begin{align*}
      \mathbb{Z}_7^\times&=\{[2^{\alpha_1}3^{\alpha_2}4^{\alpha_3}5^{\alpha_5}6^{\alpha_5}]_7:\alpha_1,\alpha_2,\alpha_3,\alpha_4,\alpha_5\in\mathbb{Z}\}\\
      5^6&=25\cdot5^4=4\cdot5^4=20\cdot5^3=6\cdot5^3=30\cdot5^2=2\cdot5^2=10\cdot5=3\cdot 5=15=1\\
      \mathbb{Z}_7^\times&=\{[\left(5^4\right)^{\alpha_1}\left(5^5\right)^{\alpha_2}\left(5^2\right)^{\alpha_3}5^{\alpha_4}\left(5^3\right)^{\alpha_5}]_7:\alpha_1,\alpha_2,\alpha_3,\alpha_4,\alpha_5\in\mathbb{Z}\}=\langle5\rangle\cong\mathbb{Z}_6\\
      \mathbb{Z}_{10}^\times&=\left\{\left[3^{\alpha_1}7^{\alpha_2}9^{\alpha_3}\right]_{10}:\alpha_1,\alpha_2,\alpha_3\in\mathbb{Z}\right\}\\
      3^4&=9\cdot3^2=27\cdot3=7\cdot3=21=1\\
      \mathbb{Z}_{10}^\times&=\{[3^{\alpha_1}\left(3^3\right)^{\alpha_2}\left(3^2\right)^{\alpha_3}]_7:\alpha_1,\alpha_2,\alpha_3\in\mathbb{Z}\}=\langle3\rangle\cong\mathbb{Z}_4\\
      \mathbb{Z}_{12}^\times&=\{5^{\alpha_1}7^{\alpha_2}11^{\alpha_3}:\alpha_1,\alpha_2,\alpha_3\in\mathbb{Z}\}\\
      5^2&=25=1\quad\quad\quad\;\;\,
      7^2=49=1\quad\quad\;\;\,
      11^2=121=1\\
      5\cdot7&=35=11\quad\quad
      7\cdot11=77=5\quad\quad
      5\cdot11=55=7\\
      Z_{12}^\times&=\{1,5\}\times\{1,7\}\cong\mathbb{Z}_2\times\mathbb{Z}_2\\
      Z_{14}^\times&=\{[3^{\alpha_1}5^{\alpha_2}9^{\alpha_3}11^{\alpha_4}13^{\alpha_5}]:\alpha_1,\alpha_2,\alpha_3,\alpha_4,\alpha_5\in\mathbb{Z}\}\\
      3^6&=9\cdot3^4=27\cdot3^3=13\cdot3^3=39\cdot3^2=11\cdot3^2=33\cdot3=5\cdot3=15=1\\
      Z_{14}^\times&=\{[3^{\alpha_1}\left(3^5\right)^{\alpha_2}\left(3^2\right)^{\alpha_3}\left(3^4\right)^{\alpha_4}\left(3^3\right)^{\alpha_5}]:\alpha_1,\alpha_2,\alpha_3,\alpha_4,\alpha_5\in\mathbb{Z}\}=\langle3\rangle\cong\mathbb{Z}_6
    \end{align*}
    So $\mathbb{Z}_7^\times$ and $\mathbb{Z}_{14}^\times$ are isomorphic.
  \setcounter{enumii}{14}
  %3.5 #15
  \item
    Prove that any finite cyclic group with more than two elements has at least two different generators.

    Lets call our group $G$ with generator $a$. So $G=\langle a\rangle$. Furthermore the order of $\langle a\rangle$ is finite, and more than two, and so $\text{ord}(\langle a\rangle)=\text{ord}(a)=n$ where $2<n\in \mathbb{N}$. That is to say $G=\{e,a,a^2,\dots,a^{n-1}\}$.

    Now because $n>2$ we know that $1\ne n-1$ and because the order of $a$ is $n$ we know that $a\ne a^{n-1}\ne e$. So lets see what happens if we apply the group operation $n-1$ times to the element $a^{n-1}$.
    \[\left(a^{n-1}\right)^{n-1}=
      a^{(n-1)^2}=a^{n^2-2n+1}=
      \left(a^n\right)^n\left(a^n\right)^{-2}a^1=e^ne^{-2}a=a
    \]
    And so $a^{n-1}$ generates $a$ and $a$ generates $G$. The immediate consequence of this fact is that $a^{n-1}$ generates $G$.

    {\em Note:} I had some problems with this because all I have shown is that if $a$ generates a group, then so does it's inverse, which seems kind of too obvious and maybe even the same statement. But then I considered the smallest group which fits the definition: $G=\{e,a,a^2\}$. Now $a^{-1}=a^2$ and obviously $e$ can not generate $G$, so the only other possible element to generate $G$ is $a^{-1}$. So that's what I went with in my proof.

%    Let's assume $n$ is odd, then there is some $k\in \mathbb{N}$ such that $n=2k+1$ by definition. Now, because there are more than two elements in $G$ we know that $G$ contains at least $e,a$ and crucially $a^2$. Lets take this element and apply our group operation to it $k+1$ times.
%  \[(a^2)^{k+1}=a^{2k+2}=a^{(2k+1)+1}=a^{2k+1}a=a^na=a\]
%  Of course we can generate all the elements in $G$ by taking $a$ to the powers $0$ through $n-1$ and because $(a^2)^{k+1}=a$ it follows that we can generate $G$ from powers of $a^2$ also.

%  Now lets assume $n$ is even, then there is some $k\in \mathbb{N}$ such that $n=2k$ by definition. Well we know that $e,a$ and $a^2$ are in $G$. But that is an odd number of elements so then $a^3\in G$ for sure. Now $n>2$ and so $a^{n-2}$ is also in our group and $a^{n-2}\ne e$. And closure tells us that if we perform our group operation on these elements, then the result is also in our group.
  \end{enumerate}
\item
  \begin{enumerate}
  \setcounter{enumii}{6}
  %3.6 #7
  \item
    Find the order of each element of $D_6$.

    \begin{align*}
      e&=e&
      (a^1)^6&=e&
      (a^2)^3&=e\\
      (a^3)^2&=e&
      (a^4)^3&=e&
      (a^5)^6&=e\\
      b^2&=e&
      (ba)^2&=(baa^{-1}b)=e\\
      \intertext{by assumption}ba^n&=a^{-n}b\\
      \intertext{by induction}a^{-1}ba^n&=a^{-1}a^{-n}b&
      baa^n&=ba^{n+1}=a^{-(n+1)}b\\
      \intertext{and then}(ba^n)^2&=ba^na^{-n}b=e
    \end{align*}
    And so we have $\text{ord}(e)=1$ (duh), $\text{ord}(a)=\text{ord}(a^5)=6, \text{ord}(a^2)=\text{ord}(a^4)=3$ and $\text{ord}(ba^k)=2\quad\forall 0\le k\le 5\in\mathbb{Z}$
  \setcounter{enumii}{19}
  %3.6 #20
  \item
    Let the dihedral group $D_n$ be given by elements $a$ of order $n$ and $b$ of order 2, where $ba=a^{-1}b$. Find the smallest subgroup of $D_n$ that contains $a^2$ and $b$.

    {\em Hint: }Consider two cases, depending on whether $n$ is odd or even.

    The group specified is $\langle a^2\rangle\times\langle b\rangle$. We know that $\text{ord}(a)=n$ and so if $k<n$ then $a^k\ne e$.

    Assume $n$ is even. Then $(a^2)^{\frac{n}{2}}=e$ and $\forall0<k<\frac{n}{2}$ we know that $(a^2)^{k}\ne e$ and so the subgroup we are looking for consists of $\{a^{2j}b^k:0\le j\le\frac{n}{2},0\le k\le1\text{ and }j,k\in\mathbb{Z}\}$

    Now lets assume $n$ is odd. Then $n=2k+1$ and for all $0<j\le k$ we know that $a^{2j}\ne e$. And further $a^{2(k+1)}=a^{2k+1}a=a$. That is to say $\langle a^2\rangle=\langle a\rangle$. And so it follows that the subgroup we are looking for is $\langle a\rangle\times\langle b\rangle$ which is the original group $D_n$.
  \end{enumerate}
\end{enumerate}
\end{document}
