\documentclass[letterpaper]{article}

\usepackage{fullpage}
\usepackage{nopageno}
\usepackage{amsmath}
\usepackage{amssymb}
\allowdisplaybreaks

\newcommand{\abs}[1]{\left\lvert #1 \right\rvert}

\begin{document}
\title{Notes}
\date{November 3, 2014}
\maketitle
\section*{Section 4.1: 3, 5(d), 6, 19.}
\subsection*{3}
$f(x)=x^{3}+x^2-2x+1\to f(x)=q(x)(x-1)+f(1)$
\subsection*{5d}
$x^3+2x+3;c=2;F=\mathbb{Z}_5$
\begin{align*}
  x^3+2x+3&=q(x)(x-c)+f(c)\\
  &=q(x)(x-2)+f(2)\\
  &=q(x)(x-2)+8+4+3\\
  &=q(x)(x-2)
\end{align*}
\begin{align*}
  2&|1\quad0\quad2\quad 3\\
  &\vdots\quad2\quad4\quad 2\\
  \hline
  &1\quad2\quad1\quad0
\end{align*}
\subsection*{6}
everything but zero is a root
from 1.4.11
\begin{align*}
  a^{\varphi(n)}\equiv1\mod n
\end{align*}
for $(a,n)=1$
if $n$ is print and $(a,p)=1$ then $a^{p-1}\equiv 1\mod p$ that is $a^{p-1}=1$ in $\mathbb{Z}_p$. and $x(x^{p-1}-1)=x^p-x$ has a zero for all elements in $\mathbb{Z}_p$

corollary $p$ is prime means that $a^p\equiv a\mod p$ (1.4.12)

\section*{4.2}
theorem 4.2.1
$f(x)=q(x)g(x)+r(x)$ where $\deg r<\deg g$ or $\deg r=0$
proof, if $\deg f<\deg g$ then $q=0, r=f$

now assume $\deg f\ge \deg g$. easily see that if $\deg f=0$ is true

now assume that $\deg f=m$ and $\deg g=$. 


\subsection*{4.2.2}
$I$ is an {\bfseries ideal} of $F[x]$
notation $I=d(x)K[x]=(d(x))$ where the last one is ideal notation
\end{document}


