\documentclass[letterpaper]{article}

\usepackage{fullpage}
\usepackage{nopageno}
\usepackage{amsmath}
\usepackage{amssymb}
\allowdisplaybreaks

\newcommand{\abs}[1]{\left\lvert #1 \right\rvert}

\begin{document}
\title{Notes}
\date{August 29, 2014}
\maketitle

last time assigned gcd stuff and assigned to read results

\section*{gcd definition}
$a,b$ not both 0, notation $\text{gcd}(a,b)=(a,b)$

\subsection*{facts}
\begin{enumerate}
\item
gcd exists and  is unique

follows from assigned theorems.

\subsubsection*{example}
$\text{gcd}(6,14)=2$
\item
the gcd of $a$ and $b$ is a linear combination of $a$ and $b$. ie, there exist $m,n\in\mathbb{Z}$ such that $(a,b)=ma+nb$

in fact $\text{gcd}(a.b)$ is the smallest positive inegerthat is a linear combination of $a$ and $b$
\begin{align*}
  \{ma+nb\mid m,n\in\mathbb{Z}\}
\end{align*}
\end{enumerate}

\subsection*{euclidean algorithm}
\begin{align*}
  (a,b)&=(\abs{a},\abs{b})
\end{align*}
we may assume that $a\ge b\ge 0$. $a=bq_1+r_1$.

\subsubsection*{claim}
\begin{align*}
  (a,b)&=(b,r_1)\\
  d_1=(a,b),&\quad d_2=(b,r_1)\\
  d_1\vert a&\rightarrow d_1\vert(bq_1+r_1)\\
  d_1\vert b&\rightarrow d_1\vert r_1\\
  &\rightarrow d_1\vert d2 \text{ because } d_2=(b,r_1)
\end{align*}
similarly show that $d2\vert d1$ hence d1=d2

now we see
\begin{align*}
  a&=bq_1+r_1\\
  b&=rq_2+r_2\\
  r_1&=r_2q3+r_3\\
  &\vdots\\
  r_n&=\text{zero remainder because remainders are shrinking}
\end{align*}
so $(a,b)=(r_{n-1},0)=r_{n-1}$
\subsection*{example}
find $(33,9)$
\begin{align*}
  33&=9*3+6\\
  9&=6*1+3\\
  6&=3*2+0\\
  (33,9)&=3
  3&=9-1*6\\
  &=9-1*(33-3*9)\\
  &=9-1*33+3*9\\
  &=4*9+(-1)*33
\end{align*}
can also use euclidean algorithm to generate linear combination from gcd
\section*{1.2 prime numbers}
\subsection*{definition}
$p>1$ is prime if the only positive divisors of $p$ are $1$ and $p$

$p>1$ is prime if the only divisors of $p$ are $\pm1$ and $p$

\subsection*{definition}
we say that $a$ and $b$ are relatively prime if $\text{gcd}(a,b)=1$
\subsection*{proposition}
let $p>1,p\in\mathbb{Z}$ then $p$ is prime iff the following property holds:

$a,b\in\mathbb{Z}$ and $p\vert ab$ then $p\vert a$ or $p\vert b$

only true if p is prime, ${4}{\not\vert}{6\cdot6}$
\subsubsection*{proof}
assume $p$ is prime.  assume $p\vert ab$, then $(p,a)=1$ or $(p,a)=p$. this is because the only divisor of $p$ is $p$ or 1.

case 1, $(p,a)=p$. then $p|a$ and we are done.

case 2, $(p,a)=1$. then there exists $m,n\in\mathbb{Z}$ such that $mp+na=1$.
\begin{align*}
  mp+na&=1\\
  bmp+bna&=b\\
  p|ab&\rightarrow p|ab
  p|bmp
\end{align*}
since $p|bmp$ and $p|bna$ therefore $p|b$

conversely

assume $\alpha|p$ with $\alpha>0$. Nee to prove that $\alpha=1$ or $\alpha=p$

$\alpha|p\rightarrow p=\alpha\cdot\beta$ with  $\beta\in\mathbb{Z}$

by the property satisfied $p|\alpha or p|\beta$. if $p|\alpha$ since $\alpha|p$ we have $\alpha=p$. if $p|\beta$, since $\beta|p$ we have $\beta=p$

if $\beta=p$ then $\alpha=1$
\end{document}
