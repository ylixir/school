\documentclass[letterpaper]{article}

\usepackage{fullpage}
\usepackage{nopageno}
\usepackage{amsmath}
\usepackage{amssymb}
\allowdisplaybreaks

\newcommand{\abs}[1]{\left\lvert #1 \right\rvert}

\begin{document}
\title{Homework}
\date{September 17, 2014}
\author{Jon Allen}
\maketitle
Section 1.3: \#12, 20
Section 2.1: \# 18, 8
\renewcommand{\labelenumi}{1.\arabic{enumi}}
\renewcommand{\labelenumii}{\arabic{enumii}.}
\renewcommand{\labelenumiii}{(\alph{enumiii})}
\begin{enumerate}
\setcounter{enumi}{2}
\item
\begin{enumerate}
\setcounter{enumii}{11}
%1.2 12
\item
Show that $4\cdot(n^2+1)$ is never divisible by 11.
\subsubsection*{proof}
%We prove the result by assuming that 11 divides $4\cdot(n^2+1)$ and finding a contradiction.
First we note that $\gcd(4,11)=1$ and 11 is prime, so then if $11|4\cdot(n^2+1)$ then by the fundamental theorem of arithmetic, $11|(n^2+1)$.
In other words, there exists an integer $a$ such that $11a=n^2+1$.
Note that 
In other words $n^2+1\equiv0\mod11$.

Tweaking a bit, we have $n^2\equiv10\mod11$.
This tell us that 11 does not divide $n^2$ and then by the fundamental theorem of arithmetic, 11 must not divide $n$. So we can say that $n\equiv10b\mod11$. That is to say  $11|(n-10b)$. 
And tweaking a little more, we have $n^2-9\equiv1\mod11$.
%This means that there exists some $a$ such that $11a=n^2+1$, so then $n\cdot n-11a=1$ and then because we have a linear combination of $n$ and $11$ that equals one, $\gcd(11,n^2+1)=1$.
%Remember that $11a=n^2+1$, and $\gcd(11,11a)\ge11$.
%Of course $11\ne1$ so $4\cdot(n^2+1)$ is never divisible by 11.
%We can go further, even and say that if $\gcd(11,b)=1$ then $b\cdot(n^2+1)$ is never divisible by 11.
$\Box$
\setcounter{enumii}{19}
%1.2 20
\item
Solve the following system of congruences.
\begin{align*}
  2x&\equiv 5\mod 7&3x&\equiv4\mod 8
\end{align*}
\emph{Hint:} First reduce to the usual form.
\begin{align*}
  2x&\equiv 5\mod 7&3x&\equiv4\mod 8\\
  \gcd(2,7)&=1&\gcd(3,8)&=1\\
  \intertext{So both congruencies have one solution}
  c\cdot2&\equiv1\mod 7&c\cdot3&\equiv1\mod8\\
  4\cdot2&\equiv1\mod 7&3\cdot3&\equiv1\mod8\\
  x&\equiv5\cdot4\mod7&x&\equiv3\cdot4\mod8\\
  x&\equiv6\mod7&x&\equiv4\mod8\\
\end{align*}
Now because $\gcd(7,8)=1$ we can apply the Chinese Remainder Theorem.
\begin{align*}
  7a+8b&=1\\
  7(-1)+8(1)&=1\\
  4(7)(-1)+6(1)(8)&=48-28=20\text{ is a specific solution}\\
  20+7\cdot8t&=20+56t\text{ is all solutions}
\end{align*}
\end{enumerate}
\renewcommand{\labelenumi}{2.\arabic{enumi}}
\setcounter{enumi}{0}
\item
\begin{enumerate}
\setcounter{enumii}{7}
%2.1 8
\item
Which of the following formulas define functions from the set of rational numbers into itself? (Assume in each case the $n,m$ are integersn and that $n$ is nonzero.)
\begin{enumerate}
\item
$\displaystyle f\left(\frac{m}{n}\right)=\frac{m+1}{n+1}$

Not a function from $\mathbb{Q}\to\mathbb{Q}$ because when $n=-1$ there is no image.
\item
$\displaystyle g\left(\frac{m}{n}\right)=\frac{2m}{3n}$

This is a function because rational numbers are closed under multiplication so for any $q\in\mathbb{Q}$ we know that $\frac{2}{3}q\in\mathbb{Q}$
\item
$\displaystyle h\left(\frac{m}{n}\right)=\frac{m+n}{n^2}$

This is not a function. Counterexample: $\frac{1}{2}=\frac{2}{4}$. $\frac{1+2}{2^2}=\frac{3}{4}\ne\frac{2+4}{4^2}=\frac{6}{16}=\frac{3}{8}$. $\frac{1}{2}$ has more than one image so the map is not well defined and not a function.
\item
$\displaystyle k\left(\frac{m}{n}\right)=\frac{(m-n)^2}{n^2}$

$\displaystyle \frac{(m-n)^2}{n^2}=\frac{m^2-2mn+n^2}{n^2}=\left(\frac{m}{n}\right)^2-2\frac{m}{n}+1$. Looks like a good function. It will have the same result independent of representation of the rational number, and has an image for every element of $\mathbb{Q}$.
\item
$\displaystyle p\left(\frac{m}{n}\right)=\frac{4m^2}{7n^2}-\frac{m}{n}$

Is a function of rationals. They are closed under multiplication and subtraction. all equivalent elements will have the same image, regardless if their representation in terms of $m,n$.
\item
$\displaystyle q\left(\frac{m}{n}\right)=\frac{m+1}{m}$

Not a function. No representation of zero has an image. For example $\frac{0}{1}$ does not have an image as $\frac{1}{0}$ is undefined.
\end{enumerate}
\setcounter{enumii}{17}
%2.1 18
\item
Let $A$ be a nonempty set, and let $f:A\to B$ be a function. Prove that $f$ is one-to-one if and only if there exists a function $g:B\to A$ such that $g\circ f=1_A$
\subsubsection*{proof}
Lets start by assuming that $f$ is a one to one function. Because $f$ is a function, we know that for every $x\in A$ there exists some $x'\in B$. Furthermore, because $f$ is one to one, we know that $x'$ is unique. Now we simply define $g:x'\to x$. If $B$ has more elements than $A$ then we can define those elements that aren't images of $A$ under $f$ to map to random $a\in A$. Now we see that $g(f(x))=g(x')=x$ and so we've found a function that satisfies our result.

Now let us assume that the function $f$ is not one to one. Because $f$ is a function, we can't have any elements of $A$ map to more than one element in $B$. Therefore $\left\lvert f\right\rvert\le \left\lvert A\right\rvert$. Now because $f$ is not one to one, we know that there are two elements in $A$ that have the same image in $B$. This makes our cardinality inequality strict: $\left\lvert f\right\rvert<\left\lvert A\right\rvert$. This means that if we have a function $g:B\to A$ and feed it the images created by $f$ it will only be able to spit out at most $\left\lvert A\right\rvert-1$ images of it's own. So we have $\left\lvert g\circ f\right\rvert<\left\lvert A\right\rvert$. Because $\left\lvert g\circ f\right\rvert\ne\left\lvert A\right\rvert$ it is certain that $g\circ f\ne1_A$
$\Box$
\end{enumerate}
\end{enumerate}
\end{document}
