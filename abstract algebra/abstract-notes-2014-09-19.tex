\documentclass[letterpaper]{article}

\usepackage{fullpage}
\usepackage{nopageno}
\usepackage{amsmath}
\usepackage{amssymb}
\allowdisplaybreaks

\newcommand{\abs}[1]{\left\lvert #1 \right\rvert}

\begin{document}
\title{Notes}
\date{September 19, 2014}
\maketitle
\section*{assignment}
number of cycles of length  $n$ in $S_n$ is $(n-1)!$ because you fix the first entry to eliminate duplicates.

number of cycles of length $m$ of $S_n$. pick $\binom{n}{m}$ for the first element. $\binom{n}{m}m!/m=\frac{n!}{(n-m)!}\frac{1}{m}$

\#12

(ab) is odd, length three cycles are even, two evens multiplied together is even.

\section*{3.1 groups}
$S$ is a set.
\subsubsection*{definiton}a binary operation on $S$ is a function $S\times S\to S$, or $(x,y)\to x*y$

interesting binary operations satisfy: associativity, identity($\exists e\in S$ such that $x*e=e*x=x$), inverse ($a*b=b*a=e$). if an element has an inverse, we say that it is invertible.

\section*{example}
$\mathbb{Z}$ with binary operation is usual addition, $(\mathbb{Z},+)$, then it is associative, $0$ is identity, and all elements are invertible.

$(2\mathbb{Z},+)$, nothing is different

$(2\mathbb{Z},\cdot)$. No identity element.o

$(2\mathbb{Z}+1,+)$, this operation is not closed, it's not a binary operation.

\section*{propostion}
let $*$ be an associative opertion on $S$, let $a,b,c\in S$ be invertible elements. then
\begin{enumerate}
\item
the $*$ operation has at most one identity element
\item
if it has an identity elemen, then an element a in S has at most one inverse.
\end{enumerate}
\subsubsection*{proof}
assume that $e,e'$ identity elements, $x\star e=e\star x=x$ and $x\star e'=e'\star x=x$for all $x\in S$.

take $x=e'$ then $e'=e'\star e=e$

now if $a$ has two inverses $b,b'$ then $b=b\star e=b\star (a\star b')=(b\star a)\star b'=e\star b'=b'$

\section*{propostion}
let $*$ be an associative opertion on $S$, let $a,b,c\in S$ be invertible elements. then
\begin{enumerate}
\item
$a^{-1}$ is invertible

$a\star a^{-1}=a^{-1}\star a=e$
\item
$a\star b$ is invertible and $(a\star b)^{-1}=a^{-1}\star b^{-1}$.

$(a*b)*(b^{-1}*a^{-1})=a*e*a^{-1}=e$ and similarly $(b^{-1}*a^{-1})*(a*b)=e$
\end{enumerate}

\section*{definition of group}
let $G$ be a set and $\star$ be a binary operation on $G$. we say that $(G,\star)$ is a group if
\begin{enumerate}
\item
$\star$ is associative
\item
$\star$ as an identity element
\item
every element of $G$ is invertible.
\end{enumerate}
\subsubsection*{examples}
$(\mathbb{Z},+)$ is a group, $(\mathbb{Z},\cdot)$ is not, $(\mathbb{Q},\cdot)$ is not because zero is not invertible, $(\mathbb{Q}*,\cdot)$ is because the $*$ means throw out zero. $(S_n,\circ)$ where $\circ$ is a permutation, is a group, $(\mathbb{Z}_n,+)$ is a group. $\mathbb{Z}_n^*$ is all the elements of $\mathbb{Z}_n^*$ is all elements of $\mathbb{Z}_n$ that are invertible

\section*{proposition}
$(G,*)$ is a group, and $a,b,c\in G$ thenn if $ab=ac$ then $b=c$ and if $ba=ca$ then $b=c$

$a^{-1}ab=a^{-1}ac=b=c$

\section*{abelian groups (commutative groups)}
a group $(G,*)$ is called abelian if $*$ is commutative.

for example $(S_n,\circ)$ is not abelian.

another example is $GL_n(\mathbb{R})$ the invertible $n\times n$ matrices with entries in $\mathbb{R}$. $(GL_n(\mathbb{R}),\cdot)$ is not commutative.


\end{document}

