\documentclass[letterpaper]{article}

\usepackage{fullpage}
\usepackage{nopageno}
\usepackage{amsmath}
\usepackage{amssymb}
\allowdisplaybreaks

\newcommand{\abs}[1]{\left\lvert #1 \right\rvert}

\begin{document}
\title{Notes}
\date{September 3, 2014}
\maketitle
\section*{1.1 \#11}
if $a>0$ then $(ab,ac)=a(b,c)$

assume $a,b,c\in\mathbb{Z}, a>0$
\begin{align*}
  d&=(ab,ac)\\
  d&=mab+mac\text{ theorem 1.1.6}\\
  &=a(mb+nc)\text{ this is a linear combination of gcd for }a,b\\
  mb+nc&\in\gcd(b,c)\mathbb{Z}\\
  d&=ad_1
\end{align*}
now prove that $d|ad_1$.
\begin{align*}
  d_1&=m'b+n'c\text{ for some }m',n'\in\mathbb{Z}\\
  ad_1&=m'ab+n'ac\\
  d&=(ab,ac)\rightarrow d|m'ab+n'ac\rightarrow d|ad_1
\end{align*}
\section*{$3x+7$ divisible by 11 (problem 22)}
$x=11+5k,k\in\mathbb{Z}$, note there are infinitely many solutions, and the difference between any two solutions is 11

if $x=5+11k$ for $k\in\mathbb{Z}$ then $3x+7=3(5+11k)+7$

assume $3x+7$ is divisible by 11. $11q+r=x, 0\le r<10$ then $3x+7=3(11q+r)+7=33q+3r+7$ so $3r+7$ is divisible by 11. we know that $0\le r<11$ so $7\le 3r+7\le37$, $3r+7\in\{11,22,33\}\rightarrow r=5$ 
\section*{fundamental  theorem of arithmetic}
any integer $a>1$ can be factored uniquely as a product of prime numbers. $a=p_1^{\alpha_1}p_2^{\alpha_2}...p_n^{\alpha_n}...$ with $p_1<p_2<...<p_n$ and $\alpha_1,\alpha_2,...\alpha_n$ positive integers
\section*{least common multiple}
given $a,b\in\mathbb{Z}^+$, we say that the positive integer $m$ is the lcm of $a$ and $b$ if
\begin{enumerate}
\item
$a|m$ and $b|m$
\item
if $a|c$ and $b|c$ then $m|c$
\end{enumerate}
\subsubsection*{fact}
$a=p_1^{\alpha_1}p_2^{\alpha_2}...p_n^{\alpha_n}$

$b=p_1^{\beta_1}p_2^{\beta_2}...p_n^{\beta_n}$

$p_1<p_2<...<p_n, \alpha_i,\beta_i\ge 0$

then $(a,b)=p_1^{\min\{\alpha_1,\beta_1\}}...p_n^{\min\{\alpha_n,\beta_n\}}$

then $[a,b]=p_1^{\max\{\alpha_1,\beta_1\}}...p_n^{\max\{\alpha_n,\beta_n\}}$
\subsubsection*{example}
\begin{align*}
  6&=2^13^15^0\\
  15&=2^03^15^1\\
  (6,15)&=2^03^15^0=3\\
  [6,15]&=2^13^15^1=30
\end{align*}
\subsubsection*{observe}
$(a,b)[a,b]=ab$

least common multiple of $a,b$ is $ab$
\section*{congruences}
\subsection*{definition}
given $a,b\in\mathbb{Z}$ and $n\in\mathbb{Z}, n>0$ we say that $a\equiv b \mod n$ if $a$ and $b$ give the same remainder when divided by $n$ 

exercise from last time showed $a\equiv b \mod n\Leftrightarrow n|(a-b)$
\subsubsection*{properties}
\begin{enumerate}
\item
\begin{align*}
  a\equiv b \mod n\\
  c\equiv d \mod n\\
  \intertext{implies}
  a\pm c\equiv b\pm d \mod n\\
  \intertext{and}
  ac\equiv bd \mod n\\
\end{align*}
\subsubsection*{proof}
we prove that $ac\equiv bd \mod n$

we know that $n|(a-b)$ and $n|(c-d)$. write that $a-b=n\alpha, \alpha\in\mathbb{Z}$ and $c-a=n\beta, \beta\in\mathbb{Z}$ then $ac-bd=(b+n\alpha)(d+m\beta)-bd=$multiple of $n \Box$

\item
$a\in\mathbb{Z}, n>1, n\in\mathbb{Z}$ then there exist $b\in\mathbb{Z}$ such that $ab\equiv 1 \mod n$ if and only if $(a,n)=1$ 

\emph{note} $3x+7$ divisible by 11 is like saying $3x\equiv -7\equiv4 \mod 11$. $12x\equiv-28 \mod 11$
\subsubsection*{proof}
$\Rightarrow$

assume $ab\equiv1\mod n$ for some $b\in\mathbb{Z}$. Then $ab-1=n\alpha$ for some $\alpha\in\mathbb{Z}$ and $ab+n\alpha=1\rightarrow d=(a,b)$ so $d|1\rightarrow d=1$ 

$\Leftarrow$

assume $(a,n)=1$. there exist $\alpha,\beta\in\mathbb{Z}$ such that $a\alpha+n\beta=1$ and then $a\alpha\equiv1\mod n$
\end{enumerate}
\end{document}
