\documentclass[letterpaper]{article}

\usepackage{fullpage}
\usepackage{nopageno}
\usepackage{amsmath}
\usepackage{amssymb}
\allowdisplaybreaks

\newcommand{\abs}[1]{\left\lvert #1 \right\rvert}

\begin{document}
\title{Notes}
\date{November 24, 2014}
\maketitle
\section*{5.1\#7}
$1=(1-a)^n=(1-a)(1+a+a^2+\dots+a^(n-1)$

$u-a=a(1-u^{-1}a)$ and note that $u^{-1}a=b$ is nilpotent and so $(1-b)$ is invertible and then $u-a$ is invertible
\section*{5.1\#8}
$a+a=(a+a)^2=a^2+a^2+a^2+a^2=a+a+a+a\to0=a+a$

\section*{ring homomorphisms}
for $R,S$ commutative rings, $\varphi:R\to S$ is a ring homomorphism if
\begin{enumerate}
\item
$\varphi(a+b)=\varphi(a)+\varphi(b)$
\item
$\varphi(ab)=\varphi(a)\varphi(b)$
\item
$\varphi(1_R)=1_S$ (not always defined with this property, but are in this class)
\end{enumerate}

\subsection*{prop}
given that $\varphi:R\to S$ is a ring homomorphism then
\begin{enumerate}
\item
$\varphi(0_R)=0_S$
\item
$\varphi(-a)=-\varphi(a)$

because $\varphi(0+(-1)a)=\varphi(0)+\varphi(-1)\varphi(a)$?
\end{enumerate}

\subsection*{def}
we say the $\varphi$ is a isomorphism if it is bijective

\section*{exercises}
if $\varphi$ is an isomorphism then $\varphi^{-1}$ is also an isomorphism

also composition of isomorphisms are isomorphisms (transitivity)
\section*{examples}
$i:\mathbb{C}\to\mathbb{C}[x]$ where $i(a)=a$ is a ring homomorphism

$\mathbb{Z}\to\mathbb{Z}_n$ is not injective because it is an infinite set onto a finite set (pigeonhole principle)

$K\to K[x]/\langle f(x)\rangle$ where $\varphi(a)=[a]$

$K[x]\to K$. Fix $\alpha\in K$ and for each $\alpha$ we define $\varphi_\alpha:K[x]\to K$ and so $\varphi_{\alpha}(f(x))=f(\alpha)$ is called evaluation function.

\subsection*{def}
given a ring homomorphism where $\varphi:R\to S$ then $\ker \varphi=\{x\in R:\varphi(x)=0\}$


\section*{proposition}
if $R$ and $S$ are commutative rings and we have $\varphi$ a ring homomorphism, then
\begin{enumerate}
\item
for every $a,b\in\ker\varphi$ we have that $a-b$ and $a+b$ are also elements in $\ker\varphi$.
\item
for every $r\in R$ and every $a\in\ker\varphi$ we have $ra\in\ker\varphi$
\end{enumerate}
\subsubsection*{proof}
1 follows because $\ker\varphi$ is an additive subgroup of the abelian group $(R,+)$.

2 folows because $\varphi(ra)=\varphi(r)\varphi(a)=\varphi(r)\cdot0=0$

\section*{construction}
two rings and a ring homomorphism $\varphi:R\to S$ and on $R$ we define an equivalence relation $x\sim_\varphi y\Leftrightarrow \varphi(x)=\varphi(y)$ where $[x]$ is the equivalence class of $x$ where $R/\ker\varphi=\{[x]:x\in R\}$. On $R/\ker\varphi$ we have the well defined operation $[x]+[y]=[x+y]$

now we define a new operation on the set $[x][y]=[xy]$

is this well defined?

$[x]=[x']\to\varphi(x)=\varphi(x')$ and similarly with $y$ and so $[x'][y']=\varphi(x')\varphi(y')=\varphi(x)\varphi(y)=\varphi(xy)=[xy]=[x'y']$
\end{document}


