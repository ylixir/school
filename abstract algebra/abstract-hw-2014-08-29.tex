\documentclass[letterpaper]{article}

\usepackage{fullpage}
\usepackage{nopageno}
\usepackage{amsmath}
\usepackage{amssymb}
\allowdisplaybreaks

\newcommand{\abs}[1]{\left\lvert #1 \right\rvert}

\begin{document}
\title{Homework}
\date{August 29, 2014}
\maketitle
Read Theorems 1.1.4, 1.1.6  and their proofs from Section 1.1

Exercises: 2(d), 7, 8, 16, 17.

\begin{enumerate}
\setcounter{enumi}{1}
\item
Find the quotient and remainder when $a$ is divided by $b$
  \begin{enumerate}
  \setcounter{enumii}{3}
  \item
  $a=-1017, b=99$

  \begin{align*}
    \text{gcd}(-1017,99)&=\text{gcd}(1017,99)\\
    1017&=99*10+27\\
    \text{gcd}(-1017,99)&=\text{gcd}(99,27)\\
    99=27*3+18\\
    \text{gcd}(-1017,99)&=\text{gcd}(27,18)\\
    27=18*1+9\\
    \text{gcd}(-1017,99)&=\text{gcd}(18,9)\\
    9\vert18&\rightarrow\text{gcd}(-1017,99)=9
  \end{align*}
  \end{enumerate}
\setcounter{enumi}{6}
\item
Let $a,b,c\in\mathbb{Z}$. Prove these facts about divisors:
  \begin{enumerate}
  \item if $b\vert a$, then $b\vert ac$

  if $b\vert a$ then $\exists q\in\mathbb{Z}$ such that $a=bq$. Then $ac=bqc$ and so $b$ divides $ac$ $\Box$
  \item if $b\vert a$ and $c\vert b$, then $c\vert a$

  if $b\vert a$ then $\exists q_1\in\mathbb{Z}$ such that $a=bq_1$. If $c\vert b$ then $\exists q_2\in\mathbb{Z}$ such that $b=cq_2$. We see then that $a=bq_1=(cq_2)q_1=c(q_1q_2)$ and therefore $c\vert a$ $\Box$
  \item if $c\vert a$, and $c\vert b$, then $c\vert (ma+nb)$ for any integers $m,n$

  if $c\vert a$ and $c\vert b$ then $\exists q_1,q_2\in\mathbb{Z}$ such that $a=cq_1$ and $b=cq_2$. So we see that $(ma+nb)=(mcq_1+ncq_2)=c(mq_1+nq_2)$ and $c\vert (ma+nb)$
  \end{enumerate}
\item
Let $a,b,c$ be integers such that $a+b+c=0$. Show that if $n$ is an integer which is a divisor of two of the three integers, then it is also a divisor of the third.

Because $a+b+c=0$ we can say that $a+b=(-1)c$. Now let us say that $n\vert a$ and $n\vert b$. Then there exists $q_1, q_2\in\mathbb{Z}$ such that $a+b=n(q_1+q_2)=(-1)c$.
\setcounter{enumi}{15}
\item
Let $a,b,c\in\mathbb{Z}$ with $b>0,c>0,$ and let $q$ be the quotient and $r$ the remainder when $a$ is divided by $b$.
  \begin{enumerate}
  \item Show that $q$ is the qotient and $rc$ is the remainder when $ac$ is divided by $bc$
  \item Show that if $q'$ is the quotient when $q$ is divided by $c$, then $q'$ is the quotient when $a$ is divided by $bc$. (Do not assume that the remainders are 0.)
  \end{enumerate}
\item
Let $a,b,n$ be integers with $n>1$. Suppose that $a=nq_1+r_1$ with $0\le r_1<n$ and $b=nq_2+r_2$ with $0\le r_2<n$. Prove that $n\vert (a-b)$ if and only if $r_1=r_2$

\begin{align*}
  a-b&=nq_1+r_1-(nq_2+r_2)\\
  &=n(q_1-q_2)+(r_1-r_2)
\end{align*}
$0\le r_1<n$ and $0\le r_2<n$ therefore $\left\lvert r1-r2\right\rvert<n$. So by the division algorithm $n|(a-b)$ iff $r_1-r_2=0$

class proof

\begin{align*}
  a&=nq_1+r_1,0\le r_1< n\\
  b&=nq_2+r_2,0\le r_2<n\\
  a-b&=nq\\
  \intertext{q=some integer}
  nq_1+r_1-(nq_2+r_2)&=nq\\
  n(q_1-q_2)+(r_1-r_2)&=nq\\
  r_1-r_2&=nq-n(q_1-q_2)\\
  n\vert (r_1-r_2)\\
  \left\lvert r_1-r_2\right\rvert<n
\end{align*}
since $r_i<n, r_1-r_2=0$


other direction
if $r_1=r_2=r_0$ $a=nq_1+r_1$ and $b=nq_2+r_2$
\begin{align*}
  a-nq_1=r_0=r-nq_2\\
  a-b=nq_1-nq_2\\
  a-b=n(q_1-q_2)\\
  n\vert(a-b)
\end{align*}
\end{enumerate}
\end{document}
