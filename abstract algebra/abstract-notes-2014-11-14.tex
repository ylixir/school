\documentclass[letterpaper]{article}

\usepackage{fullpage}
\usepackage{nopageno}
\usepackage{amsmath}
\usepackage{amssymb}
\allowdisplaybreaks

\newcommand{\abs}[1]{\left\lvert #1 \right\rvert}

\begin{document}
\title{Notes}
\date{November 14, 2014}
\maketitle
\section*{\#10}
$\mathbb{Q}[x]/\langle x^2+2\rangle\cong \mathbb{Q}[x]/\langle x^2+1\rangle$ 

no, can't find $\beta\in\mathbb{Q}[x]/\langle x^2+2\rangle$ where $\beta=-[2]$

note that if we replace $\mathbb{Q}$ with $\mathbb{R}$ then we get an isomorphism.

\subsubsection*{is $\mathbb{Z}_2[x]/\langle x^2+x+1\rangle$ a field?} is $x^2+x+1\in \mathbb{Z}_2[x]$ irreducible?

$[0],[1]$ are not roots and degree is $\le3$ and so it is a field.
\subsection*{question}
if $f(x)\in \mathbb{Z}_p[x]$ where $p$ is prime and $f(x)$ is irreducible. how many elements does $\mathbb{Z}_p/f(x)$ have? $\{a_0+a_1x+\dots+a_{n-1}x^{n-1}:a_i\in\mathbb{Z}_p\}$  and so $p^n$ elements.

\section*{\#24}
you can always find such an $f(x)$ (of any degree).

\section*{thm}
assume $f(x)=g(x)h(x)$ where $f(x)\in \mathbb{Z}[x]$ and $h(x),g(x)\in\mathbb{Q}[x]$. then we can factor $f(x)$ into poly with integer coefficients of the same degree.
\subsubsection*{proof}
$g(x)\in\mathbb{Q}[x]$ with $g(x)=\frac{1}{b}g_1(x)$ where $g_1(x)\in\mathbb{Z}[x]$ and $b\in\mathbb{Z}$ and $g(x)=\frac{c}{b}g_2(x)$ where $g_2(x)\in\mathbb{Z}[x]$ and $b,c\in\mathbb{Z}$ and $g_2(x)$ is a primitive polynomial

say $\gcd(m,n)=1$

$f(x)=\frac{m}{n}\cdot\frac{s}{t}g_2(x)h_2(x)$.

multiplying two primitive polynomials gives a primitive polynomial

\section*{thm}
eisenstein's irreducibility criterion)

\section*{corollary 4.4.7}
\end{document}


