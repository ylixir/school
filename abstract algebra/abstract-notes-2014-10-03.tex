\documentclass[letterpaper]{article}

\usepackage{fullpage}
\usepackage{nopageno}
\usepackage{amsmath}
\usepackage{amssymb}
\allowdisplaybreaks

\newcommand{\abs}[1]{\left\lvert #1 \right\rvert}

\begin{document}
\title{Notes}
\date{October 3, 2014}
\maketitle
if $G$ is a group, and $A\subseteq G$ and $B\subseteq G$ then $AB=\{ab|a\in A, b\in B\}\subseteq G$.

\section*{proposition}

let $G$ be a group, then $H,K$ subgroups of $G$. Assume that $h^{-1}kh\in K$ for all $h\in H$, $k\in K$ then $HK$ is a subgroup of $G$ that contain s both $H$ and $K$, in fact, $HK$ is the smallest subgroup of $G$ that contains both $H$ and $K$. Assumption only important if we are not dealing with abelian groups.

\subsubsection*{proof}
$a,b\in HK$. Write $a=h_1k_1,b=h_2k_2$ with $h_i\in H,k_i\in K$ then $a\cdot b=h_1k_1h_2k_2=h_1h_2(h_2^{-1}k_1h_2)k_2\in HK$

$a=hk, a^{-1}=(hk)^{-1}=k^{-1}h^{-1}=h^{-1}(hk^{-1}h^{-1})\in HK$

\subsection*{examples}
$S_3, H=\{(1),(12)\}, K=\{(1),(123),(132)\}, (12)(123)=(23)\in HK, (12)(132)=(13)\in HK$ so $HK=G$ and is therefore contained by G

$(\mathbb{Z},+)$, $H=a\mathbb{Z}, k=b\mathbb{Z}$, let $d=(a,b)$

claim: $a\mathbb{Z}+b\mathbb{Z}=d\mathbb{Z}$. clearly $a\mathbb{Z}\subseteq d\mathbb{Z}$, $b\mathbb{Z}\subseteq d\mathbb{Z}$. 

$a\mathbb{Z}+b\mathbb{Z}$ is the smallest subgroup that contains both $a\mathbb{Z}$ and $b\mathbb{Z}$. so $a\mathbb{Z}+b\mathbb{Z}\subseteq d\mathbb{Z}$.

$d=\gcd(a,b)$ so we can write $d=ma+nb$. let $\alpha\in d\mathbb{Z}$ and write $\alpha=dt, t\in \mathbb{Z}$ then $\alpha=dt=mat+nbt\in a\mathbb{Z}+b\mathbb{Z}$. so $d\mathbb{Z}\subseteq a\mathbb{Z}+b\mathbb{Z}$ 
\section*{thm subgroup gen by a subset}
$G$ is a group, if $a\in G$ $<a>=\{a^i|i\in \mathbb{Z}\}$ is the smallest subgroupthat contains $a$.

\subsubsection*{proof}
let $S\subseteq G$, let $<S>=\{\underbrace{a_1a_2\dots a_k}_{\text{word}}|a_i\in S\text{ or }{a_i}^{-1}\in S, k\in \mathbb{N}\}$ then $<S>$ is a subgroup, $<S>=\cap \forall H$ where $S\subseteq H\subseteq G$, and $H$ is a subgroup of $G$, $<S>$ is the smallest subgroup of $G$that contains $S$.

so it is closed under multiplication, identity is in it, and the inverse of all words are in it.

show containment both ways, one is clear because we have words of length 1 that span $S$ and so $S$ is one of the elements of our $H$ intersection.

\section*{example}
$a,b\in G, S=\{a,b\}\subseteq G, <S>=\{a_1a_2\dots a_k|a_i\in\{a,a^{-1},b,b^{-1}\}\}$

if $ab=ba$ then $<S>=\{a^{i}b^{j}|i\in \mathbb{Z}, j\in \mathbb{Z}\}$


\section*{maps}
studied groups, subgroups. now we are going to talk about maps

if we have groups $G_1,G_2$ and $\varphi:G_1\to G_2$ is a group homomorphism provided $x\to\varphi(x), y\to\varphi$ means that $\varphi(x*y)=\varphi(x)*\varphi(y)$ for all $x,y\in G_1$.
\section*{examples}
identity: $x\to x$

$(\mathbb{R},+)=G_1, (\mathbb{R}^+,\cdot)=G_2$. $\varphi(x)=e^x$. ie $\varphi(x+y)=e^{x+y}=e^xe^y=\varphi(x)\varphi(y)$.

\section*{notation}
let $\varphi:G_1\to G_2$ be a group homomorphism, then $\ker\varphi=\{x\in G_1|\varphi(x)=e\}$

homomorphism always takes the identity in $G_1$ to $G_2$.

$\varphi(e_1)\varphi(e_1)^{-1}=\varphi(e_1e_1)\varphi(e_1)^{-1}=\varphi(e_1)\varphi(e_1)\varphi(e_1)^{-1}=e_2=\varphi(e_1)$

prove that $\ker \varphi$ is a subgroup

now we say that $\varphi$ is an isomorphism if $\varphi$ is a group homomorphism and $\varphi$ is bijective.

both of the previous examples are isomorphisms.

so from an algebraic point of view, there is no difference between addition on the reals and multiplication on the positive reals.

\section*{proposition}
let $\varphi$ be an isomorphism. the following are true
\begin{enumerate}
\item
$\varphi^{-1}$ which is the map from $G_2$ to $G_1$ is also an isomorphism.
\item
if $G_1$ is abelian, then $G_2$ is abelian.
\item
if $G_1$ is cyclic then so is $G_2$
\item
if $a\in G_1$ then $\text{ord}(a)=\text{ord}(\varphi(a))$
\end{enumerate}

\begin{enumerate}
\item
need to prove $\varphi^{-1}(\alpha\beta)=\varphi^{-1}(\alpha)\varphi^{-1}(\beta)$ for all $\beta\in G_2$, but $\varphi$ is injective so it is enough to prove that $\varphi(\varphi^{-1}(\alpha\beta))=\varphi(\varphi^{-1}(\alpha)\varphi^{-1}(\beta)=\varphi(\varphi^{-1}(\alpha))\varphi(\varphi^{-1}(\beta))=\alpha\beta$
\item
assume $G_1$ is abelian
\begin{align*}
  \alpha\beta=\varphi(\varphi^{-1}\left(\alpha\right)\varphi^{-1}(\beta))
\end{align*}
\item
hint: assume that $G_1=<a>$ for some $a\in G_1$ and then prove that $G_2=<\varphi(a)$
\item
no hint
\end{enumerate}
\section*{example}
\begin{align*}
  \mathbb{Z}_4\not\equiv\mathbb{Z}_2\times\mathbb{Z}_2
\end{align*}
by contradiction, assume that there exists an isomorphism $\varphi$ from z4 to z2+z2. $[1]\in \mathbb{Z}_4$ and $\text{ord}[1]=4$ so then $\text{ord}\varphi([1])=4$. But all elements of $\mathbb{Z}_2\times\mathbb{Z}_2$ has no elements of order 4, so there is no isomorphisms. however, if $\gcd(m,n)=1$ then $\mathbb{Z}_{mn}\equiv\mathbb{Z}_m\times\mathbb{Z}_n$

\begin{align*}
  \varphi:[x]_{mn}\to[x]_m[x]_n
\end{align*}
\section*{what does well defined mean?}
same input gives same output, ie if $[x]=[y]$ then $\varphi[x]=\varphi[y]$

\section*{3.4 \#13}
$(\mathbb{R}^*,\cdot), C_2=\{\pm1\}\subseteq \mathbb{R}^*$. $C_2$ is a subgroup of $\mathbb{R}^*$. prove that $\mathbb{R}^*\cong \mathbb{R}^+\times C_2$

we construct an isomorphism $\theta:\mathbb{R}^*\to R^{+}\times C_2$.

$x\to(|x|,\frac{x}{|x|})$. prove that $\theta$ is a group homomorphism and bijective.

$\theta(xy)=(|xy|,\frac{xy}{|xy|})=(|x|,\frac{x}{|x|})(|y|,\frac{y}{|y|})=\theta(x)\theta(y)$

bijectivity is exercise, but a number is uniquely identified by sign and magnitude (absolute value)

\section*{example}
prove that $\text{ord}(aba^{-1})=\text{ord}(b)$ for every $a,b\in G$ where $G$ is a group.

$m=\text{ord}(x)$ means $x^m=e$ and $x^k=e$ means that $m|k$.

if given $n=ord(x)$ and $m=ord(y)$ then best way is to show that $m=n$ is $m|n$ and $n|m$. this all works for finit.

this question is trivial if the group is abelian.

so let $m=ord(aba^{-1}), n=ord(b)$

\subsection*{case 1}
$n$ is finite, $b^n=e$. consider $(aba^{-1})^n=aba^{-1}aba^{-1}\dots aba^{-1}=ab^{n}a^{-1}=aea^{-1}=e$ so $ord(aba^{-1})$ is finite, also $m|n$

$b^m=a^{-1}\underbrace{(aba^{-1})(aba^{-1})\dots(aba^{-1})(aba^{-1})}_{m\text{ times}}a=a(aba^{-1})^ma=a^{-1}ea=e$ so $b^m=e$ and $n|m$
\subsection*{case 2}
$n$ is infinite. then we must prove that $m$ is infinite. by contradiction, assume $m<\infty$.

then $b^m=a^{-1}\underbrace{(aba^{-1})(aba^{-1})\dots(aba^{-1})(aba^{-1})}_{m\text{ times}}a=a(aba^{-1})^ma=a^{-1}ea=e$ so $b^m=e$.

and so the order is finite and we have our contradiction. so $m$ must be finite then
\subsection*{consider}
$ord(a^{-1})=ord(a)$ and $ord(ab)=ord(ba)$

by previous part $ord(ba)=(abaa^{-1})=ord(ab)$

\section*{3.2 \# 25 from class}
we note that if $x\in G$ has the required order, then $x^{-1}\in G$ also has the required order. note that the $x\ne x^{-1}$ because the order is greater than $2$.

\section*{!3.3 \#9}

\section*{other things in mind}
lets take a group $G$ and $H,K$ subgroups. then $\left\lvert HK\right\rvert=\frac{|H||K|}{|H\cap K|}$

recall: $HK=\{hk|h\in H, k\in K\}$

\begin{align*}
  g\in HK\\
  g=hk\\
  h\in H\quad |H|=m\\
  k\in K\quad |K|=n\\
\end{align*}
question? how many ways can $g=hk=h'k'$?

\begin{align*}
  hk=h'k'\to(h')^{-1}hk=k'\\
  (h')^{-1}h=k'k^{-1}\in H\cap K
\end{align*}
so $k'=\alpha k, h'=h\alpha^{-1}$

that is to say $hk=(h'\alpha)(\alpha^{-1}k')$
there are $|H\cap K|$ ways to choose alpha
\end{document}


