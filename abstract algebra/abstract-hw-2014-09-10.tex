\documentclass[letterpaper]{article}

\usepackage{fullpage}
\usepackage{nopageno}
\usepackage{amsmath}
\usepackage{amssymb}
\allowdisplaybreaks

\newcommand{\abs}[1]{\left\lvert #1 \right\rvert}

\begin{document}
\title{Homework}
\date{September 10, 2014}
\author{Jon Allen}
\maketitle
Section 1.1: \# 14, 16, 21
Section 1.2: \# 12, 23.
Math 620 students: in addition to the above problems, turn in

Section 1.2: \# 21, 25.
\renewcommand{\labelenumi}{1.\arabic{enumi}}
\renewcommand{\labelenumii}{\arabic{enumii}.}
\renewcommand{\labelenumiii}{(\alph{enumiii})}
\begin{enumerate}
\item
\begin{enumerate}
\setcounter{enumii}{13}
\item
For what positive integers $n$ is it true that $(n,n+2)=2$? Prove your claim.
\subsubsection*{claim}
The statement is true for all even integers.
\subsubsection*{proof}
First lets note that by the definition of even and odd integers, two does not divide an odd integer. Because two is not a divisor of an odd integer, it cannot be a greatest common divisor with any other integer. It follows then that $n$ must be even. That is to say there exists some $k\in\mathbb{Z}$ such that $n=2k$.

Now we show that the fact that $n$ is even  is enough for $\gcd(n,n+2)=2$. Let's take an integer $d\ne 2$ where $d=\gcd(n,n+2)$. Now by definition $d|n$. Because $n$ is even we have $d|2k$. Because $d\ne2$ there must exist some integer $a$ such that $k=ad$ and then $d|2ad$. Now because $d$ also divides $n+2$ we have $d|2ad+2$ which is to say there exists some integer $b$ such that $bd=2ad+2$. This is true only if $d|2$. Because $d\ne2$ we see that $d=1$ However we defined $d=\gcd(n,n+2)$ and we know that $2|n$ and $2|(n+2)$ and obviously $2>1$ so we have a contradiction. Therefore if $n$ is even, then $\gcd(n,n+2)=2$. $\Box$
\setcounter{enumii}{15}
\item
Let $a,b,c$ be integers, with $b>0, c>0,$ and let $q$ be the quotient and $r$ the remainder when $a$ is divided by $b$.
\begin{enumerate}
\item
Show that $q$ i s the quotient and $rc$ is the remainder when $ac$ is divided by $bc$

\subsubsection*{proof}
We observe that dividing $a$ by $b$ we are guaranteed to be able to find $a=bq+r$ such that $0\le r<b$. Let's multiply this by $c$. We get $c(a)=c(bq+r)$. Now distributing and rearranging we see that $ac=(bc)q+rc$. We see from this that $q$ is a quotient and $rc$ is a remainder. Observe however that these may not be the quotient and remainder that are guaranteed by the division algorithm.

To guarantee this, we must show that $rc<bc$. Note that $c>0$ and $0\le r<b$. Again multiplying by $c$ we have $0\le rc<bc$ and therefore $rc$ is the remainder guaranteed by the division algorithm. By extension $q$ must also be the quotient guaranteed by the division algorithm. $\Box$

\item
Show that if $q'$ is the quotient when $q$ is divided by $c$ then $q'$ is the quotient when $a$ is divided by $bc$. (Do not assume that the remainders are zero.)
\subsubsection*{proof}
We let $q=cq'+r'$ where $r'$ is the remainder when $q$ is divided by $c$. Let us multiply both sides of the equation by $b$ to get $bq=bcq'+br'$. We note that $a=bq+r$ and substitute in $bq$ to obtain $a=bcq'+br'+r$. Now we see that $q'$ is a quotient when $a$ is divided by $bc$. We must show that $0\le br'+r<bc$ to prove that $br'+r$is the remainder guaranteed by the division algorithm and by extension, $q'$ is the quotient guaranteed by the division algorithm. We know $0\le r'<c$ and $0\le r<b$ so $r'$ is at most $c-1$ and $r$ is at most $b-1$. This means that $br'+r\le b(c-1)+b-1$. Simplifying we have $br'+r\le bc-1$ or $br'+r<bc$. Similarly $0\le b(0)+0=0$ so $0\le br'+r<bc$ and we have our result. $\Box$
\end{enumerate}
\setcounter{enumii}{20}
\item
Prove that the sum of the cubes of any three consecutive positive integers is divisible by 3.
\subsubsection*{proof}
Let $a\in\mathbb{Z}^+$. We seek to show that there exists some integer $b$ such that $3b=a^3+(a+1)^3+(a+2)^3$.
\begin{align*}
  3b&=a^3+(a+1)^3+(a+2)^3\\
  &=a^3+a^3+3a^2+3a+1+a^3+3\cdot2a^2+3\cdot2^2a+2^3\\
  &=3a^3+9a^2+15a+9\\
  &=3(a^3+3a^2+5a+3)\\
  b&=a^3+3a^2+5a+3
\end{align*}
Note that this works with all $a\in\mathbb{Z}$, it does not require $a\in\mathbb{Z}^+$. $\Box$
\end{enumerate}
\item
\begin{enumerate}
\setcounter{enumii}{11}
\item
Let $a,b,c$ be nonzero integers. Show that $\gcd(a,b)=1$ and $\gcd(a,c)=1$ if and only if $\gcd(a,\text{lcm[b,c])}=1$
\subsubsection*{proof}
Let $a,b,$ and $c$ be represented by their prime factorizations $a=p_1^{\alpha_1}p_2^{\alpha_2}\dots p_n^{\alpha_n}$, $b=p_1^{\beta_1}p_2^{\beta_2}\dot a p_n^{\beta_n}$, and $c=p_1$
\setcounter{enumii}{22}
\item
Show that if $n$ is a positive integer such that $2^n+1$ is prime, then $n$ is a power of 2.
\subsubsection*{proof}
Because $n$ is a power of two, it's prime factorization is $2^m$ for some integer $m\ge 0$ and all of its factors are therefore even. Let us assume that $n$ is not a power of two, then it must have some odd prime factor $p=2k+1$. We'll say that $n=(2k+1)q$.
\begin{align*}
  2^n+1&=2^{q(2k+1)}+1\\
  &=(2^q)^{2k+1}+1\\
  &=(2^q+1)(2^{q2k}-2^{q(2k-1)}+\dots+1)
\end{align*}
Now because we defined $p$ to be prime we know that $p\ge 2$ and therefore $k\ge1$. This means that we have found a factor for $2^n+1$ when $n$ is not a power of two. Therefore, if $n$ is a power of two, then $2^n+1$ must not have any factors and is therefore prime. $\Box$
%\setcounter{enumii}{20}
%\item
%\setcounter{enumii}{24}
%\item
\end{enumerate}
\end{enumerate}
\end{document}
