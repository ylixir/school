\documentclass[letterpaper]{article}

\usepackage{fullpage}
\usepackage{nopageno}
\usepackage{amsmath}
\usepackage{amssymb}
\allowdisplaybreaks

\newcommand{\abs}[1]{\left\lvert #1 \right\rvert}

\begin{document}
\title{Notes}
\date{November 26, 2014}
\maketitle
\section*{fundamental theorem of ring homomorphism}

\subsection*{observation}
let $\varphi:R\to S$ be a given ring homomorphism (R,S are commutative rings)

we want to define $\varepsilon:R[x]\to S$ ring homomorphism such that $\begin{cases}\varphi(r)=\varepsilon(r)&\forall r\in R\\\varepsilon(x)=s&s\text{ fixed }\in S\end{cases}$

then $\varepsilon(a_0+a_1x+\dots+a_nx^n)=\varphi(a_0)+\phi(a_1)s+\dots+\varphi(a_n)s^n$

\subsection*{construction}
we define $R_1+R_2+\dots+R_n=\{(a_1,a_2,\dots,a_n):a_i\in R_i\}$ with component wise operations. this is a ring.

\subsection*{definition}
given $R$ a commutative ring, we define the characteristic of$R$ to be $\text{char } R$ to be the smallest $n$ such that $1_1+1_2+...+1_n=0$. If it doesn't exists we say $\text{char } R$ is zero.

\subsection*{prop}
the characteristic of an integral domain is either $0$ or prime.

\subsubsection*{proof}
if 0 then done, lets, assume that it's not prime.

then $\text{char } R=n=\alpha\beta$ and $0=1_1+1_2+\dots+1_n=1_1+1_2+\dots+1_\alpha+1_1+\dots+1_\beta=\alpha\beta$. Because we are in an integral domain then $\alpha=0$ or $\beta=0$ and so we have a contradiction because $\alpha<n$ and the same for $\beta$ and so we have a contradiction because $n$ is the smallest to be 0 and so $n$ is prime.

\subsection*{corollary}
if $K$ is a field then characteristic is 0 or prime
\end{document}


