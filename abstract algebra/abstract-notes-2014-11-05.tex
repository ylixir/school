\documentclass[letterpaper]{article}

\usepackage{fullpage}
\usepackage{nopageno}
\usepackage{amsmath}
\usepackage{amssymb}
\allowdisplaybreaks

\newcommand{\abs}[1]{\left\lvert #1 \right\rvert}

\begin{document}
\title{Notes}
\date{November 5, 2014}
\maketitle
\section*{4.2 2a,8,9}
\subsection*{9}
show that the remainder when $f(x)$ is divided by $(x-a)^2$ is $f'(a)(x-a)+f(a)$

$f(x)=(x-a)^2q(x)+r(x)$ and $\deg r<2$ $r(x)=\alpha x+\beta$.
$f(a)=r(a)=\alpha a+\beta$.
$f'(x)=2(x-a)q(x)+(x-a)^2q'(x)+\alpha$
$f'(a)=\alpha$

$f(a)=f'(a)a+\beta\to\beta=f(a)-af'(a)$.o

$r(x)=f'(a)x+f(a)-af'(a)=f'(a)(x-a)+f(a)$
\section*{proposition}
let$I\subseteq K[x]$ such that
\begin{enumerate}
\item
$I$ contains a non-zero polynomial
\item
$f(x),g(x)\in I\Rightarrow f(x)+g(x)\in I$
\item
$f(x)\in I, g(x)\in K[x]\Rightarrow f(x)g(x)\in I$
\end{enumerate}
this is the ideal of $K[x]$
let $d(x)\in I$ of minimal degree. then $I=\{a(x)f(x):f(x)\in K[x]\}$

\subsubsection*{proof}
let $h(x)\in I$ write $h(x)=d(x)q(x)+r(x)$ with $r(x)=0$ or $\deg r>\deg d$

then $r(x)=h(x)+d(x)[-q(x)]\in I$ by 2 above. by choice of $d(x)$ we have $r(x)=0$ and so $h(x)\in I$


\section*{def}
$f(x),g(x)\in K[x]$ where $K$ is a field. a monic polynomial $a(x)\in K[x]$ is called gcd of $f(x),g(x)$ if
\begin{enumerate}
\item
$a(x)|f(x)$ and $a(x)|g(x)$
\item
if $t(x)|f(x)$ and $t(x)|g(x)$ then $t(x)|a(x)$
\end{enumerate}

monic means that the leading coefficient is 1


\section*{thm}
if we have $f(x),g(x)$ non-zero, then $\exists\gcd(f(x),g(x))$ and $\gcd(f(x),g(x))$ can be expressed in the form $\alpha(x)f(x)+\beta(x)g(x)$.
\subsubsection*{proof}

let ideal $I=\{\alpha(x)f(x)+\beta(x)g(x)\}$. Check that $I$ satisfies all conditions of earlier proposition.

let $d(x)\in I$ of minimal degree and without loss of generality assume $d(x)$ is monic. we can do this because multiplying by a constant is multiplying by a polynomial, so it's still in $I$.

claim $d(x)$ is a $\gcd$ of $f(x),g(x)$

$I=\{d(x)h(x):h(x)\in K[x]\}$ in particular $d(x)f(x)$ and $d(x)g(x)$ are both in $I$. now if $t(x)|f(x)$ and $t(x)|g(x)$. $\exists\alpha(x),\beta(x)$ such that $d(x)=\alpha(x)f(x)+\beta(x)g(x)$. then $t(x)|d(x)$. 

\section*{thm}
the gcd is unique. lets assume that $d_1(x)$ and $d_2(x)$ are gcd of $f(x)$ and $g(x)$. $d_2(x)|f(x)$ and $d_2(x)|g(x)$. $d_1$ is gcd so $d_1|d_2$ and $d_1(x)|f(x)$ and $d_1(x)|g(x)$ so because $d_2$ is gcd then $d_2|d_1$. we said our gcd was monic.

$d_1(x)=d_2(x)\alpha_1(x)$ and $d_2(x)=d_1(x)\alpha_2(x)$. now $d_1(x)\alpha_1(x)\alpha_2(x)\to d_2=d_1(x)\alpha_1(x)\alpha_2(x)$

\section*{thm}
if $p(x)|f(x)g(x)$ and $\gcd(p(x),f(x))=1$ then $p(x)|g(x)$.

\section*{def}
$f(x)\in K[x]$. we say that $f(x)$ is irreducible over the field $K$ if $f(x)$ cannot be factored into a product of two polynomials of degree lower than $\deg f(x)$.
\subsection*{example}
$x^2+1$ is irreducible over $\mathbb{R}$
\subsection*{example}
$x^2+1\in \mathbb{C}[x]$ is reducible (not irreducible) over $\mathbb{C}$

\section*{prop}
$f(x)\in K[x]$, $\deg f(x)$ is $2$ or $3$ and $f(x)$ has no roots in $K$. then $f(x)$ is irreducible.

\subsubsection*{proof}
we choose $\deg 2,3$ because at least one of the factors is $\deg 1$

we assume the factors exist. then $f(x)=(\alpha x+\beta)h(x)$ and $f(-\beta\alpha^{-1})=0$ and it has a root.

\section*{thm}
every non-zero polynomial in  $K[x]$ can be written uniquely as a product of irreducible polynomials.

proof is inductive argument

\section*{def}
$f(x)\in K[x], c\in K$ we say that $c$ is a root of multiplicity $m$ of $f(x)$ if $(x-c)^m$ divides $f(x)$ and $(x-c)^{m+1}\not| f(x)$

\section*{prop}
$f(x)\in \mathbb{R}[x]$, $\deg f(x)\ge 1$ then $f(x)$ has no repeatable factors iff the $gcd$ of $f(x)$ and $f'(x)$ is one
\end{document}


