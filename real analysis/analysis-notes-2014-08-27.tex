\documentclass[letterpaper]{article}

\usepackage{fullpage}
\usepackage{nopageno}
\usepackage{amsmath}
\usepackage{amssymb}
\allowdisplaybreaks

\newcommand{\abs}[1]{\left\lvert #1 \right\rvert}

\begin{document}
\title{Notes}
\date{August 27, 2014}
\maketitle
\section*{real numbers}
we define a real number as $\sum\limits_{i=0}^{\inf}{a_i\cdot 10^{-i}}=a_0.a_1a_2a_3a_4, 0\le a_i\le 9, a_i\in \mathbb{Z}$
\subsection*{2.2e}
\begin{align*}
  x&=2.1357
\end{align*}
\subsection*{example}
$0.99999...=\sum\limits_{i=0}^{\inf}{9\cdot 10^{-i}}=9\left(\sum\limits_{i=1}^{\inf}{10^{-1}}\right)=9\left(\frac{\frac{1}{10}}{1-\frac{1}{10}}\right)=\frac{9}{9}=1$
\section*{not in book}
\subsection*{the real line axiomatic definition}
the real numbers are the set $\mathbb{R}$ with two operations $+,\cdot$ multiplication and addition that are closed in $\mathbb{R}$. if $a,b\in \mathbb{R}$ then $a+b\in \mathbb{R}$ and $a+b\in \mathbb{R}$

and verify the following axioms:
\begin{enumerate}
\item
for all a,b in R, a+b=b+a and a*b=b*a
\item
for all a,b,c in R, (a+b)+c=a+(b+c) and a(bc)=(ab)c
\item
for all a,b,c in R, a(b+c)=a*b+a*c
\item
give a,b in R, there exists c in R such that a+c=b (each c is denoted by (b-a)

in particular the real number a-a is independent of a and is denoted by 0 and the real number 0-a is denoted by -a and called the negative of a
\item
\begin{enumerate}
\item
there is a non-zero element in R
\item
given x,y in R, x not 0 then there exists z in R such that xz=y. such z is denoted by y/x

in particular, x/x is independant of x in R, x not 0 and denoted by 1, and 1/x is called the reciprocal of x ($x^{-1}$.
\end{enumerate}
\end{enumerate}
actually these are properties of fields, not just reals. \emph{field axioms}. this includes rationals.

now for the things that differentiate from rationals
\subsection*{order axioms}
no order in complex numbers, unlike reals and rationals
\begin{enumerate}
\item
for all x,y in R, exactly one of these hold:
\[x<y, x=y, x>y\]
\item
if $x<y$ then for all z in R $x+z<y+z$
\item
if $x>0$ and $y>0$ then $xy>0$
\item
if $x>y$ and $y>z$ then $x>z$
\end{enumerate}
now we have an ordered field defined, both reals and rationals (Q) are ordered fields
\subsection*{completeness axiom}
every non-empty set $S\subset R$ (not proper subset) that is bounded above has a least upper bound
\section*{moving on}
prove that if $x<y$ and $z<0$ then $xz<yz$
\subsection*{lemma}
if $z<0$ then $-z>0$

assume that $z<0$. by 6 we know that $-z$ and 0 we get that exactly one of these happen:

\[-z<0, -z=0,-z>0\]
we know that $z<0$ and $z+(-z)=0$. use 7. $z<0$ so $z+(-z)<0+(-z)$ and $0<-z$
\subsection*{case 1, x is positive}
then by transitivity property (9) y is also positive and by 8 $x(-z)<y(-z)$.
by 7 $x(-z)+xz<y(-z)+xz$ and $0<y(-z)+xz$ and similarly $yz<xz$

\section*{2.3.1 definition}
a set S contained by R is bounded above if there exists M in R such that $x\le M$ for all $x\in S$. M is called an upper bound for S.

A set S contained by R is bounded below if there exists m in R such that x is greater than or equal to m for all x in S. m is called a lower bound for S

a set S contained by R is bounded if it is bounded above and below

Let S be a bounded subset of R

L is the supremum or least upper bound of S if 1) L is an upper bound of S and 2) if M is another upper bound of S then L is less than or equal to M

l is the infimum or greatest lower bound of S if 1) l is a lower bound of S and 2) if m is another lower bound of 
\subsection*{examples}
\begin{align*}
  S&=(1,5]\\
  sup(S)&=5\\
  inf(S)&=1\\
\end{align*}
6 is an upper bound and 0 or -4324 are lower bounds for s.

the sup and inf need not necessarily be elements of s


let s be contained in R and bounded. if sup(s) is in s then we call it the maximum of s. if inf(s) is in s then we call it the minimum of s.
\subsection*{examples}
find sup, inf, max and min(if they  exist)

if S=\{2/n,n in N, $n\ge 1$\} then sup(S)=2=max S and inf(S)=0

if S=\{$\frac{(-1)^nn}{n+1}$,$n\in N$\} then sup(S)=1, inf(S)=-1, no max or min

if S=\{$x\in Q:x^2<2$\} then sup(S)=$\sqrt{2}$

if we view s as a subset only of q and forget about R, s has no supremum. any q in Q, q squared greater than 2 is an upper bound for s but there is not a smallest such q in Q
\end{document}
