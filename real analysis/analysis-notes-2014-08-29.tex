\documentclass[letterpaper]{article}

\usepackage{fullpage}
\usepackage{nopageno}
\usepackage{amsmath}
\usepackage{amssymb}
\allowdisplaybreaks

\newcommand{\abs}[1]{\left\lvert #1 \right\rvert}

\begin{document}
\title{Notes}
\date{August 29, 2014}
\maketitle

given a set S contained in R bounded abouve, the supremum of S or least upper bound is a real number L such that
\begin{enumerate}
\item
for all x in S, $x\le L$
\item
if there is a number M such that $x\le M$ for all $x\in S$ then $L\le M$
\end{enumerate}

\section*{approximation property of the supremum theorem}
let S be a subset of R, $S\ne 0$ bounded above, Let $b=\text{sup}(S)$ then for all $a<b,\exists x\in S$ such that $a<x\le b$
\subsection*{proof}
if there is no x in S such that $a<x\le b$ then a is an upper bound for S contradicting that b is the least upper bound$\Box$
\section*{2.3.3 Least upper bound principle}
tenth axiom from yesterday

every non-empty set of real numbers that is bounded above has a supremum
\subsection*{proof}
required because book defines real numbers as decimal expansions, not axiomatic definition

\begin{enumerate}
\item
observation: this is equivalent to proving that any non-empty set of reals that is bbounded below has an infimum. Why? homework: let S be a set of Reals, let -S=$\{-x:x\in S\}$. Then you will have to prove that sup(-S)=-(inf S)

so we will prove infimum statement
\end{enumerate}

S is bounded below. let $m$ be a lower bound. $m=m_0.m_1m_2m_3m_4m_5m_5...$ wher $m_0\in\mathbb{Z}, m_0>0$ without loss of generality and $m_i, i>0$ is 0-9 digit. clearly $m_0$ is also a lower bound for S.

Consider all integers that are lower bounds for S, (there is at least $m_0$). Take the biggest of such integers ($n_0)$.

$n_0$ is a lower bound for S, but $n_0+1$ is not. we build the infimum with $n_0.\_\_\_\_$. Now pick the gretest ineger $n_1$ such that $n_0+\frac{n_1}{10}$ is a lower bound for S. Since $n_0$ is a lower bound, $0\le n_1$.  Since $n_0+1$ is not a lower bound, $n_1<10$

Now pick the reatest integer $n_2$ such that $n_0+\frac{n_1}{10}+\frac{n_2}{100}$ is still a lower bound for S. Claim $n_0.n_1n_2n_3n_4...$ is inf(S)$\Box$

\section*{properties of the supremum}
let $A,B$ be subset of $\mathbb{R}$, nonempty, let $C=\{a+b:a\in A, b\in B\}$ if $A,B$ have a supremum, then so does $A+B$ and sup($A+B$)=sup$A$+sup$B$
\subsubsection*{proof}
let $z\in C$, then $z=a+b$, where $a\in A$, $b\in B$

let $L_1=\text{sup}A, L_2=\text{sup}B$ then $a\le L_1, b\le L_2$ and then $z\le L_1+L_2$ for all $z\in C$. This shows that $L_1+L_2$ is an upper bound for $C$. choose $\epsilon>0$ and $x\in A, y\in B$ such that $L_1-\epsilon<x, L_2-\epsilon$  by important property  of sup.

$L_1+L_2-2\epsilon<x+x\le L_1+L_2$, $x+y\in C$, since for all tilde $\epsilon>0$ there exists $z\in C$ such that $L_1+L_2-\epsilon<z\le L_1+L_2$ so $L_1+L_2$ is the supremum of C
\subsection*{2}
Let $S,T$ be subsets of R, nonempty, bounded above. if for all s in S and t in T, s is less than or equal to t then supremum of S is less than or equal to supremum of T (exercise)

\subsubsection*{proposition}
$\mathbb{Z}^+$ is unbounded above.
\subsubsection*{proof}
$\mathbb{Z}^+$ is a subset of R nonempty, if Z+ were bounded above it would have a supremum m. by the important property of supremum there exists som x in Z+ such that m-1 is less than x is less than or equal to m, but then m is less than x+1 which  is in Z+ so we have a contradiction
\subsubsection*{corollary}
for all x in R there exists an n in Z+ such  that x is less than or equal to n.
\subsubsection*{proposition}
archimedean property of R. page 12

for all x greater than 0, y in R there exists some n in Z+ such that nx is greater than y.
\subsubsection*{proof}
apply previous corollary, with x replaced by $\frac{y}{x}\Box$
\subsection*{defintion of absolut value}
also on page 12
$\abs{x}=\{x,x\ge0, -x,x<0$
\subsubsection*{properties}
\begin{enumerate}
\item
if $a\ge0, \abs{x}\le a$ iff $-a\le x\le a$
\item
for all $x,y\in \mathbb{R}, \abs{x+y}\le\abs{x}+\abs{y}$ (triangle inequality)
\item
same as above but with more than two numbers
\item
reverse triangle $\abs{\abs{a}-\abs{b}}\le \abs{a-b}$
\item
$\abs{xy}=\abs{x}\abs{y}$, $\abs{x^{-1}}=\abs{x}^{-1}$
\end{enumerate}
\subsubsection*{proof 1}
assume $\left\lvert x\right\rvert\le a$

cases
\begin{enumerate}
\item
$x\ge 0$ then $\abs{x}=x$ so $0\le x\le a$ since $x\ge 0$ and $-a<0, -a\le x$ so $-a\le x\le a$
\item
$x<0$ then $\left\lvert x\right\rvert=-x\le a$ so $x\ge -a$ since $x<0$ and $a>0$ $x\le a$ so $-a\le x\le a$$\Box$
\end{enumerate}
on the way back

assume $-a\le x\le a$ if $x\ge 0$ $x=\left\lvert x\right\rvert$ hence $-a\le\abs{x}\le a$ then in particular $\abs{x}\le a$

if $x<0$

\section*{cauchy-schwartz inequality}
for every $a_k,b_k\in\mathbb{R}$
\begin{align*}
  \left(\sum\limits_{k=1}^n{a_kb_k}\right)^2\le\left(\sum\limits_{k=1}^n{{a_k}^2}\right)^{\frac{1}{2}}\left(\sum\limits_{k=1}^n{{b_k}^2}\right)^{\frac{1}{2}}
\end{align*}
with equality iff $\exists x \in \mathbb{R}$ such that $a_kx+b_k=0$ for all $k=1,...,n$
\subsubsection*{proof}
\begin{align*}
  \sum\limits_{k=1}^n{\left(a_kx+b_k\right)^2}\ge 0, \forall x \in \mathbb{R}
\end{align*}
$Ax^2+Bx+C\ge0$, $A=\sum\limits_{k=1}^n{a_k^2}$
\end{document}
