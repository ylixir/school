\documentclass[letterpaper]{article}

\usepackage{fullpage}
\usepackage{nopageno}
\usepackage{amsmath}
\usepackage{amssymb}
\allowdisplaybreaks

\newcommand{\abs}[1]{\left\lvert #1 \right\rvert}

\begin{document}
\title{Notes}
\date{September 10, 2014}
\maketitle
\section*{2.6.3 the nested intervals lemma}
$I_n$'s need to be closed:
\begin{align*}
  I_n=(0,\frac{1}{n})\\
\end{align*}
this is not closed
\subsection*{2.6.I}
Let $\{a_n\}_{n=1}^\infty$ be a bounded sequence. we define $b_n=\sup_{k\ge n}\{a_k\}$. $b_1=\sup_{i\ge 1}\{a_i\},b_2=\sup_{i\ge 2}\{a_i\},\dots$. Let $X,Y$ be sets, $X\subseteq Y$. $\sup X\le \sup Y$, $\inf X\ge \inf Y$. Because $\{a_n\}_{n=1}^\infty$ is bounded, $\{b_n\}_{n=1}^\infty$ is bounded. $b_n$ is a decreasing sequence. $\lim_{n\to\infty}b_n=\inf b_n$.

definition: the limsup of a bounded sequence $\{a_n\}$ is $\inf_{n\in\mathbb{N}}\left(\sup_{k\ge n}\{a_k\}\right)=\lim_{n\to\infty\left(\sup_{k\ge n}\{a_k\}\right)}$ and it always exists.

\subsubsection*{example}
let $a_n=(-1)^n$. $b_n=1\forall n$ and $\lim_{n\to\infty}b_n=1$. $\limsup_{n\to\infty}(-1)^n=1$

definition: the $\liminf_{n\to\infty}a_n=\sup_{n\in\mathbb{N}}\left(\inf_{k\ge n}a_n\right)$ always exists.

we can extend this concept to unbounded sequences and allow the values $\pm\infty$ for $\limsup$ and $\liminf$ 

increasing sequence of supremum negative infinity: all negative infinity

\subsubsection*{proposition}
let $\{a_n\}_{n=1}^\infty$ be a sequence. then $\lim_{n\to\infty}a_n$ exists $\Leftrightarrow\liminf_{n\to\infty}a_n$

we are assuming that $\lim_{n\to\infty}a_n=L\in\mathbb{R}, |a_n-L|<\epsilon\forall\epsilon>0$. for $\epsilon>0\exists N, a_n\in(L-\epsilon,L+\epsilon)$ if $n\ge N$
\end{document}
