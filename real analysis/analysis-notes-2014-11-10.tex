\documentclass[letterpaper]{article}

\usepackage{fullpage}
\usepackage{nopageno}
\usepackage{amsmath}
\usepackage{amssymb}
\usepackage[utf8]{inputenc}
\allowdisplaybreaks

\newcommand{\abs}[1]{\left\lvert #1 \right\rvert}

\begin{document}
\title{Notes}
\date{November 11, 2014}
\maketitle
$f:\mathbb{R}^n\to\mathbb{R}^m$ is continous iff $\forall C$ is closed in $R^{m}$ then $f^{-1}(C)$ is closed in $\mathbb{R}^n$.o

assume $f$ is continuous. let $c\in \mathbb{R}^n$ be closed. we want to show $f^{-1}(C)$ is closed.

first we pick a sequence $x_k\in f^{-1}(C)$. or $f(x_k)\in C$. we are picking $x_k$ so that it converges to something (say $y$). now because $f(x)$ is continous, then $f(x_k)\to f(y)$. and because $f(y)\in C$ then $C$ is closed.

\section*{true false}
$f:\mathbb{R}^n\to\mathbb{R}^m$ is continous
\begin{enumerate}
\item
fi $A\subseteq \mathbb{R}^n$ is open then $f(A)$ is open
\item
fi $A\subseteq \mathbb{R}^n$ is closed then $f(A)$ is closed
\item
fi $A\subseteq \mathbb{R}^n$ is compact then $f(A)$ is compact
\item
fi $A\subseteq \mathbb{R}^m$ is open then $f^{-1}(A)$ is open
\item
fi $A\subseteq \mathbb{R}^m$ is compact then $f(A)$ is compact
\end{enumerate}

unions, intersections of $f$or $f^{-1}$ over two sets. $f(A^{c})=(f(A))^{c}$ and same with $f^{-1}$.

\section*{5.5.A}
note that $g'(x)$ is unbounded on $(0,\infty)$. so we cannot sed the ``bounded slope '' argueent for uniform continuity. need to show $\exists r=r(\varepsilon)$ such that if $|x-y|<r$ then $|\sqrt{x}-\sqrt{y}|<\varepsilon$

if we prove this then $\sqrt{x}-\sqrt{y}\le \sqrt{x-y}$ $\sqrt{r}=\varepsilon$

\begin{align*}
  x-y=(\sqrt{x}-\sqrt{y})(\sqrt{x}+\sqrt{y})\ge (\sqrt{x}-\sqrt{y})^2\\
  \sqrt{x-y}\ge \sqrt{x}-\sqrt{y}
\end{align*}
\section*{5.5.D}
show that $f(x)=x^p$ is not uniformally continuous on $\mathbb{R}$ is $p>1$

\begin{align*}
  \frac{x^p-y^p}{x-y}=|xx^{p-1}-y^p|=xx^{p-1}-x^{p-1}y+x^{p-1}y-y^{p}|le |x^{p-1}||x-y|+|y||x^{p-1}-y^{p-1}|
\end{align*}
\section*{T is continouous, A is closed, it T(A) closed?}
no, hint, look for $T:\mathbb{R}^2\to \mathbb{R}$ or $\mathbb{R}^2$
 where $T$ is a linear transform (matrix), example $f(x)=\frac{1}{x^2+1}$ on $\mathbb{R}$, $f(\mathbb{R})=(0.1]$

 \section*{5.4.J}
 let $A$ be compact in $\mathbb{R}^n$. show that $\forall x\in \mathbb{R}^n\exists a\in A$ such that $a$ is the closest point from $A$ so $x$. that is $||x-a||\le||x-y||\forall y \in A$.
 \subsubsection*{proof}
 $x$ is a fixed point. $f(y)=||x-y||$. $A$ is compact, if we showthat $f$ is continous then $f$ has  min on $A$.

 if $||z-y||<r$ then $|f(z)-f(y)|<\varepsilon$. $| ||x-z||-||x-y|| |\le ||z-y||$. need $r=\epsilon$
\section*{5.5.HI}
let $f$ be continuous on $(0,1]$ show $f$ is uniformallly continuous iff $\lim_{x\to 0^+}f(x)$ exists.

asssuming $f$ is uniformally continuous, $|f(x)-f(y)|<\varepsilon$ whenever $|x-y|<r$. take $x_k$ any decreasing sequence converging to zero. we use the cauchy condition. $\forall r\exists N$ such that $|x_k-x_l|<r$ if $k,l\ge N$. so $|f(x_k)-f(x_l)|<\varepsilon$ then $f(x_k)$ is cauchy. and so $\lim_{k\to\infty}f(x_k)=L=\lim_{x\to0^+}f(x)$

other way, we need to shwo that $r$ does not depend on $\varepsilon$
\end{document}
