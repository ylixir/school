%shell-escape
\documentclass[letterpaper]{article}

\usepackage{fullpage}
\usepackage{nopageno}
\usepackage{amsmath}
\usepackage{amssymb}
\usepackage{gnuplottex}
\allowdisplaybreaks

\newcommand{\abs}[1]{\left\lvert #1 \right\rvert}

\begin{document}
\title{Homework 4}
\date{September 26, 2014}
\author{Jon Allen}
\maketitle
Section 2.7: G, H, J*

2.8: B, C, E*, H (for C, you can use what you know about series from Calculus 2).
\renewcommand{\labelenumi}{2.\arabic{enumi}}
\renewcommand{\labelenumii}{\Alph{enumii}.}
\renewcommand{\labelenumiii}{(\alph{enumiii})}
\begin{enumerate}
\setcounter{enumi}{6}
\item
  \begin{enumerate}
  \setcounter{enumii}{6}
  %2.7G
  \item
  Let $(x_{n})_{n=1}^\infty$ be a sequence of real numbers.
  Suppose that there is a real number $L$ such that $L=\lim\limits_{n\to\infty}{x_{3n-1}}=\lim\limits_{n\to\infty}{x_{3n+1}}=\lim\limits_{n\to\infty}{x_{3n}}$.
  Show that $\lim\limits_{n\to\infty}{x_n}$ exists and equals $L$.

  Let us take any $\varepsilon>0$.
  Now we find $N_1$ such that $\left\lvert x_{3n}-L\right\rvert<\varepsilon$ for all $n\ge N_1$.
  And find $N_2$ such that $\left\lvert x_{3n+1}-L\right\rvert<\varepsilon$ for all $n\ge N_2$. 
  Finally find $N_3$ such that $\left\lvert x_{3n-1}-L\right\rvert<\varepsilon$ for all $n\ge N_3$.
  My brain hurts and I don't feel like working out the logic for an optimal $N$ so let's take $N=3\cdot\max\{N_1,N_2,N_3\}+1$. Then $\left\lvert x_n-L\right\rvert<\varepsilon$ for all $n\ge N$ and we have our result.
  %2.7H
  \item
  Let $(x_{n})_{n=1}^\infty$ be a sequence in $\mathbb{R}$.
  Suppose that there is a number $L$ such that every subsequence $\left(x_{n_k}\right)_{k=1}^\infty$ has a subsubsequence $\left(x_{n_{k(l)}}\right)_{l=1}^\infty$ with $\lim\limits_{n\to\infty}{x_{n_{k(l)}}=L}$. Show that the whole sequence converges to $L$. {\scshape  Hint:} If not, you could find a subsequence bounded away from $L$.

  If $(x_n)$ does not converge to $L$ then there would be a subsequence that doesn't converge on $L$ and which does not contain a subsequence that converges to $L$. Because we can always find such a subsubsequence we know the whole thing must converge.
  \setcounter{enumii}{9}
  %2.7j
  \item
  Suppose $(x_n)_{n=1}^\infty$ is a sequence in $\mathbb{R}$, and that $L_k$ are real numbers with $\lim\limits_{k\to\infty}{L_k}=L$.
  If for each $k\ge1$, there is a subsequence of $(x_n)_{n=1}^\infty$ converging to $L_k$, show that some subsequence converges to $L$. {\scshape Hint:} Find an increasing sequence $n_k$ such that $\left\lvert x_{n_k}-L\right\rvert<\frac{1}{k}$.

  We choose $(x_{n_k})$ from $(x_n)$ such that as $k$ increases, $x_{n_k}$ comes from the subsequence of $(x_n)$ that approaches $L_k$ and is bigger than $x_{n_{k-1}}$. So then as $k$ increases, $L_k$ approaches $L$ and so will $x_{n_k}$.
  \end{enumerate}
\item
  \begin{enumerate}
  \setcounter{enumii}{1}
  %2.8B
  \item
  Give a sequence $(a_n)$ such that $\lim\limits_{n\to\infty}\left\lvert a_n-a_{n+1}\right\rvert=0$, but the sequence does not converge.

  $a_n=\log n$
  %2.8C
  \item
  Let $(a_n)$ be a sequence such that $\lim\limits_{N\to\infty}{\sum_{n=1}^N{\left\lvert a_n-a_{n+1}\right\rvert}}<\infty$.
  Show that $(a_n)$ is Cauchy.

  First we note that $\lim\limits_{N\to\infty}{\sum_{n=1}^N{\left\lvert a_n-a_{n+1}\right\rvert}}\ge0$ because we are adding only nonnegative terms.
  So our limit is in the reals (particularly it's not $-\infty)$.
  Lets call it $L$.
  \begin{align*}
    \lim\limits_{N\to\infty}{\sum_{n=1}^N{\left\lvert a_n-a_{n+1}\right\rvert}}&=L
  \end{align*}
  So then for all $\varepsilon\in\mathbb{R}$ where $\varepsilon>0$ there exists some $N_1\in\mathbb{N}$ such that for all $N\ge N_1$ we have
  \[\left\lvert{\sum_{n=1}^N{\left\lvert a_n-a_{n+1}\right\rvert}}-L\right\rvert<\varepsilon\]
  And so then playing a bit when $N>N_1$ we come up with
  \begin{align*}
    \left\lvert\sum_{n=1}^{N_1}{\left\lvert a_n-a_{n+1}\right\rvert}+\sum_{n=N_1+1}^N{\left\lvert a_n-a_{n+1}\right\rvert}-L\right\rvert&<\varepsilon\\
    \left\lvert\sum_{n=N_1+1}^{N}{\left\lvert a_n-a_{n+1}\right\rvert}-\left(L-\sum_{n=1}^{N_1}{\left\lvert a_n-a_{n+1}\right\rvert}\right)\right\rvert&<\varepsilon\\
  \end{align*}
  Note that because our series is weakly increasing, $L$ must be greater than our series for all $N$. This leads us to the following observations.
  \begin{align*}
    L-\sum_{n=1}^{N_1}{\left\lvert a_n-a_{n+1}\right\rvert}&<\varepsilon\\
    \sum_{n=1}^{N_1}{\left\lvert a_n-a_{n+1}\right\rvert}+\sum_{n=N_1+1}^N{\left\lvert a_n-a_{n+1}\right\rvert}&<L\\
    \sum_{n=N_1+1}^N{\left\lvert a_n-a_{n+1}\right\rvert}&<L-\sum_{n=1}^{N_1}{\left\lvert a_n-a_{n+1}\right\rvert}\\
    \sum_{n=N_1+1}^N{\left\lvert a_n-a_{n+1}\right\rvert}&<L-\sum_{n=1}^{N_1}{\left\lvert a_n-a_{n+1}\right\rvert}<\varepsilon\\
    \sum_{n=N_1+1}^N{\left\lvert a_n-a_{n+1}\right\rvert}&<\varepsilon\\
    \left\lvert\sum_{n=N_1+1}^N{ a_n-a_{n+1}}\right\rvert<\sum_{n=N_1+1}^N{\left\lvert a_n-a_{n+1}\right\rvert}&<\varepsilon\\
    \left\lvert(a_{N_1+1}-a_{N_1+2})+(a_{N_1+2}-a_{N_1+3})+\dots+(a_{N-1}-a_{N})\right.&\left.+(a_{N}-a_{N+1})\right\rvert<\varepsilon\\
    \left\lvert a_{N_1+1}-a_{N+1}\right\rvert&<\varepsilon\\
  \end{align*}
  And so we see that for every $\varepsilon>0$ there is an integer $N$ such that $\left\lvert a_n-a_m\right\rvert<\epsilon$ for all $m,n\ge N$.
  $\Box$
  \setcounter{enumii}{4}
  %2.8E
  \item
  Suppose that $(a_n)$ is a sequence such that $a_{2n}\le a_{2n+2}\le a_{2n+3}\le a_{2n+1}$ for all $n\ge0$. Show that this sequence is Cauchy if and only if $\lim\limits_{n\to\infty}\left\lvert a_n-a_{n+1}\right\rvert=0$

  Lets partition our sequence into even and odd elements.
  We notice that $a_{2n}\le a_{2N+1}$ for any $N\in \mathbb{N}$ and for all $n\ge N$.
  Further we notice that $a_{2n+1}\ge a_{2N}$ for any $N\in\mathbb{N}$ and for all $n\ge N$.
  We also observe that the evens are monotone increasing ($a_{2n}\le a_{2n+2}$) while the odds are monotone decreasing ($a_{2n+1}\ge a_{2n+3}$).

  Now because the evens are bounded above by $a_{1}$ and the odds are bounded below by $a_0$ we know that each of our sequences has a limit.
  Lets say $\lim\limits_{n\to\infty}a_{2n}=L_2$ and $\lim\limits_{n\to\infty}a_{2n+1}=L_1$.
  By definition then for all $\frac{\varepsilon}{2}>0$ there exists $N\in \mathbb{N}$ such that for all $n\ge N$
  \begin{align*}
    \left\lvert a_{2n}-L_2\right\rvert&<\frac{\varepsilon}{2}\\
    \left\lvert a_{2n+1}-L_1\right\rvert&<\frac{\varepsilon}{2}\\
    \left\lvert a_{2n}-L_2\right\rvert+\left\lvert a_{2n+1}-L_1\right\rvert&<\varepsilon\\
    \left\lvert a_{2n}-L_2\right\rvert+\left\lvert L_1-a_{2n+1}\right\rvert&<\varepsilon\\
    \left\lvert a_{2n}-L_2+L_1-a_{2n+1}\right\rvert\le\left\lvert a_{2n}-L_2\right\rvert+\left\lvert L_1-a_{2n+1}\right\rvert&<\varepsilon\\
    \left\lvert (a_{2n}-a_{2n+1})-(L_2-L_1)\right\rvert&<\varepsilon\\
    \left\lvert \left\lvert a_{2n}-a_{2n+1}\right\rvert -\left\lvert L_2-L_1\right\rvert \right\rvert\le\left\lvert (a_{2n}-a_{2n+1})-(L_2-L_1)\right\rvert&<\varepsilon\\
    \left\lvert \left\lvert a_{2n}-a_{2n+1}\right\rvert -\left\lvert L_2-L_1\right\rvert \right\rvert&<\varepsilon\\
  \end{align*}
  So then $\lim\limits_{n\to\infty}\left\lvert a_{2n}-a_{2n+1}\right\rvert=\left\lvert L_2-L_1\right\rvert$.
  Similarly we can show $\lim\limits_{n\to\infty}\left\lvert a_{2n+1}-a_{2n+2}\right\rvert=\left\lvert L_2-L_1\right\rvert$.
  And so we have that $\lim\limits_{n\to\infty}\left\lvert a_n-a_{n+1}\right\rvert=\left\lvert L_2-L_1\right\rvert$.

  Let us assume that $\lim\limits_{n\to\infty}\left\lvert a_n-a_{n+1}\right\rvert=0$. Then $L_1=L_2=L$. Now because the odds provide an upper bound for the evens, and the evens provide a lower bound for the odds, we know from the squeeze theorem that $\lim\limits_{n\to\infty}a_n=L$. So our sequence is convergent and is therefore Cauchy.

  Now let's assume that our sequence is Cauchy. Then for every $\varepsilon>0$ there is an integer $N$ such that $\left\lvert a_n-a_m\right\rvert<\varepsilon$ for all $m,n\ge N$. In particular
  \begin{align*}
    \left\lvert a_n-a_{n+1}\right\rvert&<\varepsilon\\
    \left\lvert |a_n-a_{n+1}|-0\right\rvert\le\left\lvert a_n-a_{n+1}\right\rvert&<\varepsilon
  \end{align*}
  And so $\lim\left\lvert a_n-a_{n+1}\right\rvert=0$.
  $\Box$
  \setcounter{enumii}{7}
  %2.8H
  \item
  Let $a_0=0$ and set $a_{n+1}=\cos(a_{n})$ for $n\ge0$. Try this on your calculator (use radian mode!).
    \begin{enumerate}
    \item
    Show that $a_{2n}\le a_{2n+2}\le a_{2n+3}\le a_{2n+1}$ for all $n\ge 0$.

    We know that $a_{n+1}=\cos a_n$. First we note that $a_n$ is bounded by $[0,1]$. We know this because $\cos 0$ gives us a $1$ back, which is the maximum value $\cos a_n$ can give us. Furthermore, $0\le\cos x\le1$ for all $0\le x\le \pi/2$. So $a_n$ can never escape the $[0,1]$ bounds. Lets assume that $a_n<L$ where $0=L-\cos L$. Then $\cos a_n>L$. Similarly, if $a_n>L$ then $\cos a_n<L$. So if $a_n<L$ then $a_{n+1}=\cos a_n>L$ and $a_{n+2}=\cos \cos a_n<L$. Because $a_0=0$ we know that $a_0<L$ and by extension $a_{2n}<L$ while $a_{2n+1}>L$ for all $n$.

    Now we assume that $a_2n<a_{2n+2}$. So $a_{2n}<\cos \cos a_{2n}$ and $\cos a_{2n}>\cos \cos \cos a_{2n}$ because in our domain, as the angle gets smaller, the cosine gets larger. Thus $a_{2n+1}>\cos \cos a_{2n+1}=a_{2n+3}$ and $a_{2n+2}<a_{2n+4}$.
    Notice that $\cos 0=1$ and $\cos 1>\cos \pi/2=0$. So then $a_0<a_2$. Good enough for a basis.
    And induction then gives us our result for all elements in the sequence.
    \item
    Use the Mean Value Theorem to find an explicit number $r<1$ such that $\left\lvert a_{n+2}-a_{n+1}\right\rvert\le r\left\lvert a_n-a_{n+1}\right\rvert$ for all $n\ge0$. Hence show that this sequence is Cauchy.

    \begin{align*}
      \frac{\left\lvert a_{n+2}-a_{n+1}\right\rvert}{\left\lvert a_n-a_{n+1}\right\rvert}&=
      \left\lvert \frac{\cos \cos a_{n}-\cos a_{n}}{\cos a_n-a_n}\right\rvert\\
      1&\ge\left\lvert \frac{(\sin x +1)\cos(\cos(x))+\sin(x)(\cos x -x)\sin(\cos x)-\cos x-x\sin x}{(x-\cos x)^2}\right\rvert\\
    \end{align*}
    And solve for $x$

    I'd like to note that the result in part (a) combined with the squeeze theorem actually already shows that it has a limit and is Cauchy. It even tells us what the limit is, as shown in the next part.
    \item
    Describe the limit geometriclly as the intersection point of two curves.

    It is the intersection of $\cos x$ and $x$

    \begin{gnuplot}
      plot [0:1] x, cos(x)
    \end{gnuplot}
    \end{enumerate}
  \end{enumerate}
\end{enumerate}
\end{document}
