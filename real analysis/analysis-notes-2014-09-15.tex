\documentclass[letterpaper]{article}

\usepackage{fullpage}
\usepackage{nopageno}
\usepackage{amsmath}
\usepackage{amssymb}
\usepackage[utf8]{inputenc}
\allowdisplaybreaks

\newcommand{\abs}[1]{\left\lvert #1 \right\rvert}

\begin{document}
\title{Notes}
\date{September 15, 2014}
\maketitle
\section*{2.7 subsequences}
subsequences have infinite increasing sequence of index n.

\subsection*{deciding convergence}
\begin{enumerate}
\item
squeeze thm.
\item
monotone convergence thm: if a sequence is monotone increasing/decreasing and bounded above/below, then it converges.
\item
if it is the sum, product, square root, quotient, etc, of convergent sequences, then it is convergent.
\item
cesàro sums converge ``better'' than original sequence
\end{enumerate}
\subsection*{2.7.2 bolzano-weierstrass theorem}
every  bounded sequence of real numbers has a convergent subsequence.

think of it as a fixup of the monotone convergent sequence, when the sequence isn't monotone

sequence $\{a_n\}$ is bounded by $B\in\mathbb{R}$ so $-B\le a_n\le B$.

if $\{a_n\}$ only takes finitely many values, then nescessarily, one of them can be taken innfinitely many times. this is our constant subsequence, which is convergent.

if $\{a_n\}$ takes infinitely many values, then split $[-B,B]$ into halves, $[-B,0],[0,B]$. One of these halves contains infinitely many values of the sequence.
Call this half $I_1$. Split $I_1$ into halves, call the one with infinitely many values $I_2$ and so on. $\{I_n\}$ is a sequence of invervals and each $I_n$ contains infinitely many values of $a_n$. $I_{n+1}\subseteq I_n$. $\left\lvert I_n\right\rvert=\frac{B}{2^{n-1}}{\rightarrow}({n\to\infty})\to 0$

by nestedintevals theorem $\cap_{n=1}^\infty I_n\ne\emptyset$, $[a,b]\subseteq\cap_{n=1}^\infty I_n$ Because $|I_n|\to0$ then $a=b$ and it convergences on this point. because each $I_n$ contains some $x_n\in\{a_n\}$ and we can choose them such that 
\end{document}
