\documentclass[letterpaper]{article}

\usepackage{fullpage}
\usepackage{nopageno}
\usepackage{amsmath}
\usepackage{amssymb}
\usepackage[utf8]{inputenc}
\allowdisplaybreaks

\newcommand{\abs}[1]{\left\lvert #1 \right\rvert}

\begin{document}
\title{Notes}
\date{November 24, 2014}
\maketitle
\section*{example 5.6.E}
let $T(x)$ be the temperature at point $x$ on surface of planet where we assume that the planet is a sphere. $T$ is a continuous function

Then $\exists x$ on the surface of the planet such that $T(x)=T(-x)$.
\subsection*{proof}
$f(x)=T(x)-T(-x)$ and $f(-x)=-f(x)$ for all $x$ on surface. either $f(x)=0\forall $ or $\exists x_0$ such that $f(x_0\ne0$ assume $f(x_0)>0$ then $f(-x_0)<0$ and there is a point in between where $f$ is 0.

\section*{example 5.6.f}
map from $[0,2\pi)\to\text{ unit circle}\to \mathbb{R}$
\end{document}
