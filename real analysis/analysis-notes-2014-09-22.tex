\documentclass[letterpaper]{article}

\usepackage{fullpage}
\usepackage{nopageno}
\usepackage{amsmath}
\usepackage{amssymb}
\usepackage[utf8]{inputenc}
\allowdisplaybreaks

\newcommand{\abs}[1]{\left\lvert #1 \right\rvert}

\begin{document}
\title{Notes}
\date{September 22, 2014}
\maketitle
\section*{cont'd example from last time}
$a_{n+1}=1+\frac{1}{a_n}$ and is fibanocci sequence. terms are back and forth but converging. it is clear that $a_n>1\forall n$.
\begin{align*}
  \left\lvert a_{n+1}-a_n\right\rvert&=\left\lvert 1+\frac{1}{a_n}-\left(1+\frac{1}{a_{n-1}}\right)\right\rvert\\
  &=\left\lvert\frac{1}{a_n}-\frac{1}{a_{n-1}}\right\rvert\\
  &=\frac{\left\lvert a_n-a_{n-1}\right\rvert}{a_n\cdot a_{n-1}}\text{with $a_n>1$ so}\\
  \left\lvert a_{n+1}-a_n\right\rvert&<\left\lvert a_{n}-a_{n-1}\right\rvert\\
\end{align*}
does this mean we have a limit?
contractive sequence has the property
$|a_{n+1}-a_n|<r|a_n-a_{n-1}|$ wher $r\in(0,1)$

no, it is possible in principle that $\lim|a_{n+1}-a_n|=b>0$ and that would mean that $\{a_n\}$ is not convergent

\begin{align*}
  a_na_{n+1}&=a_n(1+\frac{1}{a_n}=a_n+1>2\\
  \left\lvert a_{n+1}-a_n\right\rvert&<\frac{1}{2}\left\lvert a_{n}-a_{n-1}\right\rvert\\
\end{align*}
so it is convergent because $\frac{1}{2}\in(0,1)$

\section*{2.8.D}
pick a sequence of $\varepsilon$, $\varepsilon_n=\{\frac{1}{2^n}\}$. given $\varepsilon_1=\frac{1}{2}$ there exists $N+1\in\mathbb{N}$ such that $|a_m-a_n<\varepsilon_1$ if $m,n\ge N_1$, $\varepsilon_2=\frac{1}{4}, \exists N_2$ st $\left\lvert a_m-a_n\right\rvert<\varepsilon_2$ and so on.

so $|a_{\mathbb{N}_{n+1}}-a_{N_{n}}|<\frac{1}{2^n}$ and so the sum is less than 1 and we win.
\section*{convergent series}
given $(a_n)$ we consider the series $\sum\limits_{n=1}^\infty{a_n}$ let $s_n=\sum\limits_{k=1}^n{a_k}$ be the $n$th partial sum of the series. if $\lim s_n$ exists, we say that the series $\sum\limits_{k=1}^\infty{a_k}$is convergent

the following are equivalent
\begin{itemize}
\item
$\sum a_n$ is convergent
\item
$\forall\varepsilon>0\exists N$ st if $n\ge N, \left\lvert\sum\limits_{k=n+1}^\infty{a_k}\right\rvert<\varepsilon$
\item
$\forall\varepsilon>0\exists N$ st if $m,n\ge N, \left\lvert\sum\limits_{k=n+1}^m{a_k}\right\rvert<\varepsilon$
\end{itemize}
proofs are in the book

\section*{note}
if $\sum a_k<\infty$ then $\lim a_k=0$ but the converse is false, as shown by the harmonic series.

telescoping and geometric series are basically the only ones where we know how to find the sums

\section*{3.1.c}
if $\sum t_k$ is a convergent series of positive terms and $p>1$ show that $\sum t_k^p$ is convergent

the necessary condition is that $\lim t_k=0$. by the necessary condition $\exists N$ st $0\le t_k\le1\forall k\ge N$. Therefore $\sum\limits_{k=N}^\infty{t_k^p}\le\sum\limits_{k=N}^\infty{t_k}<\infty$ 
\section*{3.1.d}
if $\lim |a_n|=0$ then there exists $\sum a_{n_k}<\infty$

example argument: harmonic series
\end{document}

