\documentclass[letterpaper]{article}

\usepackage{fullpage}
\usepackage{nopageno}
\usepackage{amsmath}
\usepackage{amssymb}
\usepackage[utf8]{inputenc}
\allowdisplaybreaks

\newcommand{\abs}[1]{\left\lvert #1 \right\rvert}

\begin{document}
\title{Notes}
\date{September 29, 2014}
\maketitle
if $\liminf x_n=L$ then there exists $\{x_{n_k}\}$ such that $\lim x_{n_k}=L$

$l=\liminf x_n=\lim(inf\{\underbrace{x_{n_1},x_{n_2},x_{n_3},\dots}_{c_n}\}$

why not just let $c_n$ be the subseuence? because $c_n$ may not be equal to any of the $x_k$ in the sequence

$c_n=\inf\{x_{n_1},x_{n_2},\dots\}$ give $\varepsilon=2^{-n}$ there exists $x_{n_k}\in\{x_{n_1},x_{n+1},x_{n+2},\dots\}$ such that $\left\lvert c_n-x_{n_k}\right\rvert<2^{-n}$ by def of infinum

we has a sequence $\{c_n\}$ given $\varepsilon>0$ there exists $N$ such that $\left\lvert c_n-L\right\rvert<\varepsilon$ if $n\ge N$. we approximate each $c_n$ by some $x_{n_k}$ from the oricinal sequence sutch that ....

\section*{convergence test for series}
first we talk about series with positive terms $\sum\limits_{k=1}^\infty{a_k}$, $s_n=\sum\limits_{k=1}^n{a_k}$. So if $s_n$ is bounded about then the series is convergent. and if not, it is divergent.

geometric series $\sum\limits_{n=0}^\infty{r^n}$ is convergent if $\left\lvert r\right\rvert<1$. $s_n=\sum\limits_{k=0}n{r^k}=1+r+r^2+\dots+r^n, rs_n=r+r^2+r^3+\dots, sn-rSn=1-r^{n+1}$

$s_n=\frac{1-r^{n+1}}{1-r}\to\frac{1}{1-r}$


\section*{comparison test}
if $\forall n, |a_n|\le b_n$
\begin{itemize}
\item
  if $\sum\limits_{n=1}^\infty{b_n}$ is convergent then $\sum\limits_{n=1}^\infty{a_n}$ is convergent,
  \item
  if $\sum\limits{a_n}$ is divergent, so is $\sum\limits{b_n}$.
\end{itemize}
\section*{3.2.b}
show that if $\left(|a_n|\right)_{n=1}^\infty$ is summable then so is $\left(a_n\right)_{n=1}^\infty$.
\begin{align*}
  \sum\limits_{k=n+1}^m{|a_k|}&<\varepsilon\text{ for all }N\le n\le m \text{ because is is summable}\\
  \left\lvert\sum\limits_{k=n+1}^m{a_k}\right\rvert&\le\sum\limits_{k=n+1}^m{|a_k|}<\varepsilon
\end{align*}
so then $\sum\limits{a_k}$ is also cauchy and summable

\section*{cauchy-schwartz inequality}
$\sum\limits_{k=1}^n{a_kb_k}\le \left(\sum\limits_{k=1}^n{a_k^2}\right)^{1/2}\left(\sum\limits_{k=1}^n{b_k^2}\right)^{1/2}$

\section*{3.2.f}

\section*{leibniz test for alternating series}
if $\{a_n\}$ is a monotone decreasing sequence of positive terms with the $\lim a_n=0$ then $\sum\limits_{n=1}^\infty{(-1)^na_n}$ is convergent
\end{document}

