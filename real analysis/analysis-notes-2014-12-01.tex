\documentclass[letterpaper]{article}

\usepackage{fullpage}
\usepackage{nopageno}
\usepackage{amsmath}
\usepackage{amssymb}
\usepackage[utf8]{inputenc}
\allowdisplaybreaks

\newcommand{\abs}[1]{\left\lvert #1 \right\rvert}

\begin{document}
\title{Notes}
\date{December 1, 2014}
\maketitle
\section*{note on homework}
differentiable with bounded $f'$ means lipschitz and differentiable

lipschitz mean continuous and differentiable means continuous

continuous means continuous at $x_0$ and differentiable at $x_0$ means continous at $x_0$

analytic($c^\omega$)$\to c^\infty\to\dots\to c^{m+1}\to c^m\to\dots\to c^2\to c^1\to$ differentiable

\section*{riemann integrable}
$f:[a,b]\to\mathbb{R}$ bounded. then $f$ is Riemann integrable if $\sup L(f,p)=\inf U(f,p)=\int_a^b{f(x)\;\mathrm{d}x}$

\subsection*{reimann condition}
if $f$ is riemann integrable $\Leftrightarrow\forall \varepsilon>0\exists P_\varepsilon$ such that $U(f,P_{\varepsilon}-L(f,P_\varepsilon)<\varepsilon$.

\subsection*{6.3.6 theorem}
not covered, just variations on the riemann condition

\section*{examples}
which are integrable, which are not?

$f(x)=\chi_{\mathbb{Q}_n[0,1]}=\begin{cases}1&x\in\mathbb{Q}_{n[0,1]}\\0&\text{otherwise}\end{cases}$

because any interval contains a rational then inf is 1 and since any interval has an irrational then sup is 0.

there are other ways to do integrals such at riemann-stieltjes, lebesque, denjoy,feynmann

\section*{thm 6.3.7}
if $f$ is monotone on $[a,b]$ then $f$ is integrable on $[a,b]$
\section*{thm 6.3.8}
any continous function on $[a,b]$ is riemann integrable on $[a,b]$

\subsubsection*{proof}
function is compact, and so uniformally continuous. definition of unif cont.

choose a partition $p_\varepsilon$ such that $\delta_i<r$ ($r=$radius of continuity ball) $\forall i=0,\dots,n-1$ then on each $[x_i,x_{i+1}]$ we have $\max f(x)-\min f(x)<\varepsilon$ and so back to book proof


\section*{fundamental thm of calculus}
if $f$ is riemann integrable on$[a,b]$ then $F(x)=\int_a^x{f(x)\;\mathrm{d}x}$ is continuous. when $f$ is continuous at $x_0$ then $F$ is differentiable at $x_0$ and $F'(x_0)=f(x_0)$
\end{document}
