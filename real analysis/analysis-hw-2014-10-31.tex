%shell-escape
\documentclass[letterpaper]{article}

\usepackage{fullpage}
\usepackage{nopageno}
\usepackage{amsmath}
\usepackage{amssymb}
\usepackage{gnuplottex}
\allowdisplaybreaks

\newcommand{\abs}[1]{\left\lvert #1 \right\rvert}

\begin{document}
\title{Homework 7}
\date{October 31, 2014}
\author{Jon Allen}
\maketitle
Section 4.4  A, B, E*, I*
\renewcommand{\labelenumi}{4.\arabic{enumi}}
\renewcommand{\labelenumii}{\Alph{enumii}.}
\renewcommand{\labelenumiii}{(\alph{enumiii})}
\begin{enumerate}
\setcounter{enumi}{3}
\item
  \begin{enumerate}
  %4.4A
  \item
    Which of the following sets are compact?
    \begin{enumerate}
    \item
      $\{(x,y)\in\mathbb{R}^2:2x^{2}-y^{2}\le 1\}$

      If we rewrite the condition for the set a little $2x^2\le y^2+1$ it becomes painfully obvious that this set is unbounded and therefore, it can not be compact.
      If ``obvious'' isn't a convincing argument, we could take the sequence $\boldsymbol{a}_n=(0,n), n\in \mathbb{N}$ which is contained by our set. Note that this sequence is unbounded and so can not have a convergent subsequence (it's essentially the same as $\mathbb{N}$ in example 4.4.2).
    \item
      $\{\boldsymbol{x}\in\mathbb{R}^n:2\le||\boldsymbol{x}||\le4\}$

      This set is basically  a donut around the origin. It is bounded by $B_4(0)$. Lets look at $\boldsymbol{a}_n=(2-\frac{1}{n},0)$. Note that this sequence lies in the complement of our set, but has a limit of $(2,0)$ which is outside of the complement of our set. So  then the complement of our set is open, and so our set is closed. Being closed and bounded, it is compact.
    \item
      $\{(e^{-x}\cos x,e^{-x}\sin x):x\ge 0\}\cup\{(x,0):0\le x\le 1\}$
    \item
      $\{(e^{-x}\cos \theta,e^{-x}\sin \theta):x\ge0,0\le\theta\le2\pi\}$

      If we fix $x=0$ and vary $\theta$ we see that this set is just the cube $[-1,1]^2$. Cubes are compact subsets of $\mathbb{R}^n$.
    \end{enumerate}
  %4.4B
  \item
    Give an example to show that Cantor's Intersection Theorem would not be true if compact sets were replaced by closed sets.

  \setcounter{enumii}{4}
  %4.4E
  \item
    \begin{enumerate}
    \item
      Show that the sum of a closed subset and a compact subset of $\mathbb{R}^n$ is closed. Recall that $A+B=\{\boldsymbol{a}+\boldsymbol{b}:\boldsymbol{a}\in A\text{ and }\boldsymbol{b}\in B\}$.
    \item
      Is this true for the sum of two compact sets and a closed set?
    \item
      Is this true for the sum of two closed sets?
    \end{enumerate}
  \setcounter{enumii}{8}
  %4.4I
  \item
    Let $A$ and $B$ be {\em disjoint} closed subsets of $\mathbb{R}^n$. Define
    \[d(A,B)=\inf\{||\boldsymbol{a-b}||:\boldsymbol{a}\in A,\boldsymbol{b}\in B\}.\]
    \begin{enumerate}
    \item
      If $A=\{\boldsymbol{a}\}$ is a singleton, show that $d(A,B)>0$.
    \item
      If $A$ is compact, show that $d(A,B)>0$.
    \item
      Find an example of two disjoint closed sets in $\mathbb{R}^2$ with $d(A,B)=0$
    \end{enumerate}
  \end{enumerate}
\end{enumerate}
\end{document}
