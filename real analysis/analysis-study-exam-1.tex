\documentclass[letterpaper]{article}

\usepackage{fullpage}
\usepackage{nopageno}
\usepackage{amsmath}
\usepackage{amssymb}
\usepackage[utf8]{inputenc}
\allowdisplaybreaks

\newcommand{\abs}[1]{\left\lvert #1 \right\rvert}

\begin{document}
\title{Study Notes}
\date{October 6, 2014}
\maketitle
\section*{definitions}
\subsection*{bounds}
\subsubsection*{bounded above}
a set $S\subset \mathbb{R}$ is {\bfseries bounded above} if there is a real number $M$ such that $s\le M$ for all $s\in S$.
\subsection*{upper bound}
If $M\ge s, \forall s\in S\subset\mathbb{R}$ then $M$ is an {\bfseries upper bound}
\subsubsection*{supremum}
if $L$ is the lowest upper bound such that $M\ge L\ge s\forall s\in S\subset\mathbb{R}$ where $M$ is any upper bound of $S$, then $L$ is the supremum.

\section*{least upper bound principle}
\subsection*{proof}
\section*{squeeze theorem}
\subsection*{proof}
\section*{define limit}
\section*{thm}
If $(a_n)$ is a convergent sequence of real numbers, then the set $\{a_n:n\in \mathbb{N}\}$ is bounded
\section*{2.5.2 arithmetic operations of limits, addition, multiplication, constant multiplication and inversion}
\end{document}

