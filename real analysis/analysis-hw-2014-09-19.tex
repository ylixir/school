%shell-escape
\documentclass[letterpaper]{article}

\usepackage{fullpage}
\usepackage{nopageno}
\usepackage{amsmath}
\usepackage{amssymb}
\usepackage{gnuplottex}
\allowdisplaybreaks

\newcommand{\abs}[1]{\left\lvert #1 \right\rvert}

\begin{document}
\title{Homework 3}
\date{September 19, 2014}
\author{Jon Allen}
\maketitle
\renewcommand{\labelenumi}{2.\arabic{enumi}}
\renewcommand{\labelenumii}{\Alph{enumii}.}
\renewcommand{\labelenumiii}{(\alph{enumiii})}
\begin{enumerate}
\setcounter{enumi}{5}
\item
  \begin{enumerate}
  \setcounter{enumii}{5}
  %2.6F
  \item
  Let $a,b$ be positive real numbers. Set $x_0=a$ and $x_{n+1}=({x_n}^{-1}+b)^{-1}$ for $n\ge0$.
    \begin{enumerate}
    \item
    Prove that $x_n$ is monotone decreasing.
    \subsubsection*{proof}
    If $x_n$ is monotone decreasing, then $x_n\ge x_{n+1}$ for all $n\ge0$.
    \[
      x_{n+1}=({x_n}^{-1}+b)^{-1}
      =\frac{1}{\frac{1+bx_n}{x_n}}
      =\frac{x_n}{1+bx_n}
    \]
    Note that if $x_n$ and $b$ are positive, then so is $x_{n+1}$.
    Now we are told that $x_0$ and $b$ are positive, so we know that all $x_n$ are positive.
    This means of course that $1+bx_n>1$ which in turn means that $x_n>\frac{x_n}{1+bx_n}=x_  n+1$. Indeed it appears that not only is $x_n$ monotone decreasing, it is strictly monotone decreasing.
    $\Box$
    \item
    Prove that the limit exists and find it.
    \subsubsection*{proof}
    As we noted in the previous proof, $x_n$ is positive for all $n\ge0$. This implies that $x_n>0$ and is therefore bounded from below.
    Because $x_n$ is monotone decreasing and bounded from below, it has a limit.
    $\Box$
    \subsubsection*{solution}
    \[
    L=\lim_{n\to\infty}x_{n+1}=\lim_{n\to\infty}\left({x_n}^{-1}+b\right)^{-1}=\left(\left({\lim_{n\to\infty}x_n}\right)^{-1}+b\right)^{-1}=\left(L^{-1}+b\right)^{-1}\\
    \]
    \begin{align*}
      L&=\frac{1}{\frac{1}{L}+b}\\
      1&=1+bL\\
      0&=bL
    \end{align*}
    So then $\lim\limits_{n\to\infty}{x_n}=0$.
    \end{enumerate}
  %2.6G
  \item
  Let $\displaystyle a_n=\left(\sum_{k=1}^n{1/k}\right)-\log n$ for $n\ge 1$. {\bfseries Euler's constant} is defined as $\gamma=\lim\limits_{n\to\infty}a_n$. Show that $(a_n)_{n=1}^\infty$ is decreasing and bounded below by zero, and so this limit exists. \uppercase{Hint}: Prove that $1/(n+1)\le\log(n+1)-\log n\le1/n$
  \subsubsection*{argument}
  Now the trick is to realize that $\displaystyle\int_b^a{\frac{1}{x}\mathrm{d}x}=\log a-\log b$.
  In other words, the area under $\displaystyle\frac{1}{x}$ where $x$ goes from $b$ to $a$ is $\log a-\log b$
  So then $\displaystyle\int_1^n{\frac{1}{x}\mathrm{d}x}=\log n$ and $\displaystyle\int_n^{n+1}{\frac{1}{x}\mathrm{d}x}=\log(n+1)-\log n$.

  Now lets imagine a series of boxes with a base of width one, and height $\displaystyle\frac{1}{k}$ positioned such that the left edge of the base is at $k$ and the right edge is at $k+1$.
  So each of these boxes will have the area of their height, namely $\frac{1}{k}$.
  Further, the sum of the areas of these boxes will be $\displaystyle \sum\limits_{k=1}^n{\frac{1}{k}}$.
  And our sequence then is the area of these boxes, with everything under the continuous line $1/x$ cut out.

  \begin{gnuplot}
    set xrange [0:10]
    set yrange [0:1.1]
    set xtics 1
    f(x) = 1/x
    set parametric
    set trange [1:10]
    set samples 91
    plot 10*(t-1)+1/2., f(10*(t-1)) with boxes,t,f(t)
  \end{gnuplot}

  Now we see that when $n=1$ then we have a $1\times1$ box (area one).
  For $n=2$ we add a box of size $\frac{1}{2}$ and cut out the curved section under $\frac{1}{x}$ from 1 to 2.
  Note that this line starts at the top left corner of the box of height 1 and ends at the point where the $\frac{1}{2}$ height box touches the height 1 box.
  Notice that we have cut out an area that is larger than $\frac{1}{2}$ but smaller than $1$.

  Lets generalize this observation a little:
  \begin{align*}
    a_{n+1}&=\left(\sum\limits_{k=1}^{n+1}{\frac{1}{k}}\right)-\log (n+1)\\
    &=\frac{1}{n+1}+\left(\sum\limits_{k=1}^{n}{\frac{1}{k}}\right)-\log n-\log \left(1+\frac{1}{n}\right)\\
    &=\frac{1}{n+1}+a_n-(\log \left(n+1\right)-\log n)\\
    &=a_n+\frac{1}{n+1}-\int_{n}^{n+1}{\frac{1}{x}\mathrm{d}x}
  \end{align*}
  Visually, every time we increase $n$ we add a $\frac{1}{n+1}$ box, but carve out the the portion of the $\frac{1}{n}$ box that is under the line $1/x$.
  Note that the line goes from the top left of the $\frac{1}{n}$ box to where the $\frac{1}{n}$ box touches $\frac{1}{n+1}$. This means that while we are adding $\frac{1}{n+1}$, we are subtracting a number that is bigger than $\frac{1}{n+1}$ (and less than $\frac{1}{n}$). And so we see that our sequence decreases.

  Futhermore, because $\log(n+1)-\log n$ is less than $\frac{1}{n}$ we always subtract less in the $n+1$ element of the sequence than we added in the $n$ element of the sequence.
  Therefore, every element of the sequence will be greater than zero.
  Visually the boxes are always above the $\frac{1}{x}$ line so the area is always more than zero.
  \setcounter{enumii}{12}
  %2.6M
  \item
  Suppose that $(a_n)_{n=1}^\infty$ has $a_n>0$ for all $n$. Show that $\limsup {a_n}^{-1}=(\liminf a_n)^{-1}$.
  \subsubsection*{proof}
  First we take a look at when $a_n$ is unbounded.
  In this case we have defined $\liminf a_n=-\infty$.
  Naturally in this case $(\liminf a_n)^{-1}$ really has no meaning.
  We will then focus on the case where $a_n$ is bounded.
  
  Lets take some $i,j$ such that $a_i\ge a_j$.
  Then if $a_i\ge a_j$ we know $\frac{1}{a_j}\ge \frac{1}{a_i}$.
  We define $b_n=\sup\{a_k:k\ge n\}$ for $n\ge 1$.
  Now we know that $b_n\ge a_k\forall k\ge n$.
  This implies that ${b_n}^{-1}\le {a_k}^{-1}\forall k\ge n$.
  Another way of saying that is $\displaystyle{b_n}^{-1}=\inf\limits_{k\ge n}{{a_k}^{-1}}$.
  So then
  \begin{align*}
    \lim\limits_{n\to\infty}{b_n}^{-1}&=\lim_{n\to\infty}\left(\inf_{k\ge n}{a_k}^{-1}\right)\\
    \left(\lim\limits_{n\to\infty}{b_n}\right)^{-1}&=\liminf{a_n}^{-1}\\
    \left(\lim\limits_{n\to\infty}{\sup_{k\ge n}a_k}\right)^{-1}&=\liminf{a_n}^{-1}\\
    \left(\limsup a_k\right)^{-1}&=\liminf{a_n}^{-1}\\
  \end{align*}
  Boom. $\Box$
  \end{enumerate}
\end{enumerate}
\end{document}
