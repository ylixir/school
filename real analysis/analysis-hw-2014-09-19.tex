\documentclass[letterpaper]{article}

\usepackage{fullpage}
\usepackage{nopageno}
\usepackage{amsmath}
\usepackage{amssymb}
\allowdisplaybreaks

\newcommand{\abs}[1]{\left\lvert #1 \right\rvert}

\begin{document}
\title{Homework 3}
\date{September 19, 2014}
\author{Jon Allen}
\maketitle
\renewcommand{\labelenumi}{2.\arabic{enumi}}
\renewcommand{\labelenumii}{\Alph{enumii}.}
\renewcommand{\labelenumiii}{(\alph{enumiii})}
\begin{enumerate}
\setcounter{enumi}{5}
\item
  \begin{enumerate}
  \setcounter{enumii}{5}
  %2.6F
  \item
  Let $a,b$ be positive real numbers. Set $x_0=a$ and $x_{n+1}=({x_n}^{-1}+b)^{-1}$ for $n\ge0$.
    \begin{enumerate}
    \item
    Prove that $x_n$ is monotone decreasing.
    \subsubsection*{proof}
    If $x_n$ is monotone decreasing, then $x_n\ge x_{n+1}$ for all $n\ge0$.
    \[
      x_{n+1}=({x_n}^{-1}+b)^{-1}
      =\frac{1}{\frac{1+bx_n}{x_n}}
      =\frac{x_n}{1+bx_n}
    \]
    Note that if $x_n$ and $b$ are positive, then so is $x_{n+1}$.
    Now we are told that $x_0$ and $b$ are positive, so we know that all $x_n$ are positive.
    This means of course that $1+bx_n>1$ which in turn means that $x_n>\frac{x_n}{1+bx_n}=x_  n+1$. Indeed it appears that not only is $x_n$ monotone decreasing, it is strictly monotone decreasing.
    $\Box$
    \item
    Prove that the limit exists and find it.
    \subsubsection*{proof}
    As we noted in the previous proof, $x_n$ is positive for all $n\ge0$. This implies that $x_n>0$ and is therefore bounded from below.
    Because $x_n$ is monotone decreasing and bounded from below, it has a limit.
    $\Box$
    \subsubsection*{solution}
    \[
    L=\lim_{n\to\infty}x_{n+1}=\lim_{n\to\infty}\left({x_n}^{-1}+b\right)^{-1}=\left(\left({\lim_{n\to\infty}x_n}\right)^{-1}+b\right)^{-1}=\left(L^{-1}+b\right)^{-1}\\
    \]
    \begin{align*}
      L&=\frac{1}{\frac{1}{L}+b}\\
      1&=1+bL\\
      0&=bL
    \end{align*}
    So then $\lim\limits_{n\to\infty}{x_n}=0$.
    \end{enumerate}
  %2.6G
  \item
  Let $\displaystyle a_n=\left(\sum_{k=1}^n{1/k}\right)-\log n$ for $n\ge 1$. {\bfseries Euler's constant} is defined as $\gamma=\lim\limits_{n\to\infty}a_n$. Show that $(a_n)_{n=1}^\infty$ is decreasing and bounded below by zero, and so this limit exists. \uppercase{Hint}: Prove that $1/(n+1)\le\log(n+1)-\log n\le1/n$
  \subsubsection*{proof}
  \begin{align*}
    a_{n+1}&=\left(\sum\limits_{k=1}^{n+1}{\frac{1}{k}}\right)-\log (n+1)\\
    &=\frac{1}{n+1}+\left(\sum\limits_{k=1}^{n}{\frac{1}{k}}\right)-\log n-\log \left(1+\frac{1}{n}\right)
  \end{align*}
  \setcounter{enumii}{12}
  %2.6M
  \item
  Suppose that $(a_n)_{n=1}^\infty$ has $a_n>0$ for all $n$. Show that $\limsup {a_n}^{-1}=(\liminf a_n)^{-1}$.
  \subsubsection*{proof}
  Lets take some $i,j$ such that $a_i\ge a_j$.
  The fact that $a_n>0$ implies that if $a_i\ge a_j$ then $\frac{1}{a_j}\ge \frac{1}{a_i}$.

  \end{enumerate}
\end{enumerate}
\end{document}
