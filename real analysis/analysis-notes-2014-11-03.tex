\documentclass[letterpaper]{article}

\usepackage{fullpage}
\usepackage{nopageno}
\usepackage{amsmath}
\usepackage{amssymb}
\usepackage[utf8]{inputenc}
\allowdisplaybreaks

\newcommand{\abs}[1]{\left\lvert #1 \right\rvert}

\begin{document}
\title{Notes}
\date{November 3, 2014}
\maketitle
\section*{5.3 properties of continuous functions}
let $f:S\to \mathbb{R}^m$ be a function. the following are equivalent
\begin{enumerate}
\item
  f is continuous on S
\item
  if $\{a_k\}$ is a sequence of elements in S such that $\lim a_k=v\in S$ then $\lim f(a_k)=f(v)$
\item
  if $U\subseteq \mathbb{R}^m$ is open them $f^{-1}(U)$ is an open set in $S$
\end{enumerate}

a set $V$ is open in $S$ if $V=U\cap S$ where $U$ is open in $\mathbb{R}^n$ 

\subsection*{examples}
let $S=[0,1]$
then $[0,\frac{1}{2})$ is open in $S$ because $[0,\frac{1}{2})=(-\frac{1}{2},\frac{1}{2})\cap S$.

$(0,\frac{1}{2})=(0,\frac{1}{2})\cap S$.

\subsubsection*{proof of above thrm}
by definition of limit, given any small $\delta>0$ there exists $N$ such that $|a_k-V|<\delta$ if $k\ge N$. We want to study $|f(a_k)-f(v)|$. Because $f$ is continuous $|f(a_k-f(v)|<\varepsilon$ if $|a_k-v|<r$ there exists $r(\varepsilon,v)>0$. pick $r=\delta$. then $|a_k-v|<r$ for all $k\ge N$ implies $|f(a_k)-f(v)|< \varepsilon$ and so $\lim f(a_k)=f(v)$

for $2\to1$ assume that $f$ is not continuous. we will prove that $2$ fails.

if $f$ is not continuous, then there exists some pint $p\in S$ and $\varepsilon>0$ such that $\forall r>0$ then $|x-p|<r$ and $|f(x)-f(p)|\ge \varepsilon$. consider $a_k\in S$ such that $\lim a_k=p$. by hypothesis $\lim f(a_k)=f(p)$. but then we have $|a_k-p|<r$ and $|f(a_k)-f(p)|<\varepsilon$

for $1\to3$. we want to show that given and open set $U\subseteq \mathbb{R}^m, f^{-1}(U)$ is open  in $S$. pick $x\in f^{-1}(U)$. we need to find a ball $B(x,r)\subseteq f^{-1}(U)$.

$f(x)\in U$. $U$ is open, so there exists some $\epsilon>0$ such that $B(f(x),\varepsilon)\subseteq U$. hence $f^{-1}(B(f(x),\varepsilon))\subseteq f^{-1}(U)$.

given $\varepsilon>0$ there exists $r>0$ such that if $||x-y||<r\to ||f(x)-f(y)||<\varepsilon$. continuity gives us a radius $r$ such that if $y\in B(x,r)$ then $||f(x)-f(y)||<\varepsilon$ and so $f(y)\in B(f(x),\varepsilon)$. So we have found $B(x,r)\subseteq f^{-1}(B(f(x),\varepsilon)\subseteq f^{-1}(U)$

\subsubsection*{moral of the story}
continuity can be seen in terms of finding balls.

\subsubsection*{example}
$f(x)=x^2$

$f^{-1}(c,d)=(\sqrt{c},\sqrt{d})\cup (-\sqrt{c},-\sqrt{d})$

$f^{-1}(-1,1)=(-1,1)$

$f^{-1}(-2,-1)=\emptyset$

$f$ is continuous iff $f^{-1(U)}$ is open if $U$ is open, but if $U$ is open then $f(U)$ is not necessarily open.

$f(-1,1)=[0,1)$ which is not open in $\mathbb{R}$

\subsection*{prop5.3.5}
let $f:\mathbb{R}^n\to\mathbb{R}^m, g:\mathbb{R}^m\to\mathbb{R}^p$ assume $f,g$ are continuous. then $g\circ f$ is continouos.

assume that $U\in \mathbb{R}^p$ is open. we need to show that $(g\circ f)^{-1}(U)$ is open. $(g\circ f)^{-1}(U)=f^{-1}(g^{-1}(U))$. because g is continuous, $g^{-1}(U)$ is open. beccause $f$ is continuous, $f^{-1}(g^{-1}(U)$ is open.

\subsection*{prop 5.3.2}

if in addition, $f$ and $g$ are continous at $a$ then so are $f\pm g, f\cdot g, \alpha f,\frac{f}{g}$.

corollary, we know that $f(x)=x$ is continous $f:\mathbb{R}^n\t\mathbb{R}^n$ then $f(x)=x^{m}$ is continuous a $m\in \mathbb{N}$. and any  polynomial is continous. and any rational function  is continous a $f(x)/g(x)$ such that $g(x)\ne 0$.
\end{document}



