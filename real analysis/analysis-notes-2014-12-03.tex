\documentclass[letterpaper]{article}

\usepackage{fullpage}
\usepackage{nopageno}
\usepackage{amsmath}
\usepackage{amssymb}
\usepackage[utf8]{inputenc}
\allowdisplaybreaks

\newcommand{\abs}[1]{\left\lvert #1 \right\rvert}

\begin{document}
\title{Notes}
\date{December 3, 2014}
\maketitle
\section*{fundamental theorem of calculus}
we step up one degree of regularity with integration. non-continuous->continuous, or continuous->differentiable

\subsection*{example}
$f(x)=\begin{cases}1&x\in[0,1)\\2&x\in[1,2]\\0&\text{otherwise}\end{cases}$

\begin{align*}
F(x)=\int_0^x{f(t)\;\mathrm{d}t}&=
\begin{cases}
  x\in[0.1)&\int_0^x{1\;\mathrm{d}t}=x\\
  x=1&\int_0^1{1\;\mathrm{d}t}=1\\
  1\le x\le2&\int_0^1{1\;\mathrm{d}t}+\int_1^x{2\;\mathrm{d}t}=1+(2x)-2=2x-1\\
  2\le x&\int_0^1{1}+\int_1^2{2}=3
\end{cases}\\
&=
\begin{cases}
  0&x\le0\\
  x&x\in[0.1]\\
  2x-1&x\in[1,2]\\
  3&x\ge2
\end{cases}
\end{align*}
and this is not differentiable at 1
\section*{FTC 2}
\section*{integration by parts}
reverse of the product rule
\begin{align*}
  (fg)'&=f'g+fg'\\
  \intertext{by FTC}
  F(b)G(b)-F(a)G(a)&=\int_a^b{(fg)'(x)\;\mathrm{d}x}\\
  &=\int_a^b{f'g(x)\;\mathrm{d}x}+\int_a^b{fg'(x)\;\mathrm{d}x}
\end{align*}
\section*{u-substitution, change of variable}
reverse of chain rule

\section*{8.1 sequences of functions}
difference between pointwise and uniform limit is that in pointwise limit you fix $x=x_0$ and in uniform convergence, the entire function converges 

if $f_n\to f$ uniformly on $[a,b]$ then $f_n\to f$ pointwise on $[a,b]$

uniform convergence is a stronger condition

\subsection*{8.1.A}
if we fix $x_0$ then pointwise limit is 0 so we assume that $x_0\ne 0$. exponentials always win so $x_0ne^{-nx_0}=0$ and so the pointwise limit is 0.

do all the x get to zero at the same rate?

what is the maximum? $f_n'(x)=ne^{-nx}+xne^{-nx}(-n)=0$ and $x=\frac{1}{n}$ is critical point.

critical point is a maximum

$f_n(\frac{1}{n})=e^{-1}$

so max height does not depend on $n$

if $\varepsilon<\frac{e^-1}{2}$ then $\sup_{x\ge0}|f_n(x)-f(x)|\not<\varepsilon$

\section*{8.2 and 8.3}
\subsection*{uniform convergence and continuity}
if $f_n(x)\to f(x)$ uniformally and $f_n(x)$ are all continuous then $f(x)$ is continuous


uniform convergence preserves continuity. pointwise convergence does not

\subsection*{inuform convergence and integrals}
if $f_n:[a,b]\to\mathbb{R}$ are Riemann integrable and converge uniformally  to $f:[a,b]\to\mathbb{R}$ then $f$ is riemann integrable and integral of limit of $f_(x)$ is integral of $f(x)$

uniform continuity preserves integrals

pointwise continuity does not preserve integrals

\section*{the $L^\infty$ norm}
$f:[a,b]\to\mathbb{R}$ is defined as $\sup_x\in[a,b]|f(x)|=||f||_\infty$

$f_n\to f$ uniformally $\Leftrightarrow||f_n-f||_\infty\to_{n\to\infty}0$


\end{document}
