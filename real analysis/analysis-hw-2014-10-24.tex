%shell-escape
\documentclass[letterpaper]{article}

\usepackage{fullpage}
\usepackage{nopageno}
\usepackage{amsmath}
\usepackage{amssymb}
\usepackage{gnuplottex}
\allowdisplaybreaks

\newcommand{\abs}[1]{\left\lvert #1 \right\rvert}

\begin{document}
\title{Homework 6}
\date{October 24, 2014}
\author{Jon Allen}
\maketitle
\renewcommand{\labelenumi}{4.\arabic{enumi}}
\renewcommand{\labelenumii}{\Alph{enumii}.}
\renewcommand{\labelenumiii}{(\alph{enumiii})}
\begin{enumerate}
\setcounter{enumi}{1}
\item
  \begin{enumerate}
  \setcounter{enumii}{1}
  %4.2B
  \item
    If a sequence $(\boldsymbol{x}_n)_{n=1}^\infty$ in $\mathbb{R}^n$ satisfies $\sum_{n\ge1}{\left\lvert\left\lvert\boldsymbol{x}_n-\boldsymbol{x}_{n+1}\right\rvert\right\rvert}<\infty$, show that it is a Cauchy sequence.

    Let's say $\sum_{n\ge1}||\boldsymbol{x}_n-\boldsymbol{x}_{n+1}||=L$.
    Then for every $\varepsilon>0$ there exists some $N$ such that
    $\sum\limits_{n=1}^{N-1}{||\boldsymbol{x}_n-\boldsymbol{x}_{n+1}||}>L-\varepsilon$
    and by extension $\varepsilon>\sum\limits_{n=N}^\infty{||\boldsymbol{x}_n-\boldsymbol{x}_{n+1}||}
    >\sum\limits_{k=n}^{m-1}{||\boldsymbol{x}_k-\boldsymbol{x}_{k+1}||}$ for all $m>n\ge N$.
    And with the triangle inequality and the observation that our series is telescoping we quickly see that
    \[\varepsilon>\sum\limits_{k=n}^{m-1}{||\boldsymbol{x}_k-\boldsymbol{x}_{k+1}||}\ge \left\lvert\left\lvert\sum\limits_{k=n}^{m-1}{\boldsymbol{x}_k-\boldsymbol{x}_{k+1}}\right\rvert\right\rvert=||\boldsymbol{x}_n-\boldsymbol{x}_{m}||\]
    Which is the very definition of a Cauchy sequence.
    Well almost, I guess to be complete I should point out that $||\boldsymbol{x}_n-\boldsymbol{x}_n||=0<\varepsilon$ and $||\boldsymbol{x}_n-\boldsymbol{x}_m||=||\boldsymbol{x}_m-\boldsymbol{x}_n||<\varepsilon$.
    So our inequality holds for all $m,n\ge N$, not just $m>n\ge N$

  %4.2C
  \item
    \begin{enumerate}
    %4.2Ca
    \item
      Give an example of a Cauchy sequence for which the condition of Exercise B fails.
      \[a_n=\frac{(-1)^n}{n}\]
      \begin{align*}
        \sum\limits_{n\ge 1}{\left\lvert\frac{(-1)^n}{n}-\frac{(-1)^{n+1}}{n+1}\right\rvert}
        &=\sum\limits_{n\ge 1}{\left\lvert(-1)^n\left(\frac{1}{n}-\frac{-1}{n+1}\right)\right\rvert}\\
        &=\sum\limits_{n\ge 1}{\left\lvert\frac{1}{n}+\frac{1}{n+1}\right\rvert}\\
        &>\sum\limits_{n\ge 1}{\frac{1}{n}}=\infty
      \end{align*}
    %4.2Cb
    \item
      However, show that every Cauchy sequence $(\boldsymbol{x}_n)_{n=1}^\infty$ has a subsequence $(\boldsymbol{x}_{n_i})_{i=1}^\infty$ such that $\sum_{i\ge1}\left\lvert\left\lvert \boldsymbol{x}_{n_i}-\boldsymbol{x}_{n_{i+1}}\right\rvert\right\rvert<\infty$

      First we choose $\boldsymbol{x}_{N_1}$ such that $||\boldsymbol{x}_m-\boldsymbol{x}_n||<\frac{1}{2}$ for all $m,n\ge N_1$.
      We then proceed, choosing $\boldsymbol{x}_{N_i}$ such that $||\boldsymbol{x}_m-\boldsymbol{x}_n||<\frac{1}{2^i}$ for all $m,n\ge N_i$.
      Now then $\sum_{i\ge1}||\boldsymbol{x}_{N_i}-\boldsymbol{x}_{N_i}||<\sum_{i\ge1}\frac{1}{2^i}=-1+\sum_{i\ge0}\frac{1}{2^i}=-1+\frac{1}{1-\frac{1}{2}}=1<\infty$ as required.
    \end{enumerate}
  \end{enumerate}
\item
  \begin{enumerate}
  \setcounter{enumii}{1}
  %4.3B
  \item
    Let $(\boldsymbol{a}_n)_{n=1}^\infty$ be a sequence in $\mathbb{R}^k$ with $\lim\limits_{n\to\infty}\boldsymbol{a}_n=\boldsymbol{a}$. Show that $\{\boldsymbol{a}_n:n\ge 1\}\cup\{\boldsymbol{a}\}$ is closed.

    Let's say $A=\{\boldsymbol{a}_n:n\ge1\}\cup\{\boldsymbol{a}\}$.
    If $A$ is not closed, then we can form a sequence from the elements of $A$ that converge on some point $\boldsymbol{b}\in\mathbb{R}^k$ where $\boldsymbol{b}\not\in A$.

    Now lets assume that there is no element in $A$ that is closest to $\boldsymbol{b}$.
    Then for every $\varepsilon>0$ then we could find some $L$ such that $||\boldsymbol{a}_{n_l}-\boldsymbol{b}||<\varepsilon$ where $n_l\ge L$ and $\boldsymbol{a}_{n_l}$ is a subsequence of $\boldsymbol{a}_n$.
    Of course this is the definition of a limit.
    Unfortunately we know that all subsequences of $\boldsymbol{a}_n$ must converge to $\boldsymbol{a}$.
    Of course $\boldsymbol{b}\not\in A$ so $\boldsymbol{b}\ne \boldsymbol{a}$.
    This contradiction means that we can find some $\boldsymbol{a}_m\in A$ that is closest to $\boldsymbol{b}$.

    Great, now lets say the sequence that converges on $\boldsymbol{b}$ is $\boldsymbol{a}_j$.
    Now we know that the distance from any element in $A$ to $\boldsymbol{b}$ is at least $||\boldsymbol{a}_m-\boldsymbol{b}||$.
    Lets pick $\varepsilon=\frac{||\boldsymbol{a}_m-\boldsymbol{b}||}{2}$.
    Then for all $\boldsymbol{a}_j$ we have $||\boldsymbol{a}_j-\boldsymbol{b}||>\varepsilon$ and so $\boldsymbol{b}$ can not be a limit.
    And so we have closure by contradiction.
  \setcounter{enumii}{3}
  %4.3D
  \item
    If $A$ is a bounded subset of $\mathbb{R}$, show that $\sup A$ and $\inf A$ belong to $\overline{A}$.

    Well $\sup A\ge \inf A$ and so $\inf A\le a_n=\sup A-\frac{1}{n+1}(\sup A-\inf A)\le \sup A$ for all $n\in \mathbb{N}\setminus\{0\}$.
    Notice that $a_n\in A$ for all $n$ and $\lim\limits_{n\to\infty}a_n=\sup A$.
    Similarly $\inf A\le b_n=\inf A+\frac{1}{n+1}(\sup A-\inf A)\le \sup A$.
    We see that $b_n\in A$ for every $n$ and $\lim\limits_{n\to\infty}b_n=\inf A$.
    And so because $\overline{A}$ contains all the limit points of $A$ then the supremum and infimum are in the closure.
  \setcounter{enumii}{9}
  %4.3J
  \item
    Show that if $U$ is open and $A$ is closed, the $U\setminus A=\{\boldsymbol{x}\in U:\boldsymbol{x}\not\in A\}$ is open.
    What can be said about $A\setminus U$?

    If $U$ is open, then $U'$ is closed.
    And since $U'$ is closed and $A$ is closed, then $U'\cup A$ is closed.
    And the complement of $U'\cup A$ is open.
    But notice that the complement of $U'\cup A$ is $U\setminus A$.
    And so $U\setminus A$ is open as required.

    $A\setminus U$ is equal to the complement of $A'\cup U$ which is the union of two open sets.
    But we don't know anything about the closure of the union of open sets in general.
    If $A\cap U=\emptyset$ then $A\setminus U=A$ which is closed.
    But if $A=[0,2]$ and $U=[1,2)$ then $A\setminus U=[0,1)\cup\{2\}$ which is open.
  %4.3k
  \item
    Suppose that $A$ and $B$ are closed subsets of $\mathbb{R}$
    \begin{enumerate}
    \item
      Show that the product set $A\times B=\{(x,y)\in \mathbb{R}^2:x\in A\text{ and }y\in B\}$ is closed.

      Lets suppose that $A\times B$ is open.
      Then there exists some sequence $(\boldsymbol{x}_n)_{n=1}^\infty$ such that every $\boldsymbol{x}_n\in A\times B$ and $\lim\limits_{n\to\infty}\boldsymbol{x}_n=\boldsymbol{x}$ where $\boldsymbol{x}\not\in A\times B$.
      We know that $\boldsymbol{x}_n$ only converges to a point if each of it's coefficients converge.
      So if $\boldsymbol{x}=(x_1,x_2)$ then $\lim\limits_{n\to\infty}x_{k,1}=x_1$.
      Because $A$ is closed we know that $x_1\in A$.
      Similarly $\lim\limits_{n\to\infty}x_{k,2}=x_2$.
      And again, because $B$ is closed we know that $x_2\in B$.
      Well, if $x_1\in A$ and $x_2\in B$ then $\boldsymbol{x}=(x_1,x_2)\in A\times B$.
      Whoops, that contradicts our assumption.
      I guess $A\times B$ is closed after all.
    \item
      Likewise show that if both $A$ and $B$ are open, then $A\times B$ is open.

      If $A$ is open, then there exists some sequence $a_n$ where $a_n\in A$ for all $n$ but $\lim\limits_{n\to\infty}a_n=a\not\in A$.
      Similarly, if $B$ is open, then there exists some sequence $b_n$ where $b_n\in B$ for all $n$ but $\lim\limits_{n\to\infty}b_n=b\not\in B$.
      Now we define a sequence $\boldsymbol{x}_n=(a_n,b_n)$ in $A\times B$.
      We know that $\lim\limits_{n\to\infty}\boldsymbol{x}_n=\boldsymbol{x}=(a,b)\not\in A\times B$.
      And so we have found a sequence in $A\times B$ with a limit outside of $A\times B$ and then by definition $A\times B$ is open.
    \end{enumerate}
  \end{enumerate}
\end{enumerate}
\end{document}
