\documentclass{article}
\usepackage{fullpage}
\usepackage{nopageno}
\usepackage{amsmath}
\allowdisplaybreaks

\newcommand{\abs}[1]{\left\lvert #1 \right\rvert}

\begin{document}
\title{Notes}
\date{January 15, 2014}
\maketitle

grading is 70\% hw, 30\% final

speed is 1 lesson per lecture. need to read because there won't be enough class time to cover everything.
covering through chapter 4, 36 lessons.

\section*{lesson 1 introduction}
heat flow experiment. page 11
\begin{align*}
  \frac{\partial u}{\partial t},U_t\\
  \frac{\partial u}{\partial t}&=\alpha^2\frac{\partial^2 u}{\partial x^2}\intertext{lesson 2 heat equatin in 1 space variable, diffusion equation}\\
  u(t,x)\\
  U_{xx}+U_{yy}=0\intertext{this is laplace's/potential equation is 2-space variables}\\
  Au_{xx}+Bu_{xy}+Cu_{yy}+Du_x+Eu_y+FU&=0 \text{or} f(x,y)\\
  L[u]*=0 \text{or} f(x,y)
\end{align*}

this was general 2nd order linear PDE
\begin{align*}
  L[\alpha u_1+\beta u_2]=\alpha L[u_1]+\beta L[u_2]\\
\end{align*}
note: is A, B, ... are functions of x,y then it is still linear
\begin{align*}
  L&=A(x,y)\frac{\partial^2}{\partial x^2}+B(x,y)\frac{\partial^2}{\partial x\partial y}+...
\end{align*}
homogeneous and nonhomogeneous. homogeneous if it is equal to zero and nonhomogeneous if it is equal to a function of x,y

if $L[u]=0$ and $u_1,u_2$ are solutions then any linear combination is also a solution ($\alpha u_1+\beta u_2$) and here is why
\begin{align*}
  L[\alpha u_1+\beta u_2]&=\alpha L[u_1]+\beta L[u_2]=\alpha 0+\beta 0=0
\end{align*}
linear pdes are classified as parabolic (ch 2) hyperbolic (ch 3) and elliptic (ch 4).
\begin{align*}
  B^2-4AC&=0 \text{parabolic}\\
  B^2-4AC&>0 \text{hyperbolic}\\
  B^2-4AC&<0 \text{elliptic}\\
\end{align*}
looks like the discriminant for quadradic equation. we'll start to understand this in the second half of chapter 3.

page 7 contains a table of classifications

\subsection*{page 8 example exercises}
\begin{align*}
  \frac{\mathrm{d}^2u}{\mathrm{d}x^2}&=0\\
  u(x)&=\alpha x+\beta\\
  \frac{\partial u}{\partial x}&=0\\
  u&=u(x,y) on \mathbb{R}^2\\
  u&=f(y)\\
  \frac{\partial^2 u}{\partial x\partial y}&=0\\
  \frac{\partial }{\partial x}\left(\frac{\partial u}{\partial y}\right)&=0\\
  \frac{\partial u}{\partial y}&=f(y)\\
  u(x,y)&=F(y)+g(x) \text{where} F'(y)=f(y)
\end{align*}
\end{document}
