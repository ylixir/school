\documentclass{article}
\usepackage{fullpage}
\usepackage{nopageno}
\usepackage{amsmath}
\allowdisplaybreaks

\newcommand{\abs}[1]{\left\lvert #1 \right\rvert}
\newcommand{\degree}{\ensuremath{^\circ}}

\begin{document}
Jon Allen

HW 13

Lesson 7 problem 3. Solve the IC. Note the questions about steady state behavior.
\begin{align*}
  PDE&&&\quad u_t=u_{xx}&0&<x<1&0&<t<\infty\\
  BCs&&&\left\{
  \begin{aligned}
    &u_x(0,t)=0\\
    &u_x(1,t)=0
  \end{aligned}
  \right.&&&0&<t<\infty\\
  IC&&&\quad u(x,0)=x&0&\leq x\leq 1
\end{align*}
\begin{align*}
  \lambda_n&=n\pi &
  u(x,t)=\sum\limits_{n=1}^\infty&{a_ne^{-{\lambda_n}^2t}\cos(\lambda_n x)}\\
  u(x,0)=x&=\sum\limits_{n=1}^\infty{a_n\cos(\lambda_n x)}&
  \int_0^1{\xi\cos(\lambda_m\xi)\,\mathrm{d}\xi}&=\sum\limits_{n=1}^\infty{a_n\int_0^1{\cos(\lambda_n\xi)\cos(\lambda_m\xi)\,\mathrm{d}\xi}}
\end{align*}
Because $\left\{\cos(\lambda_ix)\right\}_{0\leq i\leq n}$ are orthogonal functions we can convert the above equation into the following.
\begin{align*}
  \int_0^1{\xi\cos(\lambda_m\xi)\,\mathrm{d}\xi}&=a_m\int_0^1{\cos(\lambda_m\xi)^2\,\mathrm{d}\xi}\\
  &=a_m\cdot\frac{\sin(2\lambda_m)+2\lambda_m}{4\lambda_m}\\
  &=a_m\frac{\lambda_m+\sin(\lambda_m)\cos(\lambda_m)}{2\lambda_m}\\
  \intertext{Recall that we discovered in HW 12 that $\sin(\lambda)=0$}
  &=\frac{a_m}{2}\\
  a_n&=2\int_0^1{\xi\cos(\lambda_n\xi)\,\mathrm{d}\xi}\\
  &=2\left[\frac{\lambda_nx\sin(\lambda_nx)+\cos(\lambda_nx)}{{\lambda_n}^2}\right]_0^1\\
  &=2\left[\frac{\lambda_n1\sin(\lambda_n1)+\cos(\lambda_n1)}{{\lambda_n}^2}-\frac{\lambda_n0\sin(\lambda_n0)+\cos(\lambda_n0)}{{\lambda_n}^2}\right]_0^1\\
  &=2\left[\frac{\cos(\lambda_n)}{{\lambda_n}^2}-\frac{1}{{\lambda_n}^2}\right]=2\left[\frac{\cos(\lambda_n)-1}{{\lambda_n}^2}\right]\\
  &=2\left[\frac{\cos(n\pi)-1}{(n\pi)^2}\right]\\
  &=\frac{2((-1)^{n}-1)}{(n\pi)^2}\\
  u(x,t)&=\sum\limits_{n=1}^\infty{\frac{2((-1)^{n}-1)}{(n\pi)^2}e^{-(n\pi)^2t}\cos(n\pi x)}\\
\end{align*}
\end{document}
