\documentclass{article}
\usepackage{fullpage}
\usepackage{nopageno}
\usepackage{amsmath}
\usepackage{amssymb}
\allowdisplaybreaks

\newcommand{\abs}[1]{\left\lvert #1 \right\rvert}

\begin{document}
\title{Notes}
\date{March 3, 2014}
\maketitle

\section*{lesson 13 laplace transform}
\subsection*{sample problem}
Get a transform $U(s,t)$ for a solution (last page of lesson 13) and text gives $u(x,t)$ and says ``follows from tables''. Turns out it's not in the texts, tables. Not available in Mathematica either.

We will take a couple of days on approaches to finding inverse laplace transforms. Started talking about this on the 28th.
\subsection*{elementary inversion based}
Find laplace transforms for $t^p$ and $t^ne^{at}\begin{cases}\cos(bt)\\\sin(bt)\end{cases}$ that occur in circuit analysis

the general answer is
\begin{align*}
  f(t)&=\frac{1}{2\pi i}\int_{c-i\infty}^{c+i\infty}{e^{st}F(x)\,\mathrm{d}s}
\end{align*}
this is the general inversion formula for laplace transforms. Requires essentialuse of complex analysis. the typical result is an infinite series of an integral representation for $f(t)$
\subsection*{approach 1}
expand $F(s)$ in reciprocal powers and invert termwise. $\frac{F(p+1)}{s^{p+1}}\to t^p$
\subsubsection*{example}
\begin{align*}
  F(s)&=(s^2+1)^{-1/2} \text{ in text table}\\
  \text{Find }f(t)\\
  \text{Found }f(t)&=\sum\limits_{n=0}^\infty{\frac{(-t^2/4)^n}{(n!)^2}}
\end{align*}
approach 1 give series for $f(t)$.

\emph{note} this happens to be $J_0(t)$
\subsection*{approach 2}
using $F(s)$, try to find an equation (typically differential equatin) for $f(t)$.
\subsubsection*{example}
\begin{align*}
  F(s)&=(s^2+1)^{-1/2}&&=\mathcal{L}\{f(t)\}\\
  F'(s)&=-\frac{1}{2}(s^2+1)^{-3/2}(2s)&&=\mathcal{L}\{-t\cdot f(t)\}\\
  &=-\frac{1}{s^2+1}(s^2+1)^{-1/2}\\
  &=-\frac{s}{s^2+1}F(s)\\
  (s^2+1)F'(s)+sF(s)&=0\\
  sF(s)-f(0)&&&=\mathcal{L}\{f'(t)\}\\
  F'(s)&&&=\mathcal{L}\{-t\cdot f(t)\}\\
  s^2F'(s)-s(-t\cdot f(t))_{t=0}-\left(\frac{\mathrm{d}}{\mathrm{d}t}(-t\cdot f(t))\right)&&&=\mathcal{L}\left\{\frac{\mathrm{d}^2}{\mathrm{d}t^2}(-t\cdot f(t)\right\}\\
  s^2G(s)-sg(0)-g'(0)&&&=\mathcal{L}\{g''(t)\}\\
  \mathcal{L}\{f'(t)-tf(t)-\frac{\mathrm{d}^2}{\mathrm{d}t^2}(-tf(t))\}&\\
  +f(0)\underbrace{-s(tf(t))_{t=0}}_{=0}-\frac{\mathrm{d}}{\mathrm{d}t}(tf(t))_{t=0}-(f(t)+tf'(t))_{t=0}&=0\\
  -tf'(t)\text{ as }t\to0
\end{align*}
we assume $f(0)$ exists, and that $\lim\limits_{t\to0^+}tf'(t)=0$. \emph{AFTER} solving for $f(t)$ wee can check that these conditions hold.
\begin{align*}
  \mathcal{L}\{f'(t)-tf(t)-\frac{\mathrm{d}^2}{\mathrm{d}t^2}(tf(t))-(tf''(t)+2f'(t)\}&=0\\
  tf''(t)+f'(t)+tf(t)&=0
\end{align*}
bessel de
\begin{align*}
  z^2\frac{\mathrm{d}^2w}{\mathrm{d}z^2}+z\frac{\mathrm{d}w}{\mathrm{d}z}+(z^2+\mu^2)w&=0\\
  \mu &=\text{order}\\
  J_{\mu }(z)&\text{ Bessel function of first kind (order $\mu $)}\\
  Y_{\mu }(z)&\text{ Bessel function of second kind (order $\mu $)}\\
\end{align*}

\emph{note} dlmf.nist.gov is reference for standard functions.
\end{document}
