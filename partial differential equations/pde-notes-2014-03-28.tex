\documentclass{article}
\usepackage{fullpage}
\usepackage{nopageno}
\usepackage{amsmath}
\usepackage{amssymb}
\usepackage{enumerate}
\allowdisplaybreaks

\newcommand{\abs}[1]{\left\lvert #1 \right\rvert}

\begin{document}
\title{Notes}
\date{March 28, 2014}
\maketitle
result from last time (D'Alemberts' solution)
\begin{align*}
  \text{PDE}&&u_{tt}&=c^2u_{xx}&-\infty<&x<+\infty&0<&t<\infty\\
  \text{IC}&&u(x,0)&=f(x)\\
  &&u_t(x,0)&=g(x)\\
  \hline
  &&u(x,t)&=\frac{1}{2}(f(x-ct)+f(x+ct))+\frac{1}{2c}\int_{x-ct}^{x+ct}{g(s)\mathrm{d}s}\\
\end{align*}
\section*{18 properties of this solution}
\begin{enumerate}[C{a}se 1.]
\item
\begin{align*}
  \text{IC}&&\left.\begin{aligned}u(x,0)&=f(x)\\u_t(x,0)&=0\end{aligned}\right\}\text{ Solution }u(x,t)=\frac{1}{2}\left[\underbrace{f(x-ct)}_{\text{wave moving right}}+\underbrace{f(x+ct)}_{\text{wave moving left}}\right]
\end{align*}
\item
\begin{align*}
  \text{IC}&&\left.\begin{aligned}u(x,0)&=0\\u_t(x,0)&=g(x)\end{aligned}\right\}\text{ Solution }u(x,t)=\frac{1}{2c}\int_{x-ct}^{x+ct}{g(s)\,\mathrm{d}s}
\end{align*}
value of $g$ over a widening interval

see graphs on pages 139-141 (155-157)
\end{enumerate}
\begin{align*}
  \text{PDE}&&u_{tt}&=c^2u_{xx}&0<&x<\infty&0<&t<\infty\\
  \text{BC}&&u(0,t)&=0&&&0<&t<\infty\\
  \text{IC}&&u(x,0)&=f(x)&0<&x<\infty\\
  &&u_t(x,0)&=g(x)
\end{align*}
as last time $u(x,t)=\phi(x-ct)+\psi(x+ct)$ general solution -- IC's and BC are not used.

Match IC:
\begin{align*}
  \phi(x)+\psi(x)&=f(x)&0<&x<\infty\\
  -c\phi'(x)+c\psi'(x)&=g(x)&\to&-\phi(x)++\psi(x)\\
  &&=&\frac{1}{c}\int_0^x{g(s)\,\mathrm{d}s}+K\\
  \phi(x)&=\frac{1}{2}f(x)-\frac{1}{2c}\int_0^x{g(s)\,\mathrm{d}s}+ \frac{k}{2}\\
  \psi(x)&=\frac{1}{2}f(x)+\frac{1}{2c}\int_0^x{g(s)\,\mathrm{d}s}+ \frac{k}{2}\\
  u(x,t)&=\phi(x-ct)+\psi(x+ct)
\end{align*}
k cancel out. $x+ct>0$ so $\psi(x+ct)$ is no problem. $x-ct$ changes sign.

What is $\phi(x-ct)$ when $x-ct<0$?

now lets look at the boundary condition

\begin{align*}
  \text{BC}&&u(0,t)&=0=\phi(-ct)+\psi(ct)\text{ for}&0<&t<\infty\\
  \text{for }&&-\infty&<x<0,&\phi(x)&=-\psi(-x)\\
  \text{for }&&x-ct&>0,&u(x,t)&=\phi(x-ct)+\psi(x+ct)\\
  &&&&&=\frac{1}{2}\left[f(x-ct)+f(x+ct)\right]+\frac{1}{2c}\int_{x-ct}^{x+ct}{g(s)\,\mathrm{d}s}\\
  \text{for }&&x-ct&<0,&u(x,t)&=-\psi(x-ct)+\psi(x+ct)\\
  &&&&&=\frac{1}{2}\left[f(x-ct)+f(x+ct)\right]+\frac{1}{2c}\int_{x-ct}^{x+ct}{g(s)\,\mathrm{d}s}
\end{align*}
more on page 143(159)


homework \#28 \& \#29 due next friday (first friday of april) lession 17 exercise 3 and exercise 4
\end{document}
