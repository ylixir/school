\documentclass{article}
\usepackage{fullpage}
\usepackage{nopageno}
\usepackage{amsmath}
\usepackage{amssymb}
\usepackage{enumerate}
\allowdisplaybreaks

\newcommand{\abs}[1]{\left\lvert #1 \right\rvert}

\begin{document}
\title{Notes}
\date{April 2, 2014}
\maketitle
\section*{lesson 19 (bc)}
\section*{lesson 20 solution by separation of variables}
\begin{align*}
  \text{PDE}&&u_{tt}&=c^2u_{xx}&0<&x<L&0<&t<\infty\\
  \text{BC}&&&&x=0,&x=L\\
  \text{IC}&&u(x,0)&=f(x)\\
  &&u_t(x,0)&=g(x)
\end{align*}
\section*{19}
names book gives for boundaries
\begin{enumerate}
\item
controlled endpoints
\begin{align*}
  u(0,t)&=g_1(t)\\
  u(L,t)&=g_2(t)
\end{align*}
1st kind, dirichlet
\item
force at boundaries
\begin{align*}
  u_x(0,t)&=g_1(t)\\
  u_x(L,t)&=g_2(t)
\end{align*}
snd kind newmann
\item
elastic attachement
\begin{align*}
  u_x(0,t)+cu(0,t)&=g_1(t)\\
  u_x(L,t)+cu(L,t)&=g_2(t)
\end{align*}
3rd kind, mixed or robin
\end{enumerate}
using terminology like transverse, longitudinal and torsional waves.
Transverse waves vibrate perpindicular to reference axis or direction of motion. Like a jump rope.
Longitudinal waves vibrate parallel to reference axis/direction of motion. Like a slinky.
Torsional waves represent rotational vibrations about reference axis or direction of motion.

\subsubsection*{2nd kind}
recall that vertical force$=T\cdot\sin(\theta)=T\cdot\frac{\tan(\theta)}{\sec(\theta)}=T\cdot\frac{x_x}{\sqrt{1+{u_x}^2}}$. Zero vertical force at enpoint means $T\cdot\frac{x_x}{\sqrt{1+{u_x}^2}}=0$ or $u_x=0$. So frictionless endpoints. When there is an applied force on the endpoints, then the derivative picks up something.

Note that in longitudinal displacement (slinky example) maximum displacement is at the top, because that is where the most force is. as you go down, less force, less displacement.

\subsubsection*{3rd kind}
page 150. Still frictionless, but elastic attachment.

vertical force of spring is $-hl\sin(\theta)=-hl_0\frac{\sin(\theta)}{\cos(\theta)}=-hu=T\frac{u_x}{\sqrt{1+{u_x}^2}}$. Assum $\sqrt{1+{u_x}^2}\approx 1$ then $Tu_x=-hu$. Note that $l$ is the length of the little spring contraption and $l_0$ is the initial (short) length.
\section*{20}
separation of variables
\begin{align*}
  u&=X(x)T(t)\\
  X(x)T''(t)&=c^2X''(x)T(t)\\
  \frac{T''(t)}{c^2T(t)}&=\frac{X''(x)}{X(x)}=\lambda\\
  \intertext{$\lambda$ is a constant, separation constant}
  -\infty&<\lambda<\infty\\
  T''-c^2\lambda T&=0,\quad X''-\lambda X=0\\
  0&<t<\infty\quad 0<x<L\\
  \frac{\mathrm{d}^2y}{\mathrm{d}x^2}-\mu^2&=0\quad e^{\mu x},e^{-\mu x}\\
  \frac{\mathrm{d}^2y}{\mathrm{d}x^2}+0&=0\quad 1,x\\
  \frac{\mathrm{d}^2y}{\mathrm{d}x^2}+\mu^2&=0\quad \cos(\mu x),\sin(\mu x)\\
\end{align*}
cases are
\begin{enumerate}
\item
$\lambda=\mu^2$ positive
\item
$\lambda=0$
\item
$\lambda=-\mu^2$ negative
\end{enumerate}
earlier, $\lambda=\mu^2$ was eliminated on the ground that it gave exponntially increasing solutions in time $t$. Time solutions now $e^{+\mu ct},e^{-\mu ct}$.

The real reason $\mu^2>0$ can be eliminated is that the BC for $X(x)$ cannot be satisfied. $\lambda=\mu^2>0$ $X''-\mu^2X=0$
\begin{align*}
  X&=c_1\cosh(\mu x)+c_2\sinh(\mu x)\\
  X(0)&=0=c_1\cdot1+c_2\cdot0\quad c_1=0\\
  X&=c_2\sinh(\mu x)\\
  X(L)&=0=c_2\sinh(\mu L)
\end{align*}
want nontrivial solution $c_2\ne0$. $\sinh(\mu L)=0$, only solution is $\mu=0$ and since $\mu^2>0$ we have no solutions.
\end{document}
