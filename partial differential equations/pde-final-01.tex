\documentclass{article}
%\usepackage{fullpage}
\usepackage{nopageno}
\usepackage{amsmath}
\usepackage{graphicx}
\usepackage{color}
\usepackage{tabu}
\usepackage{longtable}
\usepackage{mathrsfs}
\usepackage{enumerate}
\usepackage[margin=1in]{geometry}
\usepackage{fancyhdr}
\pagestyle{fancy}
\lhead{Final 01}
\rhead{Jon Allen}
\allowdisplaybreaks

\newcommand{\abs}[1]{\left\lvert #1 \right\rvert}
\newcommand{\degree}{\ensuremath{^\circ}}

\begin{document}
\subsubsection*{PDE A.}
\begin{align*}
  \text{PDE.}&&\frac{\partial u}{\partial t}&=\frac{\partial^2u}{\partial x^2}&&\text{for}&0&<x<1,&0&<t<\infty\\
  \text{BC.}&&u_x(0,t)&=0=u_x(1,t)&&\text{for}&&&0&<t<\infty\\
  \text{IC.}&&u(x,0)&=f(x)&&\text{for}&0&<x<1
\end{align*}

For PDE A, apply separation of variables and, for separated solutions $u=T(t)X(x)$, analyze the associateed eigenvalue problem $X''(x)=\lambda X(x)$ and determine the eigenfunctions (or their nonexistence) for the cases:
\begin{align*}
  u&=T(t)X(x)\\
  \frac{\partial u}{\partial t}&=T'(t)X(x)\\
  \frac{\partial u}{\partial x}&=X'(x)T(t)\\
  \frac{\partial^2 u}{\partial x^2}&=X''(x)T(t)\\
  T'(t)X(x)&=X''(x)T(t)\\
  \intertext{$t$, and $x$ are independant of eachother, therefore:}
  \frac{T'(t)}{T(t)}&=\frac{X''(x)}{X(x)}=\lambda\\
  T'(t)-\lambda T(t)&=0\\
  \omega(t)&=e^{\int{-\lambda\,\mathrm{d}t}}\\
  \omega(t)T(t)&=\int{0\,\mathrm{d}t}=c_3\\
  T(t)&=c_3e^{\lambda t}\\
  X''(x)-\lambda X(x)&=0\\
  X''-\lambda X=0\\
  r^2+0r-\lambda&=0\\
  r&=\frac{{-0}\pm\sqrt{0^2-4(-\lambda)}}{2}\\\
  &=\pm\sqrt{\lambda}
\end{align*}
\begin{enumerate}[(a)]
\item
$\lambda=+\mu^2>0$
\begin{align*}
  r&=\pm\mu\\
  X(x)&=c_1e^{\mu x}+c_2e^{-\mu x}\\
  u_x&=X'(x)T(t)=\left(c_1\mu e^{\mu x}-c_2\mu e^{-\mu x}\right)T(t)\\
  u_x(0,t)&=0=u_x(1,t)\\
  \left(c_1\mu-c_2\mu\right)T(t)&=0=\left(c_1\mu e^\mu-c_2\mu e^{-\mu}\right)T(t)
  \intertext{note that if $T(t)=0$ then we are dealing with the trivial case $u(x,t)=0$ which is not what we are looking for, so we say that $T(t)\ne0$}
  c_1\mu-c_2\mu&=0=c_1\mu e^{\mu}-c_2\mu e^{-\mu}&\mu\ne0\\
  c_1-c_2&=0&c_1=c_2\\
  c_1e^{\mu}-c_1e^{-\mu}&=0\\
  e^\mu&=e^{-\mu}\\
  e^{2\mu}&=1\\
  \ln(e^{2\mu})&=\ln(1)=2\mu=0\\
  \mu&=0
\end{align*}
But we have defined $\mu^2>0$ so we have no solutions.
\item
$\lambda=0$
\begin{align*}
  r&=\pm\sqrt{0}=0\\
  X(x)&=(c_1+c_2x)e^{0x}=c_1+c_2x\\
  u_x(0,t)&=0=u_x(1,t)\\
  c_2T(t)&=0=c_2T(t)\\
  \intertext{Again we take $T(t)\ne0$}
  c_2&=0\\
  X(x)&=c_1&T(t)&=c_3e^{0t}=c_3\\
  u(x,t)&=c_1\cdot c_3=c_4
\end{align*}
So we have one eigenfunction, $u(x,t)=c_0$
\item
$\lambda=-\mu^2<0$
\begin{align*}
  r&=\pm\sqrt{-\mu^2}=\pm\mu i\\
  X(x)&=c_1\cos(\mu x)+c_2\sin(\mu x)\\
  X'(x)&=-c_1\mu\sin(\mu x)+c_2\mu\cos(\mu x)\\
  u_x(0,t)&=0=u_x(1,t)\\
  [c_2\mu\cos(0)-c_1\mu\sin(0)]T(t)&=0=[c_2\mu\cos(\mu)-c_1\mu\sin(\mu)]T(t)\\
  \intertext{Taking $T(t)\ne0$}
  c_2\mu&=0=c_2\mu\cos(\mu)-c_1\mu\sin(\mu)&\mu>0&\to c_2=0\\
  -c_1\mu\sin(\mu)&=0\\
  \intertext{Avoiding the trivial solution requires $\sin(\mu)=0$}
  \mu&=n\pi&n&=1,2,3,\dots\\
  T(t)&=c_3e^{-\mu^2t}=c_3e^{-n^2\pi^2t}\\
  u_n(x,t)&=c_ne^{-n^2\pi^2t}\cos(n\pi x)
\end{align*}
\end{enumerate}
\end{document}
