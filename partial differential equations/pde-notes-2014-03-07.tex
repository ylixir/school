\documentclass{article}
\usepackage{fullpage}
\usepackage{nopageno}
\usepackage{amsmath}
\usepackage{amssymb}
\allowdisplaybreaks

\newcommand{\abs}[1]{\left\lvert #1 \right\rvert}

\begin{document}
\title{Notes}
\date{March 7, 2014}
\maketitle
homework 14-15

sturm-lionville expansions. The last stage of separation of variales (on finite intevals)

\begin{align*}
  \text{PDE}&&u_t&=u_{xx}+\underbrace{f(x,t)}_{\sin(\lambda,x)}&\text{on }0<x<1\\
  \text{BC}&&u(0,t)&=0\\
  &&u_x(1,t)+u(1,t)&=0
\end{align*}
standard approach (incorrect). theeigenfunctions are $\sin(n\pi x)=X_n(x)$ do not satisfy the BC. The problem is driven by the boundary condition.

correct answer looks like this?
\begin{align*}
  u(x,t)&=\sum\limits{c_nT_n(t)X_n(x)}
\end{align*}
where $X_n(x)$ satisfy the BC. and $X_n(x)$ satisfies the separated equation $\frac{{X_n}''}{X_n(s)}=-\lambda_n$.

In lesson 9, there is a detailed example that has
\begin{align*}
  \text{BC}&&u(0,t)&=0\\
  &&u(1,t)&=0\\
  \text{PDE}&&u_t&=u_{xx}+f(x,t)
\end{align*}
solution is written out in detail and $X_n(x)=\sin(n\pi x)$ appear (because of the BC).

In class, BC looked more like (see notes on 2/7)
\begin{align*}
  \text{PDE}&&u_t&=\alpha^2u_{xx}+f(x,t)\\
  \text{BC}&&0&=\alpha_1u_x(0,t)+\beta_1u(0,t)\\
  &&0&=\alpha_2u_x(1,t)+\beta_2u(1,t)\\
  \text{IC}&&u(x,0)&=\phi(x)
\end{align*}
see page 65-66 in text (81-82). Step 1 on page 66. set $u(x,t)=\sum\limits{c_nT_n(t)X_n(x)}$ where $X_n(x)$ are eigenfunctions for the homogeneous PDE (and homogeneous BC)

Try $T(t)X(x)$. $u_t=\alpha^2u_{xx}$ gives $\frac{T'(t)}{\alpha^2T(t)}=\frac{X''(x)}{X(x)}=-\lambda^2\le0$. So $X''(x)+\lambda^2X(x)=0$.
\begin{align*}
  \text{BC}&&X(0)&=0\\
  &&X'(1)+X(1)&=0\\
  X(x)&=a\cos(\lambda x)+b\sin(\lambda x)\\
  X(0)&=0=a\cos(0)+b\sin(0)\to a=0\\
  X'(x)&=+b\lambda\cos(\lambda x)\\
  X'(1)+X(1)&=0\\
  X_n(x)&=\sin(\lambda_n x) \text{ where } \tan(\lambda)=-\lambda\\
  u(x,t)&=\sum\limits_{n=1}^\infty{T_n(t)\sin(\lambda_nx)}\\
  f(x,t)&=\sum\limits{f_n(t)\sin(\lambda_nx)}\\
  {T_n}'(t)&=\alpha_2(-{\lambda_n}^2)T_n(t)+f_n(t)
\end{align*}
coefficient of $\sin(\lambda_nx)$

we need $\lambda\cos(\lambda)+\sin(\lambda)=0, \lambda>0$. $\cos(\lambda)=0$? no, else $\sin(\lambda)=0$. So divide out by cosine to get $\lambda+\tan(\lambda)=0$. Showed that eigenfunctions $\sin(\lambda x)$ are orthogonal.

Homework due date extended to 3/14
\end{document}
