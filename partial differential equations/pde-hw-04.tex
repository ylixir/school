\documentclass{article}
\usepackage{fullpage}
\usepackage{nopageno}
\usepackage{amsmath}
\allowdisplaybreaks

\newcommand{\abs}[1]{\left\lvert #1 \right\rvert}
\newcommand{\degree}{\ensuremath{^\circ}}

\begin{document}
Jon Allen

HW 04

\begin{align*}
  u_t&=\alpha^2u_{xx} & 0<x&<1 & 0<&t<\infty\\
  u_x(0,t)&=0 & u_x(1,t)&=0 & u(x,0)&=\sin(\pi x)\\
  u_t=0&=\alpha^2u_{xx} &
  \int{0\,\mathrm{d}x}&=\int{\alpha^2u_{xx}\,\mathrm{d}x} &
  c_1&=\alpha^2u_x\\
  \int{c_1\,\mathrm{d}x}&=\int{\alpha^2u_{x}\,\mathrm{d}x} &
  c_1x+c_2&=\alpha^2U(x) &
  U(x)&=\frac{c_1}{\alpha^2}x+\frac{c_2}{\alpha^2} \\
  \intertext{Simplify constants}
  U(x)&=c_1x+c_2 & U'(x)&=c_1 & U'(0)&=U'(1)=0\\
  c_1&=0 & U(x)&=c_2
\end{align*}
If the problem is interpreted as the temperature of a rod, then the rod is completely insulated. The amount of heat in the system never changes. We know that the amount of heat initially is $\int_0^1{\sin(\pi x)\,\mathrm{d}x}$.
Because $U(x)=c_2$ we know that when the system reaches a steady state, the amount of heat is $\int_0^1{c_2\,\mathrm{d}x}$.
\begin{align*}
  \int_0^1{c_2\,\mathrm{d}x}&=\int_0^1{\sin(\pi x)\,\mathrm{d}x}\\
  \left[c_2x\right]_0^1&=\left[-\frac{1}{\pi}\cos(\pi x)\right]_0^1\\
  c_2&=-\frac{1}{\pi}(-1)--\frac{1}{\pi}(1)=\frac{2}{\pi}
\end{align*}
So our steady state is $U(x)=\frac{2}{\pi}$
\end{document}
