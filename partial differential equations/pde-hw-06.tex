\documentclass{article}
\usepackage{fullpage}
\usepackage{nopageno}
\usepackage{amsmath}
\usepackage{graphics}
\allowdisplaybreaks

\newcommand{\abs}[1]{\left\lvert #1 \right\rvert}
\newcommand{\degree}{\ensuremath{^\circ}}

\begin{document}
Jon Allen

HW 06

\begin{align*}
  \phi(x)&=\sum\limits_{n=1}^\infty{A_n\sin(n\pi x)}&A_n&=2\int_0^1{\phi(x)\sin(n\pi x)\,\mathrm{d}x}\\
  1&=\sum\limits_{n=1}^\infty{A_n\sin(n\pi x)}&A_n&=2\int_0^1{\sin(n\pi x)\,\mathrm{d}x}\\
  u&=n\pi x,\quad \mathrm{d}u=n\pi & A_n&=\frac{2}{n\pi}\int_0^1{\sin(u)\,\mathrm{d}u}\\
  A_n&=\frac{2}{n\pi}\left[-\cos u\right]_0^1=\frac{2}{n\pi}\left[-\cos(n\pi x)\right]_0^1
  &A_n&=\frac{2}{n\pi}\left[-\cos(n\pi)+\cos(0)\right]=\frac{2}{n\pi}\left(1-\cos(n\pi)\right)\\
  A_n&=\frac{2}{n\pi}\left(1-(-1)^n\right)
\end{align*}
So $A_n$ is zero for all even $n$s and $\frac{4}{n\pi}$ for odd.
\begin{align*}
  \phi(x)&=\sum\limits_{n=1}^\infty{\frac{4}{(2n-1)\pi}\sin((2n-1)\pi x)}\\
  &=\frac{4}{\pi}\left(\sin(\pi x)+\frac{1}{3}\sin(3\pi x)+\frac{1}{5}\sin(5\pi x)+\frac{1}{7}\sin(7\pi x)+\cdots\right)
\end{align*}
%\begin{figure}[tbp]
%  \begin{center}
    \input{pde-hw-06-plot-01.tex}
%    \caption{Graph caption}
%    \label{graph:graph1}
%  \end{center}
%\end{figure}
\end{document}
