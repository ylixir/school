\documentclass{article}
\usepackage{fullpage}
\usepackage{nopageno}
\usepackage{amsmath}
\usepackage{amssymb}
\usepackage{enumerate}
\usepackage{gnuplottex}
\allowdisplaybreaks

\newcommand{\abs}[1]{\left\lvert #1 \right\rvert}

\begin{document}
\title{Notes}
\date{April 4, 2014}
\maketitle
\section*{lesson 20 (continuing from last time}
\begin{align*}
  \text{PDE}&&&\;\;u_{tt}=c^2u_{xx}&0<&x<L&0<&t<\infty\\
  \text{BC}&&&\left.\begin{aligned}
    u(0,t)&=0\\
    u(L,t)&=0
  \end{aligned}\right\}&&&0<&t<\infty\\
  \text{IC}&&&\left.\begin{aligned}
    u(x,0)&=f(x)\\
    u_t(x,0)&=g(x)
  \end{aligned}\right\}&0<&x<L\\
\end{align*}
note $c^2=T/\rho$ tension over mass density.

Separated solutions $u(x,t)=T(t)X(x)$
\begin{align*}
  \text{PDE}&&&\quad\frac{T''}{c^2T}=\frac{X''}{X}=\lambda\\
  \text{cases}&&&\begin{cases}
    \lambda=\mu^2>0&\text{only trivial solutions for $X(x)$ from BC}\\
    \lambda=0&\text{$X''=0\;X=c_1+c_2x$ BC $X(0)=0=c_1\; X(L)=0=0+c_2L\to c_2=0$}\\
    \lambda=\mu^2<0&\text{$X''+\mu^2X=0\;X=c_1\cos(\mu x)+c_2\sin(\mu x)$ BC $X(0)=0=c_1\cdot1+c_2\cdot0\to c_1=0$}\\
    &X(L)=0=c_2\sin(\mu L)\to \sin(\mu L)=0
  \end{cases}
\end{align*}

So $\mu L=n\pi$ give $\mu_n=n\pi/L$ for $n=1,2,3,\dots$

Have $X_n(x)=\sin(\mu_nx)=\sin(n\pi\frac{x}{L})$ nontrivial solutions (eigenfunctions)

Use $\frac{T''(t)}{c^2T(t)}=-{\mu_n}^2$ $T''(t)+c^2{\mu_n}^2T(t)=0$

$T(t)=a_n\cos(c\mu_nt)+b_n\sin(c\mu_nt)$

Separated solutions $u_n(x,t)=\left[a_n\cos(n\pi\frac{ct}{L})+b_n\sin(n\pi\frac{x}{L})\right]\sin(n\pi\frac{x}{L}$.

$\sin(2\pi\omega t)$ where $\omega$ is frequency (oscillations/second) Hz. Frequency $\omega_n=n\frac{c}{2L}=n\sqrt{\frac{T}{\rho}}\cdot \frac{1}{2L}$ for $n=1,2,3,\dots$

n=1
\begin{gnuplot}
plot [0:1] sin(pi*x)
\end{gnuplot}

n=2
\begin{gnuplot}
plot [0:1] sin(2*pi*x)
\end{gnuplot}

n=3
\begin{gnuplot}
plot [0:1] sin(3*pi*x)
\end{gnuplot}

All solutions have periods $\omega_1$, this is why same pitch happens when plucked in different places.

\subsubsection*{page 154}

\subsubsection*{Mersenne Laws for Strings}
mid-1600s, First person to determine frequency of a pitch. middle c is $256/2^8$ Hz.
\begin{enumerate}
\item
Frequency is proportional to root of tension $\alpha\;\sqrt{T}$
\item
Frequency is inversely proportional to length. $\alpha\;1/L$
\item
Freqency is inversely proportion to root of density. $\alpha\; \frac{1}{\sqrt{\rho}}$
\end{enumerate}
Fix $L,\rho$, and for low $T$ freq$=k_0\sqrt{\frac{T}{\rho}}\frac{1}{L}$

$\sin(\omega t)$

don't forget to add -shell-escape to the plugin
\end{document}
