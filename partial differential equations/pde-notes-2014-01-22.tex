\documentclass{article}
\usepackage{fullpage}
\usepackage{nopageno}
\usepackage{amsmath}
\allowdisplaybreaks

\newcommand{\abs}[1]{\left\lvert #1 \right\rvert}

\begin{document}
\title{Notes}
\date{January 22, 2014}
\maketitle

\section*{prototype problem}
lesson 2

PDE is $u_t=\alpha^2u_{xx}$ on $0<x<L,0<t<\infty$

initial conditions $u(x,0)=?$ on $0<x<L,t=0$

boundary conditions $u(0,t)=T_1$ and $u(L,t)=T_2$ where $x=0,L$ and $0\leq t<\infty$

other types:
\begin{align*}
  u_t&=\alpha^2 u_{xx}-\beta (u-u_0), \beta > 0\\
  u_t&=\alpha^2 u_{xx}+f(x,t)\\
  u_t&=\alpha^2 u_{xx}-\omega u_x
\end{align*}

\section*{steady state solution}
a solution independant of time. $u=U(x)$

PDE becomes $0=\alpha^2 U_{xx}$
\begin{align*}
  U''(x)&=0\\
  U(x)&=c_2x+c_1\\
  U(0)&=T_1\\
  U(L)&=T_2\\
  U(x)&=T_1+c_2x\\
  U(L)&=T_2=T_1+c_2L\\
  c_2&=\frac{T_2-T_1}{L}\\
  U(x)&=T_1+\frac{T_2-T_1}{L}x
\end{align*}
\section*{energy conservation stuff}
the temperature of the bar determines the total heat energy
\begin{align*}
  E&=\int_0^L{cu(x,t),\mathrm{d}x}\\
  c&=\text{specific heat} \left(\frac{cal}{\mathrm{d}y cm}\right)\\
  \frac{\mathrm{d}E}{\mathrm{d}t}&=\frac{\mathrm{d}}{\mathrm{d}t}\int_0^L{cu,\mathrm{d}x}=c\int_0^L{\frac{\partial u}{\partial t},\mathrm{d}x}=c\int_0^L{\alpha^2 u_{xx},\mathrm{d}x}=\alpha^2 c(u_x(L,t)-u_x(0,t))
\end{align*}
for steady state: $\frac{\mathrm{d}E}{\mathrm{d}t}=\alpha^2 c(\frac{T_2-T_1}{L}-\frac{T_2-T_1}{L})=0$
\section*{lesson 3}
fourier's law of heat flow: the rate of flow of heat energy at the position $x_0$ (in positive direction in bar) is equal to $-k\frac{\partial u}{\partial x}(x_0,t)$

see page 22 in text

rate of heat flow is $\frac{\text{calories/sec}}{\text{cm}^2}=\frac{\text{cal}}{\text{sec deg cm}}\cdot\frac{\text{deg}}{\text{cm}}$

heat flow at $x=L$ is $\alpha^2 c u_x(L,t)$ and heat flow at $x=0$ is $\alpha^2 c u_2(0,t)$
\section*{boundary conditions}
at a boundary, common conditions are:
\begin{align*}
  u&=g(t)\\
  u_x+\lambda u&=g(t)\\
  u_x&=g(t) \text{most commonly} u_x=0 \text{called insulated condition}
\end{align*}
\end{document}
