\documentclass{article}
\usepackage{fullpage}
\usepackage{nopageno}
\usepackage{amsmath}
\usepackage{amssymb}
\allowdisplaybreaks

\newcommand{\abs}[1]{\left\lvert #1 \right\rvert}

\begin{document}
\title{Notes}
\date{February 24, 2014}
\maketitle

\section*{lesson 12}
\subsection*{definition}
Given $f(x)$ on $\mathbb{R}$.
\begin{align*}
  \mathcal{F}[f]&=F(\xi)=\frac{1}{\sqrt{2\pi }}\int_{-\infty}^{+\infty}{e^{-i\xi x}f(x)\,\mathrm{d}x}
\end{align*}
Inverse transform recovers $f(x)$ from $F(\xi)$:
\begin{align*}
  f(x)&=\frac{1}{\sqrt{2\pi }}\int_{-\infty}^{+\infty}{e^{+i\xi x}F(\xi)\,\mathrm{d}\xi}&\text{Deep Theorem}
\end{align*}
\subsection*{problem}
page 93
\begin{align*}
  PDE&&u_t&=\alpha ^2u_{xx}&-\infty<&x<\infty&0<&t<\infty\\
  IC&&u(x,0)&=\phi(x)&-\infty<&x<\infty
\end{align*}
apply $\mathcal{F}$ to pde. $U(\xi,t)=\mathcal{F}[u(x,t)]$. Use property 3 (derivative): $\mathcal{F}[u_{xx}]=$

How are $\mathcal{F}[f']$ and $\mathcal{F}[f]$ related? page 91.
\begin{align*}
  \mathcal{F}[f']&=\frac{1}{\sqrt{2\pi }}\int_{-\infty}^{+\infty}{e^{-i\xi x}f'(x)\,\mathrm{d}x}\\
  &=\frac{1}{\sqrt{2\pi }}\left[e^{-i\xi x}f(x)\right]_{-\infty}^{+\infty}-\int_{-\infty}^{+\infty}{(-i\xi)e^{-i\xi x}f(x)\,\mathrm{d}x}\\
  \intertext{\emph{note:} implicit conditions on $f(x)$ to insure integrals exist $f(+\infty)=f(-\infty)=0$}
  &=0+i\xi\frac{1}{\sqrt{2\pi }}\int_{-\infty}^{+\infty}{e^{-i\xi x}f(x)\,\mathrm{d}x}\\
  \intertext{property 3}
  \mathcal{F}[f']&=i\xi\mathcal{F}[f]\\
  \mathcal{F}[f'']&=\xi^2\mathcal{F}[f]
\end{align*}
So $\mathcal{F}[u_{xx}]=-\xi^2U(\xi,t)$. $\frac{\mathrm{d}U}{\mathrm{d}t}=-\alpha ^2\xi^2U$ with $U(\xi,0)=\phi(\xi)$

\subsubsection*{step 2}
solve prolem $U(\xi,t)=\phi(\xi)e^{-\alpha ^2\xi^2 t}$.
\subsubsection*{step 3}
invert transform

\subsection*{property 4}
\subsubsection*{convolution theorem}
definition: given $f(x),gx)$ on $\mathbb{R}$
\begin{align*}
  f*g(x)&=1\frac{1}{\sqrt{2\pi }}\int_{-\infty}^{+\infty}{f(x-s)g(s)\,\mathrm{d}s}\\
  \intertext{theorem}
  \mathcal{F}[f*g(x)]&=\mathcal{F}[f]\mathcal{F}[g]=F(\xi)G(\xi)&\text{deep theorem}
\end{align*}
sample calculation: find the convolution of two functions. Text example
\begin{align*}
  \left.\begin{aligned}
    f(x)&=x\\
    g(x)&=e^{-x^2}
  \end{aligned}\right\}&f*g(x)=\frac{x}{\sqrt{2}}\\
  f*g(x)&=\frac{1}{\sqrt{2\pi }}\int_{-\infty}^{+\infty}{f(x-u)g(u)\,\mathrm{d}u}\\
  &=\frac{1}{\sqrt{2\pi }}\int_{-\infty}^{+\infty}{(x-u)e^{-u^2}\,\mathrm{d}u}\\
  &=\frac{1}{\sqrt{2\pi }}\left[x\int_{-\infty}^{+\infty}{e^{-u^2}\,\mathrm{d}u}-\int_{-\infty}^{+\infty}{ue^{-u^2}\,\mathrm{d}u}\right]&\text{odd integral}\\
  &=\frac{x}{\sqrt{2\pi }}\sqrt{\pi }
  f(x)&=e^{-x^2}\\
  g(x)&=x\\
  f*g(x)&=\frac{1}{\sqrt{\pi }}\int_{-\infty}^{+\infty}{e^{-(x-u)^2}u\,\mathrm{d}u}\\
  \intertext{property:$f*g(x)=g*f(x)$ commutativity}
  f*g(x)&=\frac{1}{\sqrt{2\pi }}\int_{-\infty}^{+\infty}{f(x-u)g(u)\,\mathrm{d}u}\\
  u&=-\infty\qquad\gets x-u\\
  x-u&=+\infty
\end{align*}
note that the definition in the book has a negative $i$ and mathematica doesn't. and then that makes the signs flipped on the inverse transform as well. CAREFUL!!!
\begin{align*}
  u(x,t)&=\int_{-\infty}^{\infty}{\phi(x-u)\frac{1}{\sqrt{\pi}}\frac{1}{2\alpha \sqrt{t}}e^{-\frac{u^2}{2/\alpha ^2t}}\,\mathrm{d}u}\\
  \int_{-\infty}^{\infty}{\frac{1}{\sqrt{\pi}}\frac{1}{2\alpha \sqrt{t}}e^{-\frac{u^2}{2/\alpha ^2t}}\,\mathrm{d}u}&=\int_{-\infty}^{\infty}{\frac{1}{\sqrt{\pi}}\frac{1}{2\alpha \sqrt{t}}e^{-\frac{u^2}{2/\alpha ^2t}}\,\mathrm{d}u}\\
  u(x,t)&=\int_{-\infty}^\infty{\phi(x)f(x-u)\,\mathrm{d}u}\text{ positive with unit area}
\end{align*}
\end{document}
