%shell-escape
\documentclass{article}
\usepackage{fullpage}
\usepackage{nopageno}
\usepackage{amsmath}
\usepackage{amssymb}
\usepackage{enumerate}
\usepackage{gnuplottex}
\allowdisplaybreaks

\newcommand{\abs}[1]{\left\lvert #1 \right\rvert}

\begin{document}
\title{Notes}
\date{April 7, 2014}
\maketitle

\section*{lesson 20}
vibrating strings on the finite interval.
\begin{align*}
  \text{PDE}&&&\quad u_{tt}=c^2u_{xx}&0<&x<L&0<t<\infty\\
  \text{BC}&&&\quad u(0,t)=0=u(L,t)&&&0<t<\infty\\
  \text{IC}&&&\begin{cases}u(x,0)&=f(x)\\u_t(x,0)&=g(x)\end{cases}&0<&x<L
\end{align*}
looked for separated solutions $u=T(t)X(x)$

found $u_n(x,t)=[a_n\sin(n\pi ct/L)+b_n\cos(n\pi ct/L)]\sin(n\pi x/L)$

text p 157

General solution $u(x,t)=\sum\limits_{n=1}^\infty{[a)n\sin(n\pi ct/L)+b_n\cos(n\pi ct/L)]\sin(n\pi x/L)}$ satisfies PDE (linear/hom) and BC(linear hom)

set $t=0$ so $u(x,t)=\sum\limits_{n=1}^\infty{b_n\sin(n\pi x/L)}=f(x)$.

Recal $\int_0^L{\sin(n\pi x/L)\sin(m\pi x/L)\mathrm{d}x}=0, m\ne n$ and $\int_0^L{\sin(n\pi x/L)}=L/2$.

\begin{align*}
  \int_0^L{\sum\limits_{n=1}^\infty{b_n\sin(n\pi x/L)}\mathrm{d}x}&=\int_0^L{f(x)\mathrm{d}x}\\
  \int_0^L{\sin(m\pi x/L)\sum\limits_{n=1}^\infty{b_n\sin(n\pi x/L)}\mathrm{d}x}&=\int_0^L{\sin(m\pi x/L)f(x)\mathrm{d}x}\\
\end{align*}
blah blah, page 157
\begin{align*}
  u(x,t)&=\sum\limits_{n=1}^\infty{(a_n\sin(n\pi ct/L)+b_n\cos(n\pi ct/L))\sin(n\pi x/L}\\
  u_t(x,t)&=\sum\limits_{n=1}^\infty{(n\pi \frac{c}{L} a_n\cos(n\pi ct/L)-n\pi\frac{c}{L} b_n\sin(n\pi ct/L))\sin(n\pi x/L}\\
  u_t(x,t)&=\sum\limits_{n=1}^\infty{n\pi \frac{c}{L} a_n\sin(n\pi x/L)}=g(x)
\end{align*}
even in something like $\sin(\mu_n \frac{x}{L}$ where $\mu+tan(\mu)=0$ orthoganality will still be present. Sturm-Lionville theory gives this.
\section*{lesson 21}
the vibrating beam (4th order PDE)
\begin{align*}
  \text{PDE}&&&\quad u_{tt}=\alpha^2u_{xxxx}&0<&x<L&0<t<\infty\\
  \text{BC}&&&\quad u(0,t)=0=u(L,t)&&&0<t<\infty\\
  &&&\quad u_{xx}(0,t)=0=u_{xx}(L,t)\\
  \text{IC}&&&\begin{cases}u(x,0)&=f(x)\\u_t(x,0)&=g(x)\end{cases}&0<&x<L
\end{align*}
HW will involve exer 1 on p 166-167
$u(0,t)=0, u_{xx}(1,t)=0, u_x(1,t)=0, u_{xxx}(1,t)=0$  free end p 166.


set $u=T(t)X(x)$
\begin{align*}
  \frac{T''}{-\alpha^2T}&=\frac{X''''}{X}=\lambda&\text{separation constant}\\
  X''''-\lambda X&=0&\text{consider bc}\\
\intertext{assume $\lambda>0$ since $\lambda\le0$ cannot satisfy bc}
\lambda&=\omega^2>0\\
X^{(4)}-\omega^2X=0\\
X=e^{rx}\\
X^{(4)}&=r^4e^{rx}-\omega^2e^{rx}\\
r^4-\omega^2&=0\\
r^2&=\pm\omega\\
r&=\pm\sqrt{\omega},\pm i\sqrt{\omega}\\
e^{\pm\sqrt{\omega}x},e^{\pm i\sqrt{\omega}x}\\
X(x)&=C\cos(\sqrt{\omega}x)+D\sin(\sqrt{\omega}x)+E\cosh(\sqrt{\omega}x)+F\sinh(\sqrt{\omega}x
\intertext{apply bc}
  X(0)&=0=C+E\\
  X''&=-C\omega\cos(\sqrt{\omega}x)-D\omega\sin(\sqrt{\omega}x)+E\omega\cosh(\sqrt{\omega}x)+F\omega\sinh(\sqrt{\omega}x)\\
  X''(0)=0=-C\omega+E\omega\\
  C&=E=0\\
  \intertext{now at $x=L$}
\end{align*}
\end{document}
