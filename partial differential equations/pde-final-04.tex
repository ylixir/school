\documentclass{article}
%\usepackage{fullpage}
\usepackage{nopageno}
\usepackage{amsmath}
\usepackage{amssymb}
\usepackage{graphicx}
\usepackage{color}
\usepackage{tabu}
\usepackage{longtable}
\usepackage{mathrsfs}
\usepackage{enumerate}
\usepackage[margin=1in]{geometry}
\usepackage{fancyhdr}
\pagestyle{fancy}
\lhead{Final 04}
\rhead{Jon Allen}
\allowdisplaybreaks

\newcommand{\abs}[1]{\left\lvert #1 \right\rvert}
\newcommand{\degree}{\ensuremath{^\circ}}

\begin{document}
\subsubsection*{PDE C.}
\begin{align*}
  \text{PDE.}&&\frac{\partial u}{\partial t}&=\frac{\partial^2u}{\partial x^2}&&\text{for}&0&<x<\infty,&0&<t<\infty\\
  \text{BC.}&&\frac{\partial u}{\partial x}(0,t)&=u(0,t)-\frac{1}{\sqrt{\pi t}}&&\text{for}&&&0&<t<\infty\\
  \text{IC.}&&u(x,0)&=0&&\text{for}&0&<x<\infty
\end{align*}

Solve PDE C completely by a Laplace transform with respect to $t$. Use the BC as stated -- do not transform to homogeneous BC. (The necessary inverse Laplace transform is not in the textbook table but is on the handout list of transforms.)

\begin{align*}
  sU(x)-0&=\frac{\mathrm{d}U}{\mathrm{d}x^2}(x)\\
  \frac{\mathrm{d}U}{\mathrm{d}x}(0)&=U(0)-\mathcal{L}\left\{\frac{1}{\sqrt{\pi t}}\right\}\\
  &=U(0)-\frac{1}{\sqrt{s}}\qquad\text{used computer}\\
  0&=\frac{\mathrm{d}U}{\mathrm{d}x^2}(x)-sU(x)\\
  e^{-sx}U(x)&=c_1\\
  U(x)&=c_1e^{sx}\\
  \frac{\mathrm{d}U}{\mathrm{d}x}(x)&=c_1xe^{sx}\\
  \frac{\mathrm{d}U}{\mathrm{d}x}(0)&=c_10e^{s0}=0\\
  U(0)-\frac{1}{\sqrt{s}}&=0=c_1-\frac{1}{\sqrt{s}}\\
  U(x)&=\frac{1}{\sqrt{s}}e^{sx}\\
  \intertext{with computer, $\theta$ is heavyside step funtion}
  u(x,t)&=\frac{\theta(t+x)}{\sqrt{\pi}\sqrt{t+x}}
\end{align*}
\end{document}
